\documentclass[11pt, reqno]{amsart}

\usepackage[spanish]{babel}
\usepackage[LGR, T1]{fontenc}
\usepackage[utf8]{inputenc}

\input{../general.tex}
% \input{../graphics.tex}

\makeatletter
\def\emailaddrname{\textit{Correo electrónico}}
\def\subtitle#1{\gdef\@subtitle{#1}}
\def\@subtitle{}

% Metadata
\def\logo#1{\gdef\@logo{#1}}
\def\@logo{}
\def\institution#1{\gdef\@institution{#1}}
\def\@institution{}
\def\department#1{\gdef\@department{#1}}
\def\@department{}
\def\professor#1{\gdef\@professor{#1}}
\def\@professor{}
\def\course#1{\gdef\@course{#1}}
\def\@course{}
\def\coursecode#1{\gdef\@coursecode{#1}}
\def\@coursecode{}

\renewcommand{\maketitle}{
\begin{center}
	\small
	\renewcommand{\arraystretch}{1.2}
	\begin{tabular}{cp{.37\textwidth}p{0.44\textwidth}}
		% \hline
		\multirow{5}{*}{\includegraphics[height=2.0cm]{\@logo}}
	  & \multicolumn{2}{c}{ \makecell{{\bfseries \@institution} \\ \@department} } \\
	  % & \multicolumn{2}{c|}{{\bfseries\@institution} \\ \@department} \\
	  \cline{2-3}
	  & \textbf{Profesor:} \@professor & \textbf{Ayudante:} \authors \\
	  % \cline{2-3}
	  & \textbf{Curso:} \@course & \textbf{Sigla:} \@coursecode \\
	  % \cline{2-3}
	  & \multicolumn{2}{l}{ \textbf{Fecha:} \@date } \\
	  % \hline
	\end{tabular}
	\\[\baselineskip]
	% {}
	% \vspace{2\baselineskip}
	{\bfseries\Large\@title}
	\ifx\@subtitle\@empty\else
		\\[1ex]
		\large\mdseries\@subtitle
	\fi
\end{center}
}
\makeatother

\usepackage{multirow, makecell}

\usepackage[
	reversemp,
	letterpaper,
	% marginpar=2cm,
	% marginsep=1pt,
	margin=2.3cm
]{geometry}
\usepackage{fontawesome}
% \makeatletter
% \@reversemargintrue
% \makeatother

% Símbolos al margen, necesitan doble compilación
\newcommand{\hard}{\marginnote{\faFire}}
\newcommand{\hhard}{\marginnote{\faFire\faFire}}

% Dependencias para los teoremas
\usepackage{xifthen}
\def\@thmdep{}
\newcommand{\thmdep}[1]{
	\ifthenelse{\isempty{#1}}
	{\def\@thmdep{}}
	{\def\@thmdep{ (#1)}}
}
\newcommand{\thmstyle}{\color{thm}\sffamily\bfseries}

% ===== Estilos de Teoremas ==========
\newtheoremstyle{axiomstyle}
	{0.3cm}
	{0.3cm}
	{\normalfont}
	{0.5cm}
	{\bfseries\scshape}
	{:}
	{4pt}
	{\thmname{#1}\thmnote{ #3}\thmnumber{ (#2)}}
\newtheoremstyle{styleC}
	{0.5cm}
	{0.5cm}
	{\normalfont}
	{0.5cm}
	{\bfseries}
	{:}
	{4pt}
	{\thmname{#1\textrm{\@thmdep}}\thmnumber{ #2}\thmnote{ (#3)}}

% ====== Teoremas (sin borde) ===========
\theoremstyle{axiomstyle}
\newtheorem*{axiom}{Axioma}

% ====== Teoremas (sin borde) ==================
\theoremstyle{styleC}
\newtheorem{thm}{Teorema}[section]
\newtheorem{mydef}[thm]{Definición}
\newtheorem{prop}[thm]{Proposición}
\newtheorem{cor}[thm]{Corolario}
\newtheorem{lem}[thm]{Lema}
\newtheorem{con}[thm]{Conjetura}

\newtheorem*{prob}{Problema}
\newtheorem*{sol}{Solución}
\newtheorem*{obs}{Observación}
\newtheorem*{ex}{Ejemplo}

% \usepackage{tcolorbox}
% \newtcbox{bluebox}[1][]{enhanced jigsaw, 
%   sharp corners,
%   frame hidden,
%   nobeforeafter,
%   listing only,
%   #1} % comando para crear cajas de colores

\expandafter\let\expandafter\oldproof\csname\string\proof\endcsname
\let\oldendproof\endproof
\renewenvironment{proof}[1][\proofname]{%
  \oldproof[\scshape Demostración:]%
}{\oldendproof} % comando para redefinir la caja de la demostración
\newenvironment{hint}[1][\proofname]{%
  \oldproof[\scshape Pista:]%
}{\oldendproof} % comando para redefinir la caja de la demostración

% colores utilizados
\definecolor{numchap}{RGB}{249,133,29}
\definecolor{chap}{RGB}{6,129,204}
\definecolor{sec}{RGB}{204,0,0}
\definecolor{thm}{RGB}{106,176,240}
\definecolor{thmB}{RGB}{32,31,31}
\definecolor{part}{RGB}{212,66,66}

% ====== Diseño de los titulares ===============
\usepackage[explicit]{titlesec} % para personalizar el documento, la opción <<explicit>> hace que el texto de los titulares sea un objeto interactuable

\titleformat{\subsection}[runin]
	{\bfseries}
	{\textrm{\S}\thesubsection}
	{1ex}
	{#1.}

\setlist[enumerate,1]{label=\arabic*., ref=\arabic*} % Enumerate standards


\title{Congruencias y funciones multiplicativas}
% \subtitle{Conociendo a Catalan, Fermat y Mordell}
\date{18 de agosto de 2023}

\author[José Cuevas]{José Cuevas Barrientos}
\email{josecuevasbtos@uc.cl}
% \urladdr{https://josecuevas.xyz/teach/2025-1-ayud/}

\logo{../puc_negro.png}
\institution{Pontificia Universidad Católica de Chile}
\department{Facultad de Matemáticas}
\course{Teoría de Números}
\coursecode{MAT2225}
\professor{Héctor Pastén}

\begin{document}

\maketitle

\section{Congruencias}
% \begin{mydef}
% 	Una \strong{ecuación diofantina} es un polinomio $f(x_1, x_2, \dots, x_n) \in \Q[x_1, \dots, x_n]$ (generalmente $f(\vec x) \in \Z[\vec x]$),
% 	y sus \strong{soluciones} son las tuplas $(a_1, \dots, a_n) \in \Q^n$ tales que $f(a_1, \dots, a_n) = 0$.
% \end{mydef}
% En este cursos principalmente estudiaremos las ecuaciones diofantinas con coeficientes \textit{enteros} (en $\Z$) y sus soluciones \textit{enteras} (en $\Z$).

\noindent
Primero un par de problemas de práctica:
\begin{enumerate}
	\item (Gersónides) Las únicas potencias consecutivas de $2$ y $3$ son $1, 2, 3, 4, 8$ y $9$.

	\item Demuestre que la ecuación diofantina $x^2 + y^2 = 4z + 3$ no tiene soluciones enteras.

		% (Ver mód 4).
	% \item Demuestre que $7 \mid 2222^{5555} + 5555^{2222}$.
\end{enumerate}
Ahora subamos de nivel:
\begin{enumerate}[resume]
	\item Considere la sucesión
		\[
			q_n := \underbrace{33\dots 33}_{n\text{ veces}} 1.
		\]
		Demuestre que contiene infinitos números compuestos.

		Lo divertido de la sucesión es que $q_1, q_2, \dots, q_7$ son todos primos y $q_8 = 17 \cdot 19607843$ es el primer número compuesto en ella.

		% \begin{proof}
		% 	En primer lugar escribamos $q_n$ de manera cerrada:
		% 	$$ q_n = 30(1 + \cdots + 10^{n-1}) + 1 = 30 \cdot \frac{10^n - 1}{10 - 1} + 1 = \frac{10}{3}(10^n - 1) + 1. $$
		% 	Ahora sabemos que $17 \mid q_8$ (?!), por lo que podemos mirar la sucesión módulo 17.
		% 	Por el pequeño teorema de Fermat $10^{16} \equiv 1 \pmod{17}$, por lo que $10^{8 + m16} \equiv 10^8 \pmod{17}$
		% 	y, por tanto, $q_{8 + m16} \equiv q_8 \equiv 0 \pmod{17}$.
		% \end{proof}
	\item
		\hard
		(Japón 1999) Sea $f(x) := x^3 + 17$.
		Demuestre que para todo natural $n \ge 2$ existe un entero $x$ tal que $3^n \mid f(x)$ pero $3^{n+1} \nmid f(x)$.
\end{enumerate}

% \textbf{Problemas abiertos:}
% \begin{itemize}
% 	\item ¿Habrá infinitos primos de Mersenne?
% 	\item ¿Existen números impares perfectos?
% 		Por computación sabemos que, de existir, han de ser mayores que $10^{1500}$ (cfr. \citeauthor{ochem2012perfect}~\cite{ochem2012perfect}).
% 	\item ¿Existen números $n$ tales que $\sigma(n) = 2n+1$?
% \end{itemize}

\section{Funciones aritméticas y multiplicativas}
Recuérdese:
\begin{mydef}
	Una \strong{función aritmética}\index{función!aritmética} es una función $f \colon \N_{\ne 0} \to \C$.
	Una función aritmética no nula se dice:
	\begin{description}
		\item[Completamente multiplicativa] Si para todo $n, m \in \N_{\ne 0}$ se cumple que $f(nm) = f(n)f(m)$.
		\item[Multiplicativa] Si para todo par de naturales $n, m \in \N_{\ne 0}$ coprimos se cumple que $f(nm) = f(n)f(m)$.
	\end{description}
\end{mydef}

\begin{enumerate}[resume]
	\item (Prueba de sanidad) Sea $f$ una función aritmética. Demuestre:
		\begin{enumerate}
			\item Si $f$ es multiplicativa, entonces $f(1) = 1$.
			\item Si $f$ es multiplicativa, entonces está totalmente determinado por los valores que toma $f(p^\alpha)$ para todo primo $p$
				y todo exponente $\alpha > 1$.
				En cuyo caso, dados $p_1, \dots, p_m$ primos distintos y $\alpha_i \in \N$ se tiene
				$$ f(p_1^{\alpha_1} \cdots p_m^{\alpha_m}) = f(p_1^{\alpha_1}) \cdot f(p_2^{\alpha_2}) \cdots f(p_m^{\alpha_m}). $$
			\item Si $f$ es completamente multiplicativa, entonces está totalmente determinado por los valores que toma $f(p)$ para todo primo $p$.
				En cuyo caso, dados $p_1, \dots, p_m$ primos distintos y $\alpha_i \in \N$ se tiene
				$$ f(p_1^{\alpha_1} \cdots p_m^{\alpha_m}) = f(p_1)^{\alpha_1} \cdot f(p_2)^{\alpha_2} \cdots f(p_m)^{\alpha_m}. $$
		\end{enumerate}

	\item Demuestre que las siguientes son funciones multiplicativas:
		\begin{enumerate}
			% \item La función $\phi$ de Euler, dada por $\phi(n)$ es la cantidad de naturales coprimos a $n$ menores que $n$.
			\item La función de Möbius, dada por
				$$ \mu(n) :=
				\begin{cases}
					(-1)^r, & \text{si $n = p_1 \cdots p_r$, con $p_i$ primos distintos,} \\
					0, & \text{si existe un primo $p$ tal que $p^2 \mid n$.}
				\end{cases} $$
			\item Las funciones de la forma:
				$$ \sigma_s(n) := \sum_{d\mid n} d^s, \qquad s \in \Z_{\ge 0}. $$
				En particular, denotamos $\tau(n) := \sigma_0(n)$ la función que cuenta la cantidad de divisores de $n$;
				y $\sigma(n) := \sigma_1(n)$ la función que suma los divisores de $n$.
		\end{enumerate}

	\item (Corolario) Sea $n \in \N$ con factorización prima $n = p_1^{\alpha_1} \cdots p_m^{\alpha_m}$.
		Concluya lo siguiente:
		\begin{enumerate}
			\item La cantidad de divisores que posee es
				$$ \tau(n) = (\alpha_1 + 1)(\alpha_2 + 1) \cdots (\alpha_m + 1). $$
			\item La suma de sus divisores es
				$$ \sigma(n) = \frac{p_1^{\alpha_1 + 1} - 1}{p_1 - 1} \cdot \frac{p_2^{\alpha_2 + 1} - 1}{p_2 - 1}
				\cdots \frac{p_m^{\alpha_m + 1} - 1}{p_m - 1}. $$
		\end{enumerate}
\end{enumerate}
\begin{mydef}
	Se dice que un número natural $n$ es \strong{perfecto} si es igual a la suma de los divisores menores que él,
	o equivalentemente, si $\sigma(n) = 2n$.
\end{mydef}
Un ejemplo de un número perfecto es el 6 pues $6 = 1 + 2 + 3$.

% \begin{center}
% 	\itshape
% 	El objetivo de los siguientes problemas es clasificar los números pares perfectos.
% \end{center}
\begin{enumerate}[resume]
	\item
		(Euclides-Euler) Demuestre que un número par es perfecto syss es de la forma $2^{n-1}(2^n - 1)$,
		donde $p := 2^n - 1$ es un número primo.
		(Los primos de la forma $2^n - 1$ se dicen \emph{primos de Mersenne}.)

		% \begin{hint}
		% 	Para <<$\implies$>> demuestre que si $m$ es impar y $\sigma(m) = 2^{n+1}a$ con $a$ impar,
		% 	entonces $m = (2^{n+1} - 1)a$
		% \end{hint}
	\item (Euler) Demuestre que un número impar perfecto es de la forma $p^r m^2$ donde $p \nmid m$ y $p \equiv r \equiv 1 \pmod 4$.
\end{enumerate}

\section*{Problemas abiertos}
\begin{itemize}
	\item ¿Habrán infinitos primos de Mersenne?
	\item ¿Existen números impares perfectos?
		Por computación sabemos que, de existir, han de ser mayores que $10^{1500}$ (cfr. \citeauthor{ochem2012perfect}~\cite{ochem2012perfect}).
		
		La expresión del enunciado 9 puede refinarse, uno puede demostrar (sin tanto esfuerzo) que $r = 1$
		(cfr. \citeauthor{desouza2018odd}~\cite{desouza2018odd}).
	\item ¿Existen números $n$ tales que $\sigma(n) = 2n+1$?
\end{itemize}

\nocite{burton:elementary, andreescu:problems}
\printbibliography[title={Referencias y lecturas adicionales}]

\end{document}
