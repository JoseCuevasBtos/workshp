\documentclass[11pt, reqno]{amsart}

\usepackage[spanish]{babel}
\usepackage[LGR, T1]{fontenc}
\usepackage[utf8]{inputenc}

../LaTeX/general.tex
% \input{../graphics.tex}

\makeatletter
\def\emailaddrname{\textit{Correo electrónico}}
\def\subtitle#1{\gdef\@subtitle{#1}}
\def\@subtitle{}

% Metadata
\def\logo#1{\gdef\@logo{#1}}
\def\@logo{}
\def\institution#1{\gdef\@institution{#1}}
\def\@institution{}
\def\department#1{\gdef\@department{#1}}
\def\@department{}
\def\professor#1{\gdef\@professor{#1}}
\def\@professor{}
\def\course#1{\gdef\@course{#1}}
\def\@course{}
\def\coursecode#1{\gdef\@coursecode{#1}}
\def\@coursecode{}

\renewcommand{\maketitle}{
\begin{center}
	\small
	\renewcommand{\arraystretch}{1.2}
	\begin{tabular}{cp{.37\textwidth}p{0.44\textwidth}}
		% \hline
		\multirow{5}{*}{\includegraphics[height=2.0cm]{\@logo}}
	  & \multicolumn{2}{c}{ \makecell{{\bfseries \@institution} \\ \@department} } \\
	  % & \multicolumn{2}{c|}{{\bfseries\@institution} \\ \@department} \\
	  \cline{2-3}
	  & \textbf{Profesor:} \@professor & \textbf{Ayudante:} \authors \\
	  % \cline{2-3}
	  & \textbf{Curso:} \@course & \textbf{Sigla:} \@coursecode \\
	  % \cline{2-3}
	  & \multicolumn{2}{l}{ \textbf{Fecha:} \@date } \\
	  % \hline
	\end{tabular}
	\\[\baselineskip]
	% {}
	% \vspace{2\baselineskip}
	{\bfseries\Large\@title}
	\ifx\@subtitle\@empty\else
		\\[1ex]
		\large\mdseries\@subtitle
	\fi
\end{center}
}
\makeatother

\usepackage{multirow, makecell}

\usepackage[
	reversemp,
	letterpaper,
	% marginpar=2cm,
	% marginsep=1pt,
	margin=2.3cm
]{geometry}
\usepackage{fontawesome}
% \makeatletter
% \@reversemargintrue
% \makeatother

% Símbolos al margen, necesitan doble compilación
\newcommand{\hard}{\marginnote{\faFire}}
\newcommand{\hhard}{\marginnote{\faFire\faFire}}

% Dependencias para los teoremas
\usepackage{xifthen}
\def\@thmdep{}
\newcommand{\thmdep}[1]{
	\ifthenelse{\isempty{#1}}
	{\def\@thmdep{}}
	{\def\@thmdep{ (#1)}}
}
\newcommand{\thmstyle}{\color{thm}\sffamily\bfseries}

% ===== Estilos de Teoremas ==========
\newtheoremstyle{axiomstyle}
	{0.3cm}
	{0.3cm}
	{\normalfont}
	{0.5cm}
	{\bfseries\scshape}
	{:}
	{4pt}
	{\thmname{#1}\thmnote{ #3}\thmnumber{ (#2)}}
\newtheoremstyle{styleC}
	{0.5cm}
	{0.5cm}
	{\normalfont}
	{0.5cm}
	{\bfseries}
	{:}
	{4pt}
	{\thmname{#1\textrm{\@thmdep}}\thmnumber{ #2}\thmnote{ (#3)}}

% ====== Teoremas (sin borde) ===========
\theoremstyle{axiomstyle}
\newtheorem*{axiom}{Axioma}

% ====== Teoremas (sin borde) ==================
\theoremstyle{styleC}
\newtheorem{thm}{Teorema}[section]
\newtheorem{mydef}[thm]{Definición}
\newtheorem{prop}[thm]{Proposición}
\newtheorem{cor}[thm]{Corolario}
\newtheorem{lem}[thm]{Lema}
\newtheorem{con}[thm]{Conjetura}

\newtheorem*{prob}{Problema}
\newtheorem*{sol}{Solución}
\newtheorem*{obs}{Observación}
\newtheorem*{ex}{Ejemplo}

% \usepackage{tcolorbox}
% \newtcbox{bluebox}[1][]{enhanced jigsaw, 
%   sharp corners,
%   frame hidden,
%   nobeforeafter,
%   listing only,
%   #1} % comando para crear cajas de colores

\expandafter\let\expandafter\oldproof\csname\string\proof\endcsname
\let\oldendproof\endproof
\renewenvironment{proof}[1][\proofname]{%
  \oldproof[\scshape Demostración:]%
}{\oldendproof} % comando para redefinir la caja de la demostración
\newenvironment{hint}[1][\proofname]{%
  \oldproof[\scshape Pista:]%
}{\oldendproof} % comando para redefinir la caja de la demostración

% colores utilizados
\definecolor{numchap}{RGB}{249,133,29}
\definecolor{chap}{RGB}{6,129,204}
\definecolor{sec}{RGB}{204,0,0}
\definecolor{thm}{RGB}{106,176,240}
\definecolor{thmB}{RGB}{32,31,31}
\definecolor{part}{RGB}{212,66,66}

% ====== Diseño de los titulares ===============
\usepackage[explicit]{titlesec} % para personalizar el documento, la opción <<explicit>> hace que el texto de los titulares sea un objeto interactuable

\titleformat{\subsection}[runin]
	{\bfseries}
	{\textrm{\S}\thesubsection}
	{1ex}
	{#1.}

\setlist[enumerate,1]{label=\arabic*., ref=\arabic*} % Enumerate standards

\DeclareFieldFormat[book]{title}{\textit{#1}\addperiod}

\title{Dinámicas aritméticas}
% \subtitle{Guía para el autoestopista diofántico}
\date{10 de noviembre de 2023}

\DeclareMathOperator{\PGL}{PGL}
\DeclareMathOperator{\Per}{Per}

\author[José Cuevas]{José Cuevas Barrientos}
\email{josecuevasbtos@uc.cl}

\logo{../puc_negro.png}
\institution{Pontificia Universidad Católica de Chile}
\department{Facultad de Matemáticas}
\course{Teoría de Números}
\coursecode{MAT2225}
\professor{Héctor Pastén}

\begin{document}

\maketitle

\section{Dinámicas aritméticas}
Para no complicarnos la vida, diremos que un morfismo $f\colon \PP^1 \to \PP^1$ es una función que en las cartas afines (es decir, sobre puntos de la forma
$[x : 1]$ e $[1 : y]$) viene dado por funciones racionales (i.e., fracciones formales de polinomios).
Si estas fracciones tienen ceros en los denominadores, entonces diremos que determina una aplicación racional $f \colon \PP^1 \dashto \PP^1$ (y se denotara con
esta flecha quebrada).
Nótese que todo morfismo está dado por funciones racionales, necesariamente del mismo grado y homogéneas (para que estén bien definidas en la recta proyectiva);
llamaremos el \strong{grado} de la aplicación racional al grado de las funciones que le definen por coordenadas.

Nótese que toda aplicación racional de $\PP^1(\C)$ de grado 1 viene dada por:
% llamadas \strong{transformaciones de Möbius}:
$$ \mu_{a,b,c,d}([x : y]) := [ax + by : cx + dy], \qquad a,b,c,d \in \C. $$
Donde excluimos la posibilidad de que $a = b = c = d = 0$.
\begin{enumerate}
	\item Demuestre que las siguientes condiciones son equivalentes:
		\begin{enumerate}
			\item $\mu_{a,b,c,d}$ es un morfismo (¡y no una aplicación racional!).
			\item $\mu_{a,b,c,d}$ es no constante.
			\item $\mu_{a,b,c,d}$ es inyectivo.
			\item $\mu_{a,b,c,d}$ es sobreyectivo.
			\item $ad - bc \ne 0$.
		\end{enumerate}
		Concluya que los automorfismos (i.e., isomorfismos de $\PP^1 \to \PP^1$) de grado 1 están en biyección con
		$$ \PGL_2(\C) := \GL_2(\C) / \C^\times. $$
		Los elementos de $\PGL_2$ se dicen \strong{transformaciones de Möbius}.
	\item Sean $(\alpha_1, \alpha_2, \alpha_3)$ y $(\beta_1, \beta_2, \beta_3)$ dos ternas de puntos distintos de $\PP^1$.
		Demuestre que existe una transformación de Möbius $\mu \in \PGL_2(\C)$ tal que cada $\mu(\alpha_i) = \beta_i$.
\end{enumerate}

\begin{mydef}
	Sea $f \colon X \to X$ una función sobre un conjunto cualquiera.
	Dado un punto $x \in X$ su \strong{órbita} es%
	\footnote{Otros textos también emplean $\mathcal{O}_f(x)$ u $O_f^+(x)$.}
	\[
		f^\N(x) := \{ f^n(x) : n \in \N \},
	\]
	donde $f^n$ denota la composición $n$ veces y donde $f^0 := \Id_X$.
	Denotaremos
	$$ \Per_n(f) := \{ x \in X : f^n(x) = x \}, \qquad \Per_n^{**}(f) := \{ x \in \Per_n(f) : \forall 0 < m < n, \quad x \notin \Per_m(f) \}. $$
	Se dice que $x$ es \strong{periódico} si $x \in \bigcup_{n=1}^\infty \Per_n(f)$.
	Se dice que $x$ es \strong{preperiódico} si su órbita $f^\N(x)$ es finita, de lo contrario se dice que $x$ es un \strong{punto errante}.
	Se dice que $x$ es \strong{estrictamente preperiódico} si es preperiódico, pero no periódico.
\end{mydef}
\begin{enumerate}[resume]
	\item Sea $\varphi(z) \in \C(z)$ una función racional de grado (geométrico)%
		\footnote{El \textit{grado geométrico} de una función racional $\varphi(z) = g(z)/h(z)$, donde $g, h \in \C[z]$ son polinomios coprimos
		es $\max\{ \deg g, \deg h \}$.}
		$d \ge 2$.
		\begin{enumerate}
			\item Demuestre que $|\Per_n(f)| \le d^n + 1$.
			\item Demuestre que $\lim_n |\Per_n(f)| = \infty$.
			\item Concluya que $\Per_n^{**}(f)$ no es vacío para infinitos $n$'s.
		\end{enumerate}

	\item Dada una curva elíptica $\mathcal{E} \colon y^2 = x^3 + Ax^2 + Bx + C$ con $A,B,C \in \Q$ en forma de Weierstrass,
		la fórmula explícita para la duplicación de un punto con coordenadas $P := [u : v : 1]$ es
		$$ x(2\cdot P) = \frac{x^4 - b_4x^2 - 2b_6 x - b_8}{4x^3 + b_2x^2 + 2b_4x + b_6}, $$
		donde los $b_i$'s son racionales en función de $A, B, C$.
		Se pueden explicitar como $b_2 = 4A, b_4 = 2B, b_6 = 4C, b_8 = B^2 - 4AC$.
		La fórmula anterior se conoce como \strong{fórmula de duplicación}.

		Demuestre que $\mathcal{E}$ posee finitos puntos \textit{racionales} de torsión cuyo orden sea de la forma $2^n$.

	\item Sea $f(x) \in \Z[x]$ un polinomio tal que el 0 sea un punto estrictamente preperiódico de $f$.
		Denotemos $\ell(f) := \mcm(f(0), f^2(0))$; para todo punto errante $x_0$ definamos:
		$$ a_n := \frac{f^n(x_0)}{\mcd( f^n(x_0), \ell(f) )}. $$
		\begin{enumerate}
			\item \hard
				Demuestre que $(a_n)_n$ es una sucesión de enteros coprimos dos a dos.
			\item Con ello dé una nueva demostración de la infinitud de primos.
		\end{enumerate}
\end{enumerate}

\section{Comentarios adicionales}
En el ejercicio 4 vimos un caso muy particular de torsión de una curva elíptica.
Uno igual puede llevar el argumento más al extremo empleando fórmulas de para calcular $n$ veces un punto $P$.
Éstas fórmulas existen y vienen dadas por los llamados \strong{polinomios de división}
(cfr. \citeauthor{silverman:elliptic}~\cite[105-106]{silverman:elliptic}, ex.~3.7).
Aunque el argumento general que se emplea es identificando a una curva elíptica (¡sobre $\C$!) con un cociente de grupos topológicos $\C/\Lambda$,
donde $\Lambda$ es un reticulado pleno (i.e., es de la forma $\Lambda = \alpha\Z + \beta\Z$, donde $\alpha, \beta \in \C^\times$ son complejos no nulos
tales que $\alpha/\beta \notin \R$); con ello no solo se concluye finitud general de la torsión, sino que la torsión (¡en $\C$!) se puede calcular
completamente y $E[m] \cong (\Z/m\Z)^2$ (cfr. \cite[106]{silverman:elliptic}, ex.~3.8).

El último ejercicio fue una idea original de \citeauthor{granville2018dynamical}~\cite{granville2018dynamical}.

\nocite{silverman:dynamical}
\printbibliography[title={Referencias y lecturas adicionales}]

\end{document}
