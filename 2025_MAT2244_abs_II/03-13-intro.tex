\documentclass[11pt, reqno]{amsart}

\usepackage[spanish]{babel}
\usepackage[LGR, T1]{fontenc}
\usepackage[utf8]{inputenc}

../LaTeX/general.tex
% \input{../graphics.tex}

\makeatletter
\def\emailaddrname{\textit{Correo electrónico}}
\def\subtitle#1{\gdef\@subtitle{#1}}
\def\@subtitle{}

% Metadata
\def\logo#1{\gdef\@logo{#1}}
\def\@logo{}
\def\institution#1{\gdef\@institution{#1}}
\def\@institution{}
\def\department#1{\gdef\@department{#1}}
\def\@department{}
\def\professor#1{\gdef\@professor{#1}}
\def\@professor{}
\def\course#1{\gdef\@course{#1}}
\def\@course{}
\def\coursecode#1{\gdef\@coursecode{#1}}
\def\@coursecode{}

\renewcommand{\maketitle}{
\begin{center}
	\small
	\renewcommand{\arraystretch}{1.2}
	\begin{tabular}{cp{.37\textwidth}p{0.44\textwidth}}
		% \hline
		\multirow{5}{*}{\includegraphics[height=2.0cm]{\@logo}}
	  & \multicolumn{2}{c}{ \makecell{{\bfseries \@institution} \\ \@department} } \\
	  % & \multicolumn{2}{c|}{{\bfseries\@institution} \\ \@department} \\
	  \cline{2-3}
	  & \textbf{Profesor:} \@professor & \textbf{Ayudante:} \authors \\
	  % \cline{2-3}
	  & \textbf{Curso:} \@course & \textbf{Sigla:} \@coursecode \\
	  % \cline{2-3}
	  & \multicolumn{2}{l}{ \textbf{Fecha:} \@date } \\
	  % \hline
	\end{tabular}
	\\[\baselineskip]
	% {}
	% \vspace{2\baselineskip}
	{\bfseries\Large\@title}
	\ifx\@subtitle\@empty\else
		\\[1ex]
		\large\mdseries\@subtitle
	\fi
\end{center}
}
\makeatother

\usepackage{multirow, makecell}

\usepackage[
	reversemp,
	letterpaper,
	% marginpar=2cm,
	% marginsep=1pt,
	margin=2.3cm
]{geometry}
\usepackage{fontawesome}
% \makeatletter
% \@reversemargintrue
% \makeatother

% Símbolos al margen, necesitan doble compilación
\newcommand{\hard}{\marginnote{\faFire}}
\newcommand{\hhard}{\marginnote{\faFire\faFire}}

% Dependencias para los teoremas
\usepackage{xifthen}
\def\@thmdep{}
\newcommand{\thmdep}[1]{
	\ifthenelse{\isempty{#1}}
	{\def\@thmdep{}}
	{\def\@thmdep{ (#1)}}
}
\newcommand{\thmstyle}{\color{thm}\sffamily\bfseries}

% ===== Estilos de Teoremas ==========
\newtheoremstyle{axiomstyle}
	{0.3cm}
	{0.3cm}
	{\normalfont}
	{0.5cm}
	{\bfseries\scshape}
	{:}
	{4pt}
	{\thmname{#1}\thmnote{ #3}\thmnumber{ (#2)}}
\newtheoremstyle{styleC}
	{0.5cm}
	{0.5cm}
	{\normalfont}
	{0.5cm}
	{\bfseries}
	{:}
	{4pt}
	{\thmname{#1\textrm{\@thmdep}}\thmnumber{ #2}\thmnote{ (#3)}}

% ====== Teoremas (sin borde) ===========
\theoremstyle{axiomstyle}
\newtheorem*{axiom}{Axioma}

% ====== Teoremas (sin borde) ==================
\theoremstyle{styleC}
\newtheorem{thm}{Teorema}[section]
\newtheorem{mydef}[thm]{Definición}
\newtheorem{prop}[thm]{Proposición}
\newtheorem{cor}[thm]{Corolario}
\newtheorem{lem}[thm]{Lema}
\newtheorem{con}[thm]{Conjetura}

\newtheorem*{prob}{Problema}
\newtheorem*{sol}{Solución}
\newtheorem*{obs}{Observación}
\newtheorem*{ex}{Ejemplo}

% \usepackage{tcolorbox}
% \newtcbox{bluebox}[1][]{enhanced jigsaw, 
%   sharp corners,
%   frame hidden,
%   nobeforeafter,
%   listing only,
%   #1} % comando para crear cajas de colores

\expandafter\let\expandafter\oldproof\csname\string\proof\endcsname
\let\oldendproof\endproof
\renewenvironment{proof}[1][\proofname]{%
  \oldproof[\scshape Demostración:]%
}{\oldendproof} % comando para redefinir la caja de la demostración
\newenvironment{hint}[1][\proofname]{%
  \oldproof[\scshape Pista:]%
}{\oldendproof} % comando para redefinir la caja de la demostración

% colores utilizados
\definecolor{numchap}{RGB}{249,133,29}
\definecolor{chap}{RGB}{6,129,204}
\definecolor{sec}{RGB}{204,0,0}
\definecolor{thm}{RGB}{106,176,240}
\definecolor{thmB}{RGB}{32,31,31}
\definecolor{part}{RGB}{212,66,66}

% ====== Diseño de los titulares ===============
\usepackage[explicit]{titlesec} % para personalizar el documento, la opción <<explicit>> hace que el texto de los titulares sea un objeto interactuable

\titleformat{\subsection}[runin]
	{\bfseries}
	{\textrm{\S}\thesubsection}
	{1ex}
	{#1.}

\setlist[enumerate,1]{label=\arabic*., ref=\arabic*} % Enumerate standards


\title{Números algebraicos}
\date{\DTMdate{2025-03-13}}

\author{José Cuevas Barrientos}
\email{josecuevasbtos@uc.cl}
\urladdr{https://josecuevas.xyz/teach/2025-1-ayud/}

\logo{../puc_negro.png}
\institution{Pontificia Universidad Católica de Chile}
\department{Facultad de Matemáticas}
\course{Álgebra abstracta II}
\coursecode{MAT2244}
\professor{Héctor Pastén Vásquez}

\begin{document}

\maketitle

A lo largo de las ayudantías trataré de incluír comentarios o problemas especiales.
Los problemas difíciles tendrán ojos asustados {\straighteyes},
los comentarios que son opcionales u omitibles tendrán ojos hastiados {\upeyes}
y los comentarios \textbf{importantes} tendrán ojos interesados {\righteyes}.

\section{Números algebraicos}
\begin{enumerate}
	\item (Examen de lucidez)
		\begin{enumerate}
			\item Pruebe que, para toda extensión $K/\Q$ cuadrática, existe un entero $d \in \Z$ libre de cuadrados
				(i.e., si un primo $p \mid d$, entonces $p^2 \nmid d$) tal que $K = \Q(\sqrt{d})$.
			\item Pruebe que si $d_1 \ne d_2$ son dos enteros libres de cuadrados distintos, entonces
				$\Q(\sqrt{d_1}) \ne \Q(\sqrt{d_2})$.
			% \item\lookst
			% 	¿La afirmación del inciso 1 también vale para extensiones cúbicas?
			% 	A decir, dada una extensión cúbica $K/\Q$, ¿existe un entero libre de cubos $d \in \Z$ tal que
			% 	$K = \Q(\sqrt[3]{d})$?
		\end{enumerate}

	\item Pruebe que si un número complejo $x + \ui y \in \C$ con $x, y \in \R$ es ($\Q$-)algebraico,
		entonces $x$ e $y$ son algebraicos.

	\item Pruebe que para todo racional $r \in \Q$, los números $\sin(r\pi), \cos(r\pi) \in \C$ son algebraicos sobre $\Q$.

	% \item Sea $k$ un cuerpo arbitrario, sea $k(t) := \Frac k[t]$ el cuerpo de fracciones del anillo de polinomios.
	% 	\begin{enumerate}
	% 		\item Pruebe que la extensión $k(t)/k$ no es algebraica.
	% 		\item Sea $f \in k(t)$ una función racional (i.e., una fracción formal de polinomios) no constante,
	% 			pruebe que la extensión $k(t)/k(f)$ es finita.
	% 		\item Calcule $[k(t) : k(f)]$.
	% 	\end{enumerate}
	% \item Un cuerpo $k$ se dice \strong{algebraicamente cerrado} si, equivalentemente:
	% 	\begin{enumerate}
	% 		\item Toda extensión algebraica $K/k$ es tal que $K = k$.
	% 		\item Todo polinomio no constante de $k$ posee una raíz en $k$.
	% 		\item Los únicos polinomios irreducibles son los lineales.
	% 	\end{enumerate}
	% 	Pruebe que un cuerpo finito no puede ser algebraicamente cerrado.

	% \item Sea $L = K(\alpha)$ una extensión finita de grado impar.
	% 	Pruebe que $L = K(\alpha^2)$.

	% \item Sea $\Omega/k$ una extensión de cuerpos con extensiones intermedias $k \subseteq K, L \subseteq \Omega$.
	% 	Pruebe que
	% 	\[
	% 		[KL : k] \le [K : k] \, [L : k],
	% 	\]
	% 	y que se alcanza igualdad cuando $[K : k]$ y $[L : k]$ son coprimos.

	% \item Sea $L/k$ una extensión algebraica (no necesariamente finita).
	% 	Pruebe que todo subanillo $k \subseteq A \subseteq L$ es, de hecho, una extensión intermedia de cuerpos.

	\item Defina $\alpha := \sqrt{2} + \sqrt{3} \in \C$.
		\begin{enumerate}
			\item Encuentre un polinomio $f(x) \in \Q[x]$ de grado 4 tal que $f(\alpha) = 0$.
			\item Verifique que $f(x)$ no tiene raíces racionales, de modo que si $p(x)$ es un polinomio irreducible tal que $p(\alpha) = 0$,
				entonces $p$ debe ser cuadrático.
			\item Verifique que $\sqrt{2} + \sqrt{3} \notin \Q(\sqrt{6})$ y, por tanto, $f(x) = p(x)$ ya era irreducible.
		\end{enumerate}
\end{enumerate}

\begin{prob}
	\lookst
	¿Toda extensión cúbica $K/\Q$ es de la forma $K = \Q(\sqrt[3]{n})$ para algún $n \in \Z$?
\end{prob}

\section{Cuerpos finitos}
\begin{enumerate}[resume]
	\item Sea $k$ un cuerpo finito. Pruebe lo siguiente:
		\begin{enumerate}
			\item Su característica $\car k = p$ es un número primo.
			\item Su cardinalidad $|k| = p^n$ es una potencia de $p = \car k$ con exponente $n \ge 1$.
			\item Cada elemento \emph{siempre} tiene raíz $p$-ésima.
		\end{enumerate}

	\item Sea $k$ un cuerpo de característica $p := \car k > 0$.
		Sea $q := p^n$ con $n \ge 1$ a elección.
		Pruebe que
		\[
			k^q := \{ \alpha^q : \alpha \in k \} \subseteq k
		\]
		es un subcuerpo de $k$.

	\item \lookright
		Pruebe que para todo cuerpo $k$ existen infinitos polinomios irreducibles con coeficientes en $k$.

	% \item Sea $K/k$ una extensión algebraica de cuerpos.
	% 	Un elemento $\alpha \in K$ se dice \strong{puramente inseparable} si su polinomio minimal $f(x) \in k[x]$
	% 	es una potencia del monomio $x - \alpha$.
	% 	Pruebe que
	% 	\[
	% 		K_{\rm ins} = \{ \alpha \in K : \alpha \text{ es puramente inseparable} \}
	% 	\]
	% 	es un subcuerpo de $K$.

	% \item Sea $k$ un cuerpo finito.
	% 	Pruebe que el grupo de unidades $k^\times$ es cíclico.
\end{enumerate}

\nocite{lang:algebra}

\printbibliography

\end{document}
