\documentclass[11pt, reqno]{amsart}

\usepackage[spanish]{babel}
\usepackage[LGR, T1]{fontenc}
\usepackage[utf8]{inputenc}

\input{../general.tex}
% \input{../graphics.tex}

\makeatletter
\def\emailaddrname{\textit{Correo electrónico}}
\def\subtitle#1{\gdef\@subtitle{#1}}
\def\@subtitle{}

% Metadata
\def\logo#1{\gdef\@logo{#1}}
\def\@logo{}
\def\institution#1{\gdef\@institution{#1}}
\def\@institution{}
\def\department#1{\gdef\@department{#1}}
\def\@department{}
\def\professor#1{\gdef\@professor{#1}}
\def\@professor{}
\def\course#1{\gdef\@course{#1}}
\def\@course{}
\def\coursecode#1{\gdef\@coursecode{#1}}
\def\@coursecode{}

\renewcommand{\maketitle}{
\begin{center}
	\small
	\renewcommand{\arraystretch}{1.2}
	\begin{tabular}{cp{.37\textwidth}p{0.44\textwidth}}
		% \hline
		\multirow{5}{*}{\includegraphics[height=2.0cm]{\@logo}}
	  & \multicolumn{2}{c}{ \makecell{{\bfseries \@institution} \\ \@department} } \\
	  % & \multicolumn{2}{c|}{{\bfseries\@institution} \\ \@department} \\
	  \cline{2-3}
	  & \textbf{Profesor:} \@professor & \textbf{Ayudante:} \authors \\
	  % \cline{2-3}
	  & \textbf{Curso:} \@course & \textbf{Sigla:} \@coursecode \\
	  % \cline{2-3}
	  & \multicolumn{2}{l}{ \textbf{Fecha:} \@date } \\
	  % \hline
	\end{tabular}
	\\[\baselineskip]
	% {}
	% \vspace{2\baselineskip}
	{\bfseries\Large\@title}
	\ifx\@subtitle\@empty\else
		\\[1ex]
		\large\mdseries\@subtitle
	\fi
\end{center}
}
\makeatother

\usepackage{multirow, makecell}

\usepackage[
	reversemp,
	letterpaper,
	% marginpar=2cm,
	% marginsep=1pt,
	margin=2.3cm
]{geometry}
\usepackage{fontawesome}
% \makeatletter
% \@reversemargintrue
% \makeatother

% Símbolos al margen, necesitan doble compilación
\newcommand{\hard}{\marginnote{\faFire}}
\newcommand{\hhard}{\marginnote{\faFire\faFire}}

% Dependencias para los teoremas
\usepackage{xifthen}
\def\@thmdep{}
\newcommand{\thmdep}[1]{
	\ifthenelse{\isempty{#1}}
	{\def\@thmdep{}}
	{\def\@thmdep{ (#1)}}
}
\newcommand{\thmstyle}{\color{thm}\sffamily\bfseries}

% ===== Estilos de Teoremas ==========
\newtheoremstyle{axiomstyle}
	{0.3cm}
	{0.3cm}
	{\normalfont}
	{0.5cm}
	{\bfseries\scshape}
	{:}
	{4pt}
	{\thmname{#1}\thmnote{ #3}\thmnumber{ (#2)}}
\newtheoremstyle{styleC}
	{0.5cm}
	{0.5cm}
	{\normalfont}
	{0.5cm}
	{\bfseries}
	{:}
	{4pt}
	{\thmname{#1\textrm{\@thmdep}}\thmnumber{ #2}\thmnote{ (#3)}}

% ====== Teoremas (sin borde) ===========
\theoremstyle{axiomstyle}
\newtheorem*{axiom}{Axioma}

% ====== Teoremas (sin borde) ==================
\theoremstyle{styleC}
\newtheorem{thm}{Teorema}[section]
\newtheorem{mydef}[thm]{Definición}
\newtheorem{prop}[thm]{Proposición}
\newtheorem{cor}[thm]{Corolario}
\newtheorem{lem}[thm]{Lema}
\newtheorem{con}[thm]{Conjetura}

\newtheorem*{prob}{Problema}
\newtheorem*{sol}{Solución}
\newtheorem*{obs}{Observación}
\newtheorem*{ex}{Ejemplo}

% \usepackage{tcolorbox}
% \newtcbox{bluebox}[1][]{enhanced jigsaw, 
%   sharp corners,
%   frame hidden,
%   nobeforeafter,
%   listing only,
%   #1} % comando para crear cajas de colores

\expandafter\let\expandafter\oldproof\csname\string\proof\endcsname
\let\oldendproof\endproof
\renewenvironment{proof}[1][\proofname]{%
  \oldproof[\scshape Demostración:]%
}{\oldendproof} % comando para redefinir la caja de la demostración
\newenvironment{hint}[1][\proofname]{%
  \oldproof[\scshape Pista:]%
}{\oldendproof} % comando para redefinir la caja de la demostración

% colores utilizados
\definecolor{numchap}{RGB}{249,133,29}
\definecolor{chap}{RGB}{6,129,204}
\definecolor{sec}{RGB}{204,0,0}
\definecolor{thm}{RGB}{106,176,240}
\definecolor{thmB}{RGB}{32,31,31}
\definecolor{part}{RGB}{212,66,66}

% ====== Diseño de los titulares ===============
\usepackage[explicit]{titlesec} % para personalizar el documento, la opción <<explicit>> hace que el texto de los titulares sea un objeto interactuable

\titleformat{\subsection}[runin]
	{\bfseries}
	{\textrm{\S}\thesubsection}
	{1ex}
	{#1.}

\setlist[enumerate,1]{label=\arabic*., ref=\arabic*} % Enumerate standards


\title{Extensiones algebraicas}
\date{\DTMdate{2025-03-20}}

\author{José Cuevas Barrientos}
\email{josecuevasbtos@uc.cl}
\urladdr{https://josecuevas.xyz/teach/2025-1-ayud/}

\logo{../puc_negro.png}
\institution{Pontificia Universidad Católica de Chile}
\department{Facultad de Matemáticas}
\course{Álgebra abstracta II}
\coursecode{MAT2244}
\professor{Héctor Pastén Vásquez}

\begin{document}

\maketitle

% \section*{Resultados a considerar}
% Principalmente emplearemos los siguientes:
% \begin{thm}[ley de torres]
% 	Sean $F/L/k$ extensiones de cuerpos.
% 	Si $F/k$ es finita, entonces
% 	\[
% 		[F : k] = [F : L] \, [L : k].
% 	\]
% \end{thm}
% % \begin{thm}[levantamiento]
	
% % \end{thm}
% \begin{thm}[lema de Zorn]
% 	Sea $(X, \le)$ un conjunto parcialmente ordenado.
% 	Supongamos que todo conjunto $C \subseteq X$ en donde el orden $\le$ es total o lineal (i.e., en donde dados $x,
% 	y \in C$ se cumple que $x \le y$ o $y\le x$) está acotado superiormente (i.e., existe $z \in X$ tal que $z \ge
% 	c$ para todo $c \in C$).
% 	Entonces $X$ posee un \emph{elemento maximal} $x$ (i.e., si $y \ge x$, entonces $y = x$).
% \end{thm}
% Este último es una conocida equivalencia del axioma de elección que será necesario en algunos ejercicios.

\section{Extensiones finitas}
\begin{enumerate}
	\item Sea $L = K(\alpha)$ una extensión finita de grado impar.
		Pruebe que $L = K(\alpha^2)$.

	\item Sea $\Omega/k$ una extensión de cuerpos con extensiones intermedias $k \subseteq K, L \subseteq \Omega$.
		Pruebe que
		\[
			[KL : k] \le [K : k] \, [L : k],
		\]
		y que se alcanza igualdad cuando $[K : k]$ y $[L : k]$ son coprimos.

	\item Sea $k$ un cuerpo arbitrario, sea $k(t) := \Frac k[t]$ el cuerpo de fracciones del anillo de polinomios.
		\begin{enumerate}
			\item Pruebe que la extensión $k(t)/k$ no es algebraica.
			\item Sea $f \in k(t)$ una función racional (i.e., una fracción formal de polinomios) no constante,
				pruebe que la extensión $k(t)/k(f)$ es finita.
		\end{enumerate}
		\begin{prob}
			En el ejercicio anterior, calcule $[k(t) : k(f)]$.
		\end{prob}
\end{enumerate}

\section{Cuerpos algebraicamente cerrados}
\begin{enumerate}[resume]
	\item \begin{enumerate}
			\item Pruebe que, para un cuerpo $k$, son equivalentes:
				\begin{enumerate}[(i)]
					\item Toda extensión algebraica $K/k$ es tal que $K = k$.
					\item Todo polinomio no constante de $k$ posee una raíz en $k$.
					\item Los únicos polinomios irreducibles son los lineales.
				\end{enumerate}
				En cuyo caso, $k$ se dice \strong{algebraicamente cerrado}.
			\item Pruebe que un cuerpo finito no puede ser algebraicamente cerrado.
		\end{enumerate}

	\item\lookright
		Una extensión algebraica $C/k$, donde $C$ es algebraicamente cerrado, se dice una \strong{clausura algebraica} de $k$.
		Pruebe que, si existen, las clausuras algebraicas son únicas salvo $k$-isomorfismo.

		\emph{Pista:}
		Para esto necesitará hacer uso del \textbf{lema de Zorn} o equivalente.
\end{enumerate}

\appendix
\section{Ejercicios propuestos}
\begin{enumerate}
	\item \lookright
		Encuentre todas las extensiones intermedias de $\Q \subseteq \Q(\sqrt[4]{2})$.

	\item Sea $L/k$ una extensión algebraica (no necesariamente finita).
		Pruebe que todo subanillo $k \subseteq A \subseteq L$ es, de hecho, una extensión intermedia de cuerpos.

		\lookup
		¿Es esto cierto si $L$ no es algebraica?

	\item En clases se vio que en una extensión finita $k/\Fp$ de característica $p$ todo elemento es una potencia $p$-ésima.
		¿Es esto cierto aún cuando la extensión es algebraica infinita?
		¿Y cuándo no es algebraica?
\end{enumerate}

% \section{Cuerpos finitos}
% \begin{enumerate}[resume]
% 	% \item Sea $K/k$ una extensión algebraica de cuerpos.
% 	% 	Un elemento $\alpha \in K$ se dice \strong{puramente inseparable} si su polinomio minimal $f(x) \in k[x]$
% 	% 	es una potencia del monomio $x - \alpha$.
% 	% 	Pruebe que
% 	% 	\[
% 	% 		K_{\rm ins} = \{ \alpha \in K : \alpha \text{ es puramente inseparable} \}
% 	% 	\]
% 	% 	es un subcuerpo de $K$.

% 	% \item Sea $k$ un cuerpo finito.
% 	% 	Pruebe que el grupo de unidades $k^\times$ es cíclico.
% \end{enumerate}

\nocite{lang:algebra}

\printbibliography

\end{document}
