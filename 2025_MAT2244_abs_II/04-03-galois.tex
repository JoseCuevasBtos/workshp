\documentclass[11pt, reqno]{amsart}

\usepackage[spanish]{babel}
\usepackage[LGR, T1]{fontenc}
\usepackage[utf8]{inputenc}

../LaTeX/general.tex
% \input{../graphics.tex}

\makeatletter
\def\emailaddrname{\textit{Correo electrónico}}
\def\subtitle#1{\gdef\@subtitle{#1}}
\def\@subtitle{}

% Metadata
\def\logo#1{\gdef\@logo{#1}}
\def\@logo{}
\def\institution#1{\gdef\@institution{#1}}
\def\@institution{}
\def\department#1{\gdef\@department{#1}}
\def\@department{}
\def\professor#1{\gdef\@professor{#1}}
\def\@professor{}
\def\course#1{\gdef\@course{#1}}
\def\@course{}
\def\coursecode#1{\gdef\@coursecode{#1}}
\def\@coursecode{}

\renewcommand{\maketitle}{
\begin{center}
	\small
	\renewcommand{\arraystretch}{1.2}
	\begin{tabular}{cp{.37\textwidth}p{0.44\textwidth}}
		% \hline
		\multirow{5}{*}{\includegraphics[height=2.0cm]{\@logo}}
	  & \multicolumn{2}{c}{ \makecell{{\bfseries \@institution} \\ \@department} } \\
	  % & \multicolumn{2}{c|}{{\bfseries\@institution} \\ \@department} \\
	  \cline{2-3}
	  & \textbf{Profesor:} \@professor & \textbf{Ayudante:} \authors \\
	  % \cline{2-3}
	  & \textbf{Curso:} \@course & \textbf{Sigla:} \@coursecode \\
	  % \cline{2-3}
	  & \multicolumn{2}{l}{ \textbf{Fecha:} \@date } \\
	  % \hline
	\end{tabular}
	\\[\baselineskip]
	% {}
	% \vspace{2\baselineskip}
	{\bfseries\Large\@title}
	\ifx\@subtitle\@empty\else
		\\[1ex]
		\large\mdseries\@subtitle
	\fi
\end{center}
}
\makeatother

\usepackage{multirow, makecell}

\usepackage[
	reversemp,
	letterpaper,
	% marginpar=2cm,
	% marginsep=1pt,
	margin=2.3cm
]{geometry}
\usepackage{fontawesome}
% \makeatletter
% \@reversemargintrue
% \makeatother

% Símbolos al margen, necesitan doble compilación
\newcommand{\hard}{\marginnote{\faFire}}
\newcommand{\hhard}{\marginnote{\faFire\faFire}}

% Dependencias para los teoremas
\usepackage{xifthen}
\def\@thmdep{}
\newcommand{\thmdep}[1]{
	\ifthenelse{\isempty{#1}}
	{\def\@thmdep{}}
	{\def\@thmdep{ (#1)}}
}
\newcommand{\thmstyle}{\color{thm}\sffamily\bfseries}

% ===== Estilos de Teoremas ==========
\newtheoremstyle{axiomstyle}
	{0.3cm}
	{0.3cm}
	{\normalfont}
	{0.5cm}
	{\bfseries\scshape}
	{:}
	{4pt}
	{\thmname{#1}\thmnote{ #3}\thmnumber{ (#2)}}
\newtheoremstyle{styleC}
	{0.5cm}
	{0.5cm}
	{\normalfont}
	{0.5cm}
	{\bfseries}
	{:}
	{4pt}
	{\thmname{#1\textrm{\@thmdep}}\thmnumber{ #2}\thmnote{ (#3)}}

% ====== Teoremas (sin borde) ===========
\theoremstyle{axiomstyle}
\newtheorem*{axiom}{Axioma}

% ====== Teoremas (sin borde) ==================
\theoremstyle{styleC}
\newtheorem{thm}{Teorema}[section]
\newtheorem{mydef}[thm]{Definición}
\newtheorem{prop}[thm]{Proposición}
\newtheorem{cor}[thm]{Corolario}
\newtheorem{lem}[thm]{Lema}
\newtheorem{con}[thm]{Conjetura}

\newtheorem*{prob}{Problema}
\newtheorem*{sol}{Solución}
\newtheorem*{obs}{Observación}
\newtheorem*{ex}{Ejemplo}

% \usepackage{tcolorbox}
% \newtcbox{bluebox}[1][]{enhanced jigsaw, 
%   sharp corners,
%   frame hidden,
%   nobeforeafter,
%   listing only,
%   #1} % comando para crear cajas de colores

\expandafter\let\expandafter\oldproof\csname\string\proof\endcsname
\let\oldendproof\endproof
\renewenvironment{proof}[1][\proofname]{%
  \oldproof[\scshape Demostración:]%
}{\oldendproof} % comando para redefinir la caja de la demostración
\newenvironment{hint}[1][\proofname]{%
  \oldproof[\scshape Pista:]%
}{\oldendproof} % comando para redefinir la caja de la demostración

% colores utilizados
\definecolor{numchap}{RGB}{249,133,29}
\definecolor{chap}{RGB}{6,129,204}
\definecolor{sec}{RGB}{204,0,0}
\definecolor{thm}{RGB}{106,176,240}
\definecolor{thmB}{RGB}{32,31,31}
\definecolor{part}{RGB}{212,66,66}

% ====== Diseño de los titulares ===============
\usepackage[explicit]{titlesec} % para personalizar el documento, la opción <<explicit>> hace que el texto de los titulares sea un objeto interactuable

\titleformat{\subsection}[runin]
	{\bfseries}
	{\textrm{\S}\thesubsection}
	{1ex}
	{#1.}

\setlist[enumerate,1]{label=\arabic*., ref=\arabic*} % Enumerate standards


\title{Grupos de Galois}
\date{\DTMdate{2025-04-03}}

\author{José Cuevas Barrientos}
\email{josecuevasbtos@uc.cl}
\urladdr{https://josecuevas.xyz/teach/2025-1-ayud/}

\logo{../puc_negro.png}
\institution{Pontificia Universidad Católica de Chile}
\department{Facultad de Matemáticas}
\course{Álgebra abstracta II}
\coursecode{MAT2244}
\professor{Héctor Pastén Vásquez}

\begin{document}

\maketitle

% \section{Números constructibles}
\begin{enumerate}
	\item Pruebe que el nonágono (= 9-gono) no es constructible con regla y compás.

	\item Sea $L/k$ una extensión normal y definamos el subconjunto:
		\[
			L_{\rm sep} := \{ \alpha \in L : \alpha \text{ es separable sobre } k \}.
		\]
		\begin{enumerate}
			\item Pruebe que $L_{\rm sep}$ es un subcuerpo de $L$.
			\item Pruebe que la extensión $L_{\rm sep}/k$ es de Galois y $L/L_{\rm sep}$ es puramente inseparable.
			\item Pruebe que la siguiente aplicación
				\[
					\rho \colon \Gal(L/k) \longrightarrow \Gal(L_{\rm sep}/k), \qquad \sigma \longmapsto \sigma|_{L_{\rm sep}}
				\]
				está bien definida y es un isomorfismo de grupos.
		\end{enumerate}

	\item\label{exr:simple_subextension_count}
		Sea $K/k$ una extensión simple de cuerpos (i.e., hay $\alpha \in K$ tal que $K = k(\alpha)$) de grado $n
		:= [K:k] < \infty$.
		Pruebe que $K/k$ tiene a lo sumo $2^n$ subextensiones (incluyendo a $K$ y $k$ mismos).
	% \item (El teorema de la base normal)
	% 	Sea $K/k$ una extensión finita de grado $n$.
	% 	Una \strong{base normal} de $K$ es una $k$-base $\{ \alpha_j \}_{1\le j\le n}$ para $K$,
	% 	donde los elementos son $k$-conjugados (i.e., comparten el mismo polinomio minimal).
	% 	\begin{enumerate}
	% 		\item Pruebe que si $K$ posee una base normal, entonces $K/k$ es de Galois.
	% 	\end{enumerate}
	% 	Para el recíproco, suponga que $K/k$ es de Galois.
	% 	\begin{enumerate}[resume]
	% 		\item Sean $\{ \sigma_1, \dots, \sigma_n \} = \Gal(K/k)$ los elementos.
	% 			Pruebe que los conjugados de $\alpha \in K$ forman una base (normal) syss la matriz $[
	% 			\sigma_i \sigma_j \alpha ]_{ij} \in \Mat_n K$ tiene determinante no nulo.
	% 		\item Pruebe que $K$ admite una base normal.
	% 	\end{enumerate}

	\item\lookright
		Sea $\zeta_n = \exp(2\pi\ui / n) \in \C$ una raíz $n$-ésima primitiva de la unidad.
		Pruebe que $\Q(\zeta_n)/\Q$ es una extensión de Galois y que $\Gal(\Q(\zeta_n)/\Q) \cong
		(\Z/n\Z)^\times$.

	\item\label{exr:inverse_gal_ab}\lookst Pruebe que para todo grupo \emph{abeliano} finito $G$ existe una extensión $K/\Q$ de Galois tal que
		$\Gal(K/\Q) \cong G$.

		\begin{hint}
			Para la prueba puede ser útil emplear el \textbf{teorema de Dirichlet} que dice que
			dado $n > 1$ entero y $a$ coprimo con $n$, existen infinitos primos $p$ tales que $p \equiv a
			\pmod n$.
		\end{hint}
\end{enumerate}

\appendix
\section{Ejercicios propuestos}
\begin{enumerate}
	\item ¿El pentágono regular es constructible con regla y compas?
	\item Cuente la cantidad de subextensiones de las siguientes extensiones:
		\begin{enumerate}
			\item $\Q(\sqrt{2}, \sqrt{3}, \sqrt{5})/\Q$.
			\item $\Q(\zeta_3, \sqrt[3]{2})/\Q$, donde $\zeta_3$ es una raíz cúbica (primitiva) de la unidad.
			\item $\Q(\sqrt[4]{2})/\Q$.
		\end{enumerate}
\end{enumerate}

\section{Comentarios adicionales}
En el ejercicio \ref{exr:simple_subextension_count} es importante que la extensión sea \emph{simple}.
Si la extensión es separable finita, el resultado también aplica, pero, por ejemplo, la extensión inseparable
$\Fp(t^{1/p}, s^{1/p}) \supseteq \Fp(t, s)$ posee infinitas subextensiones.

El ejercicio \ref{exr:inverse_gal_ab} es un caso sencillo del \emph{problema inverso de Galois}: ¿será acaso que todo
grupo finito es isomorfo a un grupo de Galois de una extensión $K/\Q$? ¿De no ser así, habrá un invariante que
permita discriminar cuáles sí o no?

\nocite{jacobson:basic, lang:algebra}
\printbibliography

\end{document}
