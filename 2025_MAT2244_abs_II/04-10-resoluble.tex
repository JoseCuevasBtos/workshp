\documentclass[11pt, reqno]{amsart}

\usepackage[spanish]{babel}
\usepackage[LGR, T1]{fontenc}
\usepackage[utf8]{inputenc}

../LaTeX/general.tex
% \input{../graphics.tex}

\makeatletter
\def\emailaddrname{\textit{Correo electrónico}}
\def\subtitle#1{\gdef\@subtitle{#1}}
\def\@subtitle{}

% Metadata
\def\logo#1{\gdef\@logo{#1}}
\def\@logo{}
\def\institution#1{\gdef\@institution{#1}}
\def\@institution{}
\def\department#1{\gdef\@department{#1}}
\def\@department{}
\def\professor#1{\gdef\@professor{#1}}
\def\@professor{}
\def\course#1{\gdef\@course{#1}}
\def\@course{}
\def\coursecode#1{\gdef\@coursecode{#1}}
\def\@coursecode{}

\renewcommand{\maketitle}{
\begin{center}
	\small
	\renewcommand{\arraystretch}{1.2}
	\begin{tabular}{cp{.37\textwidth}p{0.44\textwidth}}
		% \hline
		\multirow{5}{*}{\includegraphics[height=2.0cm]{\@logo}}
	  & \multicolumn{2}{c}{ \makecell{{\bfseries \@institution} \\ \@department} } \\
	  % & \multicolumn{2}{c|}{{\bfseries\@institution} \\ \@department} \\
	  \cline{2-3}
	  & \textbf{Profesor:} \@professor & \textbf{Ayudante:} \authors \\
	  % \cline{2-3}
	  & \textbf{Curso:} \@course & \textbf{Sigla:} \@coursecode \\
	  % \cline{2-3}
	  & \multicolumn{2}{l}{ \textbf{Fecha:} \@date } \\
	  % \hline
	\end{tabular}
	\\[\baselineskip]
	% {}
	% \vspace{2\baselineskip}
	{\bfseries\Large\@title}
	\ifx\@subtitle\@empty\else
		\\[1ex]
		\large\mdseries\@subtitle
	\fi
\end{center}
}
\makeatother

\usepackage{multirow, makecell}

\usepackage[
	reversemp,
	letterpaper,
	% marginpar=2cm,
	% marginsep=1pt,
	margin=2.3cm
]{geometry}
\usepackage{fontawesome}
% \makeatletter
% \@reversemargintrue
% \makeatother

% Símbolos al margen, necesitan doble compilación
\newcommand{\hard}{\marginnote{\faFire}}
\newcommand{\hhard}{\marginnote{\faFire\faFire}}

% Dependencias para los teoremas
\usepackage{xifthen}
\def\@thmdep{}
\newcommand{\thmdep}[1]{
	\ifthenelse{\isempty{#1}}
	{\def\@thmdep{}}
	{\def\@thmdep{ (#1)}}
}
\newcommand{\thmstyle}{\color{thm}\sffamily\bfseries}

% ===== Estilos de Teoremas ==========
\newtheoremstyle{axiomstyle}
	{0.3cm}
	{0.3cm}
	{\normalfont}
	{0.5cm}
	{\bfseries\scshape}
	{:}
	{4pt}
	{\thmname{#1}\thmnote{ #3}\thmnumber{ (#2)}}
\newtheoremstyle{styleC}
	{0.5cm}
	{0.5cm}
	{\normalfont}
	{0.5cm}
	{\bfseries}
	{:}
	{4pt}
	{\thmname{#1\textrm{\@thmdep}}\thmnumber{ #2}\thmnote{ (#3)}}

% ====== Teoremas (sin borde) ===========
\theoremstyle{axiomstyle}
\newtheorem*{axiom}{Axioma}

% ====== Teoremas (sin borde) ==================
\theoremstyle{styleC}
\newtheorem{thm}{Teorema}[section]
\newtheorem{mydef}[thm]{Definición}
\newtheorem{prop}[thm]{Proposición}
\newtheorem{cor}[thm]{Corolario}
\newtheorem{lem}[thm]{Lema}
\newtheorem{con}[thm]{Conjetura}

\newtheorem*{prob}{Problema}
\newtheorem*{sol}{Solución}
\newtheorem*{obs}{Observación}
\newtheorem*{ex}{Ejemplo}

% \usepackage{tcolorbox}
% \newtcbox{bluebox}[1][]{enhanced jigsaw, 
%   sharp corners,
%   frame hidden,
%   nobeforeafter,
%   listing only,
%   #1} % comando para crear cajas de colores

\expandafter\let\expandafter\oldproof\csname\string\proof\endcsname
\let\oldendproof\endproof
\renewenvironment{proof}[1][\proofname]{%
  \oldproof[\scshape Demostración:]%
}{\oldendproof} % comando para redefinir la caja de la demostración
\newenvironment{hint}[1][\proofname]{%
  \oldproof[\scshape Pista:]%
}{\oldendproof} % comando para redefinir la caja de la demostración

% colores utilizados
\definecolor{numchap}{RGB}{249,133,29}
\definecolor{chap}{RGB}{6,129,204}
\definecolor{sec}{RGB}{204,0,0}
\definecolor{thm}{RGB}{106,176,240}
\definecolor{thmB}{RGB}{32,31,31}
\definecolor{part}{RGB}{212,66,66}

% ====== Diseño de los titulares ===============
\usepackage[explicit]{titlesec} % para personalizar el documento, la opción <<explicit>> hace que el texto de los titulares sea un objeto interactuable

\titleformat{\subsection}[runin]
	{\bfseries}
	{\textrm{\S}\thesubsection}
	{1ex}
	{#1.}

\setlist[enumerate,1]{label=\arabic*., ref=\arabic*} % Enumerate standards


\title{Extensiones resolubles}
\date{\DTMdate{2025-04-03}}

\author{José Cuevas Barrientos}
\email{josecuevasbtos@uc.cl}
\urladdr{https://josecuevas.xyz/teach/2025-1-ayud/}

\logo{../puc_negro.png}
\institution{Pontificia Universidad Católica de Chile}
\department{Facultad de Matemáticas}
\course{Álgebra abstracta II}
\coursecode{MAT2244}
\professor{Héctor Pastén Vásquez}

\begin{document}

\maketitle

\nocite{nagata:fields}
\nocite{lang:algebra}
% \section{Números constructibles}

\begin{enumerate}
	\item Una extensión $L/k$ se dice \strong{abeliana} (resp.\ \strong{cíclico}) si existe una extensión
		$A/k$ de Galois con $\Gal(A/k)$ abeliano (resp.\ cíclico), tal que $k \subseteq L \subseteq A$.

		\begin{enumerate}
			\item Pruebe que si $L/k$ es abeliano (resp.\ cíclico), entonces $L/k$ es de Galois y
				$\Gal(L/k)$ es abeliano (resp.\ cíclico).
			\item En la extensión $F/L/k$, pruebe que si $F/k$ es abeliano (resp.\ cíclico), entonces $F/L$
				y $L/k$ son abelianos (resp.\ cíclicos).
			% \item En la extensión $F/L/k$, pruebe que $F/k$ es abeliano syss $F/L$ y $L/k$ lo son.

				\begin{prob}
					\lookup
					Si en la extensión $F/L/k$, se cumple que $F/L$ y $L/k$ son abelianos (resp.\
					cíclicos), entonces ¿$F/k$ es siempre abeliano (resp.\ cíclico)?
				\end{prob}
		\end{enumerate}

	\item Pruebe que si $L_1, \dots, L_n / k$ son extensiones abelianas (resp.\ resolubles) contenidas en otra
		extensión $\Omega/k$, entonces el composito $L_1 \cdots L_n$ es abeliano (resp.\ resoluble).

	\item\lookright
		Pruebe, sin usar las fórmulas cuadrática y de Cardano, que todo polinomio (separable) de grado
		$\le 3$ es resoluble por radicales.

	\item Sea $K/k$ una extensión finita.
		Dado $\alpha \in K$, denotemos por $m_\alpha(x) := \alpha\cdot x$ el cual determina un $k$-endomorfismo
		$m_\alpha \colon K \to K$.
		Defina la \strong{norma} y \strong{traza} de $\alpha$ como
		\[
			\galnorm_{K/k}(\alpha) := \det m_\alpha, \qquad \galtr_{K/k}(\alpha) := \tr m_\alpha.
		\]
		(Recuerde, de su curso de álgebra lineal, que el determinante y la traza de un endomorfismo no dependen
		de la elección de base.)
		\begin{enumerate}
			\item Pruebe que, para $\alpha, \beta \in K$ se cumple que
				\[
					\galtr_{K/k}(\alpha + \beta) = \galtr_{K/k}(\alpha) + \galtr_{K/k}(\beta),
					\qquad
					\galnorm_{K/k}(\alpha \cdot \beta) = \galnorm_{K/k}(\alpha) \cdot \galnorm_{K/k}(\beta).
				\]

			\item\lookst
				Sea $G = \{ \sigma \colon K \to \algcl k \}$ el conjunto de los homomorfismos de
				$K$-álgebras. Pruebe que
				\[
					\galtr_{K/k}(\alpha) = \sum_{\sigma \in G} \sigma(\alpha),
					\qquad
					\galnorm_{K/k}(\alpha) = \prod_{\sigma \in G} \sigma(\alpha).
				\]

				\begin{hint}
					Haga el caso $K = k(\alpha)$ y luego emplee que hay tantos
					$k(\alpha)$-homomorfismos $K \to \algcl{k(\alpha)} = \algcl k$ como grado $[K :
					k(\alpha)]$.
				\end{hint}

			\item Sea $L/K$ una extensión finita y sea $\gamma \in L$, pruebe que
				\[
					\galtr_{L/k} = \galtr_{K/k} \circ \galtr_{L/K},
					\qquad
					\galnorm_{L/k} = \galnorm_{K/k} \circ \galnorm_{L/K}.
				\]
		\end{enumerate}

	\item Empleando lo anterior vamos a probar que $\Q(\sqrt[n]{a}) \ne \Q(\sqrt[n]{b})$ con $a \ne b$ enteros
		\emph{coprimos} libres de potencias $n$-ésimas para $n > 1$ un entero \emph{primo}.

		\begin{hint}
			Use la traza con $K := \Q(\sqrt[n]{a})$.
			Pruebe que $\galtr_{K/\Q}(\sqrt[n]{a}) = 0$ y $\galtr_{K/\Q}(\sqrt[n]{a^j b}) = 0$ para $0 \le j
			< n$, para llegar a una contradicción.
		\end{hint}
\end{enumerate}

\appendix
\section{Ejercicios propuestos}
\begin{enumerate}
	\item\label{q:solvable_four} Pruebe, sin usar la fórmula de Ferrari, que todo polinomio de grado $\le 4$ es resoluble por radicales.
	% \item Pruebe que si $L_1, \dots, L_n / k$ son extensiones resolubles contenidas en otra extensión $\Omega/k$,
	% 	entonces el composito $L_1 \cdots L_n$ es resoluble.
	\item\label{q:D6_as_gal}\lookst
		% (Exc.~4.5.4, pág.~243)
		Construya $f \in \C(t)$ no constante de modo que la extensión $\C(t)/\C(f)$ sea de Galois y
		\[
			\Gal( \C(t)/\C(f) ) \cong D_6.
		\]
		(Para mí, $D_6$ es el diedral con 6 elementos.)

		\begin{hint}
			Primero, construya un subgrupo $G \le \Gal(\C(t) / \C)$ tal que $G \cong D_6$.
			Para lograrlo, encuentre un elemento $\sigma$ en $G$ de orden 2 (como $\sigma(t) = 1/t$) y otro
			$\tau$ de orden 3 tales que $\sigma \tau \sigma = \tau^{-1}$.
			Finalmente, explicite el cuerpo fijo $\C(t)^G$.
		\end{hint}
	% \item Generalice la estrategia en la preg.~\ref{q:D6_as_gal} para probar que para todo $n \ge 3$ se cumple que
	% 	existe $f \in \C(t)$ con $\Gal(\C(t)/\C(f)) \cong D_{2n}$.
\end{enumerate}

\section{Comentarios adicionales}
El ejercicio propuesto \ref{q:D6_as_gal} es generalizable para ver que para todo $n > 1$ existe $f$ tal que
$\Gal(\C(t)/\C(f)) \cong D_{2n}$.
El ejercicio propuesto \ref{q:solvable_four} es lo mejor posible, pues precisamente el teorema de Abel-Ruffini dice que
hay polinomios separables de grado 5 que no son resolubles por radicales, precisamente porque los grupos $A_5$ y $S_5$
no son resolubles.

En la ayudantía pasada vimos un ejemplo de extensión abeliana, las ciclotómicas $\Q(\zeta_n)/\Q$.
Por lo demás, toda subextensión de una ciclotómica es también abeliana y el composito de dos extensiones ciclotómicas es
también ciclotómica (pues $\Q(\zeta_n)\Q(\zeta_m) = \Q(\zeta_{nm})$ cuando $n, m$ son coprimos).
Es altamente interesante que un teorema de Kronecker-Weber nos da un recíproco, a decir:

\begin{thm}
	Toda extensión abeliana de $\Q$ es subextensión de una ciclotómica.
\end{thm}
Este resultado es difícil y se puede probar con teoría de cuerpos de clase.
Vea \cite{neukirch:algebraic}.

% El ejercicio \ref{exr:inverse_gal_ab} es un caso sencillo del \emph{problema inverso de Galois}: ¿será acaso que todo
% grupo finito es isomorfo a un grupo de Galois de una extensión $K/\Q$? ¿De no ser así, habrá un invariante que
% permita discriminar cuáles sí o no?

\printbibliography

\end{document}
