\documentclass[11pt, reqno]{amsart}

\usepackage[spanish]{babel}
\usepackage[LGR, T1]{fontenc}
\usepackage[utf8]{inputenc}

../LaTeX/general.tex
% \input{../graphics.tex}

\makeatletter
\def\emailaddrname{\textit{Correo electrónico}}
\def\subtitle#1{\gdef\@subtitle{#1}}
\def\@subtitle{}

% Metadata
\def\logo#1{\gdef\@logo{#1}}
\def\@logo{}
\def\institution#1{\gdef\@institution{#1}}
\def\@institution{}
\def\department#1{\gdef\@department{#1}}
\def\@department{}
\def\professor#1{\gdef\@professor{#1}}
\def\@professor{}
\def\course#1{\gdef\@course{#1}}
\def\@course{}
\def\coursecode#1{\gdef\@coursecode{#1}}
\def\@coursecode{}

\renewcommand{\maketitle}{
\begin{center}
	\small
	\renewcommand{\arraystretch}{1.2}
	\begin{tabular}{cp{.37\textwidth}p{0.44\textwidth}}
		% \hline
		\multirow{5}{*}{\includegraphics[height=2.0cm]{\@logo}}
	  & \multicolumn{2}{c}{ \makecell{{\bfseries \@institution} \\ \@department} } \\
	  % & \multicolumn{2}{c|}{{\bfseries\@institution} \\ \@department} \\
	  \cline{2-3}
	  & \textbf{Profesor:} \@professor & \textbf{Ayudante:} \authors \\
	  % \cline{2-3}
	  & \textbf{Curso:} \@course & \textbf{Sigla:} \@coursecode \\
	  % \cline{2-3}
	  & \multicolumn{2}{l}{ \textbf{Fecha:} \@date } \\
	  % \hline
	\end{tabular}
	\\[\baselineskip]
	% {}
	% \vspace{2\baselineskip}
	{\bfseries\Large\@title}
	\ifx\@subtitle\@empty\else
		\\[1ex]
		\large\mdseries\@subtitle
	\fi
\end{center}
}
\makeatother

\usepackage{multirow, makecell}

\usepackage[
	reversemp,
	letterpaper,
	% marginpar=2cm,
	% marginsep=1pt,
	margin=2.3cm
]{geometry}
\usepackage{fontawesome}
% \makeatletter
% \@reversemargintrue
% \makeatother

% Símbolos al margen, necesitan doble compilación
\newcommand{\hard}{\marginnote{\faFire}}
\newcommand{\hhard}{\marginnote{\faFire\faFire}}

% Dependencias para los teoremas
\usepackage{xifthen}
\def\@thmdep{}
\newcommand{\thmdep}[1]{
	\ifthenelse{\isempty{#1}}
	{\def\@thmdep{}}
	{\def\@thmdep{ (#1)}}
}
\newcommand{\thmstyle}{\color{thm}\sffamily\bfseries}

% ===== Estilos de Teoremas ==========
\newtheoremstyle{axiomstyle}
	{0.3cm}
	{0.3cm}
	{\normalfont}
	{0.5cm}
	{\bfseries\scshape}
	{:}
	{4pt}
	{\thmname{#1}\thmnote{ #3}\thmnumber{ (#2)}}
\newtheoremstyle{styleC}
	{0.5cm}
	{0.5cm}
	{\normalfont}
	{0.5cm}
	{\bfseries}
	{:}
	{4pt}
	{\thmname{#1\textrm{\@thmdep}}\thmnumber{ #2}\thmnote{ (#3)}}

% ====== Teoremas (sin borde) ===========
\theoremstyle{axiomstyle}
\newtheorem*{axiom}{Axioma}

% ====== Teoremas (sin borde) ==================
\theoremstyle{styleC}
\newtheorem{thm}{Teorema}[section]
\newtheorem{mydef}[thm]{Definición}
\newtheorem{prop}[thm]{Proposición}
\newtheorem{cor}[thm]{Corolario}
\newtheorem{lem}[thm]{Lema}
\newtheorem{con}[thm]{Conjetura}

\newtheorem*{prob}{Problema}
\newtheorem*{sol}{Solución}
\newtheorem*{obs}{Observación}
\newtheorem*{ex}{Ejemplo}

% \usepackage{tcolorbox}
% \newtcbox{bluebox}[1][]{enhanced jigsaw, 
%   sharp corners,
%   frame hidden,
%   nobeforeafter,
%   listing only,
%   #1} % comando para crear cajas de colores

\expandafter\let\expandafter\oldproof\csname\string\proof\endcsname
\let\oldendproof\endproof
\renewenvironment{proof}[1][\proofname]{%
  \oldproof[\scshape Demostración:]%
}{\oldendproof} % comando para redefinir la caja de la demostración
\newenvironment{hint}[1][\proofname]{%
  \oldproof[\scshape Pista:]%
}{\oldendproof} % comando para redefinir la caja de la demostración

% colores utilizados
\definecolor{numchap}{RGB}{249,133,29}
\definecolor{chap}{RGB}{6,129,204}
\definecolor{sec}{RGB}{204,0,0}
\definecolor{thm}{RGB}{106,176,240}
\definecolor{thmB}{RGB}{32,31,31}
\definecolor{part}{RGB}{212,66,66}

% ====== Diseño de los titulares ===============
\usepackage[explicit]{titlesec} % para personalizar el documento, la opción <<explicit>> hace que el texto de los titulares sea un objeto interactuable

\titleformat{\subsection}[runin]
	{\bfseries}
	{\textrm{\S}\thesubsection}
	{1ex}
	{#1.}

\setlist[enumerate,1]{label=\arabic*., ref=\arabic*} % Enumerate standards


\title{Anillos e ideales}
\date{\DTMdate{2025-04-24}}

\author{José Cuevas Barrientos}
\email{josecuevasbtos@uc.cl}
\urladdr{https://josecuevas.xyz/teach/2025-1-ayud/}

\logo{../puc_negro.png}
\institution{Pontificia Universidad Católica de Chile}
\department{Facultad de Matemáticas}
\course{Álgebra abstracta II}
\coursecode{MAT2244}
\professor{Héctor Pastén Vásquez}

\begin{document}

\maketitle

\nocite{lang:algebra}
\nocite{atiyah:commutative}
% \section{Números constructibles}

En esta ayudantía, entendemos que todo \emph{anillo} es conmutativo y unitario (i.e., con $1 \in A$).
El anillo nulo $A = 0$ sí se considera un anillo, aunque no es un cuerpo.

\begin{enumerate}
	\item (Examen de lucidez)
		Sea $A$ un anillo y considere el conjunto $R := \Func(\N_{>0}, A)$ de funciones con la suma dada coordenada por
		coordenada y el producto por \emph{convolución}:
		\[
			(f + g)(n) := f(n) + g(n), \qquad (f*g)(n) = \sum_{ab=n} f(a)g(b).
		\]
		\begin{enumerate}
			\item Pruebe que $(R, +, *)$ es un anillo cuya unidad es la función dada por
				$\varepsilon(1) = 1$ y $\varepsilon(n) = 0$ para $n > 1$.
			\item Una función $f$ se dice \strong{multiplicativa} si $f(ab) = f(a)f(b)$ cuando $a, b$ son
				coprimos.
				Pruebe que si $f, g$ son multiplicativas, entonces $f*g$ también lo es.

				\begin{prob}[fórmula de inversión de Möbius]
					\lookup
					Defina la función de Möbius $\mu$ como $\mu(n) = (-1)^m$ cuando $n = p_1 \cdots p_m$ es el
					producto de primos distintos y $\mu(n) = 0$ cuando $p^2 \mid n$ para algún primo $p$.
					Sea <<1>> la función constante $1(n) = 1$.
					Pruebe que $\mu * 1 = \varepsilon$.
				\end{prob}
		\end{enumerate}

	\item Sea $A$ un anillo y sea $a \in A$ un elemento \strong{nilpotente} (i.e., $a^n = 0$ para algún $n \in \N$).
		Pruebe que $1 + a$ es inversible.

	\item Sea $A$ un anillo en donde cada para cada $a \in A$, existe $n \ge 2$ tal que $x^n = x$.
		Pruebe que todo ideal primo es maximal.

	\item\lookright
		Diremos que un anillo $A$ es \strong{local}, si posee un único ideal maximal $\mathfrak{m} \subseteq A$.
		Pruebe que:
		\begin{enumerate}
			\item Un anillo $A$ es local syss el conjunto $\mathfrak{m} = A \setminus A^\times$ de elementos
				que no poseen inversa es un ideal; en cuyo caso, $\mathfrak{m}$ es el único ideal
				maximal.
			\item Pruebe que si $f \colon A \to B$ es un homomorfismo sobreyectivo (o \emph{epimorfismo}) de
				anillos y $A$ es local. Entonces $B$ es o bien nulo o bien un anillo local.
		\end{enumerate}

	\item El objetivo de este ejercicio es caracterizar al nilradical de un anillo.
		\begin{enumerate}
			\item Sea $S \subseteq A$ un sistema multiplicativo tal que $0 \notin S$.
				Verifique que la familia de ideales
				\[
					\mathcal{F} := \{ \mathfrak{a} \nsle A : S \subseteq A \setminus \mathfrak{a} \}
				\]
				posee un elemento $\subseteq$-maximal.
			\item Pruebe que un elemento $\subseteq$-maximal de $\mathcal{F}$ es un ideal primo de $A$.
			\item Concluya que un elemento $a \in A$ es nilpotente syss pertenece a todos los ideales
				primos\break de $A$.
		\end{enumerate}

	% \item Sea $A$ un anillo y $M$ un $A$-módulo.
	% 	Pruebe que son equivalentes:
	% 	\begin{enumerate}
	% 		\item $M = 0$.
	% 		\item $M_{\mathfrak{p}} = 0$ para todo $\mathfrak{p} \nsl A$ primo.
	% 		\item $M_{\mathfrak{m}} = 0$ para todo $\mathfrak{m} \nsl A$ maximal.
	% 	\end{enumerate}
\end{enumerate}

\appendix
\section{Ejercicios propuestos}
\begin{enumerate}
	\item Sea $k$ un cuerpo finito con $q$ elementos, y denotemos por $\psi(d)$ a la cantidad de polinomios
		irreducibles en $k[x]$ de grado $d$.
		Empleando la fórmula de inversión de Möbius, pruebe que
		\[
			n \psi(n) = \sum_{d \mid n} \mu(d)q^{n/d}.
		\]
	\item Sea $A$ un anillo y sea $\mathfrak{a} \nsle \mathfrak{N}(A)$ un ideal de nilpotentes.
		Pruebe que si $a \in A$ se proyecta en una unidad $a \mod{\mathfrak{a}} \in (A/\mathfrak{a})^\times$,
		entonces $a$ es una unidad en $A$.
	\item\label{bool_ring}
		Un anillo $A$ se dice \emph{booleano} (o \emph{de Boole}) si $a^2 = a$ para todo $a \in A$.
		Pruebe lo siguiente:
		\begin{enumerate}
			\item Para todo primo $\mathfrak{p} \nsl A$ se cumple que $A/\mathfrak{p} \cong \Fp[2]$.
			\item Todo ideal finitamente generado es principal.
		\end{enumerate}
\end{enumerate}

\section{Comentarios adicionales}
El nombre <<anillo local>> se debe a que, asociado a ciertos objetos geométricos $X$ (e.g., variedades diferenciales,
analíticas o algebraicas), uno puede construir lo que se llaman <<haces>> que consisten de anillos naturales asociados a
los abiertos de $X$.
Por ejemplo, si $X$ es una variedad diferencial (piense en un abierto de $\R^n$), al abierto $U \subseteq \R^n$ podemos
asociarle el anillo de funciones diferenciables $U \to \R$; un <<germen>> es una clase de equivalencia de dichas
funciones en vecindades de un punto $x \in U$ fijado.
El anillo de <<gérmenes en $x$>> será un anillo local.

Por lo demás, la teoría de álgebra conmutativa (vid.\ \cite{atiyah:commutative}) justifica que los anillos locales son
capaces de captar harta información algebraica (por ejemplo, un módulo será nulo si sus localizaciones lo son en
analogía a como una función es nula si sus evaluaciones lo son).

El nombre <<anillo booleano>> se debe a que corresponden, de manera elemental, a las llamadas álgebras booleanas.
Una álgebra booleana está dotada de un 0, un 1, una inversa $\neg$ y operadores binarios $\vee$ y $\wedge$ que
satisfacen la típica álgebra de proposiciones lógicas.
Por un teorema de M.~H.~Stone,%
\footnote{El mismo del <<teorema de Stone-Weierstrass>> y de las <<compactificaciones de Stone-\v Cech>>.}
un álgebra booleana corresponde a un subconjunto del conjunto potencia $\mathcal{P} S$ de un conjunto $S$, que contiene
a $\emptyset$ y $A$, y es cerrado bajo complementos, uniones e intersecciones finitas.
Estos objetos tienen su utilidad e interés en la lógica, pero también tienen interacciones con la topología:
\begin{thm}[dualidad de Stone]
	Hay una anti-equivalencia (<<las flechas se dan vuelta>>) entre la categoría de anillos booleanos y la categoría
	de espacios topológicos de Hausdorff, compactos y totalmente disconexos.%
	\footnote{Es decir, espacios en donde todo subconjunto con al menos dos puntos es disconexo. Por ejemplo, $\Q$
	es totalmente disconexo (pero no es compacto).}
\end{thm}
Para más detalles lea \cite{johnstone:stone}, en \S II.4 aparece el teorema aquí citado.

\printbibliography

\end{document}
