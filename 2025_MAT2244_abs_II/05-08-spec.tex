\documentclass[11pt, reqno]{amsart}

\usepackage[spanish]{babel}
\usepackage[LGR, T1]{fontenc}
\usepackage[utf8]{inputenc}

../LaTeX/general.tex
% \input{../graphics.tex}

\makeatletter
\def\emailaddrname{\textit{Correo electrónico}}
\def\subtitle#1{\gdef\@subtitle{#1}}
\def\@subtitle{}

% Metadata
\def\logo#1{\gdef\@logo{#1}}
\def\@logo{}
\def\institution#1{\gdef\@institution{#1}}
\def\@institution{}
\def\department#1{\gdef\@department{#1}}
\def\@department{}
\def\professor#1{\gdef\@professor{#1}}
\def\@professor{}
\def\course#1{\gdef\@course{#1}}
\def\@course{}
\def\coursecode#1{\gdef\@coursecode{#1}}
\def\@coursecode{}

\renewcommand{\maketitle}{
\begin{center}
	\small
	\renewcommand{\arraystretch}{1.2}
	\begin{tabular}{cp{.37\textwidth}p{0.44\textwidth}}
		% \hline
		\multirow{5}{*}{\includegraphics[height=2.0cm]{\@logo}}
	  & \multicolumn{2}{c}{ \makecell{{\bfseries \@institution} \\ \@department} } \\
	  % & \multicolumn{2}{c|}{{\bfseries\@institution} \\ \@department} \\
	  \cline{2-3}
	  & \textbf{Profesor:} \@professor & \textbf{Ayudante:} \authors \\
	  % \cline{2-3}
	  & \textbf{Curso:} \@course & \textbf{Sigla:} \@coursecode \\
	  % \cline{2-3}
	  & \multicolumn{2}{l}{ \textbf{Fecha:} \@date } \\
	  % \hline
	\end{tabular}
	\\[\baselineskip]
	% {}
	% \vspace{2\baselineskip}
	{\bfseries\Large\@title}
	\ifx\@subtitle\@empty\else
		\\[1ex]
		\large\mdseries\@subtitle
	\fi
\end{center}
}
\makeatother

\usepackage{multirow, makecell}

\usepackage[
	reversemp,
	letterpaper,
	% marginpar=2cm,
	% marginsep=1pt,
	margin=2.3cm
]{geometry}
\usepackage{fontawesome}
% \makeatletter
% \@reversemargintrue
% \makeatother

% Símbolos al margen, necesitan doble compilación
\newcommand{\hard}{\marginnote{\faFire}}
\newcommand{\hhard}{\marginnote{\faFire\faFire}}

% Dependencias para los teoremas
\usepackage{xifthen}
\def\@thmdep{}
\newcommand{\thmdep}[1]{
	\ifthenelse{\isempty{#1}}
	{\def\@thmdep{}}
	{\def\@thmdep{ (#1)}}
}
\newcommand{\thmstyle}{\color{thm}\sffamily\bfseries}

% ===== Estilos de Teoremas ==========
\newtheoremstyle{axiomstyle}
	{0.3cm}
	{0.3cm}
	{\normalfont}
	{0.5cm}
	{\bfseries\scshape}
	{:}
	{4pt}
	{\thmname{#1}\thmnote{ #3}\thmnumber{ (#2)}}
\newtheoremstyle{styleC}
	{0.5cm}
	{0.5cm}
	{\normalfont}
	{0.5cm}
	{\bfseries}
	{:}
	{4pt}
	{\thmname{#1\textrm{\@thmdep}}\thmnumber{ #2}\thmnote{ (#3)}}

% ====== Teoremas (sin borde) ===========
\theoremstyle{axiomstyle}
\newtheorem*{axiom}{Axioma}

% ====== Teoremas (sin borde) ==================
\theoremstyle{styleC}
\newtheorem{thm}{Teorema}[section]
\newtheorem{mydef}[thm]{Definición}
\newtheorem{prop}[thm]{Proposición}
\newtheorem{cor}[thm]{Corolario}
\newtheorem{lem}[thm]{Lema}
\newtheorem{con}[thm]{Conjetura}

\newtheorem*{prob}{Problema}
\newtheorem*{sol}{Solución}
\newtheorem*{obs}{Observación}
\newtheorem*{ex}{Ejemplo}

% \usepackage{tcolorbox}
% \newtcbox{bluebox}[1][]{enhanced jigsaw, 
%   sharp corners,
%   frame hidden,
%   nobeforeafter,
%   listing only,
%   #1} % comando para crear cajas de colores

\expandafter\let\expandafter\oldproof\csname\string\proof\endcsname
\let\oldendproof\endproof
\renewenvironment{proof}[1][\proofname]{%
  \oldproof[\scshape Demostración:]%
}{\oldendproof} % comando para redefinir la caja de la demostración
\newenvironment{hint}[1][\proofname]{%
  \oldproof[\scshape Pista:]%
}{\oldendproof} % comando para redefinir la caja de la demostración

% colores utilizados
\definecolor{numchap}{RGB}{249,133,29}
\definecolor{chap}{RGB}{6,129,204}
\definecolor{sec}{RGB}{204,0,0}
\definecolor{thm}{RGB}{106,176,240}
\definecolor{thmB}{RGB}{32,31,31}
\definecolor{part}{RGB}{212,66,66}

% ====== Diseño de los titulares ===============
\usepackage[explicit]{titlesec} % para personalizar el documento, la opción <<explicit>> hace que el texto de los titulares sea un objeto interactuable

\titleformat{\subsection}[runin]
	{\bfseries}
	{\textrm{\S}\thesubsection}
	{1ex}
	{#1.}

\setlist[enumerate,1]{label=\arabic*., ref=\arabic*} % Enumerate standards

\usepackage{tikz-cd}
\DeclareMathOperator\Gr{Gr}

\title{El espectro de un anillo}
\date{\DTMdate{2025-05-08}}

\author{José Cuevas Barrientos}
\email{josecuevasbtos@uc.cl}
\urladdr{https://josecuevas.xyz/teach/2025-1-ayud/}

\logo{../puc_negro.png}
\institution{Pontificia Universidad Católica de Chile}
\department{Facultad de Matemáticas}
\course{Álgebra abstracta II}
\coursecode{MAT2244}
\professor{Héctor Pastén Vásquez}

\begin{document}

\maketitle

\nocite{atiyah:commutative}
\nocite{matsumura:ring}
% \section{Números constructibles}

\section{Recordatorios sobre topología}
Daremos un breve recuento de las definiciones que se emplearán en la ayudantía.
% Insisto en que bastarán las definiciones, y no hará falta el uso de resultados adicionales.

Una \strong{topología} sobre un conjunto $X$ es una familia $\tau$ de subconjuntos tales que:
\begin{enumerate}[{Top}1., leftmargin=*]
	\item $\emptyset, X \in \tau$.
	\item Si $\{ U_i \}_i$ es una familia de elementos de $\tau$, entonces $\bigcup_{i\in I} U_i \in \tau$.
	\item Si $U_1, \dots, U_n \in \tau$, entonces $U_1 \cap U_2 \cap \cdots \cap U_n \in \tau$.
\end{enumerate}
Los elementos de $\tau$ se llaman \strong{abiertos} de la topología, y el complemento de un abierto se dice un
conjunto \strong{cerrado}.
El par $(X, \tau)$ se dice un \emph{espacio topológico} (usualmente obviamos a $\tau$), y a los elementos de $X$ les
llamamos \emph{puntos} del espacio.

Si $U$ es un abierto que contiene a un punto $x \in X$, decimos que es una \emph{vecindad} de $x$.
Una familia de abiertos $\mathcal{B}$ se dice una \strong{base de la topología} si para cada punto $x$ y cada vecindad
suya $U$, existe un abierto $V \in \mathcal{B}$ en la base tal que $x \in V \subseteq U$.

Una familia de abiertos $\{ U_i \}_{i\in I}$ tales que $\bigcup_{i\in I} U_i = X$ se dice un \emph{cubrimiento} del
espacio $X$.
Un espacio topológico $X$ se dice \strong{(cuasi)compacto} si todo cubrimiento $\{ U_i \}_{i\in I}$ admite un
subcubrimiento $\{ U_{i_j} \}_{j=1}^n$ finito.

Una función entre espacios topológicos $f \colon X \to Y$ se dice \strong{continua} si la preimagen de todo abierto (de
$Y$) es abierta (en $X$).
Esto puede verificarse en una base de $Y$.
Un \emph{homeomorfismo} es una biyección entre espacios topológicos que es continua y cuya inversa es también
continua.
Se dice que $f \colon X \to Y$ es un \strong{encaje cerrado} (resp.\ \strong{abierto}) si $f \colon X \to f[X]$ es un
homeomorfismo y $f[X] \subseteq Y$ es un subconjunto cerrado (resp.\ abierto).

Podrían ser útil los siguientes criterios:
\begin{prop}
	Para una función $f \colon X \to Y$ entre espacios topológicos son equivalentes:
	\begin{enumerate}
		\item $f$ es un homeomorfismo.
		\item $f$ es biyectiva y una \emph{función abierta} (i.e., la imagen de abierto es abierta).
		\item $f$ es biyectiva y una \emph{función cerrada} (i.e., la imagen de cerrado es cerrada).
	\end{enumerate}
\end{prop}

Un espacio topológico $X$ se dice \strong{disconexo} si existen dos abiertos $U, V \subseteq X$ disjuntos tales que $U
\cup V = X$.
Si, por el contrario, $X$ no es disconexo, decimos que es \emph{conexo}.

Dada una familia de espacios topológicos $(X_i)_{i\in I}$ podemos definir la suma disjunta de sus espacios como la unión
disjunta $\coprod_{i\in I} X_i$ con la topología en la que los abiertos son de la forma $\coprod_{i\in I} U_i$, donde
cada $U_i \subseteq X_i$ es abierto respectivamente.

Un espacio topológico $X$ se dice \strong{de Hausdorff} si todo par de puntos distintos $x, y \in X$ admiten vecindades
$x \in U$ e $y \in V$ disjuntas.

\newpage
\section{Espectro de Zariski}
\begin{enumerate}
	\item Dado un anillo no nulo $A$, defina $\Spec A$ como su conjunto de ideales primos.
		Dado un subconjunto $S \subseteq A$, definiremos su lugar de anulamiento como
		\[
			\VV(S) := \{ \mathfrak{p} \in \Spec A : \mathfrak{p} \supseteq S \}.
		\]
		\begin{enumerate}
			\item Pruebe que la familia $\tau := \{ \Spec A \setminus \VV(S) : S \subseteq A \}$ determina
				una topología sobre $\Spec A$.
				A este espacio le llamamos el \strong{espectro (de Zariski)} de $A$.
			\item Describa $\Spec A$ cuando $A$ es DIP (dominio de ideales principales).

				\lookup
				¿Es $\Spec A$ un espacio de Hausdorff en general?
			\item Para $f \in A$ defina
				\[
					\DD(f) := \Spec A \setminus \VV(f) = \{ \mathfrak{p} \in \Spec A : f \notin \mathfrak{p} \}.
				\]
				Pruebe que la familia $\{ \DD(f) : f\in A \}$ forma una base de la topología.
				% (i.e., para cada punto $\mathfrak{p} \in \Spec A$ y cada abierto que le contiene
				% $\mathfrak{p} \in U$, hay un $f$ tal que $\mathfrak{p} \in \DD(f) \subseteq U$).
		\end{enumerate}
		Para $f, g \in A$ pruebe que:
		\begin{enumerate}[resume]
			\item $\DD(f) = \Spec A$ syss $f$ es inversible.
			\item $\DD(f) = \emptyset$ syss $f$ es nilpotente.
			\item\lookst
				Pruebe que $\Spec A$ es compacto.
			% \item Por razones psicológicas denotaremos por $x_{\mathfrak{p}}$ a un ideal primo $\mathfrak{p}
			% 	\nsl A$ visto como un punto de $\Spec A$, y recíprocamente dado un punto $x \in \Spec A$
			% 	denotamos por $\mathfrak{p}_x$ a su ideal primo.

			% 	Muestre que $\overline{\{ x \}} = \VV(\mathfrak{p}_x)$, de modo que $y \in \overline{\{
			% 	x \}}$ syss $\mathfrak{p}_y \supseteq \mathfrak{p}_x$.
			% 	Así $x$ es un punto cerrado (i.e., $\overline{\{ x \}}$) syss $\mathfrak{p}_x$ es maximal.
			% \item Para $f \in A$ denotamos por
			% 	\[
			% 		\DD(f) := \Spec A \setminus \VV(fA) = \{ \mathfrak{p} \in \Spec A : f \notin
			% 		\mathfrak{p} \}.
			% 	\]
			% 	Pruebe que la familia $\{ \DD(f) : f \in A \}$ es una base para la topología, es decir,
			% 	que para todo punto $x \in \Spec A$ y todo abierto $x \in U \subseteq \Spec A$, existe
			% 	$f$ tal que $x \in \DD(f) \subseteq U$.
		\end{enumerate}
	\item Sea $\varphi \colon A \to B$ un homomorfismo de anillos, pruebe que
		\[
			\varphi^a \colon \Spec B \longrightarrow \Spec A, \qquad
			\mathfrak{q} \longmapsto \varphi^{-1}[\mathfrak{q}]
		\]
		es una función continua entre espacios topológicos.
		Además, pruebe que:
		\begin{enumerate}
			\item Si $\psi \colon B \to C$ es otro homomorfismo de anillos, entonces $(\psi\circ \varphi)^a
				= \varphi^a\circ \psi^a$.
				
				\lookup
				Esto sumado al hecho de que $\Id_A^a = \Id_{\Spec A}$ diría que el espectro de Zariski
				constituye un \emph{funtor contravariante} desde la categoría de anillos a la de
				espacios topológicos.
			\item\label{prob:quot_closed_emb}
				Si $\varphi$ es un epimorfismo, entonces $\varphi^a$ es un encaje cerrado que identifica a
				$\Spec B$ con el cerrado $\VV(\ker\varphi)$.
			\item Si $\varphi$ es la localización $B = A[1/f]$ para $f \in A$, entonces $\varphi^a$ es un
				encaje abierto que identifica $\Spec B$ con $\DD(f)$.
		\end{enumerate}

	\item Sean $A_1, \dots, A_n$ una tupla de anillos.
		Pruebe que los ideales primos del producto $A_1 \times \cdots \times A_n$ son de la forma
		\[
			A_1 \times \cdots \times A_{j-1} \times \mathfrak{p}_j \times A_{j+1} \times \cdots \times A_n,
		\]
		donde $\mathfrak{p}_j \nsl A_j$ es primo.

		Concluya que $\Spec(A_1 \times \cdots \times A_n) = \Spec A_1 \amalg \cdots \amalg \Spec A_n$.

		\begin{sol}
			Sea $\mathfrak{P} \nsl \prod_{j} A_j$ un primo, y denotemos por
			\[
				e_i := (0, \dots, 0, \overset{(i)}{1}, 0, \dots, 0) \in \prod_{j} A_j
			\]
			al elemento tal que $(e_i) \cong A_i$.
			Nótese que $e_i$ es un idempotente: $e_i^2 = e_i$, es decir, $e_i(e_i - 1) = 0$.
			Más aún, si $\mathfrak{P}$ contuviera a cada $e_i$, tendría a $1 = \sum_{j} e_j$, por lo que hay
			algún $e_j \notin \mathfrak{P}$, luego $1 - e_j \in \mathfrak{P}$ (pues $0 = e_j(1 - e_j)$).

			Finalmente, nótese que $e_i(1 - e_j) = e_i$ para $i \ne j$, así que $e_i \in \mathfrak{P}$ y,
			por tanto,
			\[
				\mathfrak{P} = A_1 \times \cdots \times A_{j-1} \times \mathfrak{p} \times A_{j+1} \times
				\cdots \times A_n.
			\]
			Ahora, sean $a, b \in A_j$ tales que $ab \in \mathfrak{p}$, es decir, $(ae_j)(be_j) \in
			\mathfrak{P}$, por lo que $ae_j \in \mathfrak{P}$ o $be_j \in \mathfrak{P}$; se sigue que $a\in
			\mathfrak{p}$ o $b \in \mathfrak{p}$.
			Así, $\mathfrak{p} \nsl A_j$ es primo.

			Más aún, así probamos que hay una biyección $\Spec A_j \to \VV(1 - e_j)$ que es de hecho un
			encaje cerrado por el problema~\ref{prob:quot_closed_emb}, y $\VV(1 - e_j) = \DD(e_j)$, así que
			es un abierto y cerrado en $X := \Spec(\prod_{j} A_j)$.
			En la demostración vimos que $\bigcup_{j} \VV(1 - e_j) = X$ y $\DD(e_j) \cap \DD(e_k) =
			\emptyset$ para $j \ne k$.
			Esto prueba que $X = \coprod_{j} \Spec A_j$.
		\end{sol}

	\item Pruebe que, para un anillo $A$, son equivalentes:
		\begin{enumerate}
			\item $\Spec A$ es disconexo.
			\item $A$ contiene un elemento idempotente $e$ (i.e., tal que $e^2 = e$) distinto del 0 y del 1.
			\item Existen $A_1, A_2$ no nulos tales que $A \cong A_1 \times A_2$ como anillos.
		\end{enumerate}

		\begin{sol}
			La equivalencia <<$b) \iff c)$>> es porque $A_1 := (e)$ es un anillo con $1_{A_1} = e$ y $A_2 =
			(1 - e)$ con $1_{A_2} = 1 - e$ (recuerde que $(1-e)^2 = 1-e$).
			Luego el isomorfismo es
			\[
				A \longrightarrow A_1\times A_2, \qquad a \longmapsto (ae, a(1-e))
			\]
			Recíprocamente, si $A \cong A_1\times A_2$, entonces $e := (1, 0)$ es un idempotente.

			La implicancia <<$c) \implies a)$>> es por el problema anterior.

			$a) \implies b)$.
			Sean $U = \VV(\mathfrak{a})^c$, $V := \VV(\mathfrak{b})^c$ abiertos tales que $U \cup V = \Spec
			A$ y $U \cap V = \emptyset$.
			Entonces $\VV(\mathfrak{a + b})^c = U \cup V = \Spec A$ equivale a que $\mathfrak{a + b} = (1)$;
			y $\VV(\mathfrak{a\cdot b})^c = U \cap V = \emptyset$ equivale a que $\mathfrak{a\cdot b}
			\subseteq \mathfrak{N}$.
			Así, existen $a \in \mathfrak{a}$, $b \in \mathfrak{b}$ y un entero $n \ge 1$ tales que
			\[
				a + b = 1, \qquad (ab)^n = 0.
			\]
			La primera condición ahora equivale a que $\DD(a) \cup \DD(b) = \Spec A$.
			Ahora, nótese que $\DD(x) = \DD(x^m)$ para todo $m \ge 1$ pues si $x \notin \mathfrak{p}$,
			entonces sus potencias tampoco y recíprocamente, por lo que, $\DD(a^n) \cup \DD(b^n) = \Spec A$.
			Así, existen $e \in (a^n)$ y $f \in (b^n)$ tales que
			\[
				e + f = 1, \qquad ef \in (a^nb^n) = (0).
			\]
			Finalmente, $f = 1 - e$ y $e(1-e) = e - e^2 = 0$.
		\end{sol}

	% \item Decimos que un anillo $B$ es \strong{reducido} si $\mathfrak{N}(B) = 0$, es decir, si el único nilpotente
	% 	de $B$ es 0 (e.g., todo dominio íntegro es reducido).

	% 	Sea $A$ un anillo y supongamos que para todo primo $\mathfrak{p}$ se cumple que $A_{\mathfrak{p}}$ es
	% 	reducido.
	% 	Pruebe que $A$ es reducido.
	% 	¿Si cada $A_{\mathfrak{p}}$ es un dominio íntegro, entonces $A$ también?
\end{enumerate}

\appendix
\section{Comentarios adicionales}
La topología sobre $\Spec A$ le fue sugerida por un oyente de Zariski en una charla suya.
El geómetra bielorruso Oscar Zariski popularizó el uso del álgebra conmutativa en la geometría algebraica con dos
notables ventajas: la primera es que le permitió formalizar y extender los límites de la teoría, así como generalizar
una serie de resultados ya conocidos, o simplemente \emph{algebrizarlos} (es decir, evitar métodos analíticos para poder
demostrarlos).

Inspirado en él, el francés André Weil trabajó para crear un nuevo lenguaje en la geometría algebraica en donde el
álgebra conmutativa toma un rol más predominante, lo que culminó en su libro \emph{Foundations of Algebraic Geometry};
mientras que Zariski trabajó para organizar la maquinaria algebraica, culminando en los dos volúmenes \emph{Commutative
Algebra} con Pierre Samuel.

Estos programos suyos fueron finalmente completados con la llegada de los \emph{esquemas} por el matemático Alexander
Grothendieck, quién también desarrolló otro lenguaje --capital hoy en día-- dentro del cual los espectros forman el
esqueleto fundamental.
No obstante, Grothendieck dota al $\Spec A$ de una <<estructura adicional>> (un \emph{haz}) que permite recuperar al
anillo mismo (vale decir, el espacio $\Spec A$ no es un buen invariante, ya que todo cuerpo tiene por espectro a un
punto); con ella, el $\Spec A$ pasa a ser un \emph{esquema afín} en su terminología.
Para estudiar esta teoría se recomienda ver, por ejemplo, \citeauthor{hartshorne:algebraic}~\cite{hartshorne:algebraic}
o \citeauthor{vakil:rising_sea}~\cite{vakil:rising_sea}.

\printbibliography

\end{document}
