\documentclass[11pt, reqno]{amsart}

\usepackage[spanish]{babel}
\usepackage[LGR, T1]{fontenc}
\usepackage[utf8]{inputenc}

../LaTeX/general.tex
% \input{../graphics.tex}

\makeatletter
\def\emailaddrname{\textit{Correo electrónico}}
\def\subtitle#1{\gdef\@subtitle{#1}}
\def\@subtitle{}

% Metadata
\def\logo#1{\gdef\@logo{#1}}
\def\@logo{}
\def\institution#1{\gdef\@institution{#1}}
\def\@institution{}
\def\department#1{\gdef\@department{#1}}
\def\@department{}
\def\professor#1{\gdef\@professor{#1}}
\def\@professor{}
\def\course#1{\gdef\@course{#1}}
\def\@course{}
\def\coursecode#1{\gdef\@coursecode{#1}}
\def\@coursecode{}

\renewcommand{\maketitle}{
\begin{center}
	\small
	\renewcommand{\arraystretch}{1.2}
	\begin{tabular}{cp{.37\textwidth}p{0.44\textwidth}}
		% \hline
		\multirow{5}{*}{\includegraphics[height=2.0cm]{\@logo}}
	  & \multicolumn{2}{c}{ \makecell{{\bfseries \@institution} \\ \@department} } \\
	  % & \multicolumn{2}{c|}{{\bfseries\@institution} \\ \@department} \\
	  \cline{2-3}
	  & \textbf{Profesor:} \@professor & \textbf{Ayudante:} \authors \\
	  % \cline{2-3}
	  & \textbf{Curso:} \@course & \textbf{Sigla:} \@coursecode \\
	  % \cline{2-3}
	  & \multicolumn{2}{l}{ \textbf{Fecha:} \@date } \\
	  % \hline
	\end{tabular}
	\\[\baselineskip]
	% {}
	% \vspace{2\baselineskip}
	{\bfseries\Large\@title}
	\ifx\@subtitle\@empty\else
		\\[1ex]
		\large\mdseries\@subtitle
	\fi
\end{center}
}
\makeatother

\usepackage{multirow, makecell}

\usepackage[
	reversemp,
	letterpaper,
	% marginpar=2cm,
	% marginsep=1pt,
	margin=2.3cm
]{geometry}
\usepackage{fontawesome}
% \makeatletter
% \@reversemargintrue
% \makeatother

% Símbolos al margen, necesitan doble compilación
\newcommand{\hard}{\marginnote{\faFire}}
\newcommand{\hhard}{\marginnote{\faFire\faFire}}

% Dependencias para los teoremas
\usepackage{xifthen}
\def\@thmdep{}
\newcommand{\thmdep}[1]{
	\ifthenelse{\isempty{#1}}
	{\def\@thmdep{}}
	{\def\@thmdep{ (#1)}}
}
\newcommand{\thmstyle}{\color{thm}\sffamily\bfseries}

% ===== Estilos de Teoremas ==========
\newtheoremstyle{axiomstyle}
	{0.3cm}
	{0.3cm}
	{\normalfont}
	{0.5cm}
	{\bfseries\scshape}
	{:}
	{4pt}
	{\thmname{#1}\thmnote{ #3}\thmnumber{ (#2)}}
\newtheoremstyle{styleC}
	{0.5cm}
	{0.5cm}
	{\normalfont}
	{0.5cm}
	{\bfseries}
	{:}
	{4pt}
	{\thmname{#1\textrm{\@thmdep}}\thmnumber{ #2}\thmnote{ (#3)}}

% ====== Teoremas (sin borde) ===========
\theoremstyle{axiomstyle}
\newtheorem*{axiom}{Axioma}

% ====== Teoremas (sin borde) ==================
\theoremstyle{styleC}
\newtheorem{thm}{Teorema}[section]
\newtheorem{mydef}[thm]{Definición}
\newtheorem{prop}[thm]{Proposición}
\newtheorem{cor}[thm]{Corolario}
\newtheorem{lem}[thm]{Lema}
\newtheorem{con}[thm]{Conjetura}

\newtheorem*{prob}{Problema}
\newtheorem*{sol}{Solución}
\newtheorem*{obs}{Observación}
\newtheorem*{ex}{Ejemplo}

% \usepackage{tcolorbox}
% \newtcbox{bluebox}[1][]{enhanced jigsaw, 
%   sharp corners,
%   frame hidden,
%   nobeforeafter,
%   listing only,
%   #1} % comando para crear cajas de colores

\expandafter\let\expandafter\oldproof\csname\string\proof\endcsname
\let\oldendproof\endproof
\renewenvironment{proof}[1][\proofname]{%
  \oldproof[\scshape Demostración:]%
}{\oldendproof} % comando para redefinir la caja de la demostración
\newenvironment{hint}[1][\proofname]{%
  \oldproof[\scshape Pista:]%
}{\oldendproof} % comando para redefinir la caja de la demostración

% colores utilizados
\definecolor{numchap}{RGB}{249,133,29}
\definecolor{chap}{RGB}{6,129,204}
\definecolor{sec}{RGB}{204,0,0}
\definecolor{thm}{RGB}{106,176,240}
\definecolor{thmB}{RGB}{32,31,31}
\definecolor{part}{RGB}{212,66,66}

% ====== Diseño de los titulares ===============
\usepackage[explicit]{titlesec} % para personalizar el documento, la opción <<explicit>> hace que el texto de los titulares sea un objeto interactuable

\titleformat{\subsection}[runin]
	{\bfseries}
	{\textrm{\S}\thesubsection}
	{1ex}
	{#1.}

\setlist[enumerate,1]{label=\arabic*., ref=\arabic*} % Enumerate standards

\usepackage{tikz-cd}
% \DeclareMathOperator\Gr{Gr}

\title{Localizaciones y anillos noetherianos}
\date{\DTMdate{2025-05-15}}

\author{José Cuevas Barrientos}
\email{josecuevasbtos@uc.cl}
\urladdr{https://josecuevas.xyz/teach/2025-1-ayud/}

\logo{../puc_negro.png}
\institution{Pontificia Universidad Católica de Chile}
\department{Facultad de Matemáticas}
\course{Álgebra abstracta II}
\coursecode{MAT2244}
\professor{Héctor Pastén Vásquez}

\begin{document}

\maketitle

\nocite{atiyah:commutative}
\nocite{jacobson:basic}
% \section{Números constructibles}

\section*{Preliminares}
Sea $A$ un anillo (conmutativo).
Un \strong{$A$-módulo} $M$ es un grupo abeliano aditivo $(M, +)$ con un <<producto escalar>> $\cdot \colon A\times M \to
M$ tales que para todo $a, b \in A$ y $m, n \in M$ se cumple:
\begin{enumerate}
	\item $1\cdot m = m$.
	\item $a\cdot (b\cdot m) = (ab)\cdot m$.
	\item $a\cdot(m + n) = a\cdot m + a\cdot n$.
	\item $(a + b)\cdot m = a\cdot m + b\cdot m$.
\end{enumerate}
Se sigue que $0\cdot m = 0\in M$ (recuerde que $M$ es aditivo así que tiene un <<0>>).

Dado un par de $A$-módulos $M, N$, una función $\varphi \colon M \to N$ se dice un \strong{homomorfismo de $A$-módulos}
si es un homomorfismo de grupos aditivos $(M, +) \to (N, +)$ y respeta producto escalar:
\[
	\forall a \in A, \; m\in M, \qquad \varphi(am) = a \varphi(m).
\]

Dado un conjunto multiplicativo $S \subseteq A$ y un $A$-módulo $M$ podemos construir el $S^{-1}A$-módulo $S^{-1}M$
cuyos elementos son pares $(m, s)$ con $m \in M$ y $s \in S$ bajo la equivalencia
\[
	m/s = m'/s' \iff \exists t \in S \quad t(s'm - sm) = 0 \in M.
\]
Con las sumas y producto escalar:
\[
	\frac{m}{s} + \frac{n}{t} := \frac{tm + sn}{st}, \qquad \frac{a}{t} \cdot \frac{m}{s} := \frac{am}{ts}.
\]

Dado un ideal primo $\mathfrak{p} \nsl A$, considere $S := A \setminus \mathfrak{p}$ el cual es un conjunto
multiplicativo (¿por qué?), denotaremos por $A_{\mathfrak{p}} := S^{-1}A$ a la localización.

\section{Propiedades <<locales>>}
\begin{enumerate}
	\item\lookright
		(Examen de lucidez)
		Sea $A$ un anillo.
		\begin{enumerate}
			\item Pruebe que, dado un primo $\mathfrak{p} \nsl A$, el anillo $A_{\mathfrak{p}}$ es local y
				que su único ideal maximal es
				\[
					\mathfrak{p_p} = \mathfrak{p}A_{\mathfrak{p}} =
					\left\{ \frac{p}{q} : p \in \mathfrak{p}, q \notin \mathfrak{p} \right\}.
				\]
			\item Si $A$ es dominio íntegro, describa la localización $A_{(0)}$.
		\end{enumerate}

	\item (Funtorialidad de localización)
		Sea $\varphi \colon M \to N$ un homomorfismo de $A$-módulos.
		\begin{enumerate}
			\item Pruebe que la función $M \to S^{-1}M$ dada por $m \mapsto m/1$ es un homomorfismo de
				$A$-módulos.
			\item Pruebe que la función $S^{-1}\varphi \colon S^{-1}M \to S^{-1}N$ dada por $m/s \mapsto
				\varphi(m)/s$ es un homomorfismo de $S^{-1}A$-módulos.
		\end{enumerate}

	\item Sea $A$ un anillo y $M$ un $A$-módulo.
		Pruebe que son equivalentes:
		\begin{enumerate}
			\item $M = 0$.
			\item $M_{\mathfrak{p}} = 0$ para todo $\mathfrak{p} \nsl A$ primo.
			\item $M_{\mathfrak{m}} = 0$ para todo $\mathfrak{m} \nsl A$ maximal.
		\end{enumerate}

	\item Sea $\varphi \colon M \to N$ un homomorfismo de $A$-módulos.
		Pruebe que son equivalentes:
		\begin{enumerate}
			\item $\varphi$ es inyectiva (resp.\ sobreyectiva, biyectiva).
			\item $\varphi_{\mathfrak{p}}$ es inyectiva (resp.\ sobreyectiva, biyectiva) para todo
				$\mathfrak{p} \nsl A$ primo.
			\item $\varphi_{\mathfrak{m}}$ es inyectiva (resp.\ sobreyectiva, biyectiva) para todo
				$\mathfrak{m} \nsl A$ maximal.
		\end{enumerate}

	\item Un anillo $A$ se dice \strong{reducido} si su nilradical $\nilrad(A) = 0$.
		Pruebe que $A$ es reducido syss cada localización $A_{\mathfrak{p}}$ (donde $\mathfrak{p} \nsl A$
		recorre los ideales primos) es reducida.

		\begin{prob}
			\lookup
			¿Es cierto que $A$ es un dominio íntegro syss cada localización $A_{\mathfrak{p}}$ es un dominio
			íntegro?
		\end{prob}
\end{enumerate}

\section{Anillos noetherianos}
\begin{enumerate}[resume]
	\item Sea $A$ un anillo noetheriano.
		Pruebe que toda $A$-álgebra finitamente generada es también noetheriana.
	\item Sea $A$ un anillo noetheriano. 
		Pruebe que existe un entero $n \ge 1$ tal que la potencia del nilradical $\mathfrak{N}^n = 0$.

		\begin{prob}
			\lookup
			Dé un contraejemplo de un nilradical cuyas potencias jamás son el ideal nulo.
		\end{prob}
\end{enumerate}

\appendix
\section{Ejercicios propuestos}
\begin{enumerate}
	\item\lookright
		Sea $A$ un anillo y $\mathfrak{p} \nsl A$ un ideal primo.
		Pruebe que $A_{\mathfrak{p}}/\mathfrak{p_p}$ es isomorfo al cuerpo de fracciones
		$\Frac(A/\mathfrak{p})$.

	\item Sea $A$ un anillo y $\mathfrak{p} \nsl A$ un ideal primo.
		\begin{enumerate}
			\item Describa $\Spec A_{\mathfrak{p}}$ en términos de $\Spec A$.
			\item ¿Qué sucede con $\Spec A_{\mathfrak{p}}$ cuándo $\mathfrak{p}$ es maximal?
		\end{enumerate}

		\begin{hint}
			Para este ejercicio podría resultar conveniente recordar que, al localizar con $S = \{ f^n : n
			\in \N \}$ para $f \in A$, se cumple que
			\begin{equation}
				\Spec(S^{-1}A) = \{ \mathfrak{p} : \mathfrak{p} \cap S = \emptyset \} \subseteq \Spec A.
				\tqedhere
			\end{equation}
		\end{hint}

	\item Un espacio topológico $X$ se dice \emph{noetheriano} si toda cadena descendente de cerrados
		\[
			F_0 \supseteq F_1 \supseteq F_2 \supseteq \cdots,
		\]
		se estabiliza, es decir, existe $n$ para el cual $F_n = F_{n+1} = \cdots$

		\begin{enumerate}
			\item Pruebe que si $A$ es anillo noetheriano, entonces $\Spec A$ es un espacio noetheriano.
			\item\lookup
				Dé un ejemplo de un anillo no noetheriano $A$ cuyo espectro $\Spec A$ sí es noetheriano.
		\end{enumerate}
\end{enumerate}

% \appendix
% \section{Comentarios adicionales}

\printbibliography

\end{document}
