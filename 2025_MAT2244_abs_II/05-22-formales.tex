\documentclass[11pt, reqno]{amsart}

\usepackage[spanish]{babel}
\usepackage[LGR, T1]{fontenc}
\usepackage[utf8]{inputenc}

\input{../general.tex}
% \input{../graphics.tex}

\makeatletter
\def\emailaddrname{\textit{Correo electrónico}}
\def\subtitle#1{\gdef\@subtitle{#1}}
\def\@subtitle{}

% Metadata
\def\logo#1{\gdef\@logo{#1}}
\def\@logo{}
\def\institution#1{\gdef\@institution{#1}}
\def\@institution{}
\def\department#1{\gdef\@department{#1}}
\def\@department{}
\def\professor#1{\gdef\@professor{#1}}
\def\@professor{}
\def\course#1{\gdef\@course{#1}}
\def\@course{}
\def\coursecode#1{\gdef\@coursecode{#1}}
\def\@coursecode{}

\renewcommand{\maketitle}{
\begin{center}
	\small
	\renewcommand{\arraystretch}{1.2}
	\begin{tabular}{cp{.37\textwidth}p{0.44\textwidth}}
		% \hline
		\multirow{5}{*}{\includegraphics[height=2.0cm]{\@logo}}
	  & \multicolumn{2}{c}{ \makecell{{\bfseries \@institution} \\ \@department} } \\
	  % & \multicolumn{2}{c|}{{\bfseries\@institution} \\ \@department} \\
	  \cline{2-3}
	  & \textbf{Profesor:} \@professor & \textbf{Ayudante:} \authors \\
	  % \cline{2-3}
	  & \textbf{Curso:} \@course & \textbf{Sigla:} \@coursecode \\
	  % \cline{2-3}
	  & \multicolumn{2}{l}{ \textbf{Fecha:} \@date } \\
	  % \hline
	\end{tabular}
	\\[\baselineskip]
	% {}
	% \vspace{2\baselineskip}
	{\bfseries\Large\@title}
	\ifx\@subtitle\@empty\else
		\\[1ex]
		\large\mdseries\@subtitle
	\fi
\end{center}
}
\makeatother

\usepackage{multirow, makecell}

\usepackage[
	reversemp,
	letterpaper,
	% marginpar=2cm,
	% marginsep=1pt,
	margin=2.3cm
]{geometry}
\usepackage{fontawesome}
% \makeatletter
% \@reversemargintrue
% \makeatother

% Símbolos al margen, necesitan doble compilación
\newcommand{\hard}{\marginnote{\faFire}}
\newcommand{\hhard}{\marginnote{\faFire\faFire}}

% Dependencias para los teoremas
\usepackage{xifthen}
\def\@thmdep{}
\newcommand{\thmdep}[1]{
	\ifthenelse{\isempty{#1}}
	{\def\@thmdep{}}
	{\def\@thmdep{ (#1)}}
}
\newcommand{\thmstyle}{\color{thm}\sffamily\bfseries}

% ===== Estilos de Teoremas ==========
\newtheoremstyle{axiomstyle}
	{0.3cm}
	{0.3cm}
	{\normalfont}
	{0.5cm}
	{\bfseries\scshape}
	{:}
	{4pt}
	{\thmname{#1}\thmnote{ #3}\thmnumber{ (#2)}}
\newtheoremstyle{styleC}
	{0.5cm}
	{0.5cm}
	{\normalfont}
	{0.5cm}
	{\bfseries}
	{:}
	{4pt}
	{\thmname{#1\textrm{\@thmdep}}\thmnumber{ #2}\thmnote{ (#3)}}

% ====== Teoremas (sin borde) ===========
\theoremstyle{axiomstyle}
\newtheorem*{axiom}{Axioma}

% ====== Teoremas (sin borde) ==================
\theoremstyle{styleC}
\newtheorem{thm}{Teorema}[section]
\newtheorem{mydef}[thm]{Definición}
\newtheorem{prop}[thm]{Proposición}
\newtheorem{cor}[thm]{Corolario}
\newtheorem{lem}[thm]{Lema}
\newtheorem{con}[thm]{Conjetura}

\newtheorem*{prob}{Problema}
\newtheorem*{sol}{Solución}
\newtheorem*{obs}{Observación}
\newtheorem*{ex}{Ejemplo}

% \usepackage{tcolorbox}
% \newtcbox{bluebox}[1][]{enhanced jigsaw, 
%   sharp corners,
%   frame hidden,
%   nobeforeafter,
%   listing only,
%   #1} % comando para crear cajas de colores

\expandafter\let\expandafter\oldproof\csname\string\proof\endcsname
\let\oldendproof\endproof
\renewenvironment{proof}[1][\proofname]{%
  \oldproof[\scshape Demostración:]%
}{\oldendproof} % comando para redefinir la caja de la demostración
\newenvironment{hint}[1][\proofname]{%
  \oldproof[\scshape Pista:]%
}{\oldendproof} % comando para redefinir la caja de la demostración

% colores utilizados
\definecolor{numchap}{RGB}{249,133,29}
\definecolor{chap}{RGB}{6,129,204}
\definecolor{sec}{RGB}{204,0,0}
\definecolor{thm}{RGB}{106,176,240}
\definecolor{thmB}{RGB}{32,31,31}
\definecolor{part}{RGB}{212,66,66}

% ====== Diseño de los titulares ===============
\usepackage[explicit]{titlesec} % para personalizar el documento, la opción <<explicit>> hace que el texto de los titulares sea un objeto interactuable

\titleformat{\subsection}[runin]
	{\bfseries}
	{\textrm{\S}\thesubsection}
	{1ex}
	{#1.}

\setlist[enumerate,1]{label=\arabic*., ref=\arabic*} % Enumerate standards

\usepackage{tikz-cd}
% \DeclareMathOperator\Gr{Gr}

\title{El anillo de series formales}
\date{\DTMdate{2025-05-22}}

\author{José Cuevas Barrientos}
\email{josecuevasbtos@uc.cl}
\urladdr{https://josecuevas.xyz/teach/2025-1-ayud/}

\logo{../puc_negro.png}
\institution{Pontificia Universidad Católica de Chile}
\department{Facultad de Matemáticas}
\course{Álgebra abstracta II}
\coursecode{MAT2244}
\professor{Héctor Pastén Vásquez}

\begin{document}

\maketitle

\nocite{atiyah:commutative}
\nocite{matsumura:ring}
% \section{Números constructibles}

\section{QQQ}
Sea $A$ un anillo, construiremos la álgebra de series formales $A[\![x]\!]$ como aquella en donde los elementos son series formales infinitas
\[
	f := \sum_{n=0}^{\infty} a_n x^n, \qquad g := \sum_{n=0}^{\infty} b_n x^n
\]
con la suma y producto
\[
	f+g := \sum_{n=0}^{\infty} (a_n+b_n)x^n, \qquad \sum_{n=0}^{\infty} \left( \sum_{j=0}^{n} a_jb_{n-j} \right)x^n.
\]

\begin{enumerate}
	\item\lookright
		(Examen de lucidez)
		Sea $A$ un anillo.
		\begin{enumerate}
			\item Pruebe que $A[\![x]\!]$ es un anillo.
			\item Pruebe que 
		\end{enumerate}
\end{enumerate}

% \appendix
% \section{Ejercicios propuestos}
% \begin{enumerate}
% 	\item\lookright
% 		Sea $A$ un anillo y $\mathfrak{p} \nsl A$ un ideal primo.
% 		Pruebe que $A_{\mathfrak{p}}/\mathfrak{p_p}$ es isomorfo al cuerpo de fracciones
% 		$\Frac(A/\mathfrak{p})$.

% 	\item Sea $A$ un anillo y $\mathfrak{p} \nsl A$ un ideal primo.
% 		\begin{enumerate}
% 			\item Describa $\Spec A_{\mathfrak{p}}$ en términos de $\Spec A$.
% 			\item ¿Qué sucede con $\Spec A_{\mathfrak{p}}$ cuándo $\mathfrak{p}$ es maximal?
% 		\end{enumerate}

% 		\begin{hint}
% 			Para este ejercicio podría resultar conveniente recordar que, al localizar con $S = \{ f^n : n
% 			\in \N \}$ para $f \in A$, se cumple que
% 			\begin{equation}
% 				\Spec(S^{-1}A) = \{ \mathfrak{p} : \mathfrak{p} \cap S = \emptyset \} \subseteq \Spec A.
% 				\tqedhere
% 			\end{equation}
% 		\end{hint}

% 	\item Un espacio topológico $X$ se dice \emph{noetheriano} si toda cadena descendente de cerrados
% 		\[
% 			F_0 \supseteq F_1 \supseteq F_2 \supseteq \cdots,
% 		\]
% 		se estabiliza, es decir, existe $n$ para el cual $F_n = F_{n+1} = \cdots$

% 		\begin{enumerate}
% 			\item Pruebe que si $A$ es anillo noetheriano, entonces $\Spec A$ es un espacio noetheriano.
% 			\item\lookup
% 				Dé un ejemplo de un anillo no noetheriano $A$ cuyo espectro $\Spec A$ sí es noetheriano.
% 		\end{enumerate}
% \end{enumerate}

% \appendix
% \section{Comentarios adicionales}

\printbibliography

\end{document}
