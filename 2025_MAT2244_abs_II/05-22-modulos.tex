\documentclass[11pt, reqno]{amsart}

\usepackage[spanish]{babel}
\usepackage[LGR, T1]{fontenc}
\usepackage[utf8]{inputenc}

../LaTeX/general.tex
% \input{../graphics.tex}

\makeatletter
\def\emailaddrname{\textit{Correo electrónico}}
\def\subtitle#1{\gdef\@subtitle{#1}}
\def\@subtitle{}

% Metadata
\def\logo#1{\gdef\@logo{#1}}
\def\@logo{}
\def\institution#1{\gdef\@institution{#1}}
\def\@institution{}
\def\department#1{\gdef\@department{#1}}
\def\@department{}
\def\professor#1{\gdef\@professor{#1}}
\def\@professor{}
\def\course#1{\gdef\@course{#1}}
\def\@course{}
\def\coursecode#1{\gdef\@coursecode{#1}}
\def\@coursecode{}

\renewcommand{\maketitle}{
\begin{center}
	\small
	\renewcommand{\arraystretch}{1.2}
	\begin{tabular}{cp{.37\textwidth}p{0.44\textwidth}}
		% \hline
		\multirow{5}{*}{\includegraphics[height=2.0cm]{\@logo}}
	  & \multicolumn{2}{c}{ \makecell{{\bfseries \@institution} \\ \@department} } \\
	  % & \multicolumn{2}{c|}{{\bfseries\@institution} \\ \@department} \\
	  \cline{2-3}
	  & \textbf{Profesor:} \@professor & \textbf{Ayudante:} \authors \\
	  % \cline{2-3}
	  & \textbf{Curso:} \@course & \textbf{Sigla:} \@coursecode \\
	  % \cline{2-3}
	  & \multicolumn{2}{l}{ \textbf{Fecha:} \@date } \\
	  % \hline
	\end{tabular}
	\\[\baselineskip]
	% {}
	% \vspace{2\baselineskip}
	{\bfseries\Large\@title}
	\ifx\@subtitle\@empty\else
		\\[1ex]
		\large\mdseries\@subtitle
	\fi
\end{center}
}
\makeatother

\usepackage{multirow, makecell}

\usepackage[
	reversemp,
	letterpaper,
	% marginpar=2cm,
	% marginsep=1pt,
	margin=2.3cm
]{geometry}
\usepackage{fontawesome}
% \makeatletter
% \@reversemargintrue
% \makeatother

% Símbolos al margen, necesitan doble compilación
\newcommand{\hard}{\marginnote{\faFire}}
\newcommand{\hhard}{\marginnote{\faFire\faFire}}

% Dependencias para los teoremas
\usepackage{xifthen}
\def\@thmdep{}
\newcommand{\thmdep}[1]{
	\ifthenelse{\isempty{#1}}
	{\def\@thmdep{}}
	{\def\@thmdep{ (#1)}}
}
\newcommand{\thmstyle}{\color{thm}\sffamily\bfseries}

% ===== Estilos de Teoremas ==========
\newtheoremstyle{axiomstyle}
	{0.3cm}
	{0.3cm}
	{\normalfont}
	{0.5cm}
	{\bfseries\scshape}
	{:}
	{4pt}
	{\thmname{#1}\thmnote{ #3}\thmnumber{ (#2)}}
\newtheoremstyle{styleC}
	{0.5cm}
	{0.5cm}
	{\normalfont}
	{0.5cm}
	{\bfseries}
	{:}
	{4pt}
	{\thmname{#1\textrm{\@thmdep}}\thmnumber{ #2}\thmnote{ (#3)}}

% ====== Teoremas (sin borde) ===========
\theoremstyle{axiomstyle}
\newtheorem*{axiom}{Axioma}

% ====== Teoremas (sin borde) ==================
\theoremstyle{styleC}
\newtheorem{thm}{Teorema}[section]
\newtheorem{mydef}[thm]{Definición}
\newtheorem{prop}[thm]{Proposición}
\newtheorem{cor}[thm]{Corolario}
\newtheorem{lem}[thm]{Lema}
\newtheorem{con}[thm]{Conjetura}

\newtheorem*{prob}{Problema}
\newtheorem*{sol}{Solución}
\newtheorem*{obs}{Observación}
\newtheorem*{ex}{Ejemplo}

% \usepackage{tcolorbox}
% \newtcbox{bluebox}[1][]{enhanced jigsaw, 
%   sharp corners,
%   frame hidden,
%   nobeforeafter,
%   listing only,
%   #1} % comando para crear cajas de colores

\expandafter\let\expandafter\oldproof\csname\string\proof\endcsname
\let\oldendproof\endproof
\renewenvironment{proof}[1][\proofname]{%
  \oldproof[\scshape Demostración:]%
}{\oldendproof} % comando para redefinir la caja de la demostración
\newenvironment{hint}[1][\proofname]{%
  \oldproof[\scshape Pista:]%
}{\oldendproof} % comando para redefinir la caja de la demostración

% colores utilizados
\definecolor{numchap}{RGB}{249,133,29}
\definecolor{chap}{RGB}{6,129,204}
\definecolor{sec}{RGB}{204,0,0}
\definecolor{thm}{RGB}{106,176,240}
\definecolor{thmB}{RGB}{32,31,31}
\definecolor{part}{RGB}{212,66,66}

% ====== Diseño de los titulares ===============
\usepackage[explicit]{titlesec} % para personalizar el documento, la opción <<explicit>> hace que el texto de los titulares sea un objeto interactuable

\titleformat{\subsection}[runin]
	{\bfseries}
	{\textrm{\S}\thesubsection}
	{1ex}
	{#1.}

\setlist[enumerate,1]{label=\arabic*., ref=\arabic*} % Enumerate standards

\usepackage{tikz}
\usetikzlibrary{babel,cd}
% \DeclareMathOperator\Gr{Gr}

\title{Más acerca de módulos}
\date{\DTMdate{2025-05-22}}

\author{José Cuevas Barrientos}
\email{josecuevasbtos@uc.cl}
\urladdr{https://josecuevas.xyz/teach/2025-1-ayud/}

\logo{../puc_negro.png}
\institution{Pontificia Universidad Católica de Chile}
\department{Facultad de Matemáticas}
\course{Álgebra abstracta II}
\coursecode{MAT2244}
\professor{Héctor Pastén Vásquez}

\begin{document}

\maketitle

\nocite{atiyah:commutative}
% \nocite{matsumura:ring}
% \section{Números constructibles}

\section{Sobre módulos}
\begin{enumerate}
	\item\lookright
		(Exámen de lucidez)
		Pruebe que todo grupo abeliano $G$ posee una única estructura como $\Z$-módulo.
	\item Sea $A$ un anillo y $M$ un $A$-módulo.
		Para una indeterminada $x$ definimos el conjunto $M[x]$ como aquel formado por las sumas formales
		$\sum_{j=0}^{n} m_jx^j$, donde los coeficientes $m_j \in M$.
		\begin{enumerate}
			\item Pruebe que $M[x]$ es un $A[x]$-módulo con la suma coordenada a coordenada y el producto
				escalar:
				\[
					\left( \sum_{i=0}^{p} a_ix^i \right)\left( \sum_{j=1}^{n} m_jx^j \right) =
					\sum_{\ell=0}^{p+n} \left( \sum_{i+j=\ell} a_im_j \right) x^j.
				\]
			\item Pruebe que si $N \le M$, entonces $N[x] \le M[x]$ de forma canónica.
				En particular, si $\mathfrak{a}$ es un ideal de $A$, entonces $\mathfrak{a}[x]$ es un
				ideal de $A[x]$.
			\item\lookright
				Si $\mathfrak{p}$ es un ideal primo de $A$.
				¿Es cierto que $\mathfrak{p}[x]$ es primo en $A[x]$?
				¿Y si $\mathfrak{m}$ es maximal, será que $\mathfrak{m}[x]$ también?
		\end{enumerate}

	\item Sea $0 \to M_1 \to M_2 \to M_3 \to 0$ una sucesión exacta de $A$-módulos.
		\begin{enumerate}
			\item Pruebe que si $M_1$ y $M_3$ son finitamente generados, entonces $M_2$ también.
			\item Diremos que un $A$-módulo es \strong{noetheriano} si todos sus $A$-submódulos son finitamente generados.
				Pruebe que $M_2$ es noetheriano syss $M_1$ y $M_3$ también lo son.
		\end{enumerate}

	\item Sea $M$ un $A$-módulo.
		\begin{enumerate}
			\item\label{ex:noetherian_modules}
				Pruebe que sobre un anillo noetheriano $A$, un $A$-módulo es noetheriano syss es finitamente generado.
			\item\lookup
				Pruebe que sobre \emph{todo} anillo (noetheriano o no) existe un módulo noetheriano no nulo.
				\begin{hint}
					% Para un anillo $A$, sus $A$-submódulos son, por definición, los ideales.
					Note que, por ejemplo, un $A$-módulo sería noetheriano si fuese \strong{simple},
					i.e., si sus únicos submódulos fueran $0$ y $M$; así que puede tratar de buscar
					un $A$-módulo simple.
				\end{hint}
		\end{enumerate}
\end{enumerate}

\section{Las serpientes y sus amigos}
\begin{enumerate}[resume]
	\item\lookst
		\textbf{Lema de la serpiente:}
		Considere un diagrama conmutativo de $A$-módulos
		\begin{center}
			\begin{tikzcd}
				{}     & A \dar["\alpha"] \rar["\psi"] & B \dar["\beta"] \rar["\phi"] & C \dar["\gamma"] \rar & 0 \\
				0 \rar & A'             \rar["\psi'"'] & B'            \rar["\phi'"'] & C'
			\end{tikzcd}
		\end{center}
		donde ambas filas son exactas,
		pruebe que existe un homomorfismo de $A$-módulos $\omega \colon \ker\gamma \to \coker\alpha$ tal que se
		induce la siguiente sucesión exacta:
		\begin{center}
			\begin{tikzcd}[column sep=small]
				\ker\alpha \rar & \ker\beta \rar & \ker\gamma \rar["\omega"] & \coker\alpha \rar &
				\coker\beta \rar & \coker\gamma.
			\end{tikzcd}
		\end{center}

	\item \textbf{Lema de los cinco:}
		Dado un diagrama conmutativo de $A$-módulos
		\[\begin{tikzcd}
			M_1 \rar \dar["f_1"] & M_2 \rar \dar["f_2"] & M_3 \rar \dar["f_3"] & M_4 \rar \dar["f_4"] & M_5 \dar["f_5"] \\
			N_1 \rar & N_2 \rar & N_3 \rar & N_4 \rar & N_5
		\end{tikzcd}\]
		con filas exactas.
		Pruebe que si $f_1, f_2, f_4, f_5$ son isomorfismos, entonces $f_3$ también.
		
		\begin{hint}
			Emplee el lema de la serpiente.
		\end{hint}

	\item Sea
		\begin{equation}
			\begin{tikzcd}
				0 \rar & A \rar["f"] & B \rar["g"] & C \rar & 0
			\end{tikzcd}
			\tag{$\Sigma$}
			\label{module_exact_seq}
		\end{equation}
		una sucesión exacta de $R$-módulos.
		Pruebe que son equivalentes:
		\begin{enumerate}
			\item Existe $h \colon C \to B$ tal que $g\circ h = \Id_{C}$
				(esta $h$ es ocasionalmente descrita como una <<sección de $g$>>).
			\item Existe $j \colon B \to A$ tal que $j\circ f = \Id_{A}$
				(este $j$ es ocasionalmente descrito como una <<cosección>> o <<retracción de $f$>>).
			\item Existe un isomorfismo $\phi\colon A\oplus C \to B$, de modo que $g\circ\phi \colon
				A\times C \to C$ es la proyección y $\phi^{-1}\circ f \colon A \to A\times
				C$ es la inclusión. 
		\end{enumerate}
		En cuyo caso, se dice que \eqref{module_exact_seq} \strong{se escinde} o que es una \emph{sucesión
		escindida}.
		% \begin{hint}
		% 	Hay una demostración \textquote{sencilla} asumiendo el lema de los cinco.
		% \end{hint}
\end{enumerate}

% \section{QQQ}
% Sea $A$ un anillo, construiremos la álgebra de series formales $A[\![x]\!]$ como aquella en donde los elementos son series formales infinitas
% \[
% 	f := \sum_{n=0}^{\infty} a_n x^n, \qquad g := \sum_{n=0}^{\infty} b_n x^n
% \]
% con la suma y producto
% \[
% 	f+g := \sum_{n=0}^{\infty} (a_n+b_n)x^n, \qquad \sum_{n=0}^{\infty} \left( \sum_{j=0}^{n} a_jb_{n-j} \right)x^n.
% \]

% \begin{enumerate}
% 	\item\lookright
% 		(Examen de lucidez)
% 		Sea $A$ un anillo.
% 		\begin{enumerate}
% 			\item Pruebe que $A[\![x]\!]$ es un anillo.
% 			\item Pruebe que 
% 		\end{enumerate}
% \end{enumerate}

\appendix
\section{Ejercicios propuestos}
\begin{enumerate}
	\item Describa qué debe satisfacer un grupo abeliano para tener estructura natural%
		\footnote{Si bien el adjetivo <<natural>> es un tanto ambiguo en matemáticas, aquí tiene una connotación
		precisa. El lector puede probar que si un grupo abeliano $G$ admite estructura de $\Q$-módulo, esta es única.}
		de $\Q$-módulo.

	\item\lookup
		Dé un contraejemplo de un $A$-módulo finitamente generado $M$ que no sea noetheriano, i.e., que posea un
		submódulo que no es finitamente generado.
		(Note que, en virtud del ejercicio~\ref{ex:noetherian_modules}, el anillo debe no ser noetheriano.)

	\item Sea $A$ un dominio íntegro que contiene a un subcuerpo $k \subseteq A$ y tal que $A$ es un $k$-espacio
		vectorial de dimensión finita.
		Pruebe que $A$ es un cuerpo.

	\item Se dice que un $A$-módulo $M$ es \strong{descomponible} si posee dos submódulos propios $N_1, N_2$
		tales que $M \cong N_1 \oplus N_2$.
		Claramente todo módulo simple es indescomponible, pero el recíproco no es cierto.
		\begin{enumerate}
			\item Pruebe que un módulo no nulo $M$ es descomponible syss existe un endomorfismo no nulo
				$\varphi \colon M \to M$ tal que $\varphi^2 = \varphi$.
			\item\lookright (Examen de lucidez)
				¿Para exactamente qué enteros $n > 1$ se cumple que $\Z/n\Z$ es\break ($\Z$-)indescomponible?
				¿Para cuáles es simple?
		\end{enumerate}

	\item \textbf{Lema de los cuatro:}
		Considere el diagrama conmutativo de $A$-módulos
		\[\begin{tikzcd}
			M_1 \dar["f_1", two heads] \rar & M_2 \dar["f_2"] \rar & M_3 \dar["f_3"] \rar & M_4
			\dar["f_4", hook] \\
			N_1 \rar & N_2 \rar & N_3 \rar & N_4
		\end{tikzcd}\]
		con filas exactas.
		Supongamos que $f_1$ es sobreyectivo y $f_4$ es inyectivo.
		Se cumplen:
		\begin{enumerate}
			\item Si $f_2$ es    inyectivo, entonces $f_3$ también
			\item Si $f_3$ es sobreyectivo, entonces $f_2$ también
		\end{enumerate}
\end{enumerate}

\section{Comentarios adicionales}
Las técnicas que involucran construir diagramas con filas/columnas exactas y aplicar lemas similares al de la serpiente
se conocen como <<cazando diagramas>> (eng.\ \emph{diagram chasing}).
Hay varios libros dedicados a explotar esta técnica, un ejemplo es \cite{maclane:homology}.
Las primeras dos ediciones de \citeauthor{lang:algebra}~\cite{lang:algebra} incluían este famoso y desventurado ejercicio:
\begin{displayquote}
	Coja cualquier texto de álgebra homológica y pruebe todos los teoremas que contenga sin ver las demostraciones.
	% Take any book on homological algebra, and prove all the theorems without looking at the proofs given in that book.
\end{displayquote}
La traducción al ruso del libro fue incluyendo una serie de pies de página y anotaciones por el traductor.
En este punto añade <<Sugerimos saltarse éste ejercicio en una primera lectura.>>%
\footnote{Vid.\ \url{https://mathoverflow.net/a/10909}}

El álgebra homológica fue inventada y explorada principalmente por topológos algebristas al inicio del siglo \textsc{xx}
como Samuel Eilenberg, Norman Steenrod y Jean Leray.%
\footnote{Si quiere puede rastrear influencias más antiguas a David Hilbert y Henri Poincaré.}
En un comienzo pretendía tener aplicaciones específicas a la topología, pero probó ser de gran utilidad en el álgebra
también, principalmente en el estudio de módulos con la aparición de los funtores $\Ext$ y $\Tor$.
Esta revolución fue llevada a cabo por varios matemáticos de renombre, por nombrar algunos: Eduard \v Cech, Henri
Cartan, Nobuo Yoneda, Hyman Bass y Jean-Pierre Serre.

% Hoy en día, un buen conocimiento y dominio del tema se considera obligatorio para entender los últimos avances en
% álgebra conmutativa, inclusive enunciados en dónde la influencia no es para nada obvia ni directa.
% Un ejemplo es el teorema de que <<todo anillo regular es un DFU>>, donde <<regular>>.

\printbibliography

\end{document}
