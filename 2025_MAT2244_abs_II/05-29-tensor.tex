\documentclass[11pt, reqno]{amsart}

\usepackage[spanish]{babel}
\usepackage[LGR, T1]{fontenc}
\usepackage[utf8]{inputenc}

\input{../general.tex}
% \input{../graphics.tex}

\makeatletter
\def\emailaddrname{\textit{Correo electrónico}}
\def\subtitle#1{\gdef\@subtitle{#1}}
\def\@subtitle{}

% Metadata
\def\logo#1{\gdef\@logo{#1}}
\def\@logo{}
\def\institution#1{\gdef\@institution{#1}}
\def\@institution{}
\def\department#1{\gdef\@department{#1}}
\def\@department{}
\def\professor#1{\gdef\@professor{#1}}
\def\@professor{}
\def\course#1{\gdef\@course{#1}}
\def\@course{}
\def\coursecode#1{\gdef\@coursecode{#1}}
\def\@coursecode{}

\renewcommand{\maketitle}{
\begin{center}
	\small
	\renewcommand{\arraystretch}{1.2}
	\begin{tabular}{cp{.37\textwidth}p{0.44\textwidth}}
		% \hline
		\multirow{5}{*}{\includegraphics[height=2.0cm]{\@logo}}
	  & \multicolumn{2}{c}{ \makecell{{\bfseries \@institution} \\ \@department} } \\
	  % & \multicolumn{2}{c|}{{\bfseries\@institution} \\ \@department} \\
	  \cline{2-3}
	  & \textbf{Profesor:} \@professor & \textbf{Ayudante:} \authors \\
	  % \cline{2-3}
	  & \textbf{Curso:} \@course & \textbf{Sigla:} \@coursecode \\
	  % \cline{2-3}
	  & \multicolumn{2}{l}{ \textbf{Fecha:} \@date } \\
	  % \hline
	\end{tabular}
	\\[\baselineskip]
	% {}
	% \vspace{2\baselineskip}
	{\bfseries\Large\@title}
	\ifx\@subtitle\@empty\else
		\\[1ex]
		\large\mdseries\@subtitle
	\fi
\end{center}
}
\makeatother

\usepackage{multirow, makecell}

\usepackage[
	reversemp,
	letterpaper,
	% marginpar=2cm,
	% marginsep=1pt,
	margin=2.3cm
]{geometry}
\usepackage{fontawesome}
% \makeatletter
% \@reversemargintrue
% \makeatother

% Símbolos al margen, necesitan doble compilación
\newcommand{\hard}{\marginnote{\faFire}}
\newcommand{\hhard}{\marginnote{\faFire\faFire}}

% Dependencias para los teoremas
\usepackage{xifthen}
\def\@thmdep{}
\newcommand{\thmdep}[1]{
	\ifthenelse{\isempty{#1}}
	{\def\@thmdep{}}
	{\def\@thmdep{ (#1)}}
}
\newcommand{\thmstyle}{\color{thm}\sffamily\bfseries}

% ===== Estilos de Teoremas ==========
\newtheoremstyle{axiomstyle}
	{0.3cm}
	{0.3cm}
	{\normalfont}
	{0.5cm}
	{\bfseries\scshape}
	{:}
	{4pt}
	{\thmname{#1}\thmnote{ #3}\thmnumber{ (#2)}}
\newtheoremstyle{styleC}
	{0.5cm}
	{0.5cm}
	{\normalfont}
	{0.5cm}
	{\bfseries}
	{:}
	{4pt}
	{\thmname{#1\textrm{\@thmdep}}\thmnumber{ #2}\thmnote{ (#3)}}

% ====== Teoremas (sin borde) ===========
\theoremstyle{axiomstyle}
\newtheorem*{axiom}{Axioma}

% ====== Teoremas (sin borde) ==================
\theoremstyle{styleC}
\newtheorem{thm}{Teorema}[section]
\newtheorem{mydef}[thm]{Definición}
\newtheorem{prop}[thm]{Proposición}
\newtheorem{cor}[thm]{Corolario}
\newtheorem{lem}[thm]{Lema}
\newtheorem{con}[thm]{Conjetura}

\newtheorem*{prob}{Problema}
\newtheorem*{sol}{Solución}
\newtheorem*{obs}{Observación}
\newtheorem*{ex}{Ejemplo}

% \usepackage{tcolorbox}
% \newtcbox{bluebox}[1][]{enhanced jigsaw, 
%   sharp corners,
%   frame hidden,
%   nobeforeafter,
%   listing only,
%   #1} % comando para crear cajas de colores

\expandafter\let\expandafter\oldproof\csname\string\proof\endcsname
\let\oldendproof\endproof
\renewenvironment{proof}[1][\proofname]{%
  \oldproof[\scshape Demostración:]%
}{\oldendproof} % comando para redefinir la caja de la demostración
\newenvironment{hint}[1][\proofname]{%
  \oldproof[\scshape Pista:]%
}{\oldendproof} % comando para redefinir la caja de la demostración

% colores utilizados
\definecolor{numchap}{RGB}{249,133,29}
\definecolor{chap}{RGB}{6,129,204}
\definecolor{sec}{RGB}{204,0,0}
\definecolor{thm}{RGB}{106,176,240}
\definecolor{thmB}{RGB}{32,31,31}
\definecolor{part}{RGB}{212,66,66}

% ====== Diseño de los titulares ===============
\usepackage[explicit]{titlesec} % para personalizar el documento, la opción <<explicit>> hace que el texto de los titulares sea un objeto interactuable

\titleformat{\subsection}[runin]
	{\bfseries}
	{\textrm{\S}\thesubsection}
	{1ex}
	{#1.}

\setlist[enumerate,1]{label=\arabic*., ref=\arabic*} % Enumerate standards

\usepackage{tikz}
\usetikzlibrary{babel,cd}
% \DeclareMathOperator\Gr{Gr}

\title{Tensores}
\date{\DTMdate{2025-05-29}}

\author{José Cuevas Barrientos}
\email{josecuevasbtos@uc.cl}
\urladdr{https://josecuevas.xyz/teach/2025-1-ayud/}

\logo{../puc_negro.png}
\institution{Pontificia Universidad Católica de Chile}
\department{Facultad de Matemáticas}
\course{Álgebra abstracta II}
\coursecode{MAT2244}
\professor{Héctor Pastén Vásquez}

\begin{document}

\maketitle

\nocite{aluffi:algebra}
\nocite{atiyah:commutative}
% \nocite{matsumura:ring}
% \section{Números constructibles}

\section{Calculando tensores}
A la hora de <<calcular>> un tensor $M \otimes_A N$, el lector debe sólo tener en mente que un homomorfismo $M \otimes_A
N \to T$ es lo mismo que una aplicación $A$-bilineal $M \times N \to T$ (i.e., lineal en cada coordenada).

En un tensor $M \otimes_A N$, los elementos de la forma $m \otimes n$ (con $m \in M$, $n\in N$) se dicen \strong{tensores puros}.
No todos los elementos del producto tensorial son tensores puros, pero sí todo elemento es \emph{suma} de tensores puros.

\begin{enumerate}
	\item Pruebe que $(\Z/m\Z) \otimes_\Z (\Z/n\Z) = 0$ para $m, n$ coprimos, y que $(\Z/n\Z) \otimes_\Z \Q = 0$ en
		general.
		
		\begin{sol}
			Como todo módulo de tensores $M \otimes N$ está generado por los tensores puros, bastará probar
			que estos últimos son cero.

			Sean $[a] \in \Z/m\Z$ y $[b] \in \Z/n\Z$, entonces $a \otimes b = a(1 \otimes b) = ab(1 \otimes
			1)$, así que bastará probar que $1 \otimes 1 = 0$.
			Por la identidad de Bézout, $1 = um + vn$, de modo que
			\[
				1\otimes 1 = (um + vn)\otimes 1 = vn\otimes 1 = n(v\otimes 1) = v\otimes n = v\otimes 0
				= 0,
			\]
			donde empleamos la $\Z$-bilinealidad en los cálculos.

			Similarmente dados $[a] \in \Z/n\Z$ y $b \in \Q$, tenemos que
			\[
				a\otimes b = a\otimes\left( n \cdot \frac{b}{n} \right) = n(a\otimes b/n) = (na)\otimes b/n =
				0\otimes b/n = 0.
			\]
		\end{sol}

	\item Sea $M$ un $A$-módulo.
		\begin{enumerate}
			\item\lookright
				Pruebe que toda sucesión exacta 
				\begin{tikzcd}[cramped, sep=small]
					N_1 \rar["f"] & N_2 \rar["g"] & N_3 \rar & 0
				\end{tikzcd}
				da lugar a una sucesión exacta
				\begin{equation}
					\begin{tikzcd}
						M \otimes_A N_1 \rar & M \otimes_A N_2 \rar & M \otimes_A N_3 \rar & 0.
					\end{tikzcd}
					\label{cd:tensor_rext}
				\end{equation}

				\begin{sol}
					Para probar exactitud en $M\otimes_A N_3$ basta ver que $\Id_M\otimes g$ es
					sobreyectivo.
					Para ello, sea $\sum_{j=1}^{r} m_j\otimes c_j \in M\otimes_A N_3$, donde cada
					$m_j \in M$ y $c_j \in N_3$.
					Como $g$ es sobreyectivo, existen $b_j \in N_2$ tales que $g(b_j) = c_j$, de
					modo que
					\[
						(\Id_M\otimes g)\left( \sum_{j=1}^{r} m_j\otimes b_j \right) = \sum_{j=1}^{r}
						m_j\otimes g(b_j) = \sum_{j=1}^{r} m_j\otimes c_j.
					\]

					Por otro lado, una sucesión 
					\begin{tikzcd}[cramped, sep=small]
						T \rar["f"] & N \rar & Q \rar & 0
					\end{tikzcd}
					es exacta syss $Q \cong N/\Img f$ (por abuso de notación $Q = N/T$), así que
					para probar exactitud en $M\otimes N_2$ vamos a ver que el tensor respeta
					cocientes.
					Nótese que, en este contexto, $N \otimes_A M \to (N/T)\otimes_A M$ se anula en
					$T\otimes_A M$ (pues $t\otimes m \mapsto (t\mod T)\otimes m = 0$), por lo que,
					\[
						\varphi \colon \frac{N\otimes_A M}{T\otimes_A M} \longrightarrow
						(N/T)\otimes_A M, \qquad n\otimes m \longmapsto (n\mod T)\otimes m
					\]
					es un homomorfismo bien definido.

					Recíprocamente, la función
					\[
						\psi \colon (N/T)\otimes_A M \longrightarrow \frac{N\otimes_A
						M}{T\otimes_A M}, \qquad
						(n\mod T)\otimes m \longmapsto n\otimes m \pmod{T\otimes M}
					\]
					está bien definida ya que si $n' = n + t$ para algún $t \in T$, entonces
					$\psi(n'\otimes m) = n\otimes m + t\otimes m \equiv n\otimes m$.
					Es directo ver que $\varphi$ y $\psi$ son inversas una de la otra.
				\end{sol}

			\item Pruebe, sin embargo, que si $f \colon N_1 \to N_2$ es inyectiva, entonces $\Id_M\otimes_A
				f \colon M\otimes_A N_1 \to M\otimes_A N_2$ no lo es en general.

				\begin{sol}
					Una manera fácil de construir un contraejemplo sería un submódulo $N_1 \le N_2$
					tal que $N_1 \otimes M \ne 0 = N_2 \otimes M$.
					Un ejemplo de esto es $\Z \hookto \Q$ tensorizado con $M = \Z/2\Z$.

					Otro ejemplo es considerar el homomorfismo ${}\times 2 \colon \Z \to \Z$ que al
					tensorizar por $\Z/2\Z$ nos da el homomorfismo nulo.
				\end{sol}

			\item No obstante, pruebe que dados los $A$-módulos $M, N, P$ se cumple que
				\[
					(M \oplus N)\otimes_A P \cong (M\otimes_A P)\oplus(N\otimes_A P).
				\]
				En consecuencia, si $0 \to M \to M\oplus N \to N \to 0$ es una sucesión exacta escindida,
				entonces $0 \to M\otimes_A P \to (M\oplus N)\otimes_A P \to N\otimes_A P \to 0$ también
				es exacta escindida.

				\begin{sol}
					Empleando la propiedad universal del tensor, basta ver que darse una función
					bilineal desde $(M\oplus N)\times P$ es lo mismo que darse dos funciones
					bilineales desde $M\times P$ y $N\times P$.

					Dada $\varphi \colon (M\oplus N)\times P \to T$ bilineal, construimos
					\begin{align*}
						\psi_\varphi \colon M\times P &\longrightarrow T,
									      & (m, p) &\longmapsto \varphi((m,0),p), \\
						\theta_\varphi \colon N\times P &\longrightarrow T,
										& (n, p) &\longmapsto \varphi((0,n),p).
					\end{align*}
					Recíprocamente, dados $\psi, \theta$ bilineales, construimos
					\[
						\varphi \colon (M\oplus N)\times P \longrightarrow T, \qquad
						((m,n),p) \longmapsto \varphi(m,p) + \theta(n,p).
					\]
					Y queda al lector verificar que las construcciones son una la inversa de la
					otra.
				\end{sol}
		\end{enumerate}

	\item Sean $M, N$ un par de $A$-módulos.
		\begin{enumerate}
			\item (Examen de lucidez)
				Pruebe que si $N$ está generado por un solo elemento, entonces todos los elementos del
				tensor $M \otimes_A N$ son tensores puros.

				\begin{sol}
					Sea $n \in N$ que genera.
					Ahora los elementos de $M\otimes_A N$ son de la forma $\sum_{j=1}^{r} m_j\otimes
					n_j$ donde cada $m_j\in M, n_j\in N$; luego existen $a_j \in A$ tales que $n_j =
					a_j n$ y
					\[
						\sum_{j=1}^{r} m_j\otimes n_j = \sum_{j=1}^{r} a_j(m_j\otimes n)
						= \sum_{j=1}^{r} (a_jm_j\otimes n) = \left( \sum_{j=1}^{r} a_jm_j
						\right)\otimes n.
					\]
				\end{sol}

			\item\label{tensor_commutes_ideal_quot}
				Pruebe que, dado un ideal $\mathfrak{a} \nsle A$, se cumple que $M \otimes_A
				A/\mathfrak{a} \cong M/\mathfrak{a}M$. 

				\begin{sol}
					Aquí hay dos posibles soluciones.
					La primera más funtorial es emplear la sucesión exacta 
					\begin{tikzcd}[cramped, sep=small]
						0 \rar & \mathfrak{a} \rar & A \rar & A/\mathfrak{a} \rar & 0,
					\end{tikzcd}
					tensorizar con $M$ y demostrar que $\mathfrak{a}\otimes_A M = \mathfrak{a}M$.

					La segunda es probar el isomorfismo $M\otimes A/\mathfrak{a} \cong
					M/\mathfrak{a}M$ directamente empleando la propiedad universal del tensor, es
					decir, probando que dar una aplicación bilineal $\beta\colon M\times A/\mathfrak{a} \to N$
					es lo mismo que darse un homomorfismo $\varphi\colon M/\mathfrak{a}M \to N$.

					Dado $\beta$, basta definir $\varphi_\beta(m \mod{\mathfrak{a}M}) := \beta(m, 1)$;
					y dado $\varphi$, basta definir $\beta_\varphi(m, a \mod{\mathfrak{a}}) := \varphi(am)$.
					Ambos están bien definidos ya que si $a \in \mathfrak{a}$, entonces
					$\varphi_\beta(m + am') = \beta(m, 1) + \beta(m', a) = \beta(m, 1)$
					y $\beta_\varphi(m, a) = \varphi(0) = 0$.
				\end{sol}
		\end{enumerate}

	\item Sea $A$ un anillo y sea $B$ un \strong{$A$-álgebra} (i.e., un anillo con un homomorfismo $\varphi\colon A \to B$ de anillos fijo).
		\begin{enumerate}
			\item Pruebe que si $M$ es un $A$-módulo (resp.\ si $\psi \colon A \to C$ es un $A$-álgebra),
				entonces $M \otimes_A B$ (resp.\ $C\otimes_A B$) posee estructura de $B$-módulo (resp.\
				de $B$-álgebra).

				\begin{sol}
					En ambos casos, nos piden dar una definición de <<multiplicar por escalares en
					$B$>>.
					Para agilizar notación, escribamos $M_B := M \otimes_A B$.

					Sea $b' \in B$ fijo, para ver que $M_B \to M_B$ dado por $m\otimes b \mapsto
					m\otimes bb'$ está bien definido empleamos la propiedad universal del tensor
					para notar que la función
					\[
						(m, b) \longmapsto m \otimes bb'
					\]
					es $A$-bilineal.
					Con ello, será claro que $M_B$ admite un producto escalar, el cual es
					asociativo.
					Para ver distributividad basta notar que
					\begin{align*}
						(b' + b'')(m\otimes b) &= m\otimes (b(b' + b'')) = m\otimes (bb' + bb'') \\
								       &= m\otimes bb' + m\otimes bb'' = b'(m\otimes b) + b''(m\otimes b),
					\end{align*}
					y $b'(m\otimes b + m''\otimes b'') = b'(m\otimes b) + b'(m''\otimes b'')$ es
					trivial del que ${}\times b' \colon M_B \to M_B$ sea $A$-lineal.
				\end{sol}

			\item\lookright (Propiedad universal del producto tensorial de álgebras)
				Pruebe que si $D$ es un anillo con homomorfismos de anillos $f\colon B \to D$ y $g\colon C \to D$,
				tales que $f\circ\varphi = g\circ\psi \colon A \to D$, entonces existe un único
				homomorfismo de anillos $B\otimes_A C \to D$ tal que el siguiente diagrama conmuta:
				% https://q.uiver.app/#q=WzAsNSxbMCwyLCJBIl0sWzEsMiwiQiJdLFswLDEsIkMiXSxbMSwxLCJCXFxvdGltZXNfQSBDIl0sWzIsMCwiRCJdLFswLDEsIlxcdmFycGhpIiwyXSxbMCwyLCJcXHBzaSJdLFsyLDNdLFsxLDNdLFsxLDQsImYiLDIseyJjdXJ2ZSI6MX1dLFsyLDQsImciLDAseyJjdXJ2ZSI6LTF9XSxbMyw0LCJcXGV4aXN0cyEiLDEseyJzdHlsZSI6eyJib2R5Ijp7Im5hbWUiOiJkYXNoZWQifX19XV0=
				\[\begin{tikzcd}
					&& D \\
					C & {B\otimes_A C} \\
					A & B
					\arrow["g", bend left, from=2-1, to=1-3]
					\arrow[from=2-1, to=2-2]
					\arrow["{\exists!}"{description}, dashed, from=2-2, to=1-3]
					\arrow["\psi", from=3-1, to=2-1]
					\arrow["\varphi"', from=3-1, to=3-2]
					\arrow["f"', bend right, from=3-2, to=1-3]
					\arrow[from=3-2, to=2-2]
				\end{tikzcd}\]

			\item Pruebe que si $M$ es un $A$-módulo finitamente generado, entonces $M\otimes_A B$ es un
				$B$-módulo finitamente generado.

				\begin{sol}
					Como $M$ es finitamente generado, existe un epimorfismo de $A$-módulos
					$\varphi\colon A^r \to M$, de modo que formamos la sucesión exacta
					\[\begin{tikzcd}
						0 \rar & \ker\varphi \rar & A^r \rar["\varphi"] & M \rar & 0,
					\end{tikzcd}\]
					ahora tensorizamos por $B$ y concluimos que $A^r\otimes_A B \to M\otimes_A B$
					es, en particular, sobreyectiva.
					Finalmente, como el tensor respeta sumas directas, note que
					\[
						A^{\oplus r}\otimes_A B = (A\otimes_A B)^{\oplus r} = B^r.
					\]
				\end{sol}

			\item Calcule quién es la $\C$-álgebra $\C \otimes_\R \C$.

				\begin{sol}
					Aquí hay dos soluciones.
					Podemos emplear el ejercicio~\ref{tensor_commutes_ideal_quot} para calcular que,
					en $\mathsf{Alg}_\C$:
					\[
						\C \otimes_\R \C = \C\otimes_\R \frac{\R[x]}{(x^2 + 1)} =
						\frac{\C[x]}{(x^2 + 1)} = \frac{\C[z]}{(z+i)} \times \frac{\C[z]}{(z-i)}
						= \C\times\C,
					\]
					donde en el penúltimo paso empleamos el teorema chino del resto.

					Otra solución es que, para probar que $\C\otimes_\R\C = \C\times\C$, habría que
					encontrar un idempotente $e$ distinto del $1 = 1\otimes 1$ y el 0.
					Como $\C \cong \R\oplus i\R$ (en $\mathsf{Vect}_\R$), vemos que los elementos
					son de la forma
					\[
						e = 1\otimes z + i\otimes w \in \C\otimes_\R\C.
					\]
					Ahora, expandir la ecuación $e(e - 1) = 0$ nos da
					\[
						1\otimes( z^2 - z - w^2 ) + i\otimes( w(2z - 1) ) = 0,
					\]
					lo que nos da el sistema de ecuaciones
					\[
						\left\{
						\begin{aligned}
							z^2 - z - w^2 &= 0, \\
							w(2z - 1) &= 0.
						\end{aligned}
						\right.
					\]
					La solución $w = 0$ nos obliga $z = 0$ o $z = 1$ (que corresponde a $e = 0$ y $e
					= 1$ resp.); mientras que si $w \ne 0$, obtenemos $e = 1\otimes \frac{1}{2} + i\otimes
					\frac{i}{2}$.
				\end{sol}

			\item Sea $K/k$ una extensión finita separable.
				¿Qué condición debe satisfacer una extensión $L/k$ para que $L \otimes_k K \cong
				L^{[K:k]}$ como $L$-álgebras?

				\begin{sol}
					Como $K/k$ es finita y separable, el teorema del elemento primitivo dice que
					existe $\alpha \in K$ tal que $K = k(\alpha)$.
					Con $f(x) \in k[x]$ el polinomio minimal de $\alpha$, vemos que
					\[
						L\otimes_k K = L\otimes_k \frac{k[x]}{(f(x))} \cong \frac{L[x]}{(f(x))}.
					\]
					Así, para que $L\otimes_k K \cong L^{[K:k]}$ queremos que $f$ se factorice en
					factores lineales en $L[x]$, es decir, queremos que $L$ contenga al cuerpo de
					escisión de $f$.
				\end{sol}
		\end{enumerate}
\end{enumerate}

\appendix
\section{Ejercicios propuestos}
\begin{enumerate}
	\item Sea $G$ un grupo abeliano finitamente generado.
		En virtud del teorema de clasificación, $G \cong \Z^r \oplus F$, donde $F$ es un grupo abeliano finito;
		a tal $r$ le llamamos el \strong{rango} de $G$.
		\begin{enumerate}
			\item Pruebe que $G \otimes_\Z \Q \cong \Q^r$.
			\item Pruebe además que existen infinitos primos $p$ para los cuales $G \otimes_\Z (\Z/p\Z)
				\cong (\Z/p\Z)^r$.
		\end{enumerate}
\end{enumerate}

% \section{Comentarios adicionales}

\printbibliography

\end{document}
