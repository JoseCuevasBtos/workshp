\documentclass[11pt, reqno]{amsart}

\usepackage[spanish]{babel}
\usepackage[LGR, T1]{fontenc}
\usepackage[utf8]{inputenc}

../LaTeX/general.tex
% \input{../graphics.tex}

\makeatletter
\def\emailaddrname{\textit{Correo electrónico}}
\def\subtitle#1{\gdef\@subtitle{#1}}
\def\@subtitle{}

% Metadata
\def\logo#1{\gdef\@logo{#1}}
\def\@logo{}
\def\institution#1{\gdef\@institution{#1}}
\def\@institution{}
\def\department#1{\gdef\@department{#1}}
\def\@department{}
\def\professor#1{\gdef\@professor{#1}}
\def\@professor{}
\def\course#1{\gdef\@course{#1}}
\def\@course{}
\def\coursecode#1{\gdef\@coursecode{#1}}
\def\@coursecode{}

\renewcommand{\maketitle}{
\begin{center}
	\small
	\renewcommand{\arraystretch}{1.2}
	\begin{tabular}{cp{.37\textwidth}p{0.44\textwidth}}
		% \hline
		\multirow{5}{*}{\includegraphics[height=2.0cm]{\@logo}}
	  & \multicolumn{2}{c}{ \makecell{{\bfseries \@institution} \\ \@department} } \\
	  % & \multicolumn{2}{c|}{{\bfseries\@institution} \\ \@department} \\
	  \cline{2-3}
	  & \textbf{Profesor:} \@professor & \textbf{Ayudante:} \authors \\
	  % \cline{2-3}
	  & \textbf{Curso:} \@course & \textbf{Sigla:} \@coursecode \\
	  % \cline{2-3}
	  & \multicolumn{2}{l}{ \textbf{Fecha:} \@date } \\
	  % \hline
	\end{tabular}
	\\[\baselineskip]
	% {}
	% \vspace{2\baselineskip}
	{\bfseries\Large\@title}
	\ifx\@subtitle\@empty\else
		\\[1ex]
		\large\mdseries\@subtitle
	\fi
\end{center}
}
\makeatother

\usepackage{multirow, makecell}

\usepackage[
	reversemp,
	letterpaper,
	% marginpar=2cm,
	% marginsep=1pt,
	margin=2.3cm
]{geometry}
\usepackage{fontawesome}
% \makeatletter
% \@reversemargintrue
% \makeatother

% Símbolos al margen, necesitan doble compilación
\newcommand{\hard}{\marginnote{\faFire}}
\newcommand{\hhard}{\marginnote{\faFire\faFire}}

% Dependencias para los teoremas
\usepackage{xifthen}
\def\@thmdep{}
\newcommand{\thmdep}[1]{
	\ifthenelse{\isempty{#1}}
	{\def\@thmdep{}}
	{\def\@thmdep{ (#1)}}
}
\newcommand{\thmstyle}{\color{thm}\sffamily\bfseries}

% ===== Estilos de Teoremas ==========
\newtheoremstyle{axiomstyle}
	{0.3cm}
	{0.3cm}
	{\normalfont}
	{0.5cm}
	{\bfseries\scshape}
	{:}
	{4pt}
	{\thmname{#1}\thmnote{ #3}\thmnumber{ (#2)}}
\newtheoremstyle{styleC}
	{0.5cm}
	{0.5cm}
	{\normalfont}
	{0.5cm}
	{\bfseries}
	{:}
	{4pt}
	{\thmname{#1\textrm{\@thmdep}}\thmnumber{ #2}\thmnote{ (#3)}}

% ====== Teoremas (sin borde) ===========
\theoremstyle{axiomstyle}
\newtheorem*{axiom}{Axioma}

% ====== Teoremas (sin borde) ==================
\theoremstyle{styleC}
\newtheorem{thm}{Teorema}[section]
\newtheorem{mydef}[thm]{Definición}
\newtheorem{prop}[thm]{Proposición}
\newtheorem{cor}[thm]{Corolario}
\newtheorem{lem}[thm]{Lema}
\newtheorem{con}[thm]{Conjetura}

\newtheorem*{prob}{Problema}
\newtheorem*{sol}{Solución}
\newtheorem*{obs}{Observación}
\newtheorem*{ex}{Ejemplo}

% \usepackage{tcolorbox}
% \newtcbox{bluebox}[1][]{enhanced jigsaw, 
%   sharp corners,
%   frame hidden,
%   nobeforeafter,
%   listing only,
%   #1} % comando para crear cajas de colores

\expandafter\let\expandafter\oldproof\csname\string\proof\endcsname
\let\oldendproof\endproof
\renewenvironment{proof}[1][\proofname]{%
  \oldproof[\scshape Demostración:]%
}{\oldendproof} % comando para redefinir la caja de la demostración
\newenvironment{hint}[1][\proofname]{%
  \oldproof[\scshape Pista:]%
}{\oldendproof} % comando para redefinir la caja de la demostración

% colores utilizados
\definecolor{numchap}{RGB}{249,133,29}
\definecolor{chap}{RGB}{6,129,204}
\definecolor{sec}{RGB}{204,0,0}
\definecolor{thm}{RGB}{106,176,240}
\definecolor{thmB}{RGB}{32,31,31}
\definecolor{part}{RGB}{212,66,66}

% ====== Diseño de los titulares ===============
\usepackage[explicit]{titlesec} % para personalizar el documento, la opción <<explicit>> hace que el texto de los titulares sea un objeto interactuable

\titleformat{\subsection}[runin]
	{\bfseries}
	{\textrm{\S}\thesubsection}
	{1ex}
	{#1.}

\setlist[enumerate,1]{label=\arabic*., ref=\arabic*} % Enumerate standards

\usepackage{tikz}
\usetikzlibrary{babel,cd}
% \DeclareMathOperator\Gr{Gr}

\title{Módulos planos y otros}
\date{\DTMdate{2025-06-05}}

\author{José Cuevas Barrientos}
\email{josecuevasbtos@uc.cl}
\urladdr{https://josecuevas.xyz/teach/2025-1-ayud/}

\logo{../puc_negro.png}
\institution{Pontificia Universidad Católica de Chile}
\department{Facultad de Matemáticas}
\course{Álgebra abstracta II}
\coursecode{MAT2244}
\professor{Héctor Pastén Vásquez}

\begin{document}

\maketitle

% \nocite{aluffi:algebra}
\nocite{atiyah:commutative}
\nocite{jacobson:basic}
% \section{Números constructibles}

\section{Planitud}
Un $A$-módulo $M$ se dice \strong{plano} si para toda sucesión exacta $0 \to N_1 \to N_2 \to N_3
\to 0,$ se cumple que la sucesión
\begin{equation}
	0 \to N_1\otimes_A M \to N_2\otimes_A M \to N_3\otimes_A M \to 0
	\label{cd:flat_exact}
\end{equation}
es exacta.
\begin{enumerate}
	\item Algunos ejemplos de módulos planos:
		\begin{enumerate}
			\item (Examen de lucidez) Pruebe que todo $A$-módulo libre es plano.
				Concluya que sobre un cuerpo todo módulo es libre.
			\item Pruebe que si $M, N$ son un par de módulos planos, entonces $M\otimes_A N$ también
				es plano.
			\item Pruebe que $M \otimes_A A[x] \cong M[x]$ y concluya que $A[x]$ es un $A$-módulo plano.
			\item Pruebe que $M \otimes_A S^{-1}A \cong S^{-1}M$ y concluya que $A_{\mathfrak{p}}$ es un
				$A$-módulo plano para todo $\mathfrak{p} \in \Spec A$.
		\end{enumerate}

	\item Pruebe que para un $A$-módulo $M$ son equivalentes:
		\begin{enumerate}
			\item $M$ es plano.
			\item Para todo monomorfismo $\varphi \colon T \hookto N$, el homomorfismo $\varphi\otimes\Id_M
				\colon T\otimes M \to N\otimes M$ es un monomorfismo.
			\item Para todo monomorfismo $\varphi \colon T \hookto N$ con $N$ finitamente generado, el
				homomorfismo $\varphi\otimes\Id_M \colon T\otimes M \to N\otimes M$ es un monomorfismo.
			\item Para todo ideal $\mathfrak{a} \nsle A$ se cumple que el homomorfismo $\mathfrak{a}\otimes
				M \to \mathfrak{a}M$ dado por $a\otimes m \mapsto am$ es un isomorfismo.
		\end{enumerate}

	\item
		\begin{enumerate}
			\item Pruebe que si $B$ es un $A$-álgebra plana (i.e., $B$ es plano visto como
				$A$-módulo) y $N$ es un $B$-módulo plano, entonces $N$ es plano visto como
				$A$-módulo.
			\item\lookright
				Pruebe que si $A^n \cong A^m$ para algunos $n, m \in \N$, entonces $n = m$.

				\begin{hint}
					Trate de tensorizar para reducir a algún caso conocido.
				\end{hint}
			\item Más aún, pruebe que si hay un epimorfismo $\varphi \colon A^m \epicto A^n$, entonces $m
				\ge n$.
		\end{enumerate}

	\item Sea $A$ un anillo.
		Construiremos el \strong{grupo de Grothendieck} $\mathsf{K}^0(A)$ como el grupo abeliano libre con
		generadores $[M]$, donde $M$ recorre los $A$-módulos finitamente generados, bajo la relación de que si
		existe una sucesión exacta corta $0 \to M \to N \to T \to 0$, entonces
		\[
			[N] = [M] + [T].
		\]
		En particular, $[M\oplus N] = [M] + [N]$.
		\begin{enumerate}
			\item Pruebe que si $\varphi \colon A \to B$ es un homomorfismo de anillos, entonces
				\[
					\varphi^* \colon \mathsf{K}^0(A) \longrightarrow \mathsf{K}^0(B), \qquad
					[M] \longmapsto [M\otimes_A B]
				\]
				es un homomorfismo de grupos.
			\item Pruebe que si $A = k$ es un cuerpo, entonces $\mathsf{K}^0(k) = \Z$.
			\item\lookst
				Calcule $\mathsf{K}^0(A)$, cuando $A$ es un DIP (puede asumir $A = \Z$ si prefiere).

			\item Pruebe que si $A$ es noetheriano, entonces $\mathsf{K}^0(A)$ está generado por
				$[A/\mathfrak{p}]$, donde $\mathfrak{p}$ recorre los ideales primos.
		\end{enumerate}
\end{enumerate}

\appendix
\section{Ejercicios propuestos}
\begin{enumerate}
	\item Un $A$-módulo se dice \strong{proyectivo} si para todo homomorfismo $f\colon P \to M$ y todo epimorfismo
		$g \colon N \epicto M$, hay un homomorfismo $h \colon P \to N$ tal que $f = g\circ h$.
		En diagrama:
		% https://q.uiver.app/#q=WzAsMyxbMCwxLCJQIl0sWzEsMSwiTSJdLFsxLDAsIk4iXSxbMCwxLCJmIiwyXSxbMiwxLCJnIiwwLHsic3R5bGUiOnsiaGVhZCI6eyJuYW1lIjoiZXBpIn19fV0sWzAsMiwiXFxleGlzdHMgaCIsMCx7InN0eWxlIjp7ImJvZHkiOnsibmFtZSI6ImRhc2hlZCJ9fX1dXQ==
		\[\begin{tikzcd}
			& N \\
			P & M
			\arrow["g", two heads, from=1-2, to=2-2]
			\arrow["{\exists h}", dashed, from=2-1, to=1-2]
			\arrow["f"', from=2-1, to=2-2]
		\end{tikzcd}\]
		Pruebe que son equivalentes:
		\begin{enumerate}
			\item $P$ es proyectivo.
			\item Toda sucesión exacta 
				\begin{tikzcd}[cramped, sep=small]
					0 \rar & N \rar & M \rar["f"] & P \rar & 0
				\end{tikzcd}
				se escinde (recuerde que esto significa que existe $s \colon P \to M$ tal que $f\circ s
				= \Id_M$ y, \emph{a posteriori,} que $M \cong N\oplus P$).
		\end{enumerate}

	\item Pruebe que para todo $A$-módulo proyectivo $P$ (finitamente generado) existe un módulo libre $L$
		(finitamente generado) y un módulo $N$ tales que $L \cong N\oplus P$.

	\item Pruebe que todo módulo proyectivo es plano.
\end{enumerate}

% \section{Comentarios adicionales}

\printbibliography

\end{document}
