\documentclass[11pt, reqno]{amsart}

\usepackage[spanish]{babel}
\usepackage[LGR, T1]{fontenc}
\usepackage[utf8]{inputenc}

\input{../general.tex}
% \input{../graphics.tex}

\makeatletter
\def\emailaddrname{\textit{Correo electrónico}}
\def\subtitle#1{\gdef\@subtitle{#1}}
\def\@subtitle{}

% Metadata
\def\logo#1{\gdef\@logo{#1}}
\def\@logo{}
\def\institution#1{\gdef\@institution{#1}}
\def\@institution{}
\def\department#1{\gdef\@department{#1}}
\def\@department{}
\def\professor#1{\gdef\@professor{#1}}
\def\@professor{}
\def\course#1{\gdef\@course{#1}}
\def\@course{}
\def\coursecode#1{\gdef\@coursecode{#1}}
\def\@coursecode{}

\renewcommand{\maketitle}{
\begin{center}
	\small
	\renewcommand{\arraystretch}{1.2}
	\begin{tabular}{cp{.37\textwidth}p{0.44\textwidth}}
		% \hline
		\multirow{5}{*}{\includegraphics[height=2.0cm]{\@logo}}
	  & \multicolumn{2}{c}{ \makecell{{\bfseries \@institution} \\ \@department} } \\
	  % & \multicolumn{2}{c|}{{\bfseries\@institution} \\ \@department} \\
	  \cline{2-3}
	  & \textbf{Profesor:} \@professor & \textbf{Ayudante:} \authors \\
	  % \cline{2-3}
	  & \textbf{Curso:} \@course & \textbf{Sigla:} \@coursecode \\
	  % \cline{2-3}
	  & \multicolumn{2}{l}{ \textbf{Fecha:} \@date } \\
	  % \hline
	\end{tabular}
	\\[\baselineskip]
	% {}
	% \vspace{2\baselineskip}
	{\bfseries\Large\@title}
	\ifx\@subtitle\@empty\else
		\\[1ex]
		\large\mdseries\@subtitle
	\fi
\end{center}
}
\makeatother

\usepackage{multirow, makecell}

\usepackage[
	reversemp,
	letterpaper,
	% marginpar=2cm,
	% marginsep=1pt,
	margin=2.3cm
]{geometry}
\usepackage{fontawesome}
% \makeatletter
% \@reversemargintrue
% \makeatother

% Símbolos al margen, necesitan doble compilación
\newcommand{\hard}{\marginnote{\faFire}}
\newcommand{\hhard}{\marginnote{\faFire\faFire}}

% Dependencias para los teoremas
\usepackage{xifthen}
\def\@thmdep{}
\newcommand{\thmdep}[1]{
	\ifthenelse{\isempty{#1}}
	{\def\@thmdep{}}
	{\def\@thmdep{ (#1)}}
}
\newcommand{\thmstyle}{\color{thm}\sffamily\bfseries}

% ===== Estilos de Teoremas ==========
\newtheoremstyle{axiomstyle}
	{0.3cm}
	{0.3cm}
	{\normalfont}
	{0.5cm}
	{\bfseries\scshape}
	{:}
	{4pt}
	{\thmname{#1}\thmnote{ #3}\thmnumber{ (#2)}}
\newtheoremstyle{styleC}
	{0.5cm}
	{0.5cm}
	{\normalfont}
	{0.5cm}
	{\bfseries}
	{:}
	{4pt}
	{\thmname{#1\textrm{\@thmdep}}\thmnumber{ #2}\thmnote{ (#3)}}

% ====== Teoremas (sin borde) ===========
\theoremstyle{axiomstyle}
\newtheorem*{axiom}{Axioma}

% ====== Teoremas (sin borde) ==================
\theoremstyle{styleC}
\newtheorem{thm}{Teorema}[section]
\newtheorem{mydef}[thm]{Definición}
\newtheorem{prop}[thm]{Proposición}
\newtheorem{cor}[thm]{Corolario}
\newtheorem{lem}[thm]{Lema}
\newtheorem{con}[thm]{Conjetura}

\newtheorem*{prob}{Problema}
\newtheorem*{sol}{Solución}
\newtheorem*{obs}{Observación}
\newtheorem*{ex}{Ejemplo}

% \usepackage{tcolorbox}
% \newtcbox{bluebox}[1][]{enhanced jigsaw, 
%   sharp corners,
%   frame hidden,
%   nobeforeafter,
%   listing only,
%   #1} % comando para crear cajas de colores

\expandafter\let\expandafter\oldproof\csname\string\proof\endcsname
\let\oldendproof\endproof
\renewenvironment{proof}[1][\proofname]{%
  \oldproof[\scshape Demostración:]%
}{\oldendproof} % comando para redefinir la caja de la demostración
\newenvironment{hint}[1][\proofname]{%
  \oldproof[\scshape Pista:]%
}{\oldendproof} % comando para redefinir la caja de la demostración

% colores utilizados
\definecolor{numchap}{RGB}{249,133,29}
\definecolor{chap}{RGB}{6,129,204}
\definecolor{sec}{RGB}{204,0,0}
\definecolor{thm}{RGB}{106,176,240}
\definecolor{thmB}{RGB}{32,31,31}
\definecolor{part}{RGB}{212,66,66}

% ====== Diseño de los titulares ===============
\usepackage[explicit]{titlesec} % para personalizar el documento, la opción <<explicit>> hace que el texto de los titulares sea un objeto interactuable

\titleformat{\subsection}[runin]
	{\bfseries}
	{\textrm{\S}\thesubsection}
	{1ex}
	{#1.}

\setlist[enumerate,1]{label=\arabic*., ref=\arabic*} % Enumerate standards

\usepackage{tikz}
\usetikzlibrary{babel,cd}
% \DeclareMathOperator\Gr{Gr}

\title{Representaciones, álgebras y módulos}
\date{\DTMdate{2025-06-19}}

\author{José Cuevas Barrientos}
\email{josecuevasbtos@uc.cl}
\urladdr{https://josecuevas.xyz/teach/2025-1-ayud/}

\logo{../puc_negro.png}
\institution{Pontificia Universidad Católica de Chile}
\department{Facultad de Matemáticas}
\course{Álgebra abstracta II}
\coursecode{MAT2244}
\professor{Héctor Pastén Vásquez}

\begin{document}

\maketitle

\nocite{jacobson:basic}

\section{Reinterpretación como anillos}
A lo largo de esta sección, $G$ denotará un grupo finito posiblemente no conmutativo y $K$ denotará un cuerpo
(puede suponer $K = \C$ si prefiere).
Al hablar de anillos en esta ayudantía, los asumiremos \emph{unitarios} (con neutro multiplicativo), \emph{asociativos}
y posiblemente \emph{no conmutativos}.

\begin{enumerate}
	\item (Representaciones como módulos)
		\begin{enumerate}
			\item Denotaremos por $K[G]$ al grupo abeliano de sumas formales $\sum_{g\in G} a_g g$, donde cada $a_g \in K$.
				Pruebe que $K[G]$ es un anillo con el producto
				\[
					\left( \sum_{h\in G} a_h h \right)\left( \sum_{j\in G} b_j j \right)
					= \sum_{g\in G} \left( \sum_{hj=g} a_hb_j \right) g.
				\]

			\item Sea $\rho \colon G \to \GL_K(V)$ una representación, es decir, un homomorfismo de grupos, donde
				$V$ es un $K$-espacio vectorial; denotaremos $\rho_g := \rho(g) \in \Aut(V)$ para un $g \in G$.
				Pruebe que $V$ es naturalmente un $K[G]$-módulo con la operación escalar sobre $\vec v \in V$
				dada por
				\[
					\left( \sum_{g\in G} a_g g \right)\cdot\vec v := \sum_{g\in G} a_g \rho_g(\vec v).
				\]
				Recíprocamente, pruebe que todo $K[G]$-módulo $M$ da lugar a una única representación $G \to
				\GL_K(M)$.

				\lookup
				En lenguaje sofisticado, diríamos que la categoría de representaciones y la de $K[G]$-módulos son
				equivalentes.

				\begin{prob}
					Convénzase de que la noción de subrepresentación corresponde, mediante esta
					equivalencia, a la noción de $K[G]$-submódulo.
				\end{prob}
			\item Describa qué representación corresponde al $K[G]$-módulo libre $K[G]$.
				A ésta le llamaremos la \strong{representación regular} de $G$.
		\end{enumerate}

	\item Sea $\Lambda$ un anillo.
		\begin{enumerate}
			\item Pruebe que hay una biyección entre $\Lambda$-módulos izquierdos $M$ \strong{simples} (i.e.,
				que no poseen submódulos distintos del 0 y $M$) e ideales maximales izquierdos
				$\mathfrak{m} \nsl \Lambda$ dada por $\mathfrak{m} \mapsto \Lambda/\mathfrak{m}$.

			\item Un $\Lambda$-módulo izquierdo se dice \strong{semisimple} si es suma directa de simples.
				Pruebe que si $\Lambda$ es semisimple (como módulo izquierdo), entonces todo módulo
				(izquierdo) también lo es.

				\begin{hint}
					Recuerde que todo módulo es cociente de uno libre.
				\end{hint}

			\item Pruebe que $K[G]$ satisface la condición de las cadenas \emph{descendentes} y que, si
				$\car K = 0$, entonces semisimple izquierdo, mediante la siguiente forma del teorema de
				Maschke:

				\begin{thm}[Maschke]
					Toda representación \emph{irreducible} (i.e., que no es suma directa de
					subrepresentaciones) es simple.
				\end{thm}

			\item\lookright
				Concluya que toda representación de $G$ se escribe, de manera única, como suma directa de
				las subrepresentaciones irreducibles de la representación regular.
		\end{enumerate}

	\item Sea $\varphi \colon G \to H$ un homomorfismo de grupos.
		\begin{enumerate}
			\item Pruebe que induce un homomorfismo de $K$-álgebras $K[G] \to K[H]$.
				Mediante este, toda representación de $H$ se restringe a una representación de $G$.

			\item También mediante $K[G] \to K[H]$ note que toda representación de $G$ induce una
				representación de $H$ dada por asociarle al $K[G]$-módulo izquierdo $M$ el tensor $K[H]
				\otimes_{K[G]} M$.

			\item ¿Qué condición debe satisfacer $\varphi$ para que $K[H]$ sea un $K[G]$-módulo libre?
		\end{enumerate}

	\item En este ejercicio calcularemos todas las representaciones irreducibles de un grupo abeliano finito.
		\begin{enumerate}
			\item Pruebe que $K[G \times H] \cong K[G] \times K[H]$, y describa una asociación entre
				$K[G\times H]$-módulos (izquierdos) y pares de $K[G]$ y $K[H]$-módulos.

			\item Pruebe que $\Hom(C_n, \C^\times) \cong C_n$ (el isomorfismo \emph{no} es canónico, aunque
				pueda aparentarlo).

			\item Concluya que toda representación irreducible de un grupo abeliano finito tiene dimensión 1
				y clasifique de qué tipo es.
		\end{enumerate}
\end{enumerate}

\appendix
\section{Ejercicios propuestos}
\begin{enumerate}
	\item Coja el apunte de clases --o su libro favorito de representaciones--, vea la prueba de Maschke para $K =
		\C$ y argumente por qué sigue siendo válido cuando $\car K \nmid |G|$.
	\item\lookup
		Pruebe que si $\car K$ divide al orden de $G$, entonces el resultado anterior es falso.

		\begin{hint}
			Sea $p := \car K$.
			Por Cauchy, $G$ contiene un elemento de orden $p$, de modo que $K[C_p]$ es un subanillo de
			$K[G]$.
			Luego estudie el ideal de $K[C_p]$ generado por el elemento $\sum_{g \in C_p} g$.
		\end{hint}
\end{enumerate}

\section{Comentarios adicionales}
En general, el mundo de los anillos no conmutativos puede ser bastante salvaje.
Un ejemplo de los desastres que pueden ocurrir es que dos módulos libres de rangos distintos sean isomorfos (!).
Así, en definitiva, la <<clasificación>> de anillos conmutativos no suele tener buenos análogos no conmutativos, pero
las técnicas en la teoría de módulos sí, y esa es la razón del enfoque detrás de esta ayudantía.

El libro \cite{jacobson:basic} trata con cuidado a los anillos no conmutativos.
Una larga y exhaustiva referencia es \cite{rowen:graduate_noncomm}. 

\printbibliography

\end{document}
