\documentclass[10pt]{beamer}

\usepackage[spanish]{babel}
\usepackage[LGR, T1]{fontenc}
\usepackage[utf8]{inputenc}

../LaTeX/general.tex
\usepackage{tikz}
\usetikzlibrary{babel,cd}
% \DeclareMathOperator\Gr{Gr}

\newenvironment{sol}{\textbf{Solución:}\ }{}

\title{Preparación para el examen}
\subtitle{El último tango en París}
\date{\DTMdate{2025-06-26}}

\author{José Cuevas Barrientos}
% \email{josecuevasbtos@uc.cl}
% \urladdr{https://josecuevas.xyz/teach/2025-1-ayud/}

% \logo{../puc_negro.png}
% \institution{Pontificia Universidad Católica de Chile}
% \department{Facultad de Matemáticas}
% \course{Álgebra abstracta II}
% \coursecode{MAT2244}
% \professor{Héctor Pastén Vásquez}

\tikzset{
	every picture/.prefix style={
		execute at begin picture=\shorthandoff{"}
	}
}
\listfiles % https://tex.stackexchange.com/a/211948

\beamerdefaultoverlayspecification{<+->}

\begin{document}

\maketitle

\begin{frame}{Representaciones}
	\begin{block}{Problema}
		Sea $G$ un grupo finito y sean $\rho_1, \rho_2$ dos representaciones complejas con caracteres asociados
		$\chi_1, \chi_2$ resp.
		Recordando que el tensor de dos representaciones da otra representación, pruebe que el caracter de
		$\rho_1\otimes \rho_2$ es $\chi(g) = \chi_1(g) \chi_2(g)$.
	\end{block}
\end{frame}
\begin{frame}{Solución (producto de caracteres)}
	En efecto, sea $\rho_1 \colon G \acts \C^n =: V$ y $\rho_2 \colon G \acts \C^m =: W$, donde en ambos fijamos las
	bases $\vec e_i$ y $\vec e_j^\prime$ canónicas (con $1\le i\le n$ y $1\le j\le m$).
	Escribamos la matriz de $\rho_1(g)$ y $\rho_2(g)$ como $[a_{i,j}]_{i,j}^n$ y $[b_{u,v}]_{u,v}^m$ resp.

	\pause
	Entonces en $V^{\rho_1} \otimes W^{\rho_2}$ tomamos la base ordenada $(\vec e_i \otimes \vec e_j^\prime : 1\le i\le n, \;
	1\le j\le m)$, y en ella, la matriz de $\rho_1(g)\otimes \rho_2(g)$ es $[ a_{(i,u)}b_{(j,v)} ]_{(i,j),(u,v)}$,
	por lo que su traza es
	\pause
	\[
		\chi(g) = \sum_{i=1}^{n} \sum_{j=1}^{m} a_{i,i}b_{j,j}
		= \left( \sum_{i=1}^{n} a_{i, i} \right)\cdot \left( \sum_{j=1}^{m} b_{j, j} \right)
		= \chi_1(g) \chi_2(g).
	\]
\end{frame}

\begin{frame}{Representaciones}
	\begin{block}{Problema}
		Sea $S_n$ el grupo simétrico en $n$ letras.
		Defina en $\GL_n(\C)$ la \strong{representación por permutaciones} dada por $\rho_\sigma(\vec v) =
		(v_j)_{\sigma(j)}^n$ para una permutación $\sigma \in S_n$.
		\pause
		Pruebe que admite una subrepresentación de dimensión 1 (equivalentemente, vea que hay un vector no nulo
		$\vec v$ fijo por todo $S_n$), cuyo complemento ortogonal $\langle \vec v \rangle^\perp$, llamada
		\strong{representación estándar}, es irreducible.
	\end{block}
\end{frame}
\begin{frame}{Solución (representación estándar)}
	\small
	En efecto, es fácil notar que el vector $\vec v := (1, 1, \dots, 1) \in \C^n$ está fijo por todo $S_n$
	\pause
	(alternativamente, el lector podría haber expandido las condiciones para ver que todo vector fijo por $S_n$ debe
	ser un múltiplo escalar de éste).

	\pause
	Luego calculamos el producto interno del caracter permutación $\chi_{\rm perm}$, que deviese dar 2.
	Para ello, note que $\chi_{\rm perm}(\sigma)^2$ es la traza de la acción de permutación en $\C^n \otimes \C^n$,
	así que
	\pause
	\[
		(\chi_{\rm perm}, \chi_{\rm perm}) = \frac{1}{n!} \sum_{\sigma\in S_n} \sum_{a, b}^{n} \delta_{\sigma a,
		a} \delta_{\sigma b, b},
	\]
	\pause
	intercambiamos las sumatorias y notamos que $\delta_{\sigma a, a} \delta_{\sigma b, b} = 1$ syss $\sigma$
	estabiliza al par ordenado $(a, b)$.
	\pause
	Si $a \ne b$ habrán $(n-2)!$ de esas permutaciones y, sino habrán $(n-1)!$ de ellas, por lo que
	\pause
	\begin{align*}
		(\chi_{\rm perm}, \chi_{\rm perm})
		&= \frac{1}{n!} \left( \sum_{a\ne b}^{n} \Stab(a, b) + \sum_{c=1}^{n} \Stab(c) \right) \\
		&= \frac{1}{n!} \big( (n-2)!\cdot(n^2 - n) + (n-1)!\cdot n \big) = 2.
	\end{align*}
\end{frame}

% \begin{frame}{Representaciones} % caracteres de S_4 
% 	\begin{block}{Problema}
% 		Calcular la tabla de caracteres de $S_4$. 
% 	\end{block}
% 	\begin{sol}
% 		Hay algunos <<caracteres obvios>> que ya son irreducibles:
% 		\begin{itemize}
% 			\item El caracter unitario $\chi_0$.
% 			\item Cualquier otro caracter de dimensión 1, es decir, dado por un homomorfismo no neutro $S_4
% 				\to \C^\times$.
% 				Por ejemplo, el homomorfismo signo $\sign \colon S_4 \to \{ \pm 1 \} \subseteq
% 				\C^\times$ cuyo caracter denotaremos $\varepsilon := \chi_{\sign}$.
% 			\item El caracter estándar $\chi_{\rm st}$ por el ejercicio anterior.
% 			\item El producto de un caracter por otro de dimensión 1, en particular $\varepsilon\cdot
% 				\chi_{\rm st}$.
% 		\end{itemize}
% 	\end{sol}
% \end{frame}
% \begin{frame}{Solución (caracteres de $S_4$)}
% 	\small
% 	Esto nos da la siguiente tabla:
% 	\[
% 		\begin{array}{c|*{5}{r}}
% 			{} & 1 & (12) & (12)(34) & (123) & (1234) \\
% 			\hline
% 			\chi_0 & 1 & 1 & 1 & 1 & 1 \\
% 			\varepsilon & 1 & -1 & 1 & 1 & -1 \\
% 			\chi_{\rm st} & 3 & 1 & -1 & 0 & -1 \\
% 			\varepsilon\chi_{\rm st} & 3 & -1 & -1 & 0 & 1 \\
% 			? & 
% 		\end{array}
% 	\]
% 	\pause
% 	Como hay cinco clases de conjugación, hay cinco caracteres irreducibles (conteo de caracteres), así que solo
% 	falta uno que denotaremos $\theta$.
% 	\pause
% 	Así mismo, podemos deducir su dimensión por el cor.~5.16
% 	\[
% 		4! = 24 = \chi_0(1)^2 + \varepsilon(1)^2 + \chi_{\rm st}(1)^2 + (\varepsilon\cdot\chi_{\rm st})(1)^2 +
% 		\theta(1)^2 = 2 + 18 + \theta(1)^2,
% 	\]
% 	lo que nos da $\theta(1) = 2$.
% 	\pause
% 	Empleando Ortogonalidad II, calculamos que
% 	\[
% 		0 = 1\cdot 1 + 1\cdot(-1) + 3\cdot 1 + 3\cdot(-1) + 2 \theta(12) = 0 \iff \theta(12) = 0.
% 	\]
% 	\pause
% 	Con esto rellenamos la tabla:
% 	\[
% 		\begin{array}{c|*{5}{r}}
% 			\theta & 1 & 0 & 2 & -1 & 0
% 		\end{array}
% 	\]
% \end{frame}

\begin{frame}{Representaciones} % caracteres de A_4 
	\begin{block}{Problema}
		Calcular la tabla de caracteres de $A_4$. 
	\end{block}
	\begin{sol}
		Como hay homomorfismo de grupos $A_4 \hookto S_4$, toda representación de $S_4$ se restringe a una de
		$A_4$.

		\pause
		En particular, el caracter $\psi$ de la restricción de la representación estándar es irreducible porque:
		\pause
		a) se verifica que $(\psi, \psi) = 1$ (latero, pero funciona);
		\pause
		b) si $\psi$ fuera reducible, habría otro subespacio fijo de dimensión 1, pero la acción por permutación
		$A_4 \acts \{ 1, 2, 3, 4 \}$ es transitiva.
	\end{sol}
\end{frame}
\begin{frame}{Solución (caracteres de $A_4$)}
	\small
	\begin{enumerate}[(i), wide]
		\item \textbf{Clases de conjugación:}
			Primero calculamos las clases de $A_4$ que son las siguientes:
			\begin{gather*}
				\{ 1 \}, \qquad
				\{ x := (12)(34), (13)(24), (14)(23) \}, \\
				\{ y := (123), (134), (142), (243) \}, \quad
				\{ y^2= (132), (143), (124), (234) \}
			\end{gather*}
			Así deducimos que hay cuatro representaciones irreducibles.
		\item Podemos deducir la dimensión de las representaciones restantes por cor.~5.16
			\[
				|A_4| = 12 = \chi_0(1)^2 + \psi(1)^2 + \chi_1(1)^1 + \chi_2(1)^2 = 1 + 9 + \chi_1(1)^1 + \chi_2(1)^2,
			\]
			lo que nos da que hay dos representaciones de dimensión 1.
		\item Como $x^2 = 1$, entonces $\chi_j(x) = \pm 1$.
			\pause
			Pero $(12)(34)\cdot (13)(24) = (14)(23)$, así que $\chi_j(x) = 1$.
			\pause
			Como $y^3 = 1$, entonces $\chi_j(y) = \omega^?$, donde $\omega = \zeta_3$ es la raíz cúbica
			primitiva de la unidad; como $\chi_j(y) \ne 1$, se completa la tabla:
			\pause
			\[
				\begin{array}{c|*{4}{r}}
					{} & 1 & x & y & y^2 \\
					\hline
					\chi_0 & 1 & 1 & 1 & 1 \\
					\chi_1 & 1 & 1 & \omega & \omega^2 \\
					\chi_2 & 1 & 1 & \omega^2 & \omega \\
					\psi & 3 & -1 & 0 & 0
				\end{array}
			\]
	\end{enumerate}
\end{frame}

\begin{frame}{Teoría de Galois}
	\begin{block}{Problema}
		Determine cuál de los siguientes es el grupo de Galois del cuerpo de escisión (sobre $\Q$) del polinomio
		$x^5 - x + 1$ (que puede asumir irreducible):
		\begin{enumerate}[(a)]
			\item $C_5$.
			\item $D_5$.
			\item $S_5$.
		\end{enumerate}
	\end{block}
	\pause
	Como pista, emplee el siguiente resultado:
	\begin{block}{Teorema}
		El discriminante de un polinomio irreducible $f$ de grado $n$ es un cuadrado syss
		$\Gal(\operatorname{Split}_\Q(f)/\Q) \subseteq A_n$.
	\end{block}
\end{frame}
\begin{frame}{Solución 1 (calcular Galois)}
	Empleamos el resultado y calculamos el discriminante, es decir, el siguiente determinante:
	\pause
	\[
		\footnotesize
		\begin{vmatrix}
			1 & 0 & 0 & 0 & 0 & 5 & 0 & 0 & 0 & 0 \\
			0 & 1 & 0 & 0 & 0 & 0 & 5 & 0 & 0 & 0 \\
			0 & 0 & 1 & 0 & 0 & 0 & 0 & 5 & 0 & 0 \\
			0 & 0 & 0 & 1 & 0 & 0 & 0 & 0 & 5 & 0 \\
			-1& 0 & 0 & 0 & 1 &-1 & 0 & 0 & 0 & 5 \\
			1 &-1 & 0 & 0 & 0 & 0 &-1 & 0 & 0 & 0 \\
			0 & 1 &-1 & 0 & 0 & 0 & 0 &-1 & 0 & 0 \\
			0 & 0 & 1 &-1 & 0 & 0 & 0 & 0 &-1 & 0 \\
			0 & 0 & 0 & 1 &-1 & 0 & 0 & 0 & 0 &-1 \\
			0 & 0 & 0 & 0 & 1 & 0 & 0 & 0 & 0 & 0
		\end{vmatrix} 
	\]
	\pause
	\textcolor{gray}{(Horror, lo sé.)}

	Lo cuál da 2869 y vemos que $50^2 = 2500$, por lo que es fácil comprobar que no es un cuadrado perfecto
	(a ensayo y error, vea que $53^2 = 2809$ y $54^2 = 2916$).
	\pause
	Finalmente, note que $D_5 \le S_5$ mediante $r \mapsto (12345) \in A_5$ y $s \mapsto (25)(34) \in A_5$, así que
	$D_5 \le A_5$.
	Por lo que, el grupo de Galois es el \textbf{simétrico $S_5$}.
\end{frame}

\begin{frame}{Solución 2 (calcular Galois)}
	Calculamos las raíces reales notando que la derivada es $5x^4 - 1$ la cual tiene raíces $\pm\sqrt[5]{1/5}$.
	\pause
	Así, hay dos cambios de signo, por lo que hay tres raíces reales y \textbf{dos complejas}.
	\pause
	Finalmente, el grupo de Galois $G$ contiene un 5-ciclo por el teorema de Cauchy, y contiene a una trasposición
	dada por la conjugación compleja, así que (como 5 es \textbf{primo}), tiene que darse que $G \cong S_5$.
\end{frame}

% \begin{frame}{Teoría de Galois}
% 	\begin{block}{Problema}
% 		Determine cuál de los siguientes es el grupo de Galois del cuerpo de escisión (sobre $\Q$) del polinomio
% 		$x^3 - 3x + 1$ (que puede asumir irreducible):
% 		\begin{enumerate}[(a)]
% 			\item $C_3$.
% 			\item $D_3$.
% 			\item $S_3$.
% 		\end{enumerate}
% 	\end{block}
% 	\pause
% 	\begin{sol}
		
% 	\end{sol}
% \end{frame}

\begin{frame}{Teoría de Galois}
	\small
	\begin{block}{Problema}
		Sea $L/K$ una extensión de Galois con $\Gal(L/K) \cong C_2 \times C_{12}$.
		¿Cuántas extensiones intermedias $K \subset F \subset L$ tiene tales que $[F:K] = 4$?
		¿Cuántas de estas son de Galois?
	\end{block}
	\pause
	\begin{sol}
		Por conexión de Galois, contar extensiones intermedias de grado 4 es contar subgrupos $H \le C_2 \times
		C_{12} =: G$ tales que $[G : H] = 4$.

		\pause
		\warn
		En un producto, un subgrupo \textbf{no} es un producto de subgrupos (e.g., el generado por $(1,1)$ en
		$C_2 \times C_{12}$).
		\pause
		No obstante, sea $H$ como antes, vemos que $|H| = 24/4 = 6$, por lo que $H$ contiene un elemento de
		orden 3 (teorema de Cauchy).
		\pause
		Por teorema chino del resto $C_{12} \cong C_4\times C_3$, con lo que es fácil verificar que los únicos
		elementos de $C_2 \times C_4 \times C_3$ de orden 3 son $(0, 0, \pm 1)$;
		así que $H \ge 0 \times 0 \times C_3$.
		Por correspondencia, podemos bajar a contar $H' \le C_2 \times C_4$ (mediante la proyección) de índice
		4 o, equivalentemente, de orden 2.

		\pause
		Como $H'$ tiene orden 2, solamente está generado por un elemento de orden 2.
		Hay tres de ellos: $(1, 0)$, $(0, 2)$ y $(1, 2)$.

		\pause
		Finalmente, todas ellas son extensiones de Galois puesto que todo subgrupo de un grupo abeliano es
		normal.
	\end{sol}
\end{frame}

\begin{frame}{Álgebra conmutativa}
	\begin{block}{Problema}
		Diremos que un anillo conmutativo $A$ es \strong{absolutamente plano} si todo $A$-módulo es plano.
		\begin{enumerate}
			\item Pruebe que todo cuerpo es absolutamente plano.
			\item ¿Será que todo anillo absolutamente plano es un cuerpo?
				\pause
				\textcolor{gray}{\textsc{Pista:} considere el caso de anillos \emph{booleanos} (i.e., donde $x^2 = x$
				para todo $x \in A$).}
			\item ¿Será que todo anillo \emph{local} absolutamente plano es un cuerpo?
		\end{enumerate}
	\end{block}
\end{frame}
\begin{frame}{Solución (anillos absolutamente planos)}
	\begin{enumerate}[wide]
		\item En efecto, sobre un cuerpo todo módulo (= espacio vectorial) es \textbf{libre} (i.e., posee base),
			y sabemos que todo módulo libre es plano.
		\item No. En particular, podemos tomar un producto de cuerpos.
			\pause
			Por ejemplo, podemos considerar el anillo booleano $\prod_{s\in S} \Fp[2]$ (donde $S$ es un conjunto
			cualquiera).

			\pause
			Para ver que cualquier anillo booleano $A$ es \emph{absolutamente plano}, note que un $A$-módulo
			$M$ es plano syss cada $M_{\mathfrak{p}}$ es plano sobre $A_{\mathfrak{p}}$, donde
			$\mathfrak{p}$ recorre los ideales primos (ver ayudantía).

			\pause
			Ahora bien, $A_{\mathfrak{p}}$ también es booleano y es local.
			Por la interrogación, un anillo local solo tiene por idempotentes al 0 y al 1, por lo que,
			$A_{\mathfrak{p}} = \{ 0, 1 \} = \Fp[2]$ es un cuerpo.
			Así, $M_{\mathfrak{p}}$ debe ser plano y $M$ también.
	\end{enumerate}
\end{frame}
\begin{frame}[fragile]{Solución (anillos absolutamente planos)}
	3. Sea $x \in A$ arbitrario.
	El $A$-módulo $A/(x)$ es plano y, por tanto, tenemos el siguiente diagrama conmutativo:
	\[\begin{tikzcd}[row sep=large]
		(x) \otimes_A A \rar["{1\otimes \pi}", two heads] \dar[hook] & (x) \otimes_A A/(x) = (x)/(x^2)
		\dar["\alpha", hook] \\
		A \rar[two heads] & A/(x)
	\end{tikzcd}\]
	\pause
	Como la composición es cero y $\alpha$ es inyectivo, por planitud, $1\otimes\pi = 0$, por lo que $(x) = (x^2)$.

	\pause
	Así, $x = ax^2$ y $e := ax$ es idempotente, pues $e^2 = a\cdot ax^2 = ax = e$.
	\pause
	Pero un anillo local solo tiene por idempotentes al 0 y al 1, por lo que $x$ es nulo o una unidad.
\end{frame}

\printbibliography

\end{document}
