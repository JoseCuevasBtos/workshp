\documentclass[11pt, reqno]{amsart}

\usepackage[spanish]{babel}
\usepackage[LGR, T1]{fontenc}
\usepackage[utf8]{inputenc}

\input{../general.tex}
% \input{../graphics.tex}

\makeatletter
\def\emailaddrname{\textit{Correo electrónico}}
\def\subtitle#1{\gdef\@subtitle{#1}}
\def\@subtitle{}

% Metadata
\def\logo#1{\gdef\@logo{#1}}
\def\@logo{}
\def\institution#1{\gdef\@institution{#1}}
\def\@institution{}
\def\department#1{\gdef\@department{#1}}
\def\@department{}
\def\professor#1{\gdef\@professor{#1}}
\def\@professor{}
\def\course#1{\gdef\@course{#1}}
\def\@course{}
\def\coursecode#1{\gdef\@coursecode{#1}}
\def\@coursecode{}

\renewcommand{\maketitle}{
\begin{center}
	\small
	\renewcommand{\arraystretch}{1.2}
	\begin{tabular}{cp{.37\textwidth}p{0.44\textwidth}}
		% \hline
		\multirow{5}{*}{\includegraphics[height=2.0cm]{\@logo}}
	  & \multicolumn{2}{c}{ \makecell{{\bfseries \@institution} \\ \@department} } \\
	  % & \multicolumn{2}{c|}{{\bfseries\@institution} \\ \@department} \\
	  \cline{2-3}
	  & \textbf{Profesor:} \@professor & \textbf{Ayudante:} \authors \\
	  % \cline{2-3}
	  & \textbf{Curso:} \@course & \textbf{Sigla:} \@coursecode \\
	  % \cline{2-3}
	  & \multicolumn{2}{l}{ \textbf{Fecha:} \@date } \\
	  % \hline
	\end{tabular}
	\\[\baselineskip]
	% {}
	% \vspace{2\baselineskip}
	{\bfseries\Large\@title}
	\ifx\@subtitle\@empty\else
		\\[1ex]
		\large\mdseries\@subtitle
	\fi
\end{center}
}
\makeatother

\usepackage{multirow, makecell}

\usepackage[
	reversemp,
	letterpaper,
	% marginpar=2cm,
	% marginsep=1pt,
	margin=2.3cm
]{geometry}
\usepackage{fontawesome}
% \makeatletter
% \@reversemargintrue
% \makeatother

% Símbolos al margen, necesitan doble compilación
\newcommand{\hard}{\marginnote{\faFire}}
\newcommand{\hhard}{\marginnote{\faFire\faFire}}

% Dependencias para los teoremas
\usepackage{xifthen}
\def\@thmdep{}
\newcommand{\thmdep}[1]{
	\ifthenelse{\isempty{#1}}
	{\def\@thmdep{}}
	{\def\@thmdep{ (#1)}}
}
\newcommand{\thmstyle}{\color{thm}\sffamily\bfseries}

% ===== Estilos de Teoremas ==========
\newtheoremstyle{axiomstyle}
	{0.3cm}
	{0.3cm}
	{\normalfont}
	{0.5cm}
	{\bfseries\scshape}
	{:}
	{4pt}
	{\thmname{#1}\thmnote{ #3}\thmnumber{ (#2)}}
\newtheoremstyle{styleC}
	{0.5cm}
	{0.5cm}
	{\normalfont}
	{0.5cm}
	{\bfseries}
	{:}
	{4pt}
	{\thmname{#1\textrm{\@thmdep}}\thmnumber{ #2}\thmnote{ (#3)}}

% ====== Teoremas (sin borde) ===========
\theoremstyle{axiomstyle}
\newtheorem*{axiom}{Axioma}

% ====== Teoremas (sin borde) ==================
\theoremstyle{styleC}
\newtheorem{thm}{Teorema}[section]
\newtheorem{mydef}[thm]{Definición}
\newtheorem{prop}[thm]{Proposición}
\newtheorem{cor}[thm]{Corolario}
\newtheorem{lem}[thm]{Lema}
\newtheorem{con}[thm]{Conjetura}

\newtheorem*{prob}{Problema}
\newtheorem*{sol}{Solución}
\newtheorem*{obs}{Observación}
\newtheorem*{ex}{Ejemplo}

% \usepackage{tcolorbox}
% \newtcbox{bluebox}[1][]{enhanced jigsaw, 
%   sharp corners,
%   frame hidden,
%   nobeforeafter,
%   listing only,
%   #1} % comando para crear cajas de colores

\expandafter\let\expandafter\oldproof\csname\string\proof\endcsname
\let\oldendproof\endproof
\renewenvironment{proof}[1][\proofname]{%
  \oldproof[\scshape Demostración:]%
}{\oldendproof} % comando para redefinir la caja de la demostración
\newenvironment{hint}[1][\proofname]{%
  \oldproof[\scshape Pista:]%
}{\oldendproof} % comando para redefinir la caja de la demostración

% colores utilizados
\definecolor{numchap}{RGB}{249,133,29}
\definecolor{chap}{RGB}{6,129,204}
\definecolor{sec}{RGB}{204,0,0}
\definecolor{thm}{RGB}{106,176,240}
\definecolor{thmB}{RGB}{32,31,31}
\definecolor{part}{RGB}{212,66,66}

% ====== Diseño de los titulares ===============
\usepackage[explicit]{titlesec} % para personalizar el documento, la opción <<explicit>> hace que el texto de los titulares sea un objeto interactuable

\titleformat{\subsection}[runin]
	{\bfseries}
	{\textrm{\S}\thesubsection}
	{1ex}
	{#1.}

\setlist[enumerate,1]{label=\arabic*., ref=\arabic*} % Enumerate standards


\title{Aritmética modular y DFUs}
% \subtitle{Conociendo a Catalan, Fermat y Mordell}
\date{\DTMdate{2025-08-11}}

\author[José Cuevas]{José Cuevas Barrientos}
\email{josecuevasbtos@uc.cl}
% \urladdr{https://josecuevas.xyz/teach/2025-1-ayud/}

\logo{../puc_negro.png}
\institution{Pontificia Universidad Católica de Chile}
\department{Facultad de Matemáticas}
\course{Teoría de Números}
\coursecode{MAT2814}
\professor{Ricardo Menares}

\begin{document}

\maketitle

\section*{Introducción}
A lo largo de las ayudantías incluiré ciertos símbolos:
los problemas/comentarios \textbf{recomendados} e importantes tendrán ojos interesados {\righteyes}, 
los problemas difíciles tendrán ojos asustados {\straighteyes}, y
los problemas/comentarios que son opcionales u omitibles tendrán ojos hastiados {\upeyes}.

\section{Congruencias}
\begin{enumerate}
	\item\lookright
		Pruebe que la ecuación diofántica $ax + by = c$ tiene solución syss $\mcd(a, b) \mid c$.
		Más aún, pruebe que si $(x_0, y_0) \in \Z^2$ son una solución, entonces todas las soluciones vienen parametrizadas por el
		conjunto
		\[
			\left\{ ( x_0 + bt, y_0 - at ) : t \in \Z \right\}.
		\] % \cite{hua:number}

	\item Demuestre que la ecuación diofantina $x^2 + y^2 = 4z + 3$ no tiene soluciones enteras.

	\item (Gersónides)
		Las únicas potencias consecutivas de $2$ y $3$ son $1, 2, 3, 4, 8$ y $9$.

	\item Demuestre que las únicas soluciones (en $\Z$) a la ecuación diofántica $y^2 + y = x^3$ son
		\[
			(x, y) \in \{ (0, 0), \quad (0, -1) \}.
		\] % https://kconrad.math.uconn.edu/blurbs/ringtheory/ufdapp.pdf

	% \item Considere la sucesión
	% 	\[
	% 		q_n := \underbrace{33\dots 33}_{n\text{ veces}} 1.
	% 	\]
	% 	Demuestre que contiene infinitos números compuestos.

	% 	Lo divertido de la sucesión es que $q_1, q_2, \dots, q_7$ son todos primos y $q_8 = 17 \cdot 19607843$ es el primer número compuesto en ella.

	% 	% \begin{proof}
	% 	% 	En primer lugar escribamos $q_n$ de manera cerrada:
	% 	% 	$$ q_n = 30(1 + \cdots + 10^{n-1}) + 1 = 30 \cdot \frac{10^n - 1}{10 - 1} + 1 = \frac{10}{3}(10^n - 1) + 1. $$
	% 	% 	Ahora sabemos que $17 \mid q_8$ (?!), por lo que podemos mirar la sucesión módulo 17.
	% 	% 	Por el pequeño teorema de Fermat $10^{16} \equiv 1 \pmod{17}$, por lo que $10^{8 + m16} \equiv 10^8 \pmod{17}$
	% 	% 	y, por tanto, $q_{8 + m16} \equiv q_8 \equiv 0 \pmod{17}$.
	% 	% \end{proof}

	% \item\lookright (\textquote{Elevando el exponente})
	% 	Demuestre el siguiente clásico truco de olimpiadas:
	% 	Dado un primo $p > 2$, un par de enteros $x, y \in \Z$ tales que $x \equiv y \not\equiv 0 \pmod p$ y un entero $n \ge 1$,
	% 	entonces
	% 	\[
	% 		\nu_p(x^n - y^n) = \nu_p(x - y) + \nu_p(n),
	% 	\]

	% \item Recuerde que un entero $g \in \Z$ se dice una \strong{raíz primitiva módulo $n$} si $(\Z/n\Z)^\times$ está generado (como grupo)
	% 	por $g \mod n$.
	% 	\begin{enumerate}
	% 		\item Pruebe que para un primo $p$, si $g$ es una raíz primitiva módulo $p$, entonces $g$ ó $g + p$ es una raíz
	% 			primitiva módulo $p^2$.

	% 		\item Pruebe que si $g$ es una raíz primitiva módulo $p^2$ para un primo $p > 2$, entonces $g$ es raíz primitiva
	% 			módulo $p^n$ para todo $n \ge 1$.
	% 	\end{enumerate}

	% \item
	% 	(Euclides-Euler) Demuestre que un número par es perfecto syss es de la forma $2^{n-1}(2^n - 1)$,
	% 	donde $p := 2^n - 1$ es un número primo.
	% 	(Los primos de la forma $2^n - 1$ se dicen \emph{primos de Mersenne}.)

	% 	% \begin{hint}
	% 	% 	Para <<$\implies$>> demuestre que si $m$ es impar y $\sigma(m) = 2^{n+1}a$ con $a$ impar,
	% 	% 	entonces $m = (2^{n+1} - 1)a$
	% 	% \end{hint}

	% \item (Euler) Demuestre que un número impar perfecto es de la forma $p^r m^2$ donde $p \nmid m$ y $p \equiv r \equiv 1 \pmod 4$.
\end{enumerate}

\section{El teorema fundamental de la aritmética}
El adjetivo \emph{fundamental} no es accidental, son varias las consecuencias de él y es, en muchas ocasiones, la principal herramienta para
resolver ecuaciones diofánticas.
Aquí veremos ejemplos de anillos que carecen de esta propiedad fundamental.
\begin{mydef}
	Un anillo $A$ se dice un \emph{dominio de factorización única} (abrev.\ \strong{DFU}) si es un dominio íntegro (i.e., no tiene
	divisores de cero no nulos) y todo elemento $a \in A$ admite una descomposición (\emph{existencia})
	\[
		a = u \pi_1^{e_1} \cdots \pi_n^{e_n},
	\]
	donde $u \in A^\times$ es una unidad, cada $\pi_j$ es irreducible (i.e., si $\beta \mid \pi_j$ y $\beta \notin A^\times$, entonces
	$(\beta) = (\pi_j)$) y los ideales $(\pi_j)$ son distintos dos a dos;
	además, de haber otra descomposición, entonces podemos reordenar los términos para que los ideales $(\pi_j)$ y los exponentes $e_j$
	sean los mismos (\emph{unicidad}).
\end{mydef}
Naturalmente, el <<teorema fundamental de la aritmética>> se reescribe ahora en términos de que $\Z$ es un DFU.
% Presentaremos ejemplos de anillos que no son DFU.

\begin{enumerate}[resume]
	\item Considere el siguiente conjunto
		\[
			\Z[\sqrt{-5}] = \{ a + b \sqrt{-5} : a, b \in \Z \} \subseteq \C.
		\]
		\begin{enumerate}
			\item Pruebe que $\Z[\sqrt{-5}]$ es un subanillo de $\C$ y que es un dominio íntegro.
			\item Definamos la <<norma>> de un elemento de $\Z[\sqrt{-5}]$ como
				\[
					\galnorm(x + y \sqrt{-5}) := x^2 + 5y^2 \in \Z.
				\]
				Pruebe que para $\alpha, \beta \in \Z[\sqrt{-5}]$ se cumple que $\galnorm(\alpha \cdot \beta) =
				\galnorm(\alpha) \cdot \galnorm(\beta)$.
			% \item Pruebe que si $\galnorm(\alpha)$ es un primo racional, entonces $\alpha$ era irreducible en $\Z[\sqrt{-5}]$.
			\item Empleando la factorización
				\[
					6 = 2\cdot 3 = (1 + \sqrt{-5})(1 - \sqrt{-5}),
				\]
				concluya que $\Z[\sqrt{-5}]$ no es un DFU.
		\end{enumerate}

	\item\label{ex:non_UFD_machine}\lookst
		Generalicemos el procedimiento anterior.
		Sea $\gamma \in \C$ un número algebraico
		% que es raíz de un polinomio mónico (i.e., cuyo coeficiente lider es 1) no constante con
		cuyo polinomio\break minimal (respecto a $\Q$) tiene
		coeficientes en $\Z$.
		\begin{enumerate}
			\item Pruebe que $\Z[\gamma] = \{ f(\gamma) : f(x) \in \Z[x] \}$ es un dominio íntegro.

			\item\label{ex:false_irred}
				\false{Pruebe que $\gamma \in \Z[\gamma]$ es irreducible.}

				\warn Resulta que este inciso es falso.
				El contraejemplo más sencillo es $\sqrt{-1} \in \Z[\sqrt{-1}]$ que es unidad.
				Otro ejemplo es $\sqrt{2} + 4 = \sqrt{2}(1 + 2 \sqrt{2}) \in \Z[\sqrt{2}+4] = \Z[\sqrt{2}]$.


			\item Muestre que hay un número primo (racional) $p \in \Z$ tal que $\gamma \nmid p$.

				\begin{hint}
					Para ello, pruebe que para todo $\alpha \notin \Z[\gamma]^\times$ hay \emph{finitos} ideales primos
					$\mathfrak{p} \nsl \Z[\gamma]$ tales que $\alpha \in \mathfrak{p}$, y argumente por qué hay un primo
					racional que no pertenece a alguno de ellos.
				\end{hint}

			\item \blunder{Finalmente, concluya que $\Z[n \gamma]$ no es un DFU para algún $n$.}

				Este ejercicio reposaba sobre la parte \ref{ex:false_irred} que es falsa.
				% Más aún, no es cierto que $n \gamma$ sea irreducible cuando $\gamma$ lo es en general;
				% tome $\gamma := \sqrt{2}$ y $n = 6$, entonces
				% $ 3\sqrt{8} = () $
		\end{enumerate}
\end{enumerate}

\section*{Comentarios}
% En el mundo investigativo, para los teoristas de números las ecuaciones diofánticas en dos variables de grado cuadrático están <<bien
% entendidas>> (también conocidas como \emph{cónicas planares}), de modo que el primer caso <<interesante>> sucede con cúbicas planares.
% En varios casos, hay cambios de variable mediante los cuales uno puede reducirse a estudiar ecuaciones del tipo $x^3 + ax + b = y^2 + cy +
% dxy$, que reflejan un objeto geométrico (más precisamente, una <<curva algebraica>>) llamada una \emph{curva elíptica}.
% Soluciones a coeficientes enteros de $y^2 = x^3 + a$ fueron arduamente estudiadas por el británico Louis Mordell.

Hay una razón del por qué $\Z[n \gamma]$ no es DFU en general; la estrategia que emplee para construir el contraejemplo tras bastidores está
en una aplicación de álgebra conmutativa.
Los elementos de $\C$ cuyo polinomio minimal tiene coeficientes en $\Z$ se dicen \strong{enteros algebraicos};
en general, para un anillo de enteros algebraicos, ser DFU equivale a ser DIP (esto no es trivial) y, por tanto, tal anillo debe ser
íntegramente cerrado, y esta es la propiedad que los $\Z[n \gamma]$ carecen.
(Está bien que, en una primera lectura, no entienda esta explicación; también está bien que el lector curioso la reciba de igual forma,
aunque sea vaga y poco formal.)
Para más detalles, vid.\ el primer capítulo de \citeauthor{janusz:algebraic}~\cite{janusz:algebraic}.
% o de \citeauthor{matsumura:ring}~\cite{matsumura:ring}.

No obstante, el anillo $\Z[\sqrt{-5}]$ \emph{sí} es íntegramente cerrado%
\footnote{El lector puede pensar que no existe $\gamma \in \C$ con polinomio minimal a coeficientes enteros tal que $n \gamma = \sqrt{-5}$
con $|n| > 1$.}
y es por ello que es un ejemplo clásico en la teoría de números.

Quizá el estudiante note la relación entre álgebra abstracta y teoría de números que ya se hace notar.
Esto no es coincidencia, varias de las nociones en álgebra (e.g., DIP y DFU) son abstracciones de fenómenos que ya se observan en $\Z$;
por lo demás, los conceptos más avanzados que subyacen el contraejemplo del problema~\ref{ex:non_UFD_machine} (dígase, elemento entero y
anillo íntegramente cerrado) fueron descubrimientos de teoristas de números alemanes (Dedekind, Kronecker, Gauss) que buscaban entender la
aritmética en otros anillos.
Cabe destacar que, del mismo modo que el análisis complejo facilita la teoría de integración real, la aritmética en anillos generales
facilita la teoría de números sobre $\Q$ y $\Z$.

\nocite{burton:elementary, andreescu:problems}
\printbibliography[title={Referencias y lecturas adicionales}]

\end{document}
