\documentclass[11pt, reqno]{amsart}

\usepackage{../ayud-template}
../LaTeX/general.tex
% \usepackage[spanish]{babel}
\usepackage[LGR, T1]{fontenc}
\usepackage[utf8]{inputenc}

../LaTeX/general.tex
% \input{../graphics.tex}

\makeatletter
\def\emailaddrname{\textit{Correo electrónico}}
\def\subtitle#1{\gdef\@subtitle{#1}}
\def\@subtitle{}

% Metadata
\def\logo#1{\gdef\@logo{#1}}
\def\@logo{}
\def\institution#1{\gdef\@institution{#1}}
\def\@institution{}
\def\department#1{\gdef\@department{#1}}
\def\@department{}
\def\professor#1{\gdef\@professor{#1}}
\def\@professor{}
\def\course#1{\gdef\@course{#1}}
\def\@course{}
\def\coursecode#1{\gdef\@coursecode{#1}}
\def\@coursecode{}

\renewcommand{\maketitle}{
\begin{center}
	\small
	\renewcommand{\arraystretch}{1.2}
	\begin{tabular}{cp{.37\textwidth}p{0.44\textwidth}}
		% \hline
		\multirow{5}{*}{\includegraphics[height=2.0cm]{\@logo}}
	  & \multicolumn{2}{c}{ \makecell{{\bfseries \@institution} \\ \@department} } \\
	  % & \multicolumn{2}{c|}{{\bfseries\@institution} \\ \@department} \\
	  \cline{2-3}
	  & \textbf{Profesor:} \@professor & \textbf{Ayudante:} \authors \\
	  % \cline{2-3}
	  & \textbf{Curso:} \@course & \textbf{Sigla:} \@coursecode \\
	  % \cline{2-3}
	  & \multicolumn{2}{l}{ \textbf{Fecha:} \@date } \\
	  % \hline
	\end{tabular}
	\\[\baselineskip]
	% {}
	% \vspace{2\baselineskip}
	{\bfseries\Large\@title}
	\ifx\@subtitle\@empty\else
		\\[1ex]
		\large\mdseries\@subtitle
	\fi
\end{center}
}
\makeatother

\usepackage{multirow, makecell}

\usepackage[
	reversemp,
	letterpaper,
	% marginpar=2cm,
	% marginsep=1pt,
	margin=2.3cm
]{geometry}
\usepackage{fontawesome}
% \makeatletter
% \@reversemargintrue
% \makeatother

% Símbolos al margen, necesitan doble compilación
\newcommand{\hard}{\marginnote{\faFire}}
\newcommand{\hhard}{\marginnote{\faFire\faFire}}

% Dependencias para los teoremas
\usepackage{xifthen}
\def\@thmdep{}
\newcommand{\thmdep}[1]{
	\ifthenelse{\isempty{#1}}
	{\def\@thmdep{}}
	{\def\@thmdep{ (#1)}}
}
\newcommand{\thmstyle}{\color{thm}\sffamily\bfseries}

% ===== Estilos de Teoremas ==========
\newtheoremstyle{axiomstyle}
	{0.3cm}
	{0.3cm}
	{\normalfont}
	{0.5cm}
	{\bfseries\scshape}
	{:}
	{4pt}
	{\thmname{#1}\thmnote{ #3}\thmnumber{ (#2)}}
\newtheoremstyle{styleC}
	{0.5cm}
	{0.5cm}
	{\normalfont}
	{0.5cm}
	{\bfseries}
	{:}
	{4pt}
	{\thmname{#1\textrm{\@thmdep}}\thmnumber{ #2}\thmnote{ (#3)}}

% ====== Teoremas (sin borde) ===========
\theoremstyle{axiomstyle}
\newtheorem*{axiom}{Axioma}

% ====== Teoremas (sin borde) ==================
\theoremstyle{styleC}
\newtheorem{thm}{Teorema}[section]
\newtheorem{mydef}[thm]{Definición}
\newtheorem{prop}[thm]{Proposición}
\newtheorem{cor}[thm]{Corolario}
\newtheorem{lem}[thm]{Lema}
\newtheorem{con}[thm]{Conjetura}

\newtheorem*{prob}{Problema}
\newtheorem*{sol}{Solución}
\newtheorem*{obs}{Observación}
\newtheorem*{ex}{Ejemplo}

% \usepackage{tcolorbox}
% \newtcbox{bluebox}[1][]{enhanced jigsaw, 
%   sharp corners,
%   frame hidden,
%   nobeforeafter,
%   listing only,
%   #1} % comando para crear cajas de colores

\expandafter\let\expandafter\oldproof\csname\string\proof\endcsname
\let\oldendproof\endproof
\renewenvironment{proof}[1][\proofname]{%
  \oldproof[\scshape Demostración:]%
}{\oldendproof} % comando para redefinir la caja de la demostración
\newenvironment{hint}[1][\proofname]{%
  \oldproof[\scshape Pista:]%
}{\oldendproof} % comando para redefinir la caja de la demostración

% colores utilizados
\definecolor{numchap}{RGB}{249,133,29}
\definecolor{chap}{RGB}{6,129,204}
\definecolor{sec}{RGB}{204,0,0}
\definecolor{thm}{RGB}{106,176,240}
\definecolor{thmB}{RGB}{32,31,31}
\definecolor{part}{RGB}{212,66,66}

% ====== Diseño de los titulares ===============
\usepackage[explicit]{titlesec} % para personalizar el documento, la opción <<explicit>> hace que el texto de los titulares sea un objeto interactuable

\titleformat{\subsection}[runin]
	{\bfseries}
	{\textrm{\S}\thesubsection}
	{1ex}
	{#1.}

\setlist[enumerate,1]{label=\arabic*., ref=\arabic*} % Enumerate standards


\title{Reciprocidad cuadrática y otros temas}
% \subtitle{Conociendo a Catalan, Fermat y Mordell}
\date{\DTMdate{2025-08-22}}

\author[José Cuevas]{José Cuevas Barrientos}
\email{josecuevasbtos@uc.cl}
\urladdr{https://josecuevas.xyz/teach/2025-2-num/}

\logo{../puc_negro.png}
\institution{Pontificia Universidad Católica de Chile}
\department{Facultad de Matemáticas}
\course{Teoría de Números}
\coursecode{MAT2814}
\professor{Ricardo Menares}

\begin{document}

\maketitle

\section{Ejercicios}
\begin{enumerate}
	\item Pruebe que un primo $p$ es suma de dos cuadrados syss $p = 2$ o $p \equiv 1 \pmod 4$.

		% \newpage
	\item \textbf{Criterio de Pépin:}
		Mediante reciprocidad cuadrática pruebe que si el número de Fermat $F_n := 2^{2^n} + 1$ (con $n > 0$) es primo syss
		$3^{\frac{1}{2}(F_n - 1)} \equiv -1 \pmod{F_n}$.
		\nocite{granville:masterclass}

		\begin{comment}
			\lookup
			El interés detrás de este criterio es que es computacionalmente rápido calcular potencias módulo un número.
			A mano, puede emplear el criterio de Pépin para confirmar que $F_5 = 4\, 294\, 967\, 297$ no es primo (hallazgo de
			L.~Euler) y, si lo implementa en un lenguaje como \texttt{Python}, \texttt{Sage} u otro, puede verificar que $F_5, \dots,
			F_{10}$ son todos compuestos sin problema.
			(Si, por el contrario, trata de buscar sus factores primos, dará cuenta que son bastante altos, por lo que el criterio es
			bastante más eficaz.)
		\end{comment}

		% \newpage
	\item Diremos que un dominio íntegro $A$ es \strong{íntegramente cerrado} si para todo $\alpha \in \Frac(A)$ tal que existe un
		polinomio mónico $f(x) \in A[x]$ con $f(\alpha) = 0$, se cumple que $\alpha \in A$.
		Pruebe que si $A$ es un DFU, entonces es íntegramente cerrado.
		Con ello concluya que $\Z[n \gamma]$ no es un DFU para $n > 2$ cuando $\gamma \in \C \setminus \Q$ es un entero algebraico.

		\begin{hint}
			Trate primero de probar que $\Z$ es íntegramente cerrado y generalice.
		\end{hint}

		% \newpage
	\item Sea $q$ una potencia de un primo $p$, y sea $K := \Fp[q]$.
		\begin{enumerate}
			\item Pruebe que $K^\times$ es un grupo cíclico.

			\item Pruebe que para un natural $n \in \N$ se cumple que
				\[
					\sum_{x \in K} x^n =
					\begin{cases}
						-1, & q-1\mid n, \; n \ne 0, \\
						0, & \text{en otro caso.}
					\end{cases}
				\]

			\item \textbf{Teorema de Chevalley-Warning:}
				\lookright
				Sean $\{ f_\alpha(\vec x) \}_\alpha \in K[x_1, \dots, x_n]$ un conjunto finito de polinomios tales que $\sum_{\alpha} \deg(f_\alpha) < n$.
				Sea $V := \VV(\{ f_\alpha \}) \subseteq K^n$ el conjunto de ceros comunes de los polinomios, entonces pruebe que $|V| \equiv 0 \pmod p$.

				\begin{hint}
					Note que $|V| = \sum_{\vec x \in K^n} \chi_V(\vec x)$, donde $\chi$ es la función característica.
					Ahora recupere $\chi_V$ mediante polinomios y emplee el inciso anterior.
				\end{hint}
		\end{enumerate}
		\nocite{serre:arithmetic}

		% \newpage
	\item Sea $p$ un número primo.
		\begin{enumerate}
			\item Pruebe que, dado un natural $n = n_0 + n_1p + \cdots + n_dp^d$ en base $p$, se cumple que
				\[
					(x + y)^n \equiv (x + y)^{n_0}(x^p + y^p)^{n_1} \cdots \left( x^{p^d} + y^{p^d} \right)^{n_d} \pmod p.
				\]

			\item \textbf{Teorema de Lucas:}
				Sea $m = m_0 + m_1p + \cdots + m_dp^d \le n$ otro natural en base $p$.
				Pruebe que
				\[
					\binom{n}{m} \equiv \binom{n_0}{m_0} \binom{n_1}{m_1} \cdots \binom{n_d}{m_d} \pmod p.
				\]
		\end{enumerate}
\end{enumerate}

\begin{additional}
\appendix
\section{Ejercicios adicionales}
\begin{enumerate}
	\item Pruebe que si un número de la forma $2^m + 1$ es primo, entonces necesariamente $m$ es una potencia de dos.
		Los números de la forma $F_n := 2^{2^n} + 1$ se llamarán \strong{de Fermat}.

	\item\lookright
		Pruebe que los números de Fermat son coprimos dos a dos y, con ello, dé una nueva demostración de la infinitud de los primos.
\end{enumerate}

\section{Comentarios adicionales}
\lookup
Una de las consecuencias del teorema de Chevalley-Warning (y parte del interés detrás de esta proposición) está en que muestra que los
cuerpos finitos son \emph{cuasialgebraicamente cerrados}.
En lenguaje de geometría algebraica, un cuerpo es algebraicamente cerrado si toda variedad algebraica no vacía tiene puntos racionales.
Piense el lector en la curva dada por los ceros del polinomio $x^2 + y^2 = 1$, es un círculo y eligiendo un punto base como $(1, 0)$ vemos
que tras trazar una recta de pendiente $p$ (con el convenio de que la vertical corresponde a $p = \infty$), vemos que hay una biyección con
los puntos de ésta y los de la recta proyectiva $\PP^1$.

La curva $x^2 + y^2 = -1$ tiene la misma suerte si nos paramos en el punto $(\sqrt{-1}, 0)$ que es \textquote{racional} sobre $\C$, de modo
que vemos que toda cónica no degenerada es isomorfa a la recta proyectiva.
Pero sobre $\Q$, ésta curva no tiene puntos racionales, de modo que no es una recta proyectiva, pese a que la razón tiene solo que ver con
el cuerpo base, no con la \textquote{geometría interna} de la curva.
Esta clase de variedades se llaman \strong{de Brauer-Severi}; mediante el teorema de Chevalley-Warning uno puede probar que sobre un cuerpo
finito no hay variedades de Brauer-Severi salvo por los espacios proyectivos $\PP^n$.

\printbibliography[title={Referencias y lecturas adicionales}]
\end{additional}

\end{document}
