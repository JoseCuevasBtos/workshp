\documentclass[11pt, reqno]{amsart}

\usepackage{../ayud-template}
../LaTeX/general.tex
% \usepackage[spanish]{babel}
\usepackage[LGR, T1]{fontenc}
\usepackage[utf8]{inputenc}

../LaTeX/general.tex
% \input{../graphics.tex}

\makeatletter
\def\emailaddrname{\textit{Correo electrónico}}
\def\subtitle#1{\gdef\@subtitle{#1}}
\def\@subtitle{}

% Metadata
\def\logo#1{\gdef\@logo{#1}}
\def\@logo{}
\def\institution#1{\gdef\@institution{#1}}
\def\@institution{}
\def\department#1{\gdef\@department{#1}}
\def\@department{}
\def\professor#1{\gdef\@professor{#1}}
\def\@professor{}
\def\course#1{\gdef\@course{#1}}
\def\@course{}
\def\coursecode#1{\gdef\@coursecode{#1}}
\def\@coursecode{}

\renewcommand{\maketitle}{
\begin{center}
	\small
	\renewcommand{\arraystretch}{1.2}
	\begin{tabular}{cp{.37\textwidth}p{0.44\textwidth}}
		% \hline
		\multirow{5}{*}{\includegraphics[height=2.0cm]{\@logo}}
	  & \multicolumn{2}{c}{ \makecell{{\bfseries \@institution} \\ \@department} } \\
	  % & \multicolumn{2}{c|}{{\bfseries\@institution} \\ \@department} \\
	  \cline{2-3}
	  & \textbf{Profesor:} \@professor & \textbf{Ayudante:} \authors \\
	  % \cline{2-3}
	  & \textbf{Curso:} \@course & \textbf{Sigla:} \@coursecode \\
	  % \cline{2-3}
	  & \multicolumn{2}{l}{ \textbf{Fecha:} \@date } \\
	  % \hline
	\end{tabular}
	\\[\baselineskip]
	% {}
	% \vspace{2\baselineskip}
	{\bfseries\Large\@title}
	\ifx\@subtitle\@empty\else
		\\[1ex]
		\large\mdseries\@subtitle
	\fi
\end{center}
}
\makeatother

\usepackage{multirow, makecell}

\usepackage[
	reversemp,
	letterpaper,
	% marginpar=2cm,
	% marginsep=1pt,
	margin=2.3cm
]{geometry}
\usepackage{fontawesome}
% \makeatletter
% \@reversemargintrue
% \makeatother

% Símbolos al margen, necesitan doble compilación
\newcommand{\hard}{\marginnote{\faFire}}
\newcommand{\hhard}{\marginnote{\faFire\faFire}}

% Dependencias para los teoremas
\usepackage{xifthen}
\def\@thmdep{}
\newcommand{\thmdep}[1]{
	\ifthenelse{\isempty{#1}}
	{\def\@thmdep{}}
	{\def\@thmdep{ (#1)}}
}
\newcommand{\thmstyle}{\color{thm}\sffamily\bfseries}

% ===== Estilos de Teoremas ==========
\newtheoremstyle{axiomstyle}
	{0.3cm}
	{0.3cm}
	{\normalfont}
	{0.5cm}
	{\bfseries\scshape}
	{:}
	{4pt}
	{\thmname{#1}\thmnote{ #3}\thmnumber{ (#2)}}
\newtheoremstyle{styleC}
	{0.5cm}
	{0.5cm}
	{\normalfont}
	{0.5cm}
	{\bfseries}
	{:}
	{4pt}
	{\thmname{#1\textrm{\@thmdep}}\thmnumber{ #2}\thmnote{ (#3)}}

% ====== Teoremas (sin borde) ===========
\theoremstyle{axiomstyle}
\newtheorem*{axiom}{Axioma}

% ====== Teoremas (sin borde) ==================
\theoremstyle{styleC}
\newtheorem{thm}{Teorema}[section]
\newtheorem{mydef}[thm]{Definición}
\newtheorem{prop}[thm]{Proposición}
\newtheorem{cor}[thm]{Corolario}
\newtheorem{lem}[thm]{Lema}
\newtheorem{con}[thm]{Conjetura}

\newtheorem*{prob}{Problema}
\newtheorem*{sol}{Solución}
\newtheorem*{obs}{Observación}
\newtheorem*{ex}{Ejemplo}

% \usepackage{tcolorbox}
% \newtcbox{bluebox}[1][]{enhanced jigsaw, 
%   sharp corners,
%   frame hidden,
%   nobeforeafter,
%   listing only,
%   #1} % comando para crear cajas de colores

\expandafter\let\expandafter\oldproof\csname\string\proof\endcsname
\let\oldendproof\endproof
\renewenvironment{proof}[1][\proofname]{%
  \oldproof[\scshape Demostración:]%
}{\oldendproof} % comando para redefinir la caja de la demostración
\newenvironment{hint}[1][\proofname]{%
  \oldproof[\scshape Pista:]%
}{\oldendproof} % comando para redefinir la caja de la demostración

% colores utilizados
\definecolor{numchap}{RGB}{249,133,29}
\definecolor{chap}{RGB}{6,129,204}
\definecolor{sec}{RGB}{204,0,0}
\definecolor{thm}{RGB}{106,176,240}
\definecolor{thmB}{RGB}{32,31,31}
\definecolor{part}{RGB}{212,66,66}

% ====== Diseño de los titulares ===============
\usepackage[explicit]{titlesec} % para personalizar el documento, la opción <<explicit>> hace que el texto de los titulares sea un objeto interactuable

\titleformat{\subsection}[runin]
	{\bfseries}
	{\textrm{\S}\thesubsection}
	{1ex}
	{#1.}

\setlist[enumerate,1]{label=\arabic*., ref=\arabic*} % Enumerate standards

% \includecomment{comment}

\title{Caracteres de Dirichlet}
\date{\DTMdate{2025-08-28}}

\author[José Cuevas]{José Cuevas Barrientos}
\email{josecuevasbtos@uc.cl}
\urladdr{https://josecuevas.xyz/teach/2025-2-num/}

\logo{../puc_negro.png}
\institution{Pontificia Universidad Católica de Chile}
\department{Facultad de Matemáticas}
\course{Teoría de Números}
\coursecode{MAT2814}
\professor{Ricardo Menares}

\begin{document}

\maketitle

\section{Ejercicios}
\begin{enumerate}
	\item Pruebe que las siguientes dos afirmaciones son (elementalmente) equivalentes:
		\begin{enumerate}
			\item Para cada par de enteros $a, n$ coprimos hay infinitos  primos $p$ tales que $p \equiv a \pmod n$.
			\item Para cada par de enteros $a, n$ coprimos hay al menos un primo $p$ tal   que $p \equiv a \pmod n$.
		\end{enumerate}

		\newex
	\item\lookright
		Sea $a \in \Z$.
		Pruebe que si para todo primo $p \nmid a$ se cumple que $(a/p) = 1$, entonces $a$ es un cuadrado perfecto.

		\newex
	\item Pruebe que el grupo de caracteres de Dirichlet módulo $n$ es isomorfo a $(\Z/n\Z)^\times$, no canónicamente.

		\newex
	\item Pruebe que para un primo $p$ y un natural $a \in \N_{\ge 2}$ se cumple que
		\[
			(\Z/p^a\Z)^\times \cong
			\begin{cases}
				C_2, & p = 2, \; a = 2, \\
				C_2 \times C_{2^{a-2}}, & p = 2, \; a > 2, \\
				C_{p-1} \times C_{p^{a-1}}, & p > 2, \; a \ge 2.
			\end{cases}
		\]
		\begin{hint}
			Para un primo impar $p$ equivale a buscar una raíz primitiva módulo $p$;
			para el primo $p = 2$ podemos calcular la 2 y 4-torsión del grupo.
		\end{hint}

		\newex
	\item\lookst
		\textbf{Un criterio excéntrico de primalidad:}
		Sea $p = a_db^d + \cdots + a_1b + a_0$ un primo $p > b$ en base $b \ge 3$ y
		sea $f(x) = a_dx^d + \cdots + a_1x + a_0 \in \Z[x]$.
		\begin{enumerate}
			\item Supongamos que $f(x)$ fuese reducible.
				Pruebe que existe una raíz $\alpha \in \C$ de $f$ tal que $|b - \alpha| \le 1$.

			\item Sea $\alpha$ como en el inciso anterior.
				Pruebe que $\Re(1/\alpha) > 0$, pero que $\Re(1/\alpha^j) < 0$ para algún $j$. 

			\item Pruebe que
				\[
					\Re\mathopen{}\left( \frac{f(\alpha)}{\alpha^d} \right) \ge \frac{b-3}{b-2} + a_{d-1} \Re(1/\alpha),
				\]
				y concluya, por contradicción, que $f$ debía ser irreducible.

				\begin{hint}
					Dé una cota inferior para $a_{d-n} \Re(1/\alpha^n)$ cuando $n \ge 2$.
				\end{hint}
				% \begin{proof}
				% 	Sea $f(x) = g(x)h(x)$.
				% 	Como $f(b)$ es primo, entonces podemos suponer que $g(b) = \pm 1$.
				% 	Existen $\alpha_1, \dots, \alpha_n \in \C$ tales que
				% 	$$ g(x) = c \prod_{j=1}^{n} (x - \alpha_j), $$
				% 	donde $c \in \Z$.
				% 	Nótese que, como $|c| \ge 1$, tenemos que $\prod_{j=1}^{n} |b - \alpha_j| \le g(b)$,
				% 	por lo que existe alguna raíz $\alpha$ tal que $|b - \alpha| \le 1$, es decir, $\Re\alpha \in [b - 1, b + 1],
				% 	\Re(1/\alpha) > 0$ y $|\alpha| \ge b - 1$.

				% 	Como $f(\alpha) = 0$, se sigue que
				% 	$$ 0 = \Re\left( \frac{f(\alpha)}{\alpha^d} \right) = a_d + a_{d-1}\Re\left( \frac{1}{\alpha} \right)
				% 	+ \sum_{i=2}^{d} a_{d-i} \Re\left( \frac{1}{\alpha^i} \right). $$
				% 	Como $a_{d-1}\Re(1/\alpha) > 0$ entonces algún $\Re(1/\alpha^i) < 0$, pero $a_{d-i}\Re(1/\alpha^i) \ge
				% 	-(b-1)/|\alpha|^i$. Luego
				% 	$$ 0 = \Re\left( \frac{f(\alpha)}{\alpha^d} \right) \ge 1 + 0 - (b - 1) \sum_{i=2}^{d} \frac{1}{|\alpha|^i}, $$
				% 	por lo que
				% 	$$ 1 < (b-1) \sum_{i=2}^{d} \frac{1}{|\alpha|^i} = \frac{b - 1}{|\alpha| \, (|\alpha| - 1)} \le \frac{1}{b - 2} $$
				% 	lo que es absurdo.
				% \end{proof}
		\end{enumerate}
		\nocite{granville:masterclass}

		\newex
	\item
		% Diremos que un número algebraico $\gamma \in \C$ es un \strong{entero algebraico} si su polinomio minimal tiene coeficientes
		% en $\Z$.
		Demuestre, empleando el teorema de los números primos en progresión aritmética, que para todo $n \ge 2$ existe un entero
		algebraico irracional $\gamma \in \C \setminus \Q$ de grado $n$ tal que $\gamma \in \Z[\gamma]$ es irreducible.

		\begin{thm}
			Sean $a, n$ un par de enteros coprimos y $\pi(x; a, n)$ la función que cuenta primos $p \le x$ tales que $p \equiv a
			\pmod n$.
			Entonces
			\[
				\pi(x; a, n) \sim \frac{1}{\phi(n)} \frac{x}{\log x}, \qquad x \to \infty,
			\]
			donde $\phi$ es la función de Euler.
		\end{thm}
		\begin{proof}
			Vid.\ \cite[361]{tenenbaum:analytique}.
		\end{proof}
\end{enumerate}

\begin{additional}
\newpage
\printbibliography[title={Referencias y lecturas adicionales}]
\end{additional}

\end{document}
