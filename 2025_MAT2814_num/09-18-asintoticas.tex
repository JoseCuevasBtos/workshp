\documentclass[11pt, reqno]{amsart}

\usepackage{../ayud-template}
\input{../general.tex}
% \usepackage[spanish]{babel}
\usepackage[LGR, T1]{fontenc}
\usepackage[utf8]{inputenc}

\input{../general.tex}
% \input{../graphics.tex}

\makeatletter
\def\emailaddrname{\textit{Correo electrónico}}
\def\subtitle#1{\gdef\@subtitle{#1}}
\def\@subtitle{}

% Metadata
\def\logo#1{\gdef\@logo{#1}}
\def\@logo{}
\def\institution#1{\gdef\@institution{#1}}
\def\@institution{}
\def\department#1{\gdef\@department{#1}}
\def\@department{}
\def\professor#1{\gdef\@professor{#1}}
\def\@professor{}
\def\course#1{\gdef\@course{#1}}
\def\@course{}
\def\coursecode#1{\gdef\@coursecode{#1}}
\def\@coursecode{}

\renewcommand{\maketitle}{
\begin{center}
	\small
	\renewcommand{\arraystretch}{1.2}
	\begin{tabular}{cp{.37\textwidth}p{0.44\textwidth}}
		% \hline
		\multirow{5}{*}{\includegraphics[height=2.0cm]{\@logo}}
	  & \multicolumn{2}{c}{ \makecell{{\bfseries \@institution} \\ \@department} } \\
	  % & \multicolumn{2}{c|}{{\bfseries\@institution} \\ \@department} \\
	  \cline{2-3}
	  & \textbf{Profesor:} \@professor & \textbf{Ayudante:} \authors \\
	  % \cline{2-3}
	  & \textbf{Curso:} \@course & \textbf{Sigla:} \@coursecode \\
	  % \cline{2-3}
	  & \multicolumn{2}{l}{ \textbf{Fecha:} \@date } \\
	  % \hline
	\end{tabular}
	\\[\baselineskip]
	% {}
	% \vspace{2\baselineskip}
	{\bfseries\Large\@title}
	\ifx\@subtitle\@empty\else
		\\[1ex]
		\large\mdseries\@subtitle
	\fi
\end{center}
}
\makeatother

\usepackage{multirow, makecell}

\usepackage[
	reversemp,
	letterpaper,
	% marginpar=2cm,
	% marginsep=1pt,
	margin=2.3cm
]{geometry}
\usepackage{fontawesome}
% \makeatletter
% \@reversemargintrue
% \makeatother

% Símbolos al margen, necesitan doble compilación
\newcommand{\hard}{\marginnote{\faFire}}
\newcommand{\hhard}{\marginnote{\faFire\faFire}}

% Dependencias para los teoremas
\usepackage{xifthen}
\def\@thmdep{}
\newcommand{\thmdep}[1]{
	\ifthenelse{\isempty{#1}}
	{\def\@thmdep{}}
	{\def\@thmdep{ (#1)}}
}
\newcommand{\thmstyle}{\color{thm}\sffamily\bfseries}

% ===== Estilos de Teoremas ==========
\newtheoremstyle{axiomstyle}
	{0.3cm}
	{0.3cm}
	{\normalfont}
	{0.5cm}
	{\bfseries\scshape}
	{:}
	{4pt}
	{\thmname{#1}\thmnote{ #3}\thmnumber{ (#2)}}
\newtheoremstyle{styleC}
	{0.5cm}
	{0.5cm}
	{\normalfont}
	{0.5cm}
	{\bfseries}
	{:}
	{4pt}
	{\thmname{#1\textrm{\@thmdep}}\thmnumber{ #2}\thmnote{ (#3)}}

% ====== Teoremas (sin borde) ===========
\theoremstyle{axiomstyle}
\newtheorem*{axiom}{Axioma}

% ====== Teoremas (sin borde) ==================
\theoremstyle{styleC}
\newtheorem{thm}{Teorema}[section]
\newtheorem{mydef}[thm]{Definición}
\newtheorem{prop}[thm]{Proposición}
\newtheorem{cor}[thm]{Corolario}
\newtheorem{lem}[thm]{Lema}
\newtheorem{con}[thm]{Conjetura}

\newtheorem*{prob}{Problema}
\newtheorem*{sol}{Solución}
\newtheorem*{obs}{Observación}
\newtheorem*{ex}{Ejemplo}

% \usepackage{tcolorbox}
% \newtcbox{bluebox}[1][]{enhanced jigsaw, 
%   sharp corners,
%   frame hidden,
%   nobeforeafter,
%   listing only,
%   #1} % comando para crear cajas de colores

\expandafter\let\expandafter\oldproof\csname\string\proof\endcsname
\let\oldendproof\endproof
\renewenvironment{proof}[1][\proofname]{%
  \oldproof[\scshape Demostración:]%
}{\oldendproof} % comando para redefinir la caja de la demostración
\newenvironment{hint}[1][\proofname]{%
  \oldproof[\scshape Pista:]%
}{\oldendproof} % comando para redefinir la caja de la demostración

% colores utilizados
\definecolor{numchap}{RGB}{249,133,29}
\definecolor{chap}{RGB}{6,129,204}
\definecolor{sec}{RGB}{204,0,0}
\definecolor{thm}{RGB}{106,176,240}
\definecolor{thmB}{RGB}{32,31,31}
\definecolor{part}{RGB}{212,66,66}

% ====== Diseño de los titulares ===============
\usepackage[explicit]{titlesec} % para personalizar el documento, la opción <<explicit>> hace que el texto de los titulares sea un objeto interactuable

\titleformat{\subsection}[runin]
	{\bfseries}
	{\textrm{\S}\thesubsection}
	{1ex}
	{#1.}

\setlist[enumerate,1]{label=\arabic*., ref=\arabic*} % Enumerate standards

% \includecomment{comment}

\title{Asintótica de funciones aritméticas}
\date{\DTMdate{2025-09-18}}

\author[José Cuevas]{José Cuevas Barrientos}
\email{josecuevasbtos@uc.cl}
\urladdr{https://josecuevas.xyz/teach/2025-2-num/}

\logo{../puc_negro.png}
\institution{Pontificia Universidad Católica de Chile}
\department{Facultad de Matemáticas}
\course{Teoría de Números}
\coursecode{MAT2814}
\professor{Ricardo Menares}

\begin{document}

\maketitle

\section{Ejercicios}
\begin{enumerate}
	\item\lookright
		Sea $\Fp[q]$ un cuerpo con $q < \infty$ elementos.
		\begin{enumerate}
			\item Sea $f(x) \in \Fp[q][x]$ es irreducible.
				Pruebe que $f(x) \mid x^{q^n} - x$ syss $\deg f \mid n$.

			\item Sea $M(n)$ la cantidad de polinomios irreducibles de grado $n$ en $\Fp[q][x]$ y sea $E(n)$ la cantidad de
				elementos de $\Fp[q^n]$ de grado exactamente $n$ (i.e., cuyo polinomio minimal tiene grado $n$ y, por tanto,
				que generan la extensión $\Fp[q^n]$).
				¿Qué relación hay entre $M(n)$ y $E(n)$?

			\item Pruebe que
				\[
					n \psi(n) = \sum_{d \mid n} \mu(d) q^{n/d}.
				\]
		\end{enumerate}

		\newex
	\item
		\textbf{Primer teorema de Mertens:} Demuestre que
		$$ \sum_{p\le x} \frac{\log p}{p} = \log x + O(1), $$
		(donde el subíndice $p$ siempre recorre los números primos.)

		Para facilitar el ejercicio realice los siguientes pasos:
		\begin{enumerate}[(i)]
			\item Demuestre que cuando $n$ es entero
				\[
					T(n) = \log(n!) = \sum_{p \le n} \left( \left\lfloor \frac np \right\rfloor + \left\lfloor \frac{n}{p^2} \right\rfloor
					+ \cdots \right) \log p.
				\]

			\item Demuestre que
				$$ \frac{n}{p} - 1 < \left\lfloor \frac np \right\rfloor + \left\lfloor \frac{n}{p^2} \right\rfloor + \cdots
				< \frac{n}{p} + \frac{n}{p(p - 1)}. $$

			\item Demuestre que $\sum_{p \le x} \log p \le c_2 x$.

			\item Concluya el enunciado.
		\end{enumerate}

		\newex
	\item
		\textbf{Segundo teorema de Mertens:} Demuestre que
		$$ \sum_{p\le x} \frac{1}{p} = \log\log x + M + O\left( \frac{1}{\log x} \right), $$
		donde $M$ es una constante llamada la \textit{constante de Mertens}.
		\nocite{tenenbaum:analytique}
		\begin{hint}
			Emplee fórmula de suma por partes de Abel.
		\end{hint}

		\newex
	\item\lookst
		\textbf{Postulado de Bertrand (demostración de Erd\H os):}
		El objetivo de este ejercicio es probar que para todo $n \ge 2$ existe un primo $p$ con $n \le p < 2n$.
		\begin{enumerate}[(i)]
			\item Pruebe que $\prod_{p\le x} p \le 4^x$ para todo $x \ge 2$.
			\item Pruebe que para todo entero $n \ge 1$ se cumple que
				\[
					% \binom{n}{\lfloor n/2 \rfloor} \ge \frac{2^n}{n}, \qquad
					\binom{2n}{n} \ge \frac{4^n}{2n}.
				\]
			\item Dé cotas para la valuación de un primo $p \mid \binom{2n}{n}$ en los intervalos:
				\[
					% [1, \sqrt{2n}], \qquad
					\left(\sqrt{2n}, \frac{2}{3} n\right], \qquad \left(\frac{2}{3} n, n\right], \qquad (n, 2n].
				\]
			\item Confronte cotas y pruebe que el postulado de Bertrand es cierto para $n$ suficientemente grande.
		\end{enumerate}
		\nocite{granville:masterclass}
\end{enumerate}

\begin{additional}
% \appendix
% \section{Ejercicio propuesto}
% \begin{enumerate}[resume]
% 	\item Pruebe las cotas de tipo Stirling:
% 		\[
% 			e\left( \frac{N}{e} \right)^N \le N! \le eN \left( \frac{N}{e} \right)^N.
% 		\]
% 		\begin{hint}
% 			Puede emplear que $\log(n-1) \le \int_{n-1}^{n} \log x \, \ud x \le \log n$.
% 		\end{hint}

% 	\item Generalice la prueba del postulado de Bertrand para demostrar el \textbf{teorema de Sylvester-Schur:}
% \end{enumerate}

\printbibliography[title={Referencias y lecturas adicionales}]
\end{additional}

\end{document}
