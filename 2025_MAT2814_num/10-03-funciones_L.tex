\documentclass[11pt, reqno]{amsart}

\usepackage{../ayud-template}
../LaTeX/general.tex
% \usepackage[spanish]{babel}
\usepackage[LGR, T1]{fontenc}
\usepackage[utf8]{inputenc}

../LaTeX/general.tex
% \input{../graphics.tex}

\makeatletter
\def\emailaddrname{\textit{Correo electrónico}}
\def\subtitle#1{\gdef\@subtitle{#1}}
\def\@subtitle{}

% Metadata
\def\logo#1{\gdef\@logo{#1}}
\def\@logo{}
\def\institution#1{\gdef\@institution{#1}}
\def\@institution{}
\def\department#1{\gdef\@department{#1}}
\def\@department{}
\def\professor#1{\gdef\@professor{#1}}
\def\@professor{}
\def\course#1{\gdef\@course{#1}}
\def\@course{}
\def\coursecode#1{\gdef\@coursecode{#1}}
\def\@coursecode{}

\renewcommand{\maketitle}{
\begin{center}
	\small
	\renewcommand{\arraystretch}{1.2}
	\begin{tabular}{cp{.37\textwidth}p{0.44\textwidth}}
		% \hline
		\multirow{5}{*}{\includegraphics[height=2.0cm]{\@logo}}
	  & \multicolumn{2}{c}{ \makecell{{\bfseries \@institution} \\ \@department} } \\
	  % & \multicolumn{2}{c|}{{\bfseries\@institution} \\ \@department} \\
	  \cline{2-3}
	  & \textbf{Profesor:} \@professor & \textbf{Ayudante:} \authors \\
	  % \cline{2-3}
	  & \textbf{Curso:} \@course & \textbf{Sigla:} \@coursecode \\
	  % \cline{2-3}
	  & \multicolumn{2}{l}{ \textbf{Fecha:} \@date } \\
	  % \hline
	\end{tabular}
	\\[\baselineskip]
	% {}
	% \vspace{2\baselineskip}
	{\bfseries\Large\@title}
	\ifx\@subtitle\@empty\else
		\\[1ex]
		\large\mdseries\@subtitle
	\fi
\end{center}
}
\makeatother

\usepackage{multirow, makecell}

\usepackage[
	reversemp,
	letterpaper,
	% marginpar=2cm,
	% marginsep=1pt,
	margin=2.3cm
]{geometry}
\usepackage{fontawesome}
% \makeatletter
% \@reversemargintrue
% \makeatother

% Símbolos al margen, necesitan doble compilación
\newcommand{\hard}{\marginnote{\faFire}}
\newcommand{\hhard}{\marginnote{\faFire\faFire}}

% Dependencias para los teoremas
\usepackage{xifthen}
\def\@thmdep{}
\newcommand{\thmdep}[1]{
	\ifthenelse{\isempty{#1}}
	{\def\@thmdep{}}
	{\def\@thmdep{ (#1)}}
}
\newcommand{\thmstyle}{\color{thm}\sffamily\bfseries}

% ===== Estilos de Teoremas ==========
\newtheoremstyle{axiomstyle}
	{0.3cm}
	{0.3cm}
	{\normalfont}
	{0.5cm}
	{\bfseries\scshape}
	{:}
	{4pt}
	{\thmname{#1}\thmnote{ #3}\thmnumber{ (#2)}}
\newtheoremstyle{styleC}
	{0.5cm}
	{0.5cm}
	{\normalfont}
	{0.5cm}
	{\bfseries}
	{:}
	{4pt}
	{\thmname{#1\textrm{\@thmdep}}\thmnumber{ #2}\thmnote{ (#3)}}

% ====== Teoremas (sin borde) ===========
\theoremstyle{axiomstyle}
\newtheorem*{axiom}{Axioma}

% ====== Teoremas (sin borde) ==================
\theoremstyle{styleC}
\newtheorem{thm}{Teorema}[section]
\newtheorem{mydef}[thm]{Definición}
\newtheorem{prop}[thm]{Proposición}
\newtheorem{cor}[thm]{Corolario}
\newtheorem{lem}[thm]{Lema}
\newtheorem{con}[thm]{Conjetura}

\newtheorem*{prob}{Problema}
\newtheorem*{sol}{Solución}
\newtheorem*{obs}{Observación}
\newtheorem*{ex}{Ejemplo}

% \usepackage{tcolorbox}
% \newtcbox{bluebox}[1][]{enhanced jigsaw, 
%   sharp corners,
%   frame hidden,
%   nobeforeafter,
%   listing only,
%   #1} % comando para crear cajas de colores

\expandafter\let\expandafter\oldproof\csname\string\proof\endcsname
\let\oldendproof\endproof
\renewenvironment{proof}[1][\proofname]{%
  \oldproof[\scshape Demostración:]%
}{\oldendproof} % comando para redefinir la caja de la demostración
\newenvironment{hint}[1][\proofname]{%
  \oldproof[\scshape Pista:]%
}{\oldendproof} % comando para redefinir la caja de la demostración

% colores utilizados
\definecolor{numchap}{RGB}{249,133,29}
\definecolor{chap}{RGB}{6,129,204}
\definecolor{sec}{RGB}{204,0,0}
\definecolor{thm}{RGB}{106,176,240}
\definecolor{thmB}{RGB}{32,31,31}
\definecolor{part}{RGB}{212,66,66}

% ====== Diseño de los titulares ===============
\usepackage[explicit]{titlesec} % para personalizar el documento, la opción <<explicit>> hace que el texto de los titulares sea un objeto interactuable

\titleformat{\subsection}[runin]
	{\bfseries}
	{\textrm{\S}\thesubsection}
	{1ex}
	{#1.}

\setlist[enumerate,1]{label=\arabic*., ref=\arabic*} % Enumerate standards

% \includecomment{comment}

\title{Funciones $L$ y series de Dirichlet}
\date{\DTMdate{2025-10-03}}

\author[José Cuevas]{José Cuevas Barrientos}
\email{josecuevasbtos@uc.cl}
\urladdr{https://josecuevas.xyz/teach/2025-2-num/}

\logo{../puc_negro.png}
\institution{Pontificia Universidad Católica de Chile}
\department{Facultad de Matemáticas}
\course{Teoría de Números}
\coursecode{MAT2814}
\professor{Ricardo Menares}

\begin{document}

\maketitle

\section{Ejercicios}
\begin{enumerate}
	\item \begin{enumerate}
			\item\lookright Muestre que $L(\mu^2, s) = \zeta(s)/\zeta(2s)$.
				\begin{hint}
					Escriba $\zeta(2s) = L(f, s)$ para alguna función artimética $f$.
				\end{hint}

			\item Concluya que \smash{$\displaystyle \mu^2(n) = \sum_{d^2\mid n} \mu(d)$}.

			\item Pruebe que
				\[
					\displaystyle \sum_{n\le x} \mu^2(n) = \frac{6}{\pi^2} x + O(\sqrt{x}).
				\]
		\end{enumerate}

		\newex
	\item Sea $L(f, s) := \sum_{n=1}^{\infty} f(n)/n^s$ una serie de Dirichlet y suponga que $f(1) \ne 0$ y que $L(f, s) \ne 0$ para
		todo $s \in \C$ con $\Re s > \sigma_0$.
		Entonces $L(f, s) = e^{G(s)}$, donde
		\[
			G(s) = \log f(1) + \sum_{n=2}^{\infty} \frac{(f' * f^{-1})(n)/\log n}{n^s},
		\]
		donde $f'(n) = f(n)\log(n)$ y donde $f^{-1}$ denota la inversa respecto a convolución
		(i.e.,\break $(f * f^{-1})(n) = \delta_{1, n}$).
		\nocite{apostol:analytic}

		\newex
	\item\label{ex:dirichlet_zeta} Considere la función dseta de Dirichlet
		\[
			\zeta_{\Q(\ui)}(s) = \sum_{\mathfrak{a}\ne 0} \frac{1}{\numnorm(\mathfrak{a})^s},
		\]
		donde $\mathfrak{a}$ recorre los ideales de los enteros gaussianos $\Z[\ui]$ y donde $\numnorm(\mathfrak{a}) = |\Z[i] / \mathfrak{a}|$.
		\begin{enumerate}
			\item Pruebe que si $\mathfrak{a} = \beta \Z[\ui]$, entonces $\numnorm(\mathfrak{a}) = \galnorm_{\Q(\ui)/\Q}(\beta)
				= |\beta|^2$.

			\item Pruebe que la serie que define a $\zeta_{\Q(\ui)}$ converge para $s > 1$.

			\item\lookst
				Pruebe que $\Res_{s=1} \zeta_{\Q(\ui)} := \lim_{s \to 1^+} \zeta_{\Q(\ui)}(s)/(s - 1) = \pi$.
		\end{enumerate}

		\newex
	\item\label{ex:bound_height} \lookst
		\textbf{Contar puntos de altura acotada en $\PP^1(\Q)$:}
		Recuerde que un punto racional en la recta proyectiva es un par $[x : y]$, donde $x, y \in \Q$, no son ambos nulos y $[x :
		y] = [z : w]$ syss $x = \lambda z$ e $y = \lambda w$ para algún $\lambda \in \Q^\times$.
		La \emph{altura} se define como $H([x : y]) = \max\{ |x|, |y| \}$, donde $x, y$ son enteros coprimos.

		Dado $B \ge 0$ real, demuestre que la cantidad de puntos $[x : y] \in \PP^1(\Q)$ de altura acotada $H([x : y]) \le B$ es
		$$ \frac{12}{\pi^2}B^2 + O(B \log B). $$
		\begin{hint}
			Uno puede observar que esto equivale a contar pares de coprimos y, mediante una fórmula recursiva, reducirse a calcular
			asintóticamente
			$$ \sum_{k=1}^{n} 2\phi(k), $$
			donde ésto representa la función $\phi$ de Euler.
			% Empleando inversión de Möbius podemos reescribirlo en términos de la función $\mu$ de Möbius y, finalmente, reconocer este problema de
			% la ayudantía <<$O$ grande, $o$ chica>> del 25 de agosto.
		\end{hint}
\end{enumerate}

\begin{additional}
\appendix
\section{Comentarios adicionales}
El ejercicio~\ref{ex:dirichlet_zeta} es un caso particular de la función dseta de Dirichlet que se define parecido, como suma formal con
ideales en un anillo $\mathcal{O}$ de enteros algebraicos.
En primer lugar, empleamos ideales para evitar repetición, similar a como en $\Z$ sumamos a $|n|^s = n^s$ y no a $\lvert-n\rvert^s = n^s$.
La segunda razón está en que si bien los elementos de $\mathcal{O}$ no satisfacen factorización única (por lo que no habría análogo del
producto de Euler), los ideales sí la satisfacen y, por tanto, la función dseta de Dirichlet siempre tiene un producto de Euler que ahora
recorre ideales primos.

El residuo en $s = 1$ es de sumo interés para teoristas de números ya que involucra varios invariantes del anillo $\mathcal{O}$;
con ello también quiero decir que el \textquote{$\pi$} no es casualidad.
Vid.\ \citeauthor{lang:algebraic}~\cite[259]{lang:algebraic} para más detalles.

El ejercicio~\ref{ex:bound_height} fue inspirado en el \textbf{teorema de Schanuel} que da fórmulas explícitas para el conteo de puntos
racionales de altura acotada en $\PP^n(\Q)$;
el lector puede leer más al respecto en \cite{hindry:diophantine}.

\printbibliography[title={Referencias y lecturas adicionales}]
\end{additional}

\end{document}
