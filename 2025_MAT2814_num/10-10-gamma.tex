\documentclass[11pt, reqno]{amsart}

\usepackage{../ayud-template}
\input{../general.tex}
\usepackage{tikz}
\usetikzlibrary{babel,cd}

% \usepackage[spanish]{babel}
\usepackage[LGR, T1]{fontenc}
\usepackage[utf8]{inputenc}

\input{../general.tex}
% \input{../graphics.tex}

\makeatletter
\def\emailaddrname{\textit{Correo electrónico}}
\def\subtitle#1{\gdef\@subtitle{#1}}
\def\@subtitle{}

% Metadata
\def\logo#1{\gdef\@logo{#1}}
\def\@logo{}
\def\institution#1{\gdef\@institution{#1}}
\def\@institution{}
\def\department#1{\gdef\@department{#1}}
\def\@department{}
\def\professor#1{\gdef\@professor{#1}}
\def\@professor{}
\def\course#1{\gdef\@course{#1}}
\def\@course{}
\def\coursecode#1{\gdef\@coursecode{#1}}
\def\@coursecode{}

\renewcommand{\maketitle}{
\begin{center}
	\small
	\renewcommand{\arraystretch}{1.2}
	\begin{tabular}{cp{.37\textwidth}p{0.44\textwidth}}
		% \hline
		\multirow{5}{*}{\includegraphics[height=2.0cm]{\@logo}}
	  & \multicolumn{2}{c}{ \makecell{{\bfseries \@institution} \\ \@department} } \\
	  % & \multicolumn{2}{c|}{{\bfseries\@institution} \\ \@department} \\
	  \cline{2-3}
	  & \textbf{Profesor:} \@professor & \textbf{Ayudante:} \authors \\
	  % \cline{2-3}
	  & \textbf{Curso:} \@course & \textbf{Sigla:} \@coursecode \\
	  % \cline{2-3}
	  & \multicolumn{2}{l}{ \textbf{Fecha:} \@date } \\
	  % \hline
	\end{tabular}
	\\[\baselineskip]
	% {}
	% \vspace{2\baselineskip}
	{\bfseries\Large\@title}
	\ifx\@subtitle\@empty\else
		\\[1ex]
		\large\mdseries\@subtitle
	\fi
\end{center}
}
\makeatother

\usepackage{multirow, makecell}

\usepackage[
	reversemp,
	letterpaper,
	% marginpar=2cm,
	% marginsep=1pt,
	margin=2.3cm
]{geometry}
\usepackage{fontawesome}
% \makeatletter
% \@reversemargintrue
% \makeatother

% Símbolos al margen, necesitan doble compilación
\newcommand{\hard}{\marginnote{\faFire}}
\newcommand{\hhard}{\marginnote{\faFire\faFire}}

% Dependencias para los teoremas
\usepackage{xifthen}
\def\@thmdep{}
\newcommand{\thmdep}[1]{
	\ifthenelse{\isempty{#1}}
	{\def\@thmdep{}}
	{\def\@thmdep{ (#1)}}
}
\newcommand{\thmstyle}{\color{thm}\sffamily\bfseries}

% ===== Estilos de Teoremas ==========
\newtheoremstyle{axiomstyle}
	{0.3cm}
	{0.3cm}
	{\normalfont}
	{0.5cm}
	{\bfseries\scshape}
	{:}
	{4pt}
	{\thmname{#1}\thmnote{ #3}\thmnumber{ (#2)}}
\newtheoremstyle{styleC}
	{0.5cm}
	{0.5cm}
	{\normalfont}
	{0.5cm}
	{\bfseries}
	{:}
	{4pt}
	{\thmname{#1\textrm{\@thmdep}}\thmnumber{ #2}\thmnote{ (#3)}}

% ====== Teoremas (sin borde) ===========
\theoremstyle{axiomstyle}
\newtheorem*{axiom}{Axioma}

% ====== Teoremas (sin borde) ==================
\theoremstyle{styleC}
\newtheorem{thm}{Teorema}[section]
\newtheorem{mydef}[thm]{Definición}
\newtheorem{prop}[thm]{Proposición}
\newtheorem{cor}[thm]{Corolario}
\newtheorem{lem}[thm]{Lema}
\newtheorem{con}[thm]{Conjetura}

\newtheorem*{prob}{Problema}
\newtheorem*{sol}{Solución}
\newtheorem*{obs}{Observación}
\newtheorem*{ex}{Ejemplo}

% \usepackage{tcolorbox}
% \newtcbox{bluebox}[1][]{enhanced jigsaw, 
%   sharp corners,
%   frame hidden,
%   nobeforeafter,
%   listing only,
%   #1} % comando para crear cajas de colores

\expandafter\let\expandafter\oldproof\csname\string\proof\endcsname
\let\oldendproof\endproof
\renewenvironment{proof}[1][\proofname]{%
  \oldproof[\scshape Demostración:]%
}{\oldendproof} % comando para redefinir la caja de la demostración
\newenvironment{hint}[1][\proofname]{%
  \oldproof[\scshape Pista:]%
}{\oldendproof} % comando para redefinir la caja de la demostración

% colores utilizados
\definecolor{numchap}{RGB}{249,133,29}
\definecolor{chap}{RGB}{6,129,204}
\definecolor{sec}{RGB}{204,0,0}
\definecolor{thm}{RGB}{106,176,240}
\definecolor{thmB}{RGB}{32,31,31}
\definecolor{part}{RGB}{212,66,66}

% ====== Diseño de los titulares ===============
\usepackage[explicit]{titlesec} % para personalizar el documento, la opción <<explicit>> hace que el texto de los titulares sea un objeto interactuable

\titleformat{\subsection}[runin]
	{\bfseries}
	{\textrm{\S}\thesubsection}
	{1ex}
	{#1.}

\setlist[enumerate,1]{label=\arabic*., ref=\arabic*} % Enumerate standards

% \includecomment{comment}

\title{Caracteres primitivos y la función Gamma}
\date{\DTMdate{2025-10-03}}

\author[José Cuevas]{José Cuevas Barrientos}
\email{josecuevasbtos@uc.cl}
\urladdr{https://josecuevas.xyz/teach/2025-2-num/}

\logo{../puc_negro.png}
\institution{Pontificia Universidad Católica de Chile}
\department{Facultad de Matemáticas}
\course{Teoría de Números}
\coursecode{MAT2814}
\professor{Ricardo Menares}

\begin{document}

\maketitle

\section{La función Gamma}
\begin{enumerate}
	\item
		\begin{enumerate}
			\item Pruebe que $E(z) := (1 - z)e^z$ satisface que
				\[
					\forall z \in \C : |z| < 1, \qquad
					|1 - E(z)| \le |z|^2.
				\]

			\item Pruebe que el siguiente producto infinito
				\[
					\prod_{n \ge 1} E(-z/n) = \prod_{n \ge 1} \left( 1 + \frac{z}{n} \right) e^{-z/n}
				\]
				define una función entera (i.e., holomorfa en todo $\C$).

				Para este inciso podría ser útil el siguiente criterio:
				\begin{thm}
					Un producto de funciones holomorfas $(f_n)_n \in \mathcal{H}(A)$ en un abierto conexo $A$ converge a
					una función holomorfa $F(z) = \prod_{n=1}^{\infty} f_n(z)$ si
					\[
						\sum_{n=1}^{\infty} \|1 - f(z)\|_K < \infty,
					\]
					para todo compacto $K \subseteq A$.
					(Aquí $\|\,\|_K$ denota la norma supremo en $K$.)
				\end{thm}

				\newex
			\item Defina la función de Weierstrass
				\[
					\Delta_\xi(z) := z e^{\xi z} \prod_{n \ge 1} \left( 1 + \frac{z}{n} \right) e^{-z/n},
				\]
				donde $\xi \in \C$ es una constante a elección.
				Pruebe que $\Delta_\xi(z + 1) = \frac{1}{z} \Delta_\xi(z)$ para exactamente un único número complejo $\xi$;
				tal valor es la \emph{constante de Euler-Mascheroni}
				\[
					\gamma = \lim_n \left( \sum_{k=1}^{n} \frac{1}{k} - \log n \right).
				\]

			\item Pruebe que $\Gamma(z) := 1/\Delta_\gamma(z)$ es una función meromorfa que tiene polos exclusivamente en los enteros negativos
				\[
					0, \qquad -1, \qquad -2, \qquad -3, \qquad \dots
				\]
				y sus polos son simples.
		\end{enumerate}
\end{enumerate}
\newex
Para el siguiente problema, será útil el siguiente criterio de unicidad:
\begin{thm}[Wielandt]
	Sea $f$ una función holomorfa en el semiplano derecho $\{ z \in \C : \Re z > 0 \}$ tal que $f(z + 1) = zf(z)$, entonces admite
	extensión meromorfa a todo $\C$ con polos posiblemente en los enteros negativos $\Z_{\le 0}$.
	% Más aún $u(z) := f(z)f(1 - z)$
	Si además, $f$ es acotada en la franja $\{ z \in \C : 1 \le \Re z \le 2 \}$, entonces $f(z) = f(1) \Gamma(z)$.
	(En particular, $\Gamma$ está acotada en dicha franja.)
\end{thm}
\begin{enumerate}[resume]
	\item \textbf{Fórmula de duplicación de Legendre:}
		Pruebe que la función $\Gamma$ satisface
		\[
			\Gamma(2z) \Gamma\left( \tfrac{1}{2} \right) = 2^{2z - 1} \Gamma(z) \Gamma\left( z + \tfrac{1}{2} \right).
		\]
		% \begin{hint}
		% 	Puede verificar que $\Gamma$ admite la siguiente fórmula
		% 	$ \Gamma(z) =  $
		% \end{hint}

		\newex
	\item\lookst
		\textbf{Aproximación de Stirling:}
		\begin{enumerate}
			\item Pruebe que
				\[
					\log(n!) = \left( n + \frac{1}{2} \right)\log n - n + 1 + \int_{1}^{n} \frac{P_1(t)}{t} \, \ud t,
				\]
				donde $P_1(t) := t - \lfloor t \rfloor + 1/2$ es la \textquote{función serrucho}.
				Esta función es 1-periódica y toma valores en $[-1/2, 1/2)$.

			\item Pruebe que
				\[
					\int_{1}^{n} \frac{P_1(t)}{t} \, \ud t = -\int_{1}^{n} \frac{P_1(t)^2}{t( \lfloor t \rfloor + 1/2 )}
					\, \ud t.
				\]

			\item Concluya que
				\[
					n! \sim \sqrt{2\pi} n^{n + 1/2} e^{-n}.
				\]
				\begin{hint}
					Defina
					\[
						C_n := \frac{n!}{n^{n + 1/2}e^{-n}}
					\]
					y pruebe que $C_\infty := \lim_n C_n$ existe y es un real estrictamente positivo.
					Luego considere el límite de $C_n^4/(C_{2n} C_{2n+1})$.
				\end{hint}
		\end{enumerate}
\end{enumerate}

\newex
\section{Caracteres}
\begin{enumerate}[resume]
	\item Sea $\chi \colon (\Z/m\Z)^\times \to \C^\times$ un caracter.
		Pruebe que existe un mínimo entero $f \ge 0$ tal que $f \mid m$ y existe un caracter $\bar\chi$ de modo que el siguiente diagrama conmuta
		\[\begin{tikzcd}[column sep=small]
			(\Z/m\Z)^\times \ar[rr, "\chi"] \drar[two heads] & & \C^\times \\
			{} & (\Z/f\Z)^\times \urar["\bar\chi"']
		\end{tikzcd}\]
		Dicho $f$ se conoce como el \strong{conductor} de $\chi$.
		Se dice que $\chi$ es \strong{primitivo} si $f = m$.

	\item Pruebe que si $\chi \colon (\Z/f\Z)^\times \to \C^\times$ es primitivo y $f \mid m$, entonces el caracter $\chi^* := \rho
		\circ \chi \colon (\Z/m\Z)^\times \to \C^\times$ satisface
		\[
			L(\chi^*, s) = L(\chi, s) \prod_{p \mid m} (1 - p^{-s}).
		\]
\end{enumerate}

\begin{additional}
\appendix
\section{Comentarios adicionales}
La función $\Gamma$ tiene una larga y fascinante historia, sus propiedades fueron estudiados por varios de los matemáticos más importantes
incluyendo (pero no limitado) a L.~Euler, C.F.~Gauss, K.~Weierstrass y A.-M.~Legendre.

Hay varios resultados que apuntan a la naturalidad y/o unicidad de la función $\Gamma$, incluyendo:
\begin{thm}[Bohr-Mollerup]
	Si $f \colon (0, \infty) \to (0, \infty)$ es una función continua tal que:
	\begin{enumerate}[(a)]
		\item $f(z + 1) = zf(z)$.
		\item Es log-convexa, es decir,
			\[
				\forall x, y \in (0, \infty), \; t \in [0, 1], \qquad
				f(tx + (1 - t)y) \le f(x)^t \cdot f(y)^{1-t}.
				% \log f(tx + (1 - t)x) \le t\log f(x) + (1-t)\log f(y).
			\]
	\end{enumerate}
	Entonces $f(z) = f(1) \Gamma(z)$.
\end{thm}
\begin{thm}
	Sea $f$ una función meromorfa en $\C$, que manda $(0, \infty) \to (0, \infty)$ y tal que
	\[
		f(z + 1) = zf(z), \qquad \sqrt{\pi} f(2z) = 2^{2z - 1}f(z) f\left( z + \tfrac{1}{2} \right).
	\]
	Entonces $f(z) = \Gamma(z)$.
\end{thm}
Puede leer pruebas de estos datos en \cite{remmert:classical}, \S 2.2.

El tratamiento de la función $\Gamma$ es sumamente clásico.
Puede leer al respecto en \cite{simon:complex}, \cite{lang:complex} y en el conciso libro de \citeauthor{artin:gamma}~\cite{artin:gamma}.
% \nocite{washington:cyclotomic}

Los caracteres no primitivos (y sus funciones $L$) tienen usos en la teoría de números.
Debido a que el factor de corrección es sumamente sencillo, podemos calcular residuos y otros invariantes en caracteres no primitivos.
Vea \citeauthor{washington:cyclotomic}~\cite{washington:cyclotomic}.

\printbibliography[title={Referencias y lecturas adicionales}]
\end{additional}

\end{document}
