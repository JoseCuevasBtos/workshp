\documentclass[11pt, reqno]{amsart}

\usepackage{../ayud-template}
\input{../general.tex}
% \usepackage[spanish]{babel}
\usepackage[LGR, T1]{fontenc}
\usepackage[utf8]{inputenc}

\input{../general.tex}
% \input{../graphics.tex}

\makeatletter
\def\emailaddrname{\textit{Correo electrónico}}
\def\subtitle#1{\gdef\@subtitle{#1}}
\def\@subtitle{}

% Metadata
\def\logo#1{\gdef\@logo{#1}}
\def\@logo{}
\def\institution#1{\gdef\@institution{#1}}
\def\@institution{}
\def\department#1{\gdef\@department{#1}}
\def\@department{}
\def\professor#1{\gdef\@professor{#1}}
\def\@professor{}
\def\course#1{\gdef\@course{#1}}
\def\@course{}
\def\coursecode#1{\gdef\@coursecode{#1}}
\def\@coursecode{}

\renewcommand{\maketitle}{
\begin{center}
	\small
	\renewcommand{\arraystretch}{1.2}
	\begin{tabular}{cp{.37\textwidth}p{0.44\textwidth}}
		% \hline
		\multirow{5}{*}{\includegraphics[height=2.0cm]{\@logo}}
	  & \multicolumn{2}{c}{ \makecell{{\bfseries \@institution} \\ \@department} } \\
	  % & \multicolumn{2}{c|}{{\bfseries\@institution} \\ \@department} \\
	  \cline{2-3}
	  & \textbf{Profesor:} \@professor & \textbf{Ayudante:} \authors \\
	  % \cline{2-3}
	  & \textbf{Curso:} \@course & \textbf{Sigla:} \@coursecode \\
	  % \cline{2-3}
	  & \multicolumn{2}{l}{ \textbf{Fecha:} \@date } \\
	  % \hline
	\end{tabular}
	\\[\baselineskip]
	% {}
	% \vspace{2\baselineskip}
	{\bfseries\Large\@title}
	\ifx\@subtitle\@empty\else
		\\[1ex]
		\large\mdseries\@subtitle
	\fi
\end{center}
}
\makeatother

\usepackage{multirow, makecell}

\usepackage[
	reversemp,
	letterpaper,
	% marginpar=2cm,
	% marginsep=1pt,
	margin=2.3cm
]{geometry}
\usepackage{fontawesome}
% \makeatletter
% \@reversemargintrue
% \makeatother

% Símbolos al margen, necesitan doble compilación
\newcommand{\hard}{\marginnote{\faFire}}
\newcommand{\hhard}{\marginnote{\faFire\faFire}}

% Dependencias para los teoremas
\usepackage{xifthen}
\def\@thmdep{}
\newcommand{\thmdep}[1]{
	\ifthenelse{\isempty{#1}}
	{\def\@thmdep{}}
	{\def\@thmdep{ (#1)}}
}
\newcommand{\thmstyle}{\color{thm}\sffamily\bfseries}

% ===== Estilos de Teoremas ==========
\newtheoremstyle{axiomstyle}
	{0.3cm}
	{0.3cm}
	{\normalfont}
	{0.5cm}
	{\bfseries\scshape}
	{:}
	{4pt}
	{\thmname{#1}\thmnote{ #3}\thmnumber{ (#2)}}
\newtheoremstyle{styleC}
	{0.5cm}
	{0.5cm}
	{\normalfont}
	{0.5cm}
	{\bfseries}
	{:}
	{4pt}
	{\thmname{#1\textrm{\@thmdep}}\thmnumber{ #2}\thmnote{ (#3)}}

% ====== Teoremas (sin borde) ===========
\theoremstyle{axiomstyle}
\newtheorem*{axiom}{Axioma}

% ====== Teoremas (sin borde) ==================
\theoremstyle{styleC}
\newtheorem{thm}{Teorema}[section]
\newtheorem{mydef}[thm]{Definición}
\newtheorem{prop}[thm]{Proposición}
\newtheorem{cor}[thm]{Corolario}
\newtheorem{lem}[thm]{Lema}
\newtheorem{con}[thm]{Conjetura}

\newtheorem*{prob}{Problema}
\newtheorem*{sol}{Solución}
\newtheorem*{obs}{Observación}
\newtheorem*{ex}{Ejemplo}

% \usepackage{tcolorbox}
% \newtcbox{bluebox}[1][]{enhanced jigsaw, 
%   sharp corners,
%   frame hidden,
%   nobeforeafter,
%   listing only,
%   #1} % comando para crear cajas de colores

\expandafter\let\expandafter\oldproof\csname\string\proof\endcsname
\let\oldendproof\endproof
\renewenvironment{proof}[1][\proofname]{%
  \oldproof[\scshape Demostración:]%
}{\oldendproof} % comando para redefinir la caja de la demostración
\newenvironment{hint}[1][\proofname]{%
  \oldproof[\scshape Pista:]%
}{\oldendproof} % comando para redefinir la caja de la demostración

% colores utilizados
\definecolor{numchap}{RGB}{249,133,29}
\definecolor{chap}{RGB}{6,129,204}
\definecolor{sec}{RGB}{204,0,0}
\definecolor{thm}{RGB}{106,176,240}
\definecolor{thmB}{RGB}{32,31,31}
\definecolor{part}{RGB}{212,66,66}

% ====== Diseño de los titulares ===============
\usepackage[explicit]{titlesec} % para personalizar el documento, la opción <<explicit>> hace que el texto de los titulares sea un objeto interactuable

\titleformat{\subsection}[runin]
	{\bfseries}
	{\textrm{\S}\thesubsection}
	{1ex}
	{#1.}

\setlist[enumerate,1]{label=\arabic*., ref=\arabic*} % Enumerate standards

% \includecomment{comment}

\usepackage{xcolor}
\definecolor{niceblue}{HTML}{137DBB}
\definecolor{nicered}{RGB}{255,77,0}

% ======= TikZ ======================
\usepackage{tikz, pgfplots}
\usetikzlibrary{angles, arrows, babel, calc, decorations.pathmorphing, decorations.markings, patterns, matrix, cd, spy}

\title{Función dseta}
\date{\DTMdate{2025-10-17}}

\author[José Cuevas]{José Cuevas Barrientos}
\email{josecuevasbtos@uc.cl}
\urladdr{https://josecuevas.xyz/teach/2025-2-num/}

\logo{../puc_negro.png}
\institution{Pontificia Universidad Católica de Chile}
\department{Facultad de Matemáticas}
\course{Teoría de Números}
\coursecode{MAT2814}
\professor{Ricardo Menares}

\begin{document}

\maketitle

\section{Ejercicios}
\begin{center}
	\slshape
	Para esta ayudantía convendrá recordar hechos conocidos de la función $\Gamma$, como su ecuación funcional $\Gamma(z + 1) = z \Gamma(z)$ y que $\Gamma(n + 1) = n!$.
\end{center}
\begin{enumerate}
	\item Pruebe que los residuos de la función $\Gamma$ en $\Z_{\le 0}$ son
		\[
			\Res_{z=-n} \Gamma(z) := \lim_{w \to -n} (w + n)\Gamma(w) = \frac{(-1)^n}{n!}.
		\]
		% \begin{hint}
		% 	Recuerde que calculamos que $\Gamma(1) = 1$.
		% \end{hint}

	\item Sean $u, c > 0$ números reales y $k \in \N_{\ne 0}$ un número natural.
		\begin{enumerate}
			\item Pruebe que
				\[
					\frac{u^{-z} \Gamma(z)}{\Gamma(z + k + 1)} = \frac{u^{-z}}{z(z + 1) \cdots (z + k)}.
				\]

			\item Concluya que, si $u \le 1$, para un radio real $R > k$ se cumple que
				\[
					\frac{1}{2\pi\ui}\oint_{|z|=R} \frac{u^{-z}}{z(z + 1) \cdots (z + k)} \, \ud z = \dfrac{(1 -
					u)^k}{k!}.
				\]
				\begin{hint}
					Conviene emplear la fórmula de residuos de Cauchy:
					\begin{equation}
						\frac{1}{2\pi\ui}\oint_{|z|=R} f(z) \, \ud z = \sum_{|a|<R} \Res(f; a).
						\tqedhere
					\end{equation}
				\end{hint}

			\item Pruebe que, si $R > 2k$, entonces
				\[
					\left\lvert\frac{u^{-z}}{z(z + 1)\cdots(z + k)}\right\rvert
					\le \frac{u^{-c}}{R(R/2)^k}
				\]
				para todo $z$ con $|z| = R$ y con $\Re z \ge c$ (resp.\ $\Re z < c$) si $u > 1$ (resp.\ $u \le 1$).

			\item Concluya que
				\[
					\frac{1}{2\pi\ui} \int_{c - \infty\ui}^{c + \infty\ui} \frac{u^{-z}}{z(z+1)\cdots(z+k)} \, \ud z =
					\begin{cases}
						\dfrac{(1 - u)^k}{k!}, & 0 < u \le 1, \\
						0, & u > 1.
					\end{cases}
				\]
				\begin{hint}
					Emplee el hecho de que conoce que el lado derecho son los residuos de la función y descomponga el
					contorno en una parte rectilínea como el dominio que quiere integrar, y otra parte en donde la
					integral converge a 0 cuando $R \to \infty$.
					Ojo que cuando $u > 1$ conviene tomar una región que evite a los polos en $0, -1, \dots, -k$.
				\end{hint}
		\end{enumerate}

	\item Sea $s \in \C$.
		\begin{enumerate}
			\item Pruebe que, para $\Re s > 0$, se tiene
				\[
					\zeta(s) = \sum_{n=1}^{N} \frac{1}{n^s} - s \int_{N}^{\infty} \frac{x - \lfloor x \rfloor}{x^{s+1}} \, \ud
					x + \frac{N^{1-s}}{s - 1}.
				\]
				\begin{hint}
					Para $\Re s > 1$ se sigue de despejar la integral separándola en los intervalos $[N, N + 1]$.
					Luego argumente por qué la fórmula sigue siendo válida para $\Re s > 0$.
				\end{hint}

			\item Pruebe que, para $\Re s > 0$, se tiene que
				\[
					\zeta'(s) = \sum_{n=1}^{N} \frac{\log n}{n^s} + s \int_{N}^{\infty} \frac{(x - \lfloor x \rfloor)\log
					x)}{x^{s+1}} \, \ud x - \int_{N}^{\infty} \frac{x - \lfloor x \rfloor}{x^{s+1}} \, \ud x
					- \frac{N^{1-s}\log N}{s - 1} - \frac{N^{1-s}}{(s-1)^2}.
				\]

			\item Pruebe que para todo $s \in \C$ con $\Im s =: t \ge e$ y todo $A > 0$, existe una constante $M := M(A) > 0$ tal que
				\[
					|\zeta(s)| \le M\log t, \qquad |\zeta'(s)| \le M\log^2 t,
				\]
				para todo $s$ con
				\[
					\Re s > \max\mathopen{}\left\{ \frac{1}{2}, 1 - \frac{A}{\log t} \right\}\mathclose{}.
				\]
				\begin{hint}
					Será conveniente tomar $N = \lfloor t \rfloor$ en las fórmulas previas.
				\end{hint}
		\end{enumerate}
		\nocite{apostol:analytic}

	\item Sea $s \in \C$ con $\Re s > 0$.
		\begin{enumerate}
			\item Pruebe que tenemos la fórmula
				\[
					\Gamma(s/2) n^{-s} \pi^{-s/2} = \int_{0}^{\infty} x^{\frac{1}{2} s - 1} e^{-n^2 \pi x} \, \ud x.
				\]

			\item Defina
				\[
					\omega(x) := \sum_{n=1}^{\infty} e^{-n^2 \pi x}, \qquad \theta(x) := \sum_{n=-\infty}^{\infty}
					e^{-n^2 \pi x}.
				\]
				Pruebe que $2 \omega(x) = \theta(x) - 1$ y que
				\[
					\pi^{-s/2}\Gamma\left( \tfrac{1}{2}s \right) \zeta(s) = \int_{0}^{\infty} x^{\frac{1}{2}s - 1}
					\omega(x) \, \ud x.
				\]

			% \item Concluya que $2\omega(x) = \theta(x) - 1$.
		\end{enumerate}
		\nocite{davenport:multiplicative}
\end{enumerate}

\begin{additional}
\printbibliography

\newpage
\section{Soluciones}
\begin{enumerate}
	\item Basta proceder por inducción, donde el caso base viene de que
		\[
			\lim_{z\to 0} z\Gamma(z) = \lim_{z\to 0} \Gamma(z + 1) = 1.
		\]
		Así, en general
		\begin{align*}
			\lim_{z \to -n-1} (z + n+1)\Gamma(z) &= \lim_{z\to -n-1} \frac{z + n + 1}{z} \Gamma(z + 1) \\
							     &= \lim_{w\to -n} (w + n)\Gamma(w) \frac{1}{w - 1}
							     = \frac{(-1)^n}{n!} \frac{1}{(-n-1)} = \frac{(-1)^{n+1}}{(n+1)!}.
		\end{align*}

	\item
		\begin{enumerate}
			\item Basta notar que
				\[
					\Gamma(z + k + 1) = (z + k) \Gamma(z + k) = (z + k)(z + k - 1) \cdots z \Gamma(z),
				\]
				y luego despejar.

			\item El teorema de residuos de Cauchy dice que a la derecha aparecen los residuos de $f$ que suceden necesariamente
				en los polos.
				La función
				\[
					\frac{u^{-z}}{z(z + 1)\cdots(z + k)}
				\]
				solo tiene polos en $z \in \{ 0, -1, \dots, -k \}$ y, por el inciso anterior, tenemos luego que
				\begin{align*}
					\frac{1}{2\pi\ui}\oint_{|z|=R} \frac{u^{-z}}{z(z + 1)\cdots(z + k)} \, \ud z
					&= \sum_{n=0}^{k} \Res_{z=-n} \frac{u^{-z}\Gamma(z)}{\Gamma(z + k + 1)}, \\
					\intertext{pero $u^{-z}/\Gamma(k + 1 - n)$ no tienen polo en $z \to -n$, así que una álgebra de límites dice
					que tenemos}
					&= \sum_{n=0}^{k} \frac{u^n}{\Gamma(k + 1 - n)} \Res_{z=-n}\Gamma(z), \\
					&= \sum_{n=0}^{k} \frac{u^n}{(k - n)!} \frac{(-1)^n}{n!} = \frac{1}{k!}
					\sum_{n=0}^{k}\binom{k}{n} (-u)^n = \frac{(1 - u)^k}{k!}.
				\end{align*}

			\item Basta notar que $|u^z| = u^{\Re z}|e^{\log(u)\Im(z)\ui}| = u^{\Re z}$ y que $x \mapsto (1/u)^x$ es creciente
				si $u < 1$ y decreciente si $u > 1$;
				de modo que $|u^{-z}| \le u^{-c}$ si $u > 1$ (resp.\ $u \le 1$) y $\Re z \ge c$ (resp.\ $\Re z \le c$).

				Ahora bien, como $|z| = R$, tenemos que
				\[
					|z + n| \ge |z| - n = R - n > R - k > R - \frac{1}{2}R = \frac{1}{2}R.
				\]
				Esto prueba que
				\[
					\left\lvert\frac{1}{z(z + 1)\cdots(z + k)}\right\rvert \le \frac{1}{R(R/2)^k},
				\]
				y lo juntamos con la cota anterior.

			\item Siguiendo la pista, cuando $u \le 1$ podemos considerar los dos caminos dados por la fig.~\ref{fig:apostol_path}.

				\begin{figure}[!hbtp]
					\centering
						\begin{tikzpicture}
							\draw[->] (-3, 0) -- (3, 0);
							\draw[->] (0, -3) -- (0, 3);

							\fill ({2*cos(75)}, 0) circle (.05) node[below left]{$c$};
							\draw[dashed] (0, 0) circle (2);
							\begin{scope}[thick, decoration={
									markings,
									mark=at position 0.33 with {\arrow{stealth}},
									mark=at position 0.67 with {\arrow{stealth}}
								}]
								\draw[nicered,  postaction={decorate}] (-75:2) -- node[above right]{$\beta$} (75:2);
								\draw[niceblue, postaction={decorate}] (75:2) arc (75:285:2);
							\end{scope}
							\draw[niceblue] (160:2) node[above left]{$\alpha$};
						\end{tikzpicture}
					\caption{}
					\label{fig:apostol_path}
				\end{figure}

				La unión de ambos tiene los residuos que ya calculamos y cuando $R \to \infty$, el camino $\beta$ se acerca
				a la vertical $c + \R\ui$, de modo que
				\[
					\lim_{R \to \infty} \int_{\beta(R)} \frac{u^{-z}}{z(z+1)\cdots(z+k)} \, \ud z = \int_{c -
					\infty\ui}^{c + \infty\ui} \frac{u^{-z}}{z(z+1)\cdots(z+k)} \, \ud z.
				\]
				Así, basta ver que $\lim_{R \to \infty} \int_{\alpha(R)} f(z) \, \ud z = 0$.
				Para ello, empleamos las cotas del inciso anterior
				\[
					\left\lvert \int_{\alpha(R)} \frac{u^{-z}}{z(z+1)\cdots(z+k)} \, \ud z \right\rvert
					\le \int_{\alpha(R)} \frac{u^{-c}}{R(R/2)^k} \, \ud z
					\le (2 \pi R) \frac{u^{-c}}{R(R/2)^k} \ll R^{-k} \to 0.
				\]

				En cambio, si $u > 1$, escogemos los caminos dados por la fig.~\ref{fig:apostol_path_b} y vemos que
				exactamente el mismo análisis aplica.

				\begin{figure}[!hbtp]
					\centering
					\begin{tikzpicture}
						\draw[->] (-3, 0) -- (3, 0);
						\draw[->] (0, -3) -- (0, 3);

						\fill ({2*cos(75)}, 0) circle (.05) node[below left]{$c$};
						\draw[dashed] (0, 0) circle (2);
						\begin{scope}[thick, decoration={
								markings,
								mark=at position 0.33 with {\arrow{stealth}},
								mark=at position 0.67 with {\arrow{stealth}}
							}]
							\draw[nicered,  postaction={decorate}] (-75:2) -- node[above right]{$\beta$} (75:2);
							\draw[niceblue, postaction={decorate}] (75:2) arc (75:-75:2);
						\end{scope}
						\draw[niceblue] (-40:2) node[below right]{$\alpha$};
					\end{tikzpicture}
					\caption{}
					\label{fig:apostol_path_b}
				\end{figure}
		\end{enumerate}

	\item
		\begin{enumerate}
			\item Como señala la pista, basta evaluar
				\begin{align*}
					\int_{N}^{N+1} \frac{x - \lfloor x \rfloor}{x^{s+1}} \, \ud x
					&= \int_{N}^{N+1} x^{-s} - Nx^{-s-1} \, \ud x \\
					&= \frac{1}{(1-s)}((N+1)^{1-s} - N^{1-s}) + \frac{N}{s}((N+1)^{-s} - N^{-s}) \\
					&= N^{1-s}\left( \frac{1}{s - 1} - \frac{1}{s} \right) + (N+1)^{1-s}\left( \frac{N}{(N+1)s} - \frac{1}{s - 1} \right),
				\end{align*}
				de modo que
				\[
					-s\int_{N}^{N+1} \frac{x - \lfloor x \rfloor}{x^{s+1}} \, \ud x
					= \frac{N^{1-s}}{1 - s} - \frac{(N+1)^{1-s}}{1 - s} + (N + 1)^{-s}.
				\]
				Así
				\[
					\sum_{n=1}^{N} n^{-s} -s\int_{N}^{N+1} \frac{x - \lfloor x \rfloor}{x^{s+1}} \, \ud x + \frac{N^{1-s}}{s - 1}
					= \sum_{n=1}^{N+1} n^{-s} + \frac{(N+1)^{1-s}}{s - 1},
				\]
				y por inducción obtenemos la fórmula
				\[
					\sum_{n=1}^{N} n^{-s} -s\int_{N}^{N+k} \frac{x - \lfloor x \rfloor}{x^{s+1}} \, \ud x + \frac{N^{1-s}}{s - 1}
					= \sum_{n=1}^{N+k} n^{-s} + \frac{(N+k)^{1-s}}{s - 1},
				\]
				y con $k \to \infty$ se alcanza igualdad \underline{para $\Re s > 1$.} 

				Basta notar que la fórmula obtenida determina una función analítica para $\Re s > 0$, de modo que la
				igualdad sigue siendo válida debido a la unicidad de la continuación analítica de una función.

			\item Basta tomar la fórmula anterior y derivarla, recordando que
				\[
					\frac{\ud}{\ud s} \int_{N}^{\infty} \frac{x - \lfloor x \rfloor}{x^{s+1}} \, \ud x =
					-\int_{N}^{\infty} \frac{(x - \lfloor x \rfloor)\log x}{x^{s+1}} \, \ud x.
				\]

			\item Para $s$ con $\sigma := \Re s \ge 2$, vemos que
				\[
					|\zeta(s)| \le \zeta(2), \qquad |\zeta'(s)| \le |\zeta'(2)|.
				\]
				Así, suponemos que $|\sigma| < 2$ y, como $t > 2$, tenemos
				\[
					|s| \le \sigma + t \le 2 + t \le 2t, \qquad |s - 1| \ge t.
				\]
				Luego, aplicando la fórmula del inciso anterior, tenemos
				\[
					|\zeta(s)|
					\le \sum_{n=1}^{N} n^{-\sigma} + 2t \int_{N}^{\infty} \frac{1}{x^{\sigma+1}} \, \ud x + \frac{N^{1 - \sigma}}{t}
					\le \sum_{n=1}^{N} n^{-\sigma} + \frac{2t}{\sigma N^\sigma} + \frac{N^{1 - \sigma}}{t}.
				\]
				Finalmente,
				\[
					\frac{1}{n^\sigma} = \frac{n^{1 - \sigma}}{n} = \frac{1}{n}e^{(1 - \sigma)\log n} = \frac{1}{n} e^{A
					\log n/\log t} \ll_A \frac{1}{n}.
				\]
				Hacemos ahora la elección de $N = \lfloor t \rfloor$ y obtenemos
				\[
					|\zeta(s)|
					\ll \sum_{n=1}^{N} \frac{1}{n} + \frac{t}{N\sigma} + \frac{N}{t}
					\ll \log t + \frac{1}{\sigma} + 1 \ll \log t,
				\]
				pues $\sigma \ge 1/2$.

				Similarmente con $\zeta'$ el mismo procedimiento nos dice que el único término importante será la serie del inicio.
		\end{enumerate}

	\item
		\begin{enumerate}
			\item Basta notar que tenemos la siguiente expresión para $\Gamma$ cuando $\Re s > 0$:
				\[
					\Gamma(s) = \int_{0}^{\infty} t^{s - 1}e^{-t} \, \ud t.
				\]
				(Esta fórmula es bien conocida.
				Si desea probar que coincide con la que dimos en la ayudantía pasada puede aplicar directamente el criterio de Wielandt,
				empleando que la ecuación funcional se sigue de integración por partes:
				\[
					\Gamma(s + 1) = \int_{0}^{\infty} t^s e^{-t} \, \ud t = \big[ -t^se^{-t} \big]_{t=0}^{t=\infty} + \int_{0}^{\infty} st^{s-1} e^{-t} \, \ud t
					= s \Gamma(s).)
				\]

				Luego aplicamos el cambio de variables $t = n^2 \pi x$ con $\ud t = n^2 \pi \, \ud x$ y evaluamos en $s/2$:
				\[
					\Gamma(s/2) = \int_{0}^{\infty} \pi^{\frac{1}{2}s} n^{s} x^{ \frac{1}{2}s - 1 }e^{-n^2 \pi x} \, \ud x.
				\]

			\item En efecto, que $\theta(x) = 1 + 2 \omega(x)$ se sigue de que $(-n)^2 = n^2$ y que $1 = e^0$.
				La igualdad ahora es inmediata de sumar las anteriores
				\[
					\pi^{-s/2} \Gamma( \tfrac{1}{2}s ) \zeta(s) = \sum_{n=1}^{\infty} \pi^{-s/2} \Gamma( \tfrac{1}{2}s ) n^{-s}
					= \int_{0}^{\infty} x^{ \frac{1}{2}s - 1 } \left( \sum_{n=1}^{\infty} e^{-n^2 \pi x} \right) \, \ud x.
				\]
				Como es costumbre, el intercambio suma-integral es una aplicación directa del teorema de Tonelli.
		\end{enumerate}
\end{enumerate}
\end{additional}

\end{document}
