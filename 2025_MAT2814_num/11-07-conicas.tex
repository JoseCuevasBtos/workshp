\documentclass[11pt, reqno]{amsart}

% \usepackage[spanish]{babel}
\usepackage[LGR, T1]{fontenc}
\usepackage[utf8]{inputenc}

\input{../general.tex}
% \input{../graphics.tex}

\makeatletter
\def\emailaddrname{\textit{Correo electrónico}}
\def\subtitle#1{\gdef\@subtitle{#1}}
\def\@subtitle{}

% Metadata
\def\logo#1{\gdef\@logo{#1}}
\def\@logo{}
\def\institution#1{\gdef\@institution{#1}}
\def\@institution{}
\def\department#1{\gdef\@department{#1}}
\def\@department{}
\def\professor#1{\gdef\@professor{#1}}
\def\@professor{}
\def\course#1{\gdef\@course{#1}}
\def\@course{}
\def\coursecode#1{\gdef\@coursecode{#1}}
\def\@coursecode{}

\renewcommand{\maketitle}{
\begin{center}
	\small
	\renewcommand{\arraystretch}{1.2}
	\begin{tabular}{cp{.37\textwidth}p{0.44\textwidth}}
		% \hline
		\multirow{5}{*}{\includegraphics[height=2.0cm]{\@logo}}
	  & \multicolumn{2}{c}{ \makecell{{\bfseries \@institution} \\ \@department} } \\
	  % & \multicolumn{2}{c|}{{\bfseries\@institution} \\ \@department} \\
	  \cline{2-3}
	  & \textbf{Profesor:} \@professor & \textbf{Ayudante:} \authors \\
	  % \cline{2-3}
	  & \textbf{Curso:} \@course & \textbf{Sigla:} \@coursecode \\
	  % \cline{2-3}
	  & \multicolumn{2}{l}{ \textbf{Fecha:} \@date } \\
	  % \hline
	\end{tabular}
	\\[\baselineskip]
	% {}
	% \vspace{2\baselineskip}
	{\bfseries\Large\@title}
	\ifx\@subtitle\@empty\else
		\\[1ex]
		\large\mdseries\@subtitle
	\fi
\end{center}
}
\makeatother

\usepackage{multirow, makecell}

\usepackage[
	reversemp,
	letterpaper,
	% marginpar=2cm,
	% marginsep=1pt,
	margin=2.3cm
]{geometry}
\usepackage{fontawesome}
% \makeatletter
% \@reversemargintrue
% \makeatother

% Símbolos al margen, necesitan doble compilación
\newcommand{\hard}{\marginnote{\faFire}}
\newcommand{\hhard}{\marginnote{\faFire\faFire}}

% Dependencias para los teoremas
\usepackage{xifthen}
\def\@thmdep{}
\newcommand{\thmdep}[1]{
	\ifthenelse{\isempty{#1}}
	{\def\@thmdep{}}
	{\def\@thmdep{ (#1)}}
}
\newcommand{\thmstyle}{\color{thm}\sffamily\bfseries}

% ===== Estilos de Teoremas ==========
\newtheoremstyle{axiomstyle}
	{0.3cm}
	{0.3cm}
	{\normalfont}
	{0.5cm}
	{\bfseries\scshape}
	{:}
	{4pt}
	{\thmname{#1}\thmnote{ #3}\thmnumber{ (#2)}}
\newtheoremstyle{styleC}
	{0.5cm}
	{0.5cm}
	{\normalfont}
	{0.5cm}
	{\bfseries}
	{:}
	{4pt}
	{\thmname{#1\textrm{\@thmdep}}\thmnumber{ #2}\thmnote{ (#3)}}

% ====== Teoremas (sin borde) ===========
\theoremstyle{axiomstyle}
\newtheorem*{axiom}{Axioma}

% ====== Teoremas (sin borde) ==================
\theoremstyle{styleC}
\newtheorem{thm}{Teorema}[section]
\newtheorem{mydef}[thm]{Definición}
\newtheorem{prop}[thm]{Proposición}
\newtheorem{cor}[thm]{Corolario}
\newtheorem{lem}[thm]{Lema}
\newtheorem{con}[thm]{Conjetura}

\newtheorem*{prob}{Problema}
\newtheorem*{sol}{Solución}
\newtheorem*{obs}{Observación}
\newtheorem*{ex}{Ejemplo}

% \usepackage{tcolorbox}
% \newtcbox{bluebox}[1][]{enhanced jigsaw, 
%   sharp corners,
%   frame hidden,
%   nobeforeafter,
%   listing only,
%   #1} % comando para crear cajas de colores

\expandafter\let\expandafter\oldproof\csname\string\proof\endcsname
\let\oldendproof\endproof
\renewenvironment{proof}[1][\proofname]{%
  \oldproof[\scshape Demostración:]%
}{\oldendproof} % comando para redefinir la caja de la demostración
\newenvironment{hint}[1][\proofname]{%
  \oldproof[\scshape Pista:]%
}{\oldendproof} % comando para redefinir la caja de la demostración

% colores utilizados
\definecolor{numchap}{RGB}{249,133,29}
\definecolor{chap}{RGB}{6,129,204}
\definecolor{sec}{RGB}{204,0,0}
\definecolor{thm}{RGB}{106,176,240}
\definecolor{thmB}{RGB}{32,31,31}
\definecolor{part}{RGB}{212,66,66}

% ====== Diseño de los titulares ===============
\usepackage[explicit]{titlesec} % para personalizar el documento, la opción <<explicit>> hace que el texto de los titulares sea un objeto interactuable

\titleformat{\subsection}[runin]
	{\bfseries}
	{\textrm{\S}\thesubsection}
	{1ex}
	{#1.}

\setlist[enumerate,1]{label=\arabic*., ref=\arabic*} % Enumerate standards

\usepackage{../ayud-template}
\input{../general.tex}

\usepackage{tikz}
\usetikzlibrary{babel,cd}

\title{Curvas y cónicas}
\date{\DTMdate{2025-11-07}}

\author[José Cuevas]{José Cuevas Barrientos}
\email{josecuevasbtos@uc.cl}

\logo{../puc_negro.png}
\institution{Pontificia Universidad Católica de Chile}
\department{Facultad de Matemáticas}
\course{Teoría de Números}
\coursecode{MAT2814}
\professor{Ricardo Menares}

\begin{document}

\maketitle

\section{Ejercicios}
\begin{center}
	\slshape
	A lo largo de esta ayudantía, $k$ denotará un cuerpo.
\end{center}
\begin{enumerate}
	\item Encuentre los \textquote{puntos al infinito} de las siguientes curvas afines:
		\begin{enumerate}
			\item $3x - 7y + 5 = 0$.
			\item $x^2 + xy - 2y^2 + x - 5y + 7 = 0$.
			\item $x^3 + x^2y - 3xy^2 - 3y^3 + 2x^2 - 2y + 5 = 0$.
		\end{enumerate}

	\item Para las siguientes curvas encuentre la tangente al punto $P$ (o, de no existir, declare si $P$ es singular):
		\begin{enumerate}
			\item $y^2 = x^3 - x$ en $P = (1, 0)$.
			\item $X^2 + Y^2 = Z^2$ en $P = [3 : 4 : 5]$.
			\item $x^2 + y^4 + 2xy + 2x + 2y + 1$ en $P = (-1, 0)$.
		\end{enumerate}

	\item Sea $C'$ la cónica proyectiva dada por la ecuación:
		\begin{equation}
			aX^2 + bXY + cY^2 + dXZ + eYZ + fZ^2 = 0.
		\end{equation}
		\begin{enumerate}
			\item Pruebe que $C'$ es suave
				% (o \emph{no singular})
				si el determinante
				\[
					\delta := \det
					\begin{bmatrix}
						2a & b & d \\
						b & 2c & e \\
						d & e & 2f
					\end{bmatrix}.
				\]
				es no nulo.

			\item\lookst
				Pruebe que si $C'$ es suave y tiene un punto (racional) $[x_0: y_0 : z_0] \in C(k)$,
				entonces $C' \cong \PP^1(k)$.
				\begin{hint}
					Hay una biyección entre $\PP^1(k) = k \cup \{ \infty \}$ y las pendientes de rectas racionales.
				\end{hint}
		\end{enumerate}

	\item\lookright
		Describa \emph{todos} los puntos $\Q$-racionales de $x^2 + y^2 = 2$ parándose en $(1, 1)$.

	\item Definamos el círculo afín $C(k) := \{ (x, y) \in k : x^2 + y^2 = 1 \}$.
		Para dos puntos en $C(k)$ definamos
		\[
			(x_1, y_1) \boxplus (x_2, y_2) := (x_1x_2 - y_1y_2, x_1y_2 + x_2y_1).
		\]
		Pruebe que $(C(k), \boxplus)$ determina un grupo abeliano.

	\item\label{ex:circle_group}
		\begin{enumerate}
			\item Pruebe que si $\car k \ne 2$ y existe $\sqrt{-1} \in k$, entonces $(C(k), \boxplus) \cong (k^\times, \cdot)$.
			\item Pruebe que $\tors{C(\Q)} = \{ (\pm 1, 0), (0, \pm 1) \}$.
				\begin{hint}
					Hay una contención que es clara.
					Para la recíproca, sería útil poder calcular la torsión mediante el inciso anterior,
					por lo cual conviene preguntarse qué es $k^\times_{\rm tors}$.
				\end{hint}
		\end{enumerate}

	% \item \textbf{La ecuación de Pell:}
	% 	Sea $d \in \Z$ un número libre de cuadrados.
	% 	Pruebe que si hay una solución de $x^2 - dy^2 = 1$ a coeficientes \emph{enteros} distinta de $(\pm 1, 0)$ o $(0, \pm 1)$,
	% 	entonces hay infinitas soluciones.

	% \item Encuentre una ecuación de Weierstrass para la ecuación $x^3 + y^3 = 1$.
\end{enumerate}

\appendix
\section{Cohomología de Galois y comentarios adicionales}
El ejercicio~\ref{ex:circle_group} puede mejorarse así:
\begin{enumerate}[resume]
	\item Sea $L/K$ una extensión cíclica (i.e., una extensión de Galois finita cuyo grupo es cíclico).
		\begin{enumerate}
			\item Sea $\sigma \in \Gal(L/K)$ un generador, pruebe que la sucesión
				\[\begin{tikzcd}
					1 \rar & K^\times \rar[hook] & L^\times \rar["{1-\sigma}"] & L^\times \rar["{\galnorm_{L/K}}"] & K^\times
				\end{tikzcd}\]
				es exacta
				(aquí \textquote{$1-\sigma$} denota el homomorfismo $\beta \mapsto \beta/\sigma(\beta)$).

			\item Pruebe que $\ker\galnorm_{L/K} \cong L^\times/K^\times$.
		\end{enumerate}

	\item Sea $K$ un cuerpo de $\car K \ne 2$ y suponga que el polinomio $x^2 + 1$ es irreducible en $K$.
		Pruebe que $(C(K), \boxplus) \cong (K(\sqrt{-1})^\times/K^\times, \cdot)$.
\end{enumerate}

Estructuras como el círculo $C(K)$ que son variedades algebraicas con una estructura de grupo definida por ecuaciones algebraicas, se llaman
\emph{grupos algebraicos}, y son de suma importancia en general.
% Sobre $\C$, los grupos algebraicos suaves adquieren el estatus de \emph{grupo de Lie}.
Otros ejemplos de grupos algebraicos son el grupo multiplicativo $(K^\times, \cdot)$ y las curvas elípticas.

\nocite{silverman:rational}
\printbibliography

\end{document}
