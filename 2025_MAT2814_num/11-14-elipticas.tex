\documentclass[11pt, reqno]{amsart}

% \usepackage[spanish]{babel}
\usepackage[LGR, T1]{fontenc}
\usepackage[utf8]{inputenc}

\input{../general.tex}
% \input{../graphics.tex}

\makeatletter
\def\emailaddrname{\textit{Correo electrónico}}
\def\subtitle#1{\gdef\@subtitle{#1}}
\def\@subtitle{}

% Metadata
\def\logo#1{\gdef\@logo{#1}}
\def\@logo{}
\def\institution#1{\gdef\@institution{#1}}
\def\@institution{}
\def\department#1{\gdef\@department{#1}}
\def\@department{}
\def\professor#1{\gdef\@professor{#1}}
\def\@professor{}
\def\course#1{\gdef\@course{#1}}
\def\@course{}
\def\coursecode#1{\gdef\@coursecode{#1}}
\def\@coursecode{}

\renewcommand{\maketitle}{
\begin{center}
	\small
	\renewcommand{\arraystretch}{1.2}
	\begin{tabular}{cp{.37\textwidth}p{0.44\textwidth}}
		% \hline
		\multirow{5}{*}{\includegraphics[height=2.0cm]{\@logo}}
	  & \multicolumn{2}{c}{ \makecell{{\bfseries \@institution} \\ \@department} } \\
	  % & \multicolumn{2}{c|}{{\bfseries\@institution} \\ \@department} \\
	  \cline{2-3}
	  & \textbf{Profesor:} \@professor & \textbf{Ayudante:} \authors \\
	  % \cline{2-3}
	  & \textbf{Curso:} \@course & \textbf{Sigla:} \@coursecode \\
	  % \cline{2-3}
	  & \multicolumn{2}{l}{ \textbf{Fecha:} \@date } \\
	  % \hline
	\end{tabular}
	\\[\baselineskip]
	% {}
	% \vspace{2\baselineskip}
	{\bfseries\Large\@title}
	\ifx\@subtitle\@empty\else
		\\[1ex]
		\large\mdseries\@subtitle
	\fi
\end{center}
}
\makeatother

\usepackage{multirow, makecell}

\usepackage[
	reversemp,
	letterpaper,
	% marginpar=2cm,
	% marginsep=1pt,
	margin=2.3cm
]{geometry}
\usepackage{fontawesome}
% \makeatletter
% \@reversemargintrue
% \makeatother

% Símbolos al margen, necesitan doble compilación
\newcommand{\hard}{\marginnote{\faFire}}
\newcommand{\hhard}{\marginnote{\faFire\faFire}}

% Dependencias para los teoremas
\usepackage{xifthen}
\def\@thmdep{}
\newcommand{\thmdep}[1]{
	\ifthenelse{\isempty{#1}}
	{\def\@thmdep{}}
	{\def\@thmdep{ (#1)}}
}
\newcommand{\thmstyle}{\color{thm}\sffamily\bfseries}

% ===== Estilos de Teoremas ==========
\newtheoremstyle{axiomstyle}
	{0.3cm}
	{0.3cm}
	{\normalfont}
	{0.5cm}
	{\bfseries\scshape}
	{:}
	{4pt}
	{\thmname{#1}\thmnote{ #3}\thmnumber{ (#2)}}
\newtheoremstyle{styleC}
	{0.5cm}
	{0.5cm}
	{\normalfont}
	{0.5cm}
	{\bfseries}
	{:}
	{4pt}
	{\thmname{#1\textrm{\@thmdep}}\thmnumber{ #2}\thmnote{ (#3)}}

% ====== Teoremas (sin borde) ===========
\theoremstyle{axiomstyle}
\newtheorem*{axiom}{Axioma}

% ====== Teoremas (sin borde) ==================
\theoremstyle{styleC}
\newtheorem{thm}{Teorema}[section]
\newtheorem{mydef}[thm]{Definición}
\newtheorem{prop}[thm]{Proposición}
\newtheorem{cor}[thm]{Corolario}
\newtheorem{lem}[thm]{Lema}
\newtheorem{con}[thm]{Conjetura}

\newtheorem*{prob}{Problema}
\newtheorem*{sol}{Solución}
\newtheorem*{obs}{Observación}
\newtheorem*{ex}{Ejemplo}

% \usepackage{tcolorbox}
% \newtcbox{bluebox}[1][]{enhanced jigsaw, 
%   sharp corners,
%   frame hidden,
%   nobeforeafter,
%   listing only,
%   #1} % comando para crear cajas de colores

\expandafter\let\expandafter\oldproof\csname\string\proof\endcsname
\let\oldendproof\endproof
\renewenvironment{proof}[1][\proofname]{%
  \oldproof[\scshape Demostración:]%
}{\oldendproof} % comando para redefinir la caja de la demostración
\newenvironment{hint}[1][\proofname]{%
  \oldproof[\scshape Pista:]%
}{\oldendproof} % comando para redefinir la caja de la demostración

% colores utilizados
\definecolor{numchap}{RGB}{249,133,29}
\definecolor{chap}{RGB}{6,129,204}
\definecolor{sec}{RGB}{204,0,0}
\definecolor{thm}{RGB}{106,176,240}
\definecolor{thmB}{RGB}{32,31,31}
\definecolor{part}{RGB}{212,66,66}

% ====== Diseño de los titulares ===============
\usepackage[explicit]{titlesec} % para personalizar el documento, la opción <<explicit>> hace que el texto de los titulares sea un objeto interactuable

\titleformat{\subsection}[runin]
	{\bfseries}
	{\textrm{\S}\thesubsection}
	{1ex}
	{#1.}

\setlist[enumerate,1]{label=\arabic*., ref=\arabic*} % Enumerate standards

\usepackage{../ayud-template}
\input{../general.tex}

\usepackage{tikz}
\usetikzlibrary{babel,cd}

\title{Curvas elípticas}
\date{\DTMdate{2025-11-07}}

\author[José Cuevas]{José Cuevas Barrientos}
\email{josecuevasbtos@uc.cl}

\logo{../puc_negro.png}
\institution{Pontificia Universidad Católica de Chile}
\department{Facultad de Matemáticas}
\course{Teoría de Números}
\coursecode{MAT2814}
\professor{Ricardo Menares}

\begin{document}

\maketitle

\section{Ejercicios}
\begin{enumerate}
	\item Pruebe que la curva proyectiva $X^3 + Y^3 = Z^3$ es una curva elíptica sobre un cuerpo $k$ de característica $\car k \ne 3$
		y dé una ecuación de Weierstrass corta (i.e., de la forma $y^2 = x^3 + ax + b$) cuando $\car k \nmid 6$.

		¿Qué falla exactamente en $\car k = 3$?

	\item Sea $C \colon y^2 = f(x) := x^3 + b_2x^2 + b_4x + b_6$ una curva afín con clausura proyectiva $\overline{C}$.
		\begin{enumerate}
			\item Pruebe que posee un único punto $o \in \overline{C}(k) \setminus C(k)$ \textquote{al infinito}.
			\item Calcule cuales son los posibles puntos singulares y dé un ejemplo concreto donde efectivamente los puntos que
				satisfagan esta propiedad sean los singulares.
			\item Concluya que, si $f$ no tiene raíces repetidas, entonces $C$ es suave.
		\end{enumerate}

	\item Sea $C \colon y^2 = f(x)$ una curva elíptica como antes.
		\begin{enumerate}
			\item Pruebe que si $P = (x, y)$, entonces tenemos la siguiente fórmula de duplicación
				\[
					x(2P) = \frac{x^4 - 2bx^2 - 8cx + b^2 - 4ac}{4x^3 + 4ax^2 + 4bx + 4c}.
				\]
				\begin{hint}
					Recuerde que las relaciones de Viète dicen que si un polinomio cúbico $x^3 + \alpha x^2 + \beta x +
					\gamma = 0$ tiene raíces $r_1, r_2, r_3$, entonces $\alpha = -r_1 - r_2 - r_3$.
				\end{hint}

			\item Concluya que los polinomios $x^4 - 2bx^2 - 8cx + b^2 - 4ac$ y $f(x)$ no tienen raíces comunes (en $\algcl k$).
		\end{enumerate}

	\item Más en general, la cúbica $E\colon \; X^3 + Y^3 = \alpha Z^3$ (con $\alpha \ne 0$) tiene un punto racional $o := [1 : -1 : 0] \in E(k)$ al infinito.
		\begin{enumerate}
			\item Calcule una fórmula para la suma (con neutro $o$) $P + Q$ de dos puntos afines distintos $P = (u_1, v_1)$ y $Q = (u_2, v_2)$.
			\item Encuentre una fórmula de duplicación para el punto afín $P = (u, v)$.
		\end{enumerate}

	\item Sea $C \colon y^2 = f(x)$ una curva afín con clausura proyectiva $\overline{C}$, donde $f$ es un polinomio mónico de $\deg f \ge 3$.
		\begin{enumerate}
			\item Pruebe que, en el cuerpo finito $\Fp$, tenemos la fórmula
				\[
					|C(\Fp)| = p + 1 + \sum_{j=0}^{p-1} \left(\frac{f(j)}{p}\right),
				\]
				donde $(a/p)$ es el símbolo de Legendre.
			\item Muestre que si $f(x) = x^3 + d$; entonces para $p \equiv 2 \pmod 3$, se cumple que $|C(\Fp)| = p + 1$.
				\begin{hint}
					Muestre que $x \mapsto x^3$ es un automorfismo de $\Fp^\times$.
				\end{hint}
		\end{enumerate}
\end{enumerate}

\appendix
\nocite{silverman:rational}
\printbibliography

\end{document}
