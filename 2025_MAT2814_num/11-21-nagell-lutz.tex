\documentclass[11pt, reqno]{amsart}

% \usepackage[spanish]{babel}
\usepackage[LGR, T1]{fontenc}
\usepackage[utf8]{inputenc}

\input{../general.tex}
% \input{../graphics.tex}

\makeatletter
\def\emailaddrname{\textit{Correo electrónico}}
\def\subtitle#1{\gdef\@subtitle{#1}}
\def\@subtitle{}

% Metadata
\def\logo#1{\gdef\@logo{#1}}
\def\@logo{}
\def\institution#1{\gdef\@institution{#1}}
\def\@institution{}
\def\department#1{\gdef\@department{#1}}
\def\@department{}
\def\professor#1{\gdef\@professor{#1}}
\def\@professor{}
\def\course#1{\gdef\@course{#1}}
\def\@course{}
\def\coursecode#1{\gdef\@coursecode{#1}}
\def\@coursecode{}

\renewcommand{\maketitle}{
\begin{center}
	\small
	\renewcommand{\arraystretch}{1.2}
	\begin{tabular}{cp{.37\textwidth}p{0.44\textwidth}}
		% \hline
		\multirow{5}{*}{\includegraphics[height=2.0cm]{\@logo}}
	  & \multicolumn{2}{c}{ \makecell{{\bfseries \@institution} \\ \@department} } \\
	  % & \multicolumn{2}{c|}{{\bfseries\@institution} \\ \@department} \\
	  \cline{2-3}
	  & \textbf{Profesor:} \@professor & \textbf{Ayudante:} \authors \\
	  % \cline{2-3}
	  & \textbf{Curso:} \@course & \textbf{Sigla:} \@coursecode \\
	  % \cline{2-3}
	  & \multicolumn{2}{l}{ \textbf{Fecha:} \@date } \\
	  % \hline
	\end{tabular}
	\\[\baselineskip]
	% {}
	% \vspace{2\baselineskip}
	{\bfseries\Large\@title}
	\ifx\@subtitle\@empty\else
		\\[1ex]
		\large\mdseries\@subtitle
	\fi
\end{center}
}
\makeatother

\usepackage{multirow, makecell}

\usepackage[
	reversemp,
	letterpaper,
	% marginpar=2cm,
	% marginsep=1pt,
	margin=2.3cm
]{geometry}
\usepackage{fontawesome}
% \makeatletter
% \@reversemargintrue
% \makeatother

% Símbolos al margen, necesitan doble compilación
\newcommand{\hard}{\marginnote{\faFire}}
\newcommand{\hhard}{\marginnote{\faFire\faFire}}

% Dependencias para los teoremas
\usepackage{xifthen}
\def\@thmdep{}
\newcommand{\thmdep}[1]{
	\ifthenelse{\isempty{#1}}
	{\def\@thmdep{}}
	{\def\@thmdep{ (#1)}}
}
\newcommand{\thmstyle}{\color{thm}\sffamily\bfseries}

% ===== Estilos de Teoremas ==========
\newtheoremstyle{axiomstyle}
	{0.3cm}
	{0.3cm}
	{\normalfont}
	{0.5cm}
	{\bfseries\scshape}
	{:}
	{4pt}
	{\thmname{#1}\thmnote{ #3}\thmnumber{ (#2)}}
\newtheoremstyle{styleC}
	{0.5cm}
	{0.5cm}
	{\normalfont}
	{0.5cm}
	{\bfseries}
	{:}
	{4pt}
	{\thmname{#1\textrm{\@thmdep}}\thmnumber{ #2}\thmnote{ (#3)}}

% ====== Teoremas (sin borde) ===========
\theoremstyle{axiomstyle}
\newtheorem*{axiom}{Axioma}

% ====== Teoremas (sin borde) ==================
\theoremstyle{styleC}
\newtheorem{thm}{Teorema}[section]
\newtheorem{mydef}[thm]{Definición}
\newtheorem{prop}[thm]{Proposición}
\newtheorem{cor}[thm]{Corolario}
\newtheorem{lem}[thm]{Lema}
\newtheorem{con}[thm]{Conjetura}

\newtheorem*{prob}{Problema}
\newtheorem*{sol}{Solución}
\newtheorem*{obs}{Observación}
\newtheorem*{ex}{Ejemplo}

% \usepackage{tcolorbox}
% \newtcbox{bluebox}[1][]{enhanced jigsaw, 
%   sharp corners,
%   frame hidden,
%   nobeforeafter,
%   listing only,
%   #1} % comando para crear cajas de colores

\expandafter\let\expandafter\oldproof\csname\string\proof\endcsname
\let\oldendproof\endproof
\renewenvironment{proof}[1][\proofname]{%
  \oldproof[\scshape Demostración:]%
}{\oldendproof} % comando para redefinir la caja de la demostración
\newenvironment{hint}[1][\proofname]{%
  \oldproof[\scshape Pista:]%
}{\oldendproof} % comando para redefinir la caja de la demostración

% colores utilizados
\definecolor{numchap}{RGB}{249,133,29}
\definecolor{chap}{RGB}{6,129,204}
\definecolor{sec}{RGB}{204,0,0}
\definecolor{thm}{RGB}{106,176,240}
\definecolor{thmB}{RGB}{32,31,31}
\definecolor{part}{RGB}{212,66,66}

% ====== Diseño de los titulares ===============
\usepackage[explicit]{titlesec} % para personalizar el documento, la opción <<explicit>> hace que el texto de los titulares sea un objeto interactuable

\titleformat{\subsection}[runin]
	{\bfseries}
	{\textrm{\S}\thesubsection}
	{1ex}
	{#1.}

\setlist[enumerate,1]{label=\arabic*., ref=\arabic*} % Enumerate standards

\usepackage{../ayud-template}
\input{../general.tex}

\usepackage{tikz}
\usetikzlibrary{babel,cd}

\title{Torsión y estructura de grupo}
\date{\DTMdate{2025-11-21}}

\author[José Cuevas]{José Cuevas Barrientos}
\email{josecuevasbtos@uc.cl}

\logo{../puc_negro.png}
\institution{Pontificia Universidad Católica de Chile}
\department{Facultad de Matemáticas}
\course{Teoría de Números}
\coursecode{MAT2814}
\professor{Ricardo Menares}

\begin{document}

\maketitle

\section{Reticulados}
\begin{enumerate}
	\item Sean $\omega_1, \omega_2 \in \C^\times$ dos complejos $\R$-linealmente independientes y sea $\Lambda := \omega_1\Z +
		\omega_2\Z$ su reticulado.
		\begin{enumerate}
			\item Pruebe que el área $A$ del paralelogramo de vértices $\{ 0, \omega_1, \omega_1 + \omega_2, \omega_2 \}$ no
				depende de la elección de $\Z$-base de $\Lambda$.

			\item Pruebe que
				\[
					|\{ \omega \in \Lambda : |\omega| \le R \}| = \frac{\pi}{A} R^2 + O(R).
				\]

			\item Concluya que existe $c > 0$ tal que 
				\[
					|\{ \omega \in \Lambda : R \le |\omega| < R + 1 \}| \le c R.
				\]
		\end{enumerate}

	\item Sean $\omega_1, \omega_2 \in \C^\times$ dos complejos $\R$-linealmente independientes y sea $\Lambda := \omega_1\Z +
		\omega_2\Z$ su reticulado.
		\begin{enumerate}
			\item Verifique que, para todo $s \in \C$ con $\Re s > 2$, la serie
				\[
					\sum_{\substack{\omega\in \Lambda \\ \omega\ne 0}} \frac{1}{|\omega|^s} < \infty.
				\]

			\item\lookst
				Pruebe que la serie
				\[
					\wp(z) := \frac{1}{z^2} + \sum_{\substack{\omega \in \Lambda \\ \omega \ne 0}} \left(
					\frac{1}{(\omega - z)^2} - \frac{1}{\omega^2} \right),
				\]
				converge absoluta y uniformemente en compactos del abierto $\C \setminus \Lambda$.
				Por ende, concluya que es holomorfa allí.

			\item Pruebe que $\wp$ es una función par (i.e., $\wp(z) = \wp(-z)$) y que es periódica respecto a $\Lambda$, a
				decir, $\wp(z + \omega) = \wp(z)$ para todo $\omega \in \Lambda$.
		\end{enumerate}

	\item\label{ex:endomorphism_ring}
		Recuerde que en clases vió que toda curva elíptica compleja es de la forma $\C/\Lambda$, donde $\Lambda$ es un
		\emph{reticulado} (i.e., un subgrupo abeliano libre generado por dos vectores $\R$-linealmente independientes).
		Definiremos el conjunto de endomorfismos como
		\[
			\End(\C/\Lambda) := \{ \alpha \in \C \colon \alpha\Lambda \subseteq \Lambda \}.
		\]
		\begin{enumerate}
			\item Pruebe que $\End(\C/\Lambda)$ es un subanillo de $\C$.
			\item\label{ex:complex_mult}
				Pruebe que, o bien $\End(\C/\Lambda) = \Z$, o bien $\End(\C/\Lambda)$ es un anillo de enteros de una extensión
				cuadrática imaginaria.
			\item\label{ex:automorphism_gr}
				Defina $\Aut(\C/\Lambda) = \End(\C/\Lambda)^\times$ como el grupo de unidades.
				Concluya que $\lvert\Aut(\C/\Lambda)\rvert \in \{ 2, 4, 6 \}$.
				\begin{hint}
					Para esto puede ser útil saber que el anillo de enteros algebraicos de la extensión cuadrática
					$\Q(\sqrt{-d})$ está generado como $\Z$-álgebra por el elemento $\sqrt{-d}$ si $d \not\equiv 3
					\pmod 4$ o por $\frac{1}{2}(1 + \sqrt{-d})$ si $d \equiv 3 \pmod 4$.
				\end{hint}
		\end{enumerate}
\end{enumerate}

\section{Torsión}
\begin{enumerate}[resume]
	\item Sea $E \colon y^2 = f(x) = x^3 + b_2x^2 + b_4x + b_6$ una curva elíptica.
		\begin{enumerate}
			\item Verifique que un punto $P \in E(k)$ tiene orden 3 syss es un punto de inflexión de $C$ (i.e., su tangente solo
				corta a $C$ solamente en $P$).

			\item Pruebe que
				\[
					% \displaystyle
					\frac{\ud^2 y}{\ud x^2} = \frac{2f''(x)f(x) - f'(x)^2}{4yf(x)}.
				\]

			\item Concluya que $E(\R)[3] = \Z/3\Z$.
		\end{enumerate}

	\item Pruebe que las siguientes curvas elípticas tienen infinitos puntos racionales:
		\begin{enumerate}
			\item $y^2 = x^3 - 2$. % (3, 5)
			\item $y^2 = x^3 + 8$. % (1, 3)
			% \item $y^2 = x^3 + 4x$. % (2, 4)
		\end{enumerate}
\end{enumerate}

\appendix
\section{Comentarios adicionales}
El \textquote{conjunto de endomorfismos} del ejercicio~\ref{ex:endomorphism_ring} es el honesto anillo de endomorfismos de la curva elíptica
$\C/\Lambda$ como superficie de Riemann.
La correspondencia viene probada en \citeauthor{silverman:elliptic}~\cite{silverman:elliptic}, \S VI.4.

Si una curva elíptica compleja $E = \C/\Lambda$ satisface que $\End(E) \not\cong \Z$ como en el ejercicio~\ref{ex:complex_mult}, entonces
diremos que tiene \emph{multiplicación compleja}.
Mirando las condiciones para $\Lambda$ es visible de dónde viene el nombre.
\begin{enumerate}[resume]
	\item Sea $\mathfrak{H} := \{ \tau \in \C : \Im\tau > 0 \}$ el semiplano superior.
		Pruebe que hay numerables $\tau \in \mathfrak{H}$ tales que $E_\tau := \C/\Lambda_\tau$, con $\Lambda_\tau = \Z + \tau \Z$,
		tiene multiplicación compleja.

	\item Vamos a probar la pista del ejercicio~\ref{ex:automorphism_gr}.
		Defina $\mathcal{O}_{\Q(\sqrt{-d})} \subseteq \Q(\sqrt{-d})$ como el subanillo de los elementos cuyo polinomio minimal tiene
		coeficientes en $\Z$.
		\begin{enumerate}
			\item Pruebe que $\gamma = a + b \sqrt{-d} \in \Q(\sqrt{-d})$ yace en $\mathcal{O}_{\Q(\sqrt{-d})}$ syss
				$\gamma + \overline{\gamma} = 2a, \; \gamma \cdot \overline{\gamma} = a^2 + db^2 \in \Z$.
			\item Concluya que $\sqrt{-d}$ genera a $\mathcal{O}_{\Q(\sqrt{-d})}$ cuando $d \not\equiv 3 \pmod 4$ y
				$\frac{1}{2}(1 + \sqrt{-d})$ genera cuando $d \equiv 3 \pmod 4$.
		\end{enumerate}
\end{enumerate}

\nocite{silverman:rational}
\printbibliography

\end{document}
