\documentclass[11pt, reqno]{amsart}

\usepackage[spanish]{babel}
\usepackage[LGR, T1]{fontenc}
\usepackage[utf8]{inputenc}

../LaTeX/general.tex
% \input{../graphics.tex}

\makeatletter
\def\emailaddrname{\textit{Correo electrónico}}
\def\subtitle#1{\gdef\@subtitle{#1}}
\def\@subtitle{}

% Metadata
\def\logo#1{\gdef\@logo{#1}}
\def\@logo{}
\def\institution#1{\gdef\@institution{#1}}
\def\@institution{}
\def\department#1{\gdef\@department{#1}}
\def\@department{}
\def\professor#1{\gdef\@professor{#1}}
\def\@professor{}
\def\course#1{\gdef\@course{#1}}
\def\@course{}
\def\coursecode#1{\gdef\@coursecode{#1}}
\def\@coursecode{}

\renewcommand{\maketitle}{
\begin{center}
	\small
	\renewcommand{\arraystretch}{1.2}
	\begin{tabular}{cp{.37\textwidth}p{0.44\textwidth}}
		% \hline
		\multirow{5}{*}{\includegraphics[height=2.0cm]{\@logo}}
	  & \multicolumn{2}{c}{ \makecell{{\bfseries \@institution} \\ \@department} } \\
	  % & \multicolumn{2}{c|}{{\bfseries\@institution} \\ \@department} \\
	  \cline{2-3}
	  & \textbf{Profesor:} \@professor & \textbf{Ayudante:} \authors \\
	  % \cline{2-3}
	  & \textbf{Curso:} \@course & \textbf{Sigla:} \@coursecode \\
	  % \cline{2-3}
	  & \multicolumn{2}{l}{ \textbf{Fecha:} \@date } \\
	  % \hline
	\end{tabular}
	\\[\baselineskip]
	% {}
	% \vspace{2\baselineskip}
	{\bfseries\Large\@title}
	\ifx\@subtitle\@empty\else
		\\[1ex]
		\large\mdseries\@subtitle
	\fi
\end{center}
}
\makeatother

\usepackage{multirow, makecell}

\usepackage[
	reversemp,
	letterpaper,
	% marginpar=2cm,
	% marginsep=1pt,
	margin=2.3cm
]{geometry}
\usepackage{fontawesome}
% \makeatletter
% \@reversemargintrue
% \makeatother

% Símbolos al margen, necesitan doble compilación
\newcommand{\hard}{\marginnote{\faFire}}
\newcommand{\hhard}{\marginnote{\faFire\faFire}}

% Dependencias para los teoremas
\usepackage{xifthen}
\def\@thmdep{}
\newcommand{\thmdep}[1]{
	\ifthenelse{\isempty{#1}}
	{\def\@thmdep{}}
	{\def\@thmdep{ (#1)}}
}
\newcommand{\thmstyle}{\color{thm}\sffamily\bfseries}

% ===== Estilos de Teoremas ==========
\newtheoremstyle{axiomstyle}
	{0.3cm}
	{0.3cm}
	{\normalfont}
	{0.5cm}
	{\bfseries\scshape}
	{:}
	{4pt}
	{\thmname{#1}\thmnote{ #3}\thmnumber{ (#2)}}
\newtheoremstyle{styleC}
	{0.5cm}
	{0.5cm}
	{\normalfont}
	{0.5cm}
	{\bfseries}
	{:}
	{4pt}
	{\thmname{#1\textrm{\@thmdep}}\thmnumber{ #2}\thmnote{ (#3)}}

% ====== Teoremas (sin borde) ===========
\theoremstyle{axiomstyle}
\newtheorem*{axiom}{Axioma}

% ====== Teoremas (sin borde) ==================
\theoremstyle{styleC}
\newtheorem{thm}{Teorema}[section]
\newtheorem{mydef}[thm]{Definición}
\newtheorem{prop}[thm]{Proposición}
\newtheorem{cor}[thm]{Corolario}
\newtheorem{lem}[thm]{Lema}
\newtheorem{con}[thm]{Conjetura}

\newtheorem*{prob}{Problema}
\newtheorem*{sol}{Solución}
\newtheorem*{obs}{Observación}
\newtheorem*{ex}{Ejemplo}

% \usepackage{tcolorbox}
% \newtcbox{bluebox}[1][]{enhanced jigsaw, 
%   sharp corners,
%   frame hidden,
%   nobeforeafter,
%   listing only,
%   #1} % comando para crear cajas de colores

\expandafter\let\expandafter\oldproof\csname\string\proof\endcsname
\let\oldendproof\endproof
\renewenvironment{proof}[1][\proofname]{%
  \oldproof[\scshape Demostración:]%
}{\oldendproof} % comando para redefinir la caja de la demostración
\newenvironment{hint}[1][\proofname]{%
  \oldproof[\scshape Pista:]%
}{\oldendproof} % comando para redefinir la caja de la demostración

% colores utilizados
\definecolor{numchap}{RGB}{249,133,29}
\definecolor{chap}{RGB}{6,129,204}
\definecolor{sec}{RGB}{204,0,0}
\definecolor{thm}{RGB}{106,176,240}
\definecolor{thmB}{RGB}{32,31,31}
\definecolor{part}{RGB}{212,66,66}

% ====== Diseño de los titulares ===============
\usepackage[explicit]{titlesec} % para personalizar el documento, la opción <<explicit>> hace que el texto de los titulares sea un objeto interactuable

\titleformat{\subsection}[runin]
	{\bfseries}
	{\textrm{\S}\thesubsection}
	{1ex}
	{#1.}

\setlist[enumerate,1]{label=\arabic*., ref=\arabic*} % Enumerate standards


\title{Extensiones algebraicas}
\date{\DTMdate{2025-08-13}}

\author{José Cuevas Barrientos}
\email{josecuevasbtos@uc.cl}
% \urladdr{https://josecuevas.xyz/teach/2025-1-ayud/}

\logo{../puc_negro.png}
\institution{Pontificia Universidad Católica de Chile}
\department{Facultad de Matemáticas}
\course{Álgebra II}
\coursecode{MPG3201}
\professor{José Samper}

\begin{document}

\maketitle

A lo largo de las ayudantías trataré de incluír comentarios o problemas especiales.
Los problemas difíciles tendrán ojos asustados {\straighteyes},
los comentarios que son opcionales u omitibles tendrán ojos hastiados {\upeyes}
y los comentarios \textbf{importantes} tendrán ojos interesados {\righteyes}.

\section{General}
\begin{enumerate}
		% \item Pruebe que para todo cuerpo $k$ existen infinitos polinomios irreducibles con coeficientes en $k$.
	\item Sea $k \supseteq \Fp$ una extensión algebraica (posiblemente infinita).
		Pruebe que todo elemento \emph{siempre} de $k$ tiene raíz $p$-ésima.

	\item Sea $\Fp[q]$ un cuerpo con $q < \infty$ elementos.
		\begin{enumerate}
			\item Sea $f(x) \in \Fp[q][x]$ es irreducible.
				Pruebe que $f(x) \mid x^{q^n} - x$ syss $\deg f \mid n$.

			\item Sea $\psi(d)$ la cantidad de polinomios irreducibles de grado $d$ en $\Fp[q][x]$.
				Pruebe que
				\[
					n \psi(n) = \sum_{d \mid n} \mu(d) q^{n/d},
				\]
				donde $\mu(d)$ es la \emph{función de Möbius} que vale 0 si $p^2 \mid d$ para algún primo $p$ y vale
				$(-1)^m$ si $d = p_1\cdot p_m$, donde $p_j$ son primos distintos.

				\begin{hint}
					Para el problema podría necesitar de la \href{https://es.wikipedia.org/wiki/Fórmula_de_inversión_de_Möbius}{fórmula de inversión de Möbius}.
				\end{hint}
		\end{enumerate}
\end{enumerate}

\section{Grados de extensiones}
\begin{enumerate}[resume]
	\item Sea $\Omega/k$ una extensión de cuerpos con extensiones intermedias $k \subseteq K, L \subseteq \Omega$.
		Pruebe que
		\[
			[KL : k] \le [K : k] \, [L : k],
		\]
		y que se alcanza igualdad cuando $[K : k]$ y $[L : k]$ son coprimos.

	\item Sea $f(x) \in k[x]$ un polinomio de grado $n$, sea $K$ su cuerpo de escisión.
		Pruebe que $[K : k] \mid n!$
\end{enumerate}

\section{Extensiones (in)separables}
Como se vio en clases, las extensiones en característica cero son todas separables, por lo que en esta sección $k$ será
un cuerpo de $\car k = p > 0$.
\begin{enumerate}[resume]
	\item Sea $K/k$ una extensión algebraica con $p := \car k > 0$ y sea $\alpha \in K$.
		\begin{enumerate}
			\item Pruebe que si $\alpha$ es inseparable, entonces su polinomio minimal $f(x) \in k[x]$
				satisface que $f(x) = g(x^p)$ para todo $g(x) \in k[x]$.
			\item Pruebe que $\alpha$ es separable syss $k(\alpha) = k(\alpha^p)$.
		\end{enumerate}

	\item Pruebe que si $f(x) \in k[x]$ es irreducible, entonces todas sus raíces (en su cuerpo de escisión) tienen
		la misma multiplicidad y esta es una potencia de $p$.

	\item Sea $K/k$ una extensión algebraica de cuerpos.
		Un elemento $\alpha \in K$ se dice \strong{puramente inseparable} si su polinomio minimal $f(x) \in k[x]$
		es una potencia del monomio $x - \alpha$.
		\begin{enumerate}
			\item\lookright
				Empleando el ejercicio anterior pruebe que $\alpha \in K$ es puramente inseparable syss
				$\alpha^{p^e} \in k$ para algún $e \ge 1$.
			\item Pruebe que si $a \in k \setminus k^p$, entonces el polinomio $x^{p^e} - a$ es irreducible
				para $e \in \N$.
			\item Pruebe que
				\[
					K_{\rm ins} = \{ \alpha \in K : \alpha \text{ es puramente inseparable} \}
				\]
				es un subcuerpo de $K$.
		\end{enumerate}
	% \item Sea $k$ un cuerpo de característica $p := \car k > 0$.
	% 	Sea $q := p^n$ con $n \ge 1$ a elección.
	% 	Pruebe que
	% 	\[
	% 		k^q := \{ \alpha^q : \alpha \in k \} \subseteq k
	% 	\]
	% 	es un subcuerpo de $k$.
\end{enumerate}

\nocite{lang:algebra}

\printbibliography

\end{document}
