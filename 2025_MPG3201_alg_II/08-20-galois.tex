\documentclass[11pt, reqno]{amsart}

% \usepackage[spanish]{babel}
\usepackage[LGR, T1]{fontenc}
\usepackage[utf8]{inputenc}

../LaTeX/general.tex
% \input{../graphics.tex}

\makeatletter
\def\emailaddrname{\textit{Correo electrónico}}
\def\subtitle#1{\gdef\@subtitle{#1}}
\def\@subtitle{}

% Metadata
\def\logo#1{\gdef\@logo{#1}}
\def\@logo{}
\def\institution#1{\gdef\@institution{#1}}
\def\@institution{}
\def\department#1{\gdef\@department{#1}}
\def\@department{}
\def\professor#1{\gdef\@professor{#1}}
\def\@professor{}
\def\course#1{\gdef\@course{#1}}
\def\@course{}
\def\coursecode#1{\gdef\@coursecode{#1}}
\def\@coursecode{}

\renewcommand{\maketitle}{
\begin{center}
	\small
	\renewcommand{\arraystretch}{1.2}
	\begin{tabular}{cp{.37\textwidth}p{0.44\textwidth}}
		% \hline
		\multirow{5}{*}{\includegraphics[height=2.0cm]{\@logo}}
	  & \multicolumn{2}{c}{ \makecell{{\bfseries \@institution} \\ \@department} } \\
	  % & \multicolumn{2}{c|}{{\bfseries\@institution} \\ \@department} \\
	  \cline{2-3}
	  & \textbf{Profesor:} \@professor & \textbf{Ayudante:} \authors \\
	  % \cline{2-3}
	  & \textbf{Curso:} \@course & \textbf{Sigla:} \@coursecode \\
	  % \cline{2-3}
	  & \multicolumn{2}{l}{ \textbf{Fecha:} \@date } \\
	  % \hline
	\end{tabular}
	\\[\baselineskip]
	% {}
	% \vspace{2\baselineskip}
	{\bfseries\Large\@title}
	\ifx\@subtitle\@empty\else
		\\[1ex]
		\large\mdseries\@subtitle
	\fi
\end{center}
}
\makeatother

\usepackage{multirow, makecell}

\usepackage[
	reversemp,
	letterpaper,
	% marginpar=2cm,
	% marginsep=1pt,
	margin=2.3cm
]{geometry}
\usepackage{fontawesome}
% \makeatletter
% \@reversemargintrue
% \makeatother

% Símbolos al margen, necesitan doble compilación
\newcommand{\hard}{\marginnote{\faFire}}
\newcommand{\hhard}{\marginnote{\faFire\faFire}}

% Dependencias para los teoremas
\usepackage{xifthen}
\def\@thmdep{}
\newcommand{\thmdep}[1]{
	\ifthenelse{\isempty{#1}}
	{\def\@thmdep{}}
	{\def\@thmdep{ (#1)}}
}
\newcommand{\thmstyle}{\color{thm}\sffamily\bfseries}

% ===== Estilos de Teoremas ==========
\newtheoremstyle{axiomstyle}
	{0.3cm}
	{0.3cm}
	{\normalfont}
	{0.5cm}
	{\bfseries\scshape}
	{:}
	{4pt}
	{\thmname{#1}\thmnote{ #3}\thmnumber{ (#2)}}
\newtheoremstyle{styleC}
	{0.5cm}
	{0.5cm}
	{\normalfont}
	{0.5cm}
	{\bfseries}
	{:}
	{4pt}
	{\thmname{#1\textrm{\@thmdep}}\thmnumber{ #2}\thmnote{ (#3)}}

% ====== Teoremas (sin borde) ===========
\theoremstyle{axiomstyle}
\newtheorem*{axiom}{Axioma}

% ====== Teoremas (sin borde) ==================
\theoremstyle{styleC}
\newtheorem{thm}{Teorema}[section]
\newtheorem{mydef}[thm]{Definición}
\newtheorem{prop}[thm]{Proposición}
\newtheorem{cor}[thm]{Corolario}
\newtheorem{lem}[thm]{Lema}
\newtheorem{con}[thm]{Conjetura}

\newtheorem*{prob}{Problema}
\newtheorem*{sol}{Solución}
\newtheorem*{obs}{Observación}
\newtheorem*{ex}{Ejemplo}

% \usepackage{tcolorbox}
% \newtcbox{bluebox}[1][]{enhanced jigsaw, 
%   sharp corners,
%   frame hidden,
%   nobeforeafter,
%   listing only,
%   #1} % comando para crear cajas de colores

\expandafter\let\expandafter\oldproof\csname\string\proof\endcsname
\let\oldendproof\endproof
\renewenvironment{proof}[1][\proofname]{%
  \oldproof[\scshape Demostración:]%
}{\oldendproof} % comando para redefinir la caja de la demostración
\newenvironment{hint}[1][\proofname]{%
  \oldproof[\scshape Pista:]%
}{\oldendproof} % comando para redefinir la caja de la demostración

% colores utilizados
\definecolor{numchap}{RGB}{249,133,29}
\definecolor{chap}{RGB}{6,129,204}
\definecolor{sec}{RGB}{204,0,0}
\definecolor{thm}{RGB}{106,176,240}
\definecolor{thmB}{RGB}{32,31,31}
\definecolor{part}{RGB}{212,66,66}

% ====== Diseño de los titulares ===============
\usepackage[explicit]{titlesec} % para personalizar el documento, la opción <<explicit>> hace que el texto de los titulares sea un objeto interactuable

\titleformat{\subsection}[runin]
	{\bfseries}
	{\textrm{\S}\thesubsection}
	{1ex}
	{#1.}

\setlist[enumerate,1]{label=\arabic*., ref=\arabic*} % Enumerate standards

% \usepackage{comment} % para (des)activar comentarios
% \includecomment{comment}

\usepackage{../ayud-template}
../LaTeX/general.tex
% % \input{../graphics.tex}

\usepackage{multicol}

\title{Teoría de Galois}
\date{\DTMdate{2025-08-20}}

\author{José Cuevas Barrientos}
\email{josecuevasbtos@uc.cl}
\urladdr{https://josecuevas.xyz/teach/2025-2-alg/}

\logo{../puc_negro.png}
\institution{Pontificia Universidad Católica de Chile y Universidad de Chile}
\department{Facultad de Matemáticas}
\course{Álgebra II}
\coursecode{MPG3201}
\professor{José Samper}

\begin{document}

\maketitle

\section{Grupos de Galois}
\begin{enumerate}
	% \item Considere las siguientes extensiones sobre $\Q$, determine cuáles son de Galois y calcule el grupo de Galois de su cuerpo de escisión:
	% 	\begin{enumerate}
	% 	\begin{multicols}{3}
	% 		\item $\Q(\sqrt{2 + \sqrt{2}})$.
	% 		\item $\Q(\sqrt{3 + \sqrt{5}})$.
	% 		\item $\Q(\sqrt{1 + \sqrt{2}})$.
	% 	\end{multicols}
	% 	\end{enumerate}

	\item\label{exr:inverse_gal_ab}\lookst
		Pruebe que para todo grupo \emph{abeliano} finito $G$ existe una extensión $K/\Q$ de Galois tal que
		$\Gal(K/\Q) \cong G$.

		\begin{hint}
			Para la prueba puede ser útil emplear el \textbf{teorema de Dirichlet} que dice que
			dado $n > 1$ entero y $a$ coprimo con $n$, existen infinitos primos $p$ tales que $p \equiv a
			\pmod n$.
		\end{hint}

	\item Sea $L/k$ una extensión normal y definamos el subconjunto:
		\[
			L_{\rm sep} := \{ \alpha \in L : \alpha \text{ es separable sobre } k \}.
		\]
		\begin{enumerate}
			\item Pruebe que $L_{\rm sep}$ es un subcuerpo de $L$.
			\item Pruebe que la extensión $L_{\rm sep}/k$ es de Galois y $L/L_{\rm sep}$ es puramente inseparable.
			\item Pruebe que la siguiente aplicación
				\[
					\rho \colon \Gal(L/k) \longrightarrow \Gal(L_{\rm sep}/k), \qquad \sigma \longmapsto \sigma|_{L_{\rm sep}}
				\]
				está bien definida y es un isomorfismo de grupos.
		\end{enumerate}

	\item\label{exr:simple_subextension_count}\lookright
		Sea $K/k$ una extensión simple de cuerpos de grado $n := [K:k] < \infty$.
		Pruebe que $K/k$ tiene a lo sumo $2^n$ subextensiones (incluyendo a $K$ y $k$ mismos).

		\begin{prob}
			¿Se puede alcanzar dicha cota? Dicho de otro modo, ¿cuán óptima es?
		\end{prob}

	% \item (El teorema de la base normal)
	% 	Sea $K/k$ una extensión finita de grado $n$.
	% 	Una \strong{base normal} de $K$ es una $k$-base $\{ \alpha_j \}_{1\le j\le n}$ para $K$,
	% 	donde los elementos son $k$-conjugados (i.e., comparten el mismo polinomio minimal).
	% 	\begin{enumerate}
	% 		\item Pruebe que si $K$ posee una base normal, entonces $K/k$ es de Galois.
	% 	\end{enumerate}
	% 	Para el recíproco, suponga que $K/k$ es de Galois.
	% 	\begin{enumerate}[resume]
	% 		\item Sean $\{ \sigma_1, \dots, \sigma_n \} = \Gal(K/k)$ los elementos.
	% 			Pruebe que los conjugados de $\alpha \in K$ forman una base (normal) syss la matriz $[
	% 			\sigma_i \sigma_j \alpha ]_{ij} \in \Mat_n K$ tiene determinante no nulo.
	% 		\item Pruebe que $K$ admite una base normal.
	% 	\end{enumerate}

	\item Considere la extensión \smash{$\Q\left( \sqrt{(2+\sqrt 2)(3+\sqrt 3)} \right) \supseteq \Q$} y determine:
		\begin{multicols}{2}
			\begin{enumerate}
				\item Si es de Galois.
				\item El grupo de Galois de su cuerpo de escisión.
			\end{enumerate}
		\end{multicols}
\end{enumerate}

\appendix
\section{Ejercicios adicionales:\\La correspondencia de Jacobson-Bourbaki}
En la siguiente serie de ejercicios, pretendemos probar un resultado un tanto técnico.
En la sección, $\Hom_k$ denota homomorfismos de $k$-espacios vectoriales.
\begin{enumerate}[resume]
	\item Sea $k \subseteq K \subseteq L$ una torre de extensiones, posiblemente infinitas.
		Pruebe que $K / k$ es una extensión finita syss $\Hom_k(K, L)$ es un $L$-espacio vectorial de dimensión finita (la
		suma y producto escalar son coordenada a coordenada) y, en cuyo caso, que
		\[
			[ K : k ] = [ \Hom_k(K, L) : L ].
		\]

	\item\lookst
		Sea $L$ un cuerpo.
		Note que el conjunto $\End_{\mathsf{Ab}}(L)$ de endomorfismos de $L$ como grupo abeliano es un anillo (no
		conmutativo) con la suma coordenada a coordena, y la composición como producto;
		más aún, hay una inyección de anillos $\mu\colon L \hookto \End_{\mathsf{Ab}}(L)$ que a un elemento $\alpha \in
		L$ le asigna el endomorfismo $\mu(\alpha)(x) := \alpha\cdot x$.
		% Por abuso de notación, identificaremos a $L$ con su imagen en $\End_{\mathsf{Ab}}(L)$.
		Sea $A \subseteq \End_{\mathsf{Ab}}(L)$ una $L$-subálgebra (i.e., un subanillo que contiene a la imagen de
		$L$ mediante $\mu$) tal que $n := \dim_L(A) < \infty$.
		\begin{enumerate}
			\item Pruebe que existen $\alpha_1, \dots, \alpha_n \in L$ y $\sigma_1, \dots, \sigma_n \in A$ tales que
				$\sigma_i(\alpha_j) = \delta_{ij}$, donde $\delta$ es la delta de Kronecker.

				\begin{hint}
					Hay un emparejamiento $L$-$\Z$-bilineal $A \times L \to L$ (donde $A$ tiene estructura de módulo por
					la derecha como $(\sigma\cdot x)(y) = \sigma(y)\cdot x$) dado por la evaluación mediante el
					cual ústed querrá extraer un emparejamiento $L$-bilineal no degenerado.
				\end{hint}

			\item Pruebe que
				\[
					k := \{ \alpha \in L : \forall \sigma \in A \qquad \alpha\cdot\sigma = \sigma\cdot\alpha \} \subseteq L
				\]
				es un subcuerpo de $L$ y que cada $\sigma_j \in A$ manda $\sigma_j \colon L \to k$.
				% $\alpha\cdot \sigma_j = \sigma_j\cdot \alpha$ para todo $\alpha \in k$.
				% que $[ L : k ] = n$ y que $A = \End_k(L)$.

			\item Pruebe que $\alpha_1, \dots, \alpha_n \in L$ (dados por el primer inciso) forman una $k$-base.
				% En particular, $$

			\item\textbf{Correspondencia de Jacobson-Bourbaki} (\cite{jacobson:galois}, Th.~I.2):\\
				Concluya que $[L : k] = n$ y que $A = \End_k(L)$.
		\end{enumerate}
\end{enumerate}
La razón de la inclusión de la \emph{correspondencia de Jacobson-Bourbaki} está en que, en cierto modo, generaliza la correspondencia
clásica de Galois: incluye tanto al caso finito, como ciertos casos de extensiones inseparables; vid.\ \cite{jacobson:galois}.

\begin{additional}
\section{Teoría de Galois profinita}
Hay, asimismo,
\lookup
una generalización de la teoría de Galois al caso infinito, para la cual se requiere de la noción categorial de <<límite
inverso>> (vid.\ \cite[490]{aluffi:algebra});
daremos primero un contraejemplo ilustrativo y opcional:
\begin{enumerate}[A., ref=\Alph*]
	\item\label{ex:galois_as_inv_lim}
		Sea $K/k$ una extensión algebraica de Galois posiblemente infinita.
		Vamos a considerar el \emph{conjunto dirigido} (o <<categoría de índices>>) $\mathscr{I}$ cuyos objetos son subextensiones $k
		\subseteq F \subseteq K$ de Galois \emph{finitas}, donde $F \le F'$ (o donde hay una única flecha $F \to F'$) syss
		$F \subseteq F'$.
		Tenemos el sistema inverso (o <<funtor contravariante>>) donde $\rho^{F'}_{F} \colon \Gal(F'/k) \epicto \Gal(F/k)$ es la
		restricción para $F \le F' \in \mathscr{I}$.
		Pruebe que
		\[
			\Gal(K/k) = \liminv_{F \in \mathscr{I}} \Gal(F/k),
		\]

	\item Definamos $\Z_\ell$, el anillo de enteros $\ell$-ádicos, como el límite inverso del diagrama $\rho^n_{n-1}\colon \Z/\ell^n\Z \to
		\Z/\ell^{n-1}\Z$ (dado por $n \mod{\ell^n} \mapsto n \mod{\ell^{n-1}}$) con el conjunto dirigido $(\N, \le)$.
		Pruebe que
		\[
			\Gal(\algcl\Fp/\Fp) \cong \prod_{\ell} \Z_\ell =: \hat{\Z},
		\]
		donde $\ell$ recorre todos los números primos.

	\item\label{ex:frobenius_dense}
		Pruebe, mediante un simil del argumento diagonal de Cantor, que $\Z_\ell$ es no numerable y, por tanto, concluya que existe
		$\sigma \in \Gal(\algcl\Fp/\Fp)$ que no está en el grupo generado por el automorfismo de Frobenius $\Frob_p$.
		No obstante, el cuerpo fijo por $\langle \Frob_p \rangle$ es $\Fp$, pese a que $\langle \Frob_p \rangle \ne
		\Gal(\algcl\Fp/\Fp)$.
\end{enumerate}
Esto prueba que la biyección entre subgrupos y subextensiones se rompe en el caso infinito.
¿Cómo se arregla?
Mediante el ejercicio~\ref{ex:galois_as_inv_lim}, vemos que el grupo de Galois es el límite inverso de grupos finitos,%
\footnote{De ahí el nombre \textquote{profinito.}}
con lo que lo podemos dotar de la topología del límite inverso (a veces llamada \emph{topología de Krull}).
Ahora, habrá una biyección entre subextensiones y subgrupos \emph{cerrados} del grupo de Galois.
El ejercicio~\ref{ex:frobenius_dense} muestra entonces que $\langle \Frob_p \rangle$ es denso en $\Gal(\algcl\Fp/\Fp)$ con dicha topología.
Una referencia del tema es
% \citeauthor{szamuely:fund_grp}~\cite{szamuely:fund_grp}.
\citeauthor{neukirch:algebraic}~\cite{neukirch:algebraic}, \S\S IV.1-3.
\end{additional}

\nocite{lang:algebra}

\printbibliography

\end{document}
