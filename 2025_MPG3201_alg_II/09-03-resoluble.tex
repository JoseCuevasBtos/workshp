\documentclass[11pt, reqno]{amsart}

\usepackage{../ayud-template}
../LaTeX/general.tex

\usepackage{tikz}
\usetikzlibrary{babel,cd}

\usepackage{minted}
\usemintedstyle{colorful}
\setminted{
	mathescape,
	linenos,
	autogobble,
	bgcolor=black!5,
	tabsize=2,
	fontsize=\small
}

% \usepackage{../ayud-template}
% ../LaTeX/general.tex
% % \input{../graphics.tex}

\usepackage{multicol}

\title{Extensiones resolubles}
\date{\DTMdate{2025-09-03}}

\author{José Cuevas Barrientos}
\email{josecuevasbtos@uc.cl}
\urladdr{https://josecuevas.xyz/teach/2025-2-alg/}

\logo{../puc_negro.png}
\institution{Pontificia Universidad Católica de Chile y Universidad de Chile}
\department{Facultad de Matemáticas}
\course{Álgebra II}
\coursecode{MPG3201}
\professor{José Samper}

\begin{document}

\maketitle

\section{Ejercicios}
\begin{enumerate}
	\item Sean $K \subseteq L_1, \dots, L_n \subseteq \Omega$ un conjunto de $n$ subextensiones abelianas (resp.\ resolubles).
		Pruebe que el composito $L_1 \cdots L_n \subseteq \Omega$ es abeliano (resp.\ resoluble).

	\item Sea $k$ un cuerpo de $\car k \ne 2$ y sea $f(x) \in k[x]$ un polinomio separable con raíces
		\[
			f(x) = \prod_{j=1}^{n} (x - \alpha_j) \in \algcl k[x].
		\]
		Definimos el discriminante como $\Delta := \prod_{i < j} (\alpha_i - \alpha_j)^2$ y sea $L$ el cuerpo de escisión de $f(x)$.
		Tras identificar $\Gal(L/k) \le S_n$ mediante la permutación de los $\alpha_j$'s, pruebe que $\Gal(L/k) \cap A_n = \Gal(L/k(\sqrt{\Delta}))$.

	\item\label{ex:quartic_comp} \textbf{Clasificación computacional de cuárticas:}
		Sea $k$ un cuerpo de $\car k \ne 2$ y sea
		\[
			f(x) := x^4 + ax^3 + bx^2 + cx + d = \prod_{j=1}^{4} (x - \alpha_j).
		\]
		una cuártica irreducible separable.
		\begin{enumerate}
			\item Defina el resolvente cúbico como
				\begin{equation*}
					R_3(x) := (x - (\alpha_1\alpha_2 + \alpha_3\alpha_4))(x - (\alpha_1\alpha_3 + \alpha_2\alpha_4))(x -
					(\alpha_1\alpha_4 + \alpha_2\alpha_3)).
				\end{equation*}
				Pruebe que $R_3(x) \in k[x]$ es separable.

			\item Sea $L/k$ el cuerpo de escisión de $f(x)$.
				Pruebe que
				\[
					\begin{array}{cll}
						\Delta & R_3(x) \in k[x] & \Gal(L/k) \\
						\hline
						\ne\square & \text{irreducible} & S_4 \\
						\ne\square &   \text{reducible} & D_4 \text{ o } C_4 \\
						= \square & \text{irreducible} & A_4 \\
						= \square &   \text{reducible} & K_4 \\
					\end{array}
				\]

			\item Sea $k = \Q$ y suponga que $\Delta \ne \square$ y $R_3 \in \Q[x]$ es reducible.
				Pruebe que si $\Gal(L/k) \cong C_4$, entonces $\Delta > 0$.
				(O recíprocamente, si $\Delta < 0$, entonces $\Gal(L/k) \cong D_4$.)
		\end{enumerate}
		\nocite{conrad:cubic_quartic_nochar2}

	\item Recuerde que un número real $r \in \R$ se dice \strong{constructible (con regla y compás)} si existe $\Q(r) \subseteq \Q(r_1,
		\dots, r_n)$, donde cada $r_j$ es de la forma $\sqrt{a + 1}$ para algún $a \in \Q(r_1, \dots, r_{j-1}) \cap \R_{>0}$.
		Un número complejo $z \in \R$ se dice \emph{constructible} si $\Re z$, $\Im z \in \R$ lo son.
		Una extensión se dice \emph{constructible} si todos sus elementos lo son.

		Pruebe que una extensión ciclotómica $\Q(\zeta_p)$, donde $p$ es primo, es constructible syss $p$ es un \emph{primo de
		Fermat}, es decir, de la forma $p = 2^{2^n} + 1$.
		\begin{hint}
			Emplee el calculo que ya ha hecho del grupo de Galois y el dato (que no tiene que probar) que todo primo de la forma
			$2^a + 1$ es necesariamente un primo de Fermat.
		\end{hint}
		\nocite{jacobson:basic}

	% \item Pruebe que
	% 	\[
	% 		\Gal(\algcl\Fp/\Fp) \cong \widehat\Z = \prod_{p} \Z_p.
	% 	\]

	% \item Llamemos $\Q(\zeta_\infty) := \bigcup_{n\in\N} \Q(\zeta_n) \subseteq \C$.
	% 	Pruebe que
	% 	\[
	% 		\Gal(\Q(\zeta_\infty) / \Q) \cong \widehat{\Z}^\times = \prod_{p} \Z_p^\times.
	% 	\]
\end{enumerate}

\begin{additional}
\appendix
\section{Comentarios adicionales}
Uno puede verificar que el discriminante $\Delta$ se puede calcular de manera explícita como el resultante de $f(x)$ y $f'(x)$, que es por
definición el determinante de una matriz a coeficientes en $k$.
En general, dicha matriz es bastante grande, pero por ejemplo \textsf{Sage} puede calcular rapidamente dicho número así:
\begin{minted}{sage}
R, t = QQ['t'].objgen()
f = t^3 + t^2 + t + 1
f.discriminant()
\end{minted}
Con ello, podemos determinar fácilmente el grupo de Galois de una extensión de grado 3.

Mediante el teorema de polinomios simétricos uno puede dar fórmulas explícitas para $\Delta$ y los coeficientes de $R_3(x)$; esta es
\[
	% \Delta := a^2b^2c^2 - 4a^3c^3 - 4a^2b^3d + 18a^3bcd - 27a^4d^2 - 4b^3c^2 + 18abc^3 + 16b^4d - 80ab^2cd - 6a^2c^2d + 144a^2bd^2 - 27c^4 + 144bc^2d - 128b^2d^2 - 192acd^2 + 256d^3
	R_3(x) := x^3 - bx^2 + (ac - 4d)x - (a^2d + c^2 - 4bd).
\]
El discriminante es más largo, puede obtener la fórmula general así:
\begin{minted}{sage}
R.<a, b, c, d> = QQ['a', 'b', 'c', 'd']
S.<x> = R[]
(x^4 + a*x^3 + b*x^2 + c*x + d).discriminant()
\end{minted}
Casos particulares son $\Delta(x^4 + ax + b) = -27a^4 + 256b^3$ y $\Delta(x^4 + ax^2 + b) = 16b(a^2 - 4b)^2$.

Al ser $R_3(x)$ mónico, una sencilla aplicación del teorema de las raíces racionales da un criterio computacional para su irreducibilidad;
así, en efecto, el problema~\ref{ex:quartic_comp} arroja un algoritmo para determinar (casi) completamente el grupo de Galois de una
cuártica sobre $\Q$.

\printbibliography
\end{additional}

\end{document}
