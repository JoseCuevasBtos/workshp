\documentclass[11pt, reqno]{amsart}

\usepackage{../ayud-template}
\input{../general.tex}

\usepackage{tikz}
\usetikzlibrary{babel,cd}

\title{Representaciones}
\date{\DTMdate{2025-09-24}}

\author{José Cuevas Barrientos}
\email{josecuevasbtos@uc.cl}
\urladdr{https://josecuevas.xyz/teach/2025-2-alg/}

\logo{../puc_negro.png}
\institution{Pontificia Universidad Católica de Chile y Universidad de Chile}
\department{Facultad de Matemáticas}
\course{Álgebra II}
\coursecode{MPG3201}
\professor{José Samper}

\begin{document}

\maketitle

\section{Ejercicios}
A lo largo de esta sección, $G$ denotará un grupo finito posiblemente no conmutativo y $K$ denotará un cuerpo
(puede suponer $K = \C$ si prefiere).
Al hablar de anillos en esta ayudantía, los asumiremos \emph{unitarios} (con neutro multiplicativo), \emph{asociativos}
y posiblemente \emph{no conmutativos}.

\begin{enumerate}
	\item (Representaciones como módulos)
		Denotaremos por $K[G]$ al grupo abeliano de sumas formales $\sum_{g\in G} a_g g$, donde cada $a_g \in K$.
		Entonces $K[G]$ es un anillo con el producto
		\[
			\left( \sum_{h\in G} a_h h \right)\left( \sum_{j\in G} b_j j \right)
			= \sum_{g\in G} \left( \sum_{hj=g} a_hb_j \right) g.
		\]
		\begin{enumerate}
			\item Sea $\rho \colon G \to \GL_K(V)$ una representación, es decir, un homomorfismo de grupos, donde
				$V$ es un $K$-espacio vectorial; denotaremos $\rho_g := \rho(g) \in \Aut(V)$ para un $g \in G$.
				Pruebe que $V$ es naturalmente un $K[G]$-módulo (izquierdo) con la operación escalar sobre $\vec v \in V$
				dada por
				\[
					\left( \sum_{g\in G} a_g g \right)\cdot\vec v := \sum_{g\in G} a_g \rho_g(\vec v).
				\]
				Recíprocamente, pruebe que todo $K[G]$-módulo $M$ da lugar a una única representación $G \to
				\GL_K(M)$.
				Más aún, una subrepresentación corresponde a un submódulo mediante esta construcción.

				\lookup
				En lenguaje sofisticado, diríamos que la categoría de representaciones y la de $K[G]$-módulos son
				equivalentes.

			\item Describa qué representación corresponde al $K[G]$-módulo libre $K[G]$.
				A ésta le llamaremos la \strong{representación regular} de $G$.
		\end{enumerate}

		\newex
	\item Sea $\varphi \colon H \to G$ un homomorfismo de grupos.
		\begin{enumerate}
			\item Pruebe que induce un homomorfismo de $K$-álgebras $K[H] \to K[G]$.
				Mediante este, toda representación de $G$ se restringe a una representación de $H$.

			\item ¿Bajo qué hipótesis $K[G]$ es un $K[H]$-módulo libre (es decir, cuando su representación es suma directa de las regulares)?
				En cuyo caso, ¿qué rango tiene?

			\item También mediante $K[H] \to K[G]$ note que toda representación de $H$ induce una
				representación de $G$ dada por asociarle al $K[H]$-módulo izquierdo $V$ el tensor $K[G]
				\otimes_{K[H]} M$.
				A esta le llamamos la \strong{representación inducida}.

				Describa la restricción en $H$ de la inducida cuando $H \nsl G$ es un subgrupo normal y existe $L < G$ tales que $G = N \sprod[] L$.
		\end{enumerate}

		\newex
	\item \textbf{Representaciones del grupo diedral:}
		Sea
		\[
			D_{2n} = \langle r, s : r^n = s^2 = 1, \; srs = r^{-1} \rangle
		\]
		el grupo diedral de cardinalidad $2n$.
		\begin{enumerate}
			\item Pruebe que las únicas representaciones lineales se factorizan por $\chi \colon D_{2n} \to \{ \pm 1 \} \le \C^\times$.
				Acto seguido, calcule cuáles hay.

			\item Mediante la inclusión $C_n \cong \langle r \rangle \le D_{2n}$, describa las representaciones inducidas en $D_{2n}$.
				¿Cuáles de ellas son irreducibles? ¿Cuántas de ellas hay salvo isomorfismo?

			\item Concluya cuáles son todas las representaciones irreducibles de $D_{2n}$ cuando $n$ es par.
		\end{enumerate}
		\nocite{serre:representations}

		\newex
	\item\lookup Sea $G$ un grupo finito.
		En clases vio que toda representación compleja de $G$ se descompone como suma de subrepresentaciones irreducibles;
		el lector atento podrá leer la demostración y verificar que de hecho esto también es cierto si trabajamos con
		representaciones sobre $\Q$ o incluso sobre cuerpos $K$ de $\car K \nmid |G|$.

		Pruebe, sin embargo, que si $p \mid G$ y $K$ tiene $\car K = p$, entonces existe una representación de $G$ en $K$ que no
		admite dicha descomposición.

		\newex
	\item Considere el grupo simétrico $S_3$.
		Se vio en clases que hay tres representaciones simples salvo isomorfismo:
		la trivial $\C_0$, el signo $\C_\sign$ y la representación estándar $V_{\rm st}$.
		\begin{enumerate}
			\item Pruebe que, dada una representación lineal $\C_\chi$ y una representación simple $W$ de un grupo finito
				arbitrario $G$, el tensor $\C_\chi \otimes W$ es también simple.

			\item Pruebe que
				\[
					% V_{\rm st} \otimes \C_{\sign} \cong V_{\rm st}
					% \qquad \text{y} \qquad
					V_{\rm st} \otimes V_{\rm st} \cong \C_0 \oplus \C_{\sign} \oplus V_{\rm st}.
				\]

			\item Calcule las potencias tensoriales $V_{\rm st}^{\otimes n}$.

			\item Sea $R \cong \C_0 \otimes V_{\rm st}$ la representación de permutación.
				Con lo anterior, pruebe que $\Sym^{n+6}(V_{\rm st}) \cong \Sym^n(V) \oplus R$.
		\end{enumerate}
\end{enumerate}

\begin{additional}
\appendix
\section{Comentarios adicionales}
% % Uno puede empujar la teoría de anillos suficientemente lejos para
% El ejercicio~\ref{ex:ring_interpr} muestra que conocimientos de álgebra \emph{no} conmutativa, aunque sea acerca de \textquote{anillos
% básicos}%
% \footnote{El mismo ejercicio prueba que el álgebra de representaciones $K[G]$ es artiniano bilateral y semisimple, lo cual es bastante restrictivo.}
% tiene repercusiones en la teoría de representaciones.
% Este es el enfoque de \citeauthor{jacobson:basic}~\cite{jacobson:basic}, Vol.~2, \S\S 5.1-5.3.

El lector notará que exigirle al grupo que sea \emph{finito} es, en cierto modo, una condición de pequeñez y una de sus ventajas es que
equivale a que el anillo sea noetheriano.
Así, uno está tentado a preguntarse si podemos admitir una teoría de representaciones --bajo hipótesis adicionales-- en grupos de otra
índole (ejemplos de interés podrían ser subgrupos y/o cocientes de $\GL_n(\C)$, y grupos de Galois infinitos).
En muchos casos sí, pero para ello debemos dotar al cuerpo y al grupo de estructura adicional (tradicionalmente, una topología como mínimo)
y restringirnos a trabajar con representaciones compatibles (en nuestro ejemplo, continuas); suele ser el caso que los grupos
\emph{compactos} jueguen el rol ahora de los grupos finitos.
% Resultados como la \textquote{ortogonalidad de caracteres} suelen ser tratados, de manera aparentemente irónica, por el análisis armónico
% abstracto.

\printbibliography
\end{additional}

\end{document}
