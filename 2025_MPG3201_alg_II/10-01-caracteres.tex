\documentclass[11pt, reqno]{amsart}

\usepackage{../ayud-template}
\input{../general.tex}

\usepackage{tikz}
\usetikzlibrary{babel,cd}

\usepackage{minted}
\usemintedstyle{colorful}
\setminted{
	mathescape,
	linenos,
	autogobble,
	bgcolor=black!5,
	tabsize=2,
	fontsize=\small
}

% \usepackage{../ayud-template}
% \input{../general.tex}
% % \input{../graphics.tex}

\usepackage{multicol}

\title{Caracteres}
\date{\DTMdate{2025-10-01}}

\author{José Cuevas Barrientos}
\email{josecuevasbtos@uc.cl}
\urladdr{https://josecuevas.xyz/teach/2025-2-alg/}

\logo{../puc_negro.png}
\institution{Pontificia Universidad Católica de Chile y Universidad de Chile}
\department{Facultad de Matemáticas}
\course{Álgebra II}
\coursecode{MPG3201}
\professor{José Samper}

\begin{document}

\maketitle

\section{Ejercicios}
\begin{center}
	\slshape
	A lo largo de esta ayudantía, siempre consideraremos a $\C$ como cuerpo de coeficientes.
\end{center}
\begin{enumerate}
	\item Calcule la tabla de caracteres simples del grupo de cuaterniones
		\[
			Q_8 = \langle i, j, k : i^2 = j^2 = k^2 = -1, \quad ijk = -1 \rangle.
		\]
		% \begin{hint}
		% 	Para esto podría ser útil expresar a $Q_8$ como producto semidirecto.
		% \end{hint}

		\newex
	\item Calcule la tabla de caracteres simples de $A_4$.

		\newex
	\item\lookright Recuerde que el grupo simétrico $\mathfrak{S}_n$ siempre admite la \emph{representación por permutación} $\rho \acts
		\C^n$ dada por $\sigma \cdot (v_1, \dots, v_n) = (v_{\sigma(1)}, \dots, v_{\sigma(n)})$.
		Esta fija al vector $(1, \dots, 1)$ que genera la subrepresentación trivial, el complemento de ella es la
		\strong{representación estándar} cuyo caracter es
		\[
			\chi_{\rm st}(\sigma) = \chi_{\rm perm}(\sigma) - \chi_0(\sigma) = \chi_{\rm perm}(\sigma) - 1.
		\]
		Pruebe que la representación estándar es siempre simple.
		\begin{hint}
			Ingénieselas para convertir el problema en contar dimensión del subespacio fijo de la acción por doble permutación
			$S_n \acts \C^{n^2}$ dada por $\sigma\cdot(v_{i,j})_{i,j=1}^n := (v_{\sigma(i), \sigma(j)})_{i,j}$.
		\end{hint}

		\newex
	\item\label{ex:degree_divides_order}
		Sea $G$ un grupo finito.
		\begin{enumerate}
			\item Si $C_1, \dots, C_h$ son las clases de conjugación de $G$, pruebe que los elementos
				\[
					c_j := \sum_{g\in C_j} [g] \in \C[G],
				\]
				forman una $\C$-base para el centro%
				\footnote{Relativo al producto, obvio.}
				$Z(\C[G])$.

			\item\lookst
				Sea $\chi$ un caracter simple de $G$ de grado $n := \chi(1)$, y sea $g \in C_j \subseteq G$.
				Pruebe que $|C_j| \frac{1}{n} \chi(g)$ es un entero algebraico (i.e., es raíz de un polinomio mónico con coeficientes en $\Z$).
				\begin{hint}
					Para ello, note que la representación $\rho$ que induce a $\chi$ satisface que $\rho(c_j) = b_j\Id$
					y pruebe que $b_j$ es el valor propio de una matriz a coeficientes enteros.
				\end{hint}

			\item
				Pruebe que el grado de toda representación simple divide al orden de $G$.
				\begin{hint}
					Para esto podría necesitar que los enteros algebraicos son cerrados bajo suma y productos, y que los
					enteros algebraicos de $\Q$ son exactamente los enteros $\Z$.
				\end{hint}
		\end{enumerate}
		\nocite{huppert:finite_i}

		\newex
	\item Sea $H \le G$ un subgrupo de un grupo finito y sea $\chi$ el caracter de una representación $\rho \colon H \acts V$.
		Recuerde de la ayudantía anterior (ej.~2c) que asociado a $V^\rho$ tenemos la representación inducida $\Ind^G_H(\rho)$ cuyo
		caracter denotaremos $\Ind^G_H(\chi)$.
		Pruebe que tenemos la siguiente fórmula para todo $g \in G$:
		\[
			\Ind_H^G(\chi)(g) = \frac{1}{|H|} \sum_{\substack{t\in G \\ tgt^{-1} \in H}} \chi(t^{-1}gt).
		\]
		\nocite{serre:representations}
\end{enumerate}

\begin{additional}
\appendix
\section{Comentarios adicionales breves}
Parte del objetivo del primer ejercicio está en que tras calcular la tabla de caracteres de $Q_8$ el lector puede observar que coincide con
la del grupo diedral $D_8$, de modo que dos grupos no isomorfos pueden tener la misma tabla de caracteres.
Esto es interesante porque la tabla de caracteres determina completamente a un grupo abeliano finito, por ejemplo; esto es un buen ejercicio
para el lector.

Así mismo, hay una serie de observaciones adicionales que el lector podría hacer tras calcular la tabla de un grupo.
Por ejemplo, para $Q_8$ la representación simple de grado 2 es inducida del subgrupo normal $\langle i \rangle$;
para $A_4$ todas las representaciones simples son restricciones, o bien de un cociente, o bien de $S_4$.
Esto ejemplifica la utilidad de tener criterios sencillos para la irreducibilidad de caracteres inducidos, para lo cual recomendamos al
lector leer sobre el criterio de Mackey en \cite{serre:representations}, \S 7.4.

El ejercicio~\ref{ex:degree_divides_order} lo extraje de \cite{huppert:finite_i}.
Dicha referencia incluye después de ello una serie de aplicaciones de la teoría de caracteres a preguntas exclusivamente de grupos finitos,
como el famoso teorema de Burnside de la resolubilidad de grupos de orden $p^aq^b$.
Por un largo tiempo se desconocieron pruebas con exclusivamente lenguaje de grupos de dicho teorema; además de la prueba con
representaciones, hay otra que emplea cohomología de grupos.

\printbibliography
\end{additional}

\end{document}
