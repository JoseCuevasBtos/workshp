\documentclass[11pt, reqno]{amsart}

\usepackage{../ayud-template}
../LaTeX/general.tex

\usepackage{tikz}
\usetikzlibrary{babel,cd}

\usepackage{minted}
\usemintedstyle{colorful}
\setminted{
	mathescape,
	% linenos,
	autogobble,
	bgcolor=black!5,
	tabsize=2,
	fontsize=\small
}

% \usepackage{../ayud-template}
% ../LaTeX/general.tex
% % \input{../graphics.tex}

\usepackage{multicol}

\title{Más acerca de caracteres}
\date{\DTMdate{2025-10-15}}

\author{José Cuevas Barrientos}
\email{josecuevasbtos@uc.cl}
\urladdr{https://josecuevas.xyz/teach/2025-2-alg/}

\logo{../puc_negro.png}
\institution{Pontificia Universidad Católica de Chile y Universidad de Chile}
\department{Facultad de Matemáticas}
\course{Álgebra II}
\coursecode{MPG3201}
\professor{José Samper}

\begin{document}

\maketitle

\section{Ejercicios}
\begin{center}
	\slshape
	A lo largo de esta ayudantía, siempre $G$ denotará un grupo finito y $\C$ será el cuerpo de coeficientes para sus representaciones.
\end{center}

% \begin{enumerate}
% 	\item\label{ex:degree_divides_order}
% 		% Sea $G$ un grupo finito.
% 		\begin{enumerate}
% 			\item Si $C_1, \dots, C_h$ son las clases de conjugación de $G$, pruebe que los elementos
% 				\[
% 					c_j := \sum_{g\in C_j} [g] \in \C[G],
% 				\]
% 				forman una $\C$-base para el centro%
% 				\footnote{Relativo al producto, obvio.}
% 				$Z(\C[G])$.

% 			\item Pruebe que, para todo caracter $\chi$ y todo $g \in G$, se cumple que $\chi(g)$ es un entero algebraico (i.e., es raíz de un polinomio mónico con coeficientes en $\Z$).

% 			\item\lookst
% 				Sea $\chi$ un caracter simple de $G$ de grado $n := \chi(1)$, y sea $g \in C_j \subseteq G$.
% 				Pruebe que $|C_j| \frac{1}{n} \chi(g)$ es un entero algebraico.
% 				\begin{hint}
% 					Para ello, note que la representación $\rho$ que induce a $\chi$ satisface que $\rho(c_j) = b_j\Id$
% 					y pruebe que $b_j$ es el valor propio de una matriz a coeficientes enteros.
% 				\end{hint}
% 		\end{enumerate}
% \end{enumerate}

% \newex
% Los siguientes dos ejercicios muestran aplicaciones del ejercicio anterior:
% \begin{enumerate}[resume]
% 	\item Pruebe que el grado de toda representación simple divide al orden de $G$.
% 		\begin{hint}
% 			Para esto podría necesitar que los enteros algebraicos son cerrados bajo suma y productos, y que los
% 			enteros algebraicos de $\Q$ son exactamente los enteros $\Z$.
% 		\end{hint}
% 		\nocite{huppert:finite_i}

% 	\item
% 		\begin{enumerate}
% 			\item Sea $\rho \colon G \to \GL_n(\C)$ una representación simple de grado $n$ y sea $g \in G$ un elemento cuya clase de conjugación tiene $m$ elementos.
% 				Pruebe que si $m$ y $n$ son coprimos, entonces o bien $\rho(g) = \lambda\Id$ para algún $\lambda \in \C^\times$, o bien $\tr\rho(g) = 0$.

% 			\item Pruebe que si $G$ es no abeliano y tiene una clase de conjugación $\ne \{ 1 \}$ cuya cardinalidad es la
% 				potencia de un primo, entonces no es un grupo simple.
% 		\end{enumerate}

% 	\item \textbf{El grupo aditivo de Grothendieck:}
% 		Sea $\Lambda$ un anillo.
% 		Se define el grupo de Grothendieck de $\Lambda$, denotado $\mathsf{K}^0(\Lambda)$, como el grupo abeliano libre cuyos
% 		genereadores son los $\Lambda$-módulos izquierdos finitamente generados $[M]$ cocientado por las relaciones $[M_1] + [M_3] = [M_2]$
% 		si existe una sucesión exacta $0 \to M_1 \to M_2 \to M_3 \to 0$.

% 		\begin{enumerate}
% 			\item Pruebe que, para un cuerpo $k$, se cumple que $\mathsf{K}^0(k) \cong \Z$ y que la función que a un $k$-espacio
% 				vectorial $V$ le asocia $V \mapsto [V] \in \mathsf{K}^0(k)$ corresponde a $\dim_k$ bajo este isomorfismo.

% 			\item Pruebe también que $\mathsf{K}^0(\Z) \cong \Z$ y describa qué le asocia a un grupo abeliano finitamente
% 				generado.

% 			\item Sea $G$ un grupo finito y sean $\chi_1, \dots, \chi_h$ sus caracteres simples.
% 				Pruebe que
% 				\[
% 					\mathsf{K}^0( \C[G] ) \cong \bigoplus_{j=1}^h \Z\cdot\chi_j \le \C^{G}.
% 				\]
% 		\end{enumerate}
% \end{enumerate}
\begin{enumerate}
	\item Pruebe que la multiplicidad de una representación simple en la representación regular es su grado (i.e., la dimensión del espacio vectorial donde actúa).

		\newex
	\item Sean $V, W, V', W'$ un conjunto de representaciones.
		Recuerde que $\Hom_\C(V, W)$ tiene una representación natural definida por $g\odot\varphi(v) := g \varphi(g^{-1}v)$.
		\begin{enumerate}
			\item Pruebe que $\Hom_\C(V, W) \otimes_\C \Hom_\C(V', W') \cong \Hom_\C(V \otimes V', W \otimes W')$.

			\item Denotemos por $\Hom_{\C[G]}(V, W)$ al módulo de homomorfismos \emph{de representaciones.}
				Pruebe que
				\[
					\dim\Hom_{\C[G]}(V, W) = \langle \chi_V, \chi_W \rangle.
				\]
				
			\item Concluya el \textquote{lema de Yoneda} para representaciones:
				Si $V, V'$ son tales que $\Hom_{\C[G]}(V, W) \cong \Hom_{\C[G]}(V', W)$, entonces $V \cong V'$.
		\end{enumerate}

		% \newex
	% \item \textbf{Reciprocidad de Frobenius:}
		% Sea $\varphi \colon H \to G$ un homomorfismo de grupos finitos.
		% \begin{enumerate}
		% 	\item Pruebe que dado un caracter $\chi$ de $H$ y $\psi$ de $G$, se cumple que
		% 		\[
		% 			\textstyle
		% 			\langle \chi, \Res_H^G(\psi) \rangle_H = \langle \Ind_H^G(\chi), \psi \rangle_G.
		% 		\]

		% 	\item Pruebe que, dadas representaciones $V$ sobre $G$ y $W$ sobre $H$, se cumple que
		% 		\[
		% 			\textstyle
		% 			V \otimes_\C \Ind_H^G(W) \cong \Ind_H^G(\Res_H^G(V) \otimes_\C W).
		% 		\]
		% \end{enumerate}

		\newex
	\item Sea $H \le G$ un subgrupo de un grupo finito y sea $\chi$ el caracter de una representación $\rho \colon H \acts V$.
		Recuerde de la ayudantía anterior (ej.~2c) que asociado a $V^\rho$ tenemos la representación inducida $\Ind^G_H(\rho)$ cuyo
		caracter denotaremos $\Ind^G_H(\chi)$.
		Pruebe que tenemos la siguiente fórmula para todo $g \in G$:
		\[
			\Ind_H^G(\chi)(g) = \frac{1}{|H|} \sum_{\substack{t\in G \\ tgt^{-1} \in H}} \chi(t^{-1}gt).
		\]
		\nocite{serre:representations}

% 		\newex
% 	\item un grupo finito $G$ se dice \strong{supersoluble} si existe una serie de composición
% 		\[
% 			\{ 1 \} =: G_0 \nsl G_1 \nsl \cdots \nsl G_n := G,
% 		\]
% 		en donde cada $G_j \nsl G$ y donde los cocientes sucesivos $G_j/G_{j-1}$ son cíclicos.

		\newex
	\item\label{ex:heisenberg_char_table}
		Calcule la tabla de caracteres para el grupo de Heisenberg
		\[
			H(\Fp[3]) =
			\left\{
				\begin{bmatrix}
					1 & a & b \\
					0 & 1 & c \\
					0 & 0 & 1
				\end{bmatrix} : a, b, c \in \Fp[3]
			\right\} \le \SL_2(\Fp[3]).
		\]
\end{enumerate}

\begin{additional}
\appendix
\section{Ejercicios propuestos}
\begin{enumerate}
	\item Sea $G \acts V$ una representación compleja simple.
		% Se dice que $V$ es \strong{de tipo complejo} si $V \not\cong V^*$, que es \strong{de tipo real} si posee una forma simétrica
		% no degenerada $\langle -, - \rangle$ que es invariante por $G$ (i.e., $\langle gv, gw \rangle = \langle v, w \rangle$ para
		% todo $v, w \in V$), o que es \strong{de tipo cuaterniónico} si posee una forma hermitiana (i.e., tal que
		% \[
		% 	\forall v, w \in V, \; \lambda \in \C, \qquad
		% 	\langle \lambda v, w \rangle = \langle v, \overline{\lambda} w \rangle = \lambda \langle v, w \rangle
		% \]
		% y que es no degenerada) que es invariante por $G$.
		Se dice que $V$ es
		\begin{description}
			\item[De tipo complejo] Si $V \not\cong V^*$.
			\item[De tipo real] Si posee una forma simétrica no degenerada $\langle -, - \rangle$ que es invariante por $G$
				(i.e., $\langle gv, gw \rangle = \langle v, w \rangle$ para todo $v, w \in V$).
			\item[De tipo cuaterniónico] Si posee una forma hermitiana (i.e., tal que
				\[
					\forall v, w \in V, \; \lambda \in \C, \qquad
					\langle \lambda v, w \rangle = \langle v, \overline{\lambda} w \rangle = \lambda \langle v, w \rangle
				\]
				y que es no degenerada) que es invariante por $G$.
		\end{description}
		Pruebe que el anillo de endomorfismos $\End_{\R[G]}(V)$ es $\C$ (resp.\ $\Mat_2(\R)$, $\HH$) si $V$ es de tipo complejo
		(resp.\ de tipo real, de tipo cuaterniónico).

	\item Defina el índice de Frobenius-Schur como
		\[
			\mathrm{FS}(V) =
			\begin{cases}
				\phantom{-}0, & V\text{ es de tipo complejo,} \\
				\phantom{-}1, & V\text{ es de tipo real,} \\
				-1, & V\text{ es de tipo cuaterniónico.} \\
			\end{cases}
		\]
		\begin{enumerate}
			\item Pruebe que $\mathrm{FS}(V) = \dim_\C (\Sym^2V)^G - \dim_\C(\bigwedge^2 V)^G$.
				\begin{hint}
					Para esto, emplee la descomposición de representaciones $V \otimes V = \Sym^2V \oplus \bigwedge^2 V$
					y argumente por qué el subespacio $G$-invariante de $V^{\otimes 2}$ tiene dimensión 1.
				\end{hint}

			\item \textbf{Teorema de Frobenius-Schur:}
				Pruebe que la cantidad de elementos de orden 2 en $G$ es $\sum_V \dim(V)\mathrm{FS}(V)$, donde la sumatoria
				recorre las representaciones complejas simples de $G$.
		\end{enumerate}
\end{enumerate}

\section{Implementación en \textsf{sagemath}}
El ejercicio~\ref{ex:heisenberg_char_table} ya involucra un grupo un tanto \textquote{grande}, de orden $3^3$, de modo que varios de los calculos pueden resultar tediosos,
especialmente las clases de conjugación.
Afortunadamente, ya no vivimos en eras arcaicas y hoy día las computadoras pueden ayudarnos en tal tarea:
para ello, voy a ocupar la observación de que $H(\Fp)$ está generado por
\[
	\begin{bmatrix}
		1 & 1 & 0 \\
		0 & 1 & 0 \\
		0 & 0 & 1
	\end{bmatrix}, \qquad
	\begin{bmatrix}
		1 & 0 & 1 \\
		0 & 1 & 0 \\
		0 & 0 & 1
	\end{bmatrix}, \qquad
	\begin{bmatrix}
		1 & 0 & 0 \\
		0 & 1 & 1 \\
		0 & 0 & 1
	\end{bmatrix}.
\]
Y así, puedo construir el grupo de Heisenberg en \textsf{sagemath} mediante:
\begin{minted}{sage}
MS = MatrixSpace(GF(3), 3, 3)
gens = [
        MS([[1, 1, 0], [0, 1, 0], [0, 0, 1]]),
        MS([[1, 0, 1], [0, 1, 0], [0, 0, 1]]),
        MS([[1, 0, 0], [0, 1, 1], [0, 0, 1]]),
        ]
Heis = MatrixGroup(gens)
\end{minted}
(Aquí \mintinline{sage}{GF(3)} denota el cuerpo de orden 3. Perfectamente todo lo siguiente aplica cambiándolo por \mintinline{sage}{GF(p)} con $p$ primo a elección.)

El método \mintinline{sage}{Heis.conjugacy_classes()} nos otorga las clases de conjugación, pero al imprimirlas no lo hace de manera muy legible que digamos.
Si queremos ver un representante por clase de conjugación, podemos emplear:
\begin{minted}{python}
Heis.conjugacy_classes_representatives()
\end{minted}
No obstante, si queremos ver cada clase como una lista, el siguiente código en \textsf{Python} arroja una visualización más amena:
\begin{minted}{python}
def prettyprint(L: list):
    rank = len(L[0].list())
    lines = [ [] for _ in range(rank) ]
    for matrix in L:
        for i in range(rank):
            lines[i].append( ' '.join([ str(a) for a in matrix.list()[i] ]) )
    for l in lines:
        print(l)
    print()

for cl in H.conjugacy_classes():
    prettyprint(cl.list())
\end{minted}
% \pagebreak
Finalmente, de querer verificar que el cálculo de una tabla estuvo correcto, podemos emplear el método
\begin{minted}{sage}
Heis.character_table()
\end{minted}

\printbibliography
\end{additional}

\end{document}
