\documentclass[11pt, reqno]{amsart}

\usepackage{../ayud-template}
../LaTeX/general.tex

\usepackage{tikz}
\usetikzlibrary{babel,cd}

\usepackage{minted}
\usemintedstyle{colorful}
\setminted{
	mathescape,
	% linenos,
	autogobble,
	bgcolor=black!5,
	tabsize=2,
	fontsize=\small
}

% \usepackage{../ayud-template}
% ../LaTeX/general.tex
% % \input{../graphics.tex}

\usepackage{multicol}

\title{Caracteres III: Repaso}
\date{\DTMdate{2025-10-22}}

\author{José Cuevas Barrientos}
\email{josecuevasbtos@uc.cl}
\urladdr{https://josecuevas.xyz/teach/2025-2-alg/}

\logo{../puc_negro.png}
\institution{Pontificia Universidad Católica de Chile y Universidad de Chile}
\department{Facultad de Matemáticas}
\course{Álgebra II}
\coursecode{MPG3201}
\professor{José Samper}

\begin{document}

\maketitle

\section{Ejercicios}
\begin{center}
	\slshape
	En esta ayudantía, $G$ es un grupo finito y las representaciones son complejas.
\end{center}
\begin{enumerate}
	% Ortogonalidad
	\item Sea $R$ un anillo (posiblemente no conmutativo) que admita descomposición $R = R_1 \times \cdots \times R_r$ con $R_j$ anillos no nulos.
		Pruebe que si $R$ tiene $n$ distintos módulos simples (izquierdos o derechos), entonces $n \ge r$.

	\item Sea $g \ne 1 \in G$.
		Pruebe que existe una representación irreducible cuyo caracter $\chi$ satisface que $\Re\chi(g) < 0$.

	\item
		\begin{enumerate}
			\item Sea $\psi$ el caracter de una representación $V$ tal que $\psi(g) = 0$ para todo $g \ne 1$.
				Pruebe que $|G| \mid \dim V$ y que, de hecho, $V \cong \C[G]^n$ para algún $n$.
			\item Pruebe que dada una representación $W$ arbitraria, entonces $\C[G] \otimes_\C W \cong \C[G]^{\dim W}$.
		\end{enumerate}

	\item Pruebe que en $G$ todo elemento es conjugado a su inversa syss todos sus caracteres simples tienen valores en $\R$.

	\item Pruebe que un elemento $z \in G$ está en el centro syss $|\chi(z)| = |\chi(1)|$ para todo caracter $\chi$.

	\item Encuentre la tabla de caracteres del producto semidirecto
		\[
			G = C_7 \sprod[] C_3 = \langle x, y : x^7 = y^3 = 1, \; yxy^{-1} = x^2 \rangle.
		\]
		% \begin{hint}
		% 	Como es costumbre hay que calcular: clases de conjugación $1, x, x^{-1}, x^*y, x^*y^2$ y conmutador $\langle x \rangle$.
		% 	Luego las representaciones de grado 3 vienen de inducción sobre $\langle x \rangle$.
		% \end{hint}

	\item Sea $H \le G$ un subgrupo de un grupo finito y sea $\chi$ un caracter de $H$.
		Para $g \in G$ y $h \in H$ definamos $h^g := g^{-1}hg$, $h_g := ghg^{-1}$ y $\chi^g(h^g) := \chi(h)$.
		Es decir, si $h \in H \cap H^g$, entonces $\chi^g(h) = \chi(h_g)$.
		\begin{enumerate}
			\item Pruebe que
				\[
					\Ind^G_H(\chi)(h) = \sum_{\substack{g\in G \\ h \in H \cap H_g}} \chi(h^g)
				\]
			\item Pruebe que
				\[
					\Res\nolimits_H^G\Ind^G_H(V) = \bigoplus_{g \in G} \Ind_{gHg^{-1} \cap H}^H(g^{-1} V g) 
				\]
		\end{enumerate}
		\nocite{webb:representation}
\end{enumerate}

\begin{additional}
\appendix
\printbibliography
\end{additional}

\end{document}
