\documentclass[11pt, reqno]{amsart}

\usepackage{../ayud-template}
\input{../general.tex}

\usepackage{tikz}
\usetikzlibrary{babel,cd}

\usepackage{minted}
\usemintedstyle{colorful}
\setminted{
	mathescape,
	% linenos,
	autogobble,
	bgcolor=black!5,
	tabsize=2,
	fontsize=\small
}

% \usepackage{../ayud-template}
% \input{../general.tex}
% % \input{../graphics.tex}

% \usepackage{multicol}
% \DeclareMathOperator{\LT}{\text{\textsc{lt}}}
\DeclareMathOperator{\init}{in}

\title{Eliminación e ideales radicales}
\date{\DTMdate{2025-11-12}}

\author{José Cuevas Barrientos}
\email{josecuevasbtos@uc.cl}
\urladdr{https://josecuevas.xyz/teach/2025-2-alg/}

\logo{../puc_negro.png}
\institution{Pontificia Universidad Católica de Chile y Universidad de Chile}
\department{Facultad de Matemáticas}
\course{Álgebra II}
\coursecode{MPG3201}
\professor{José Samper}

\begin{document}

\maketitle

\section{Ordenes monomiales}
\begin{enumerate}
	\item Sea $\mathfrak{a} \nsl k[\vec x]$ el ideal generado por los polinomios simétricos elementales en $n$ variables.
		Para un orden monomial $\prec$ calcule $\init_\prec(\mathfrak{a})$.

	\item \textbf{Implicitación racional:}
		Sea $K$ un cuerpo infinito y considere el morfismo racional $F \colon \A^m_K \setminus W \to \A^n_K$ dado por
		\begin{align*}
			x_1 &= \frac{f_1(t_1, \dots, t_m)}{g_1(t_1, \dots, t_m)}, \\
			    &\vdots \\
			x_n &= \frac{f_n(t_1, \dots, t_m)}{g_n(t_1, \dots, t_m)}.
		\end{align*}
		Sea $\mathfrak{b} := (g_1x_1 - f_1, \dots, g_nx_n - f_n, 1 - gy) \nsle k[y, \vec t, \vec x]$, donde $g = g_1 g_2 \cdots g_n$ y $W = \VV(g)$.
		Entonces $\VV(\mathfrak{b} \cap k[\vec x]) \subseteq \A^n_K$ es la clausura de Zariski de $\Img F$.
\end{enumerate}

\section{Radicalidad}
Sea $\mathfrak{a} \nsle A$ un ideal en un anillo, se define su radical como
\[
	\rad \mathfrak{a} := \{ \alpha \in A : \exists n \ge 1 \quad \alpha^n \in \mathfrak{a} \}.
\]
Se dice que $\mathfrak{a}$ es radical si $\rad\mathfrak{a} = \mathfrak{a}$.
\begin{enumerate}[resume]
	\item Sean $\mathfrak{a, b} \nsl k[x_1, \dots, x_n]$ un par de ideales.
		Pruebe que se tienen las siguientes inclusiones
		\[
			\rad(\mathfrak{a}) \rad(\mathfrak{b}) \subseteq \rad(\mathfrak{a\cdot b}), \qquad
			\init_\prec(\rad\mathfrak{a}) \subseteq \rad( \init_\prec\mathfrak{a} ),
		\]
		donde $\prec$ es un orden monomial.
		Muestre ejemplos donde las inclusiones sean estrictas.

	\item Pruebe que si $\mathfrak{a}$ es tal que $\init_\prec(\mathfrak{a})$ es radical, entonces $\mathfrak{a}$ también.
		¿El recíproco es cierto?

	\item Considere el morfismo $f\colon \A^1 \to \A^2$ dado por $t \mapsto (t^2, t^3)$.
		Calcule el ideal $\mathfrak{a} \nsl k[x, y]$ tal que $\VV(\mathfrak{a}) = \overline{\Img f}$.
		¿Es $\mathfrak{a}$ radical?
\end{enumerate}

\begin{additional}
\appendix
\printbibliography
\end{additional}

\end{document}
