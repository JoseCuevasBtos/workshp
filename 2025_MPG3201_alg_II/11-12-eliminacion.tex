\documentclass[11pt, reqno]{amsart}

\usepackage{../ayud-template}
\input{../general.tex}

\usepackage{tikz}
\usetikzlibrary{babel,cd}

\usepackage{minted}
\usemintedstyle{colorful}
\setminted{
	mathescape,
	% linenos,
	autogobble,
	bgcolor=black!5,
	tabsize=2,
	fontsize=\small
}

% \usepackage{../ayud-template}
% \input{../general.tex}
% % \input{../graphics.tex}

% \usepackage{multicol}
% \DeclareMathOperator{\LT}{\text{\textsc{lt}}}
\DeclareMathOperator{\init}{in}

\title{Eliminación e ideales radicales}
\date{\DTMdate{2025-11-12}}

\author{José Cuevas Barrientos}
\email{josecuevasbtos@uc.cl}
\urladdr{https://josecuevas.xyz/teach/2025-2-alg/}

\logo{../puc_negro.png}
\institution{Pontificia Universidad Católica de Chile y Universidad de Chile}
\department{Facultad de Matemáticas}
\course{Álgebra II}
\coursecode{MPG3201}
\professor{José Samper}

\begin{document}

\maketitle

\section{Ordenes monomiales}
\begin{enumerate}
	\item Sea $\mathfrak{a} \nsl k[\vec x]$ el ideal generado por los polinomios simétricos elementales en $n$ variables:
		\[
			e_1(x_1, \dots, x_n) = \sum_{i=1}^{n} x_i, \qquad
			e_k(\vec x) = \sum_{1\le i_1 < \cdots < i_k \le n} x_{i_1}\cdots x_{i_k} \quad (k \le n).
		\]
		Para un orden monomial $\prec$ calcule una base de Gröbner.

	\item \textbf{Implicitización racional:}
		Sea $K$ un cuerpo infinito y considere el morfismo racional $F \colon \A^m_K \setminus W \to \A^n_K$ dado por
		% \begin{align*}
		% 	x_1 &= \frac{f_1(t_1, \dots, t_m)}{g_1(t_1, \dots, t_m)}, \\
		% 	    &\vdots \\
		% 	x_n &= \frac{f_n(t_1, \dots, t_m)}{g_n(t_1, \dots, t_m)}.
		% \end{align*}
		\[
			x_1 = \frac{f_1(t_1, \dots, t_m)}{g_1(t_1, \dots, t_m)}, \qquad
			    \cdots, \qquad
			x_n = \frac{f_n(t_1, \dots, t_m)}{g_n(t_1, \dots, t_m)};
		\]
		con $F(\vec t) = (x_1, \dots, x_n)$.

		Sea $\mathfrak{b} := (g_1x_1 - f_1, \dots, g_nx_n - f_n, 1 - gy) \nsle k[y, \vec t, \vec x]$, donde $g = g_1 g_2 \cdots g_n$ y $W = \VV(g)$.
		Entonces $\VV(\mathfrak{b} \cap k[\vec x]) \subseteq \A^n_K$ es la clausura de Zariski de $\Img F$.
\end{enumerate}

\section{Radicalidad}
Sea $\mathfrak{a} \nsle A$ un ideal en un anillo, se define su radical como
\[
	\rad \mathfrak{a} := \{ \alpha \in A : \exists n \ge 1 \quad \alpha^n \in \mathfrak{a} \}.
\]
Se dice que $\mathfrak{a}$ es radical si $\rad\mathfrak{a} = \mathfrak{a}$.
\begin{enumerate}[resume]
	\item Sean $\mathfrak{a, b} \nsl k[x_1, \dots, x_n]$ un par de ideales.
		Pruebe $\rad\mathfrak{a}$ es un ideal y que se tienen las siguientes inclusiones
		\[
			\rad(\mathfrak{a}) \rad(\mathfrak{b}) \subseteq \rad(\mathfrak{a\cdot b}), \qquad
			\init_\prec(\rad\mathfrak{a}) \subseteq \rad( \init_\prec\mathfrak{a} ),
		\]
		donde $\prec$ es un orden monomial.
		Muestre ejemplos donde las inclusiones sean estrictas.

	\item\label{ex:twisted_cubic}\lookright
		Pruebe que si $\mathfrak{a}$ es tal que $\init_\prec(\mathfrak{a})$ es radical, entonces $\mathfrak{a}$ también.
		¿El recíproco es cierto?

	\item Considere el morfismo $f\colon \A_k^1 \to \A_k^2$ dado por $t \mapsto (t^2, t^3)$.
		Calcule el ideal $\mathfrak{a} \nsl k[x, y]$ tal que $\VV(\mathfrak{a}) = \overline{\Img f}$.
		¿Es $\mathfrak{a}$ radical?
\end{enumerate}

\begin{additional}
\appendix
\section{Ejercicios propuestos}
\begin{enumerate}[resume]
	\item Calcule la(s) ecuacion(es) que determinan la superficie en $\A^3_k$ cuyos puntos $(x, y, z)$ satisfacen
		\[
			x = ts, \qquad y = ts^2, \qquad z = s^2.
		\]

	\item Usando \textsf{sagemath}, encuentre la ecuación que define la \emph{superficie de Enneper}
		\begin{align*}
			x &= 3u + 3uv^2 - u^3, \\
			y &= 3v + 3u^2v - v^3, \\
			z &= 3u^2 - 3v^2.
		\end{align*}
		(También puede tratar de hacerlo a mano y mandarme su solución, pero le advierto que no trabajo después del 20 de diciembre.)
\end{enumerate}

\section{Comentarios y sage}
El lector familiarizado con el primer capítulo de \citeauthor{hartshorne:algebraic}~\cite{hartshorne:algebraic} reconocerá que la respuesta
al ejercicio~\ref{ex:twisted_cubic} es siempre afirmativa.
Mejor aún, $\mathfrak{a}$ siempre es \emph{primo} puesto que la clausura de la imagen es irreducible (debido a que el dominio lo es).

La manera de calcular bases de Gröbner (con el orden lexicográfico) en \textsf{sagemath} es como prosigue (para el
ej.~\ref{ex:twisted_cubic}):
\begin{minted}{sage}
from sage.rings.polynomial.toy_buchberger import buchberger
set_verbose(1) # para ver los cálculos

R.<t,x,y> = PolynomialRing(QQ, order='lex')
I = R.ideal([t^2 - x, t^3 - y])
buchberger(I)
\end{minted}

\printbibliography
\end{additional}

\end{document}
