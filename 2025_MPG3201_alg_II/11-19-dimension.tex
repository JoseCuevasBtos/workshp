\documentclass[11pt, reqno]{amsart}

\usepackage{../ayud-template}
\input{../general.tex}

\usepackage{tikz}
\usetikzlibrary{babel,cd}

% \usepackage{minted}
% \usemintedstyle{colorful}
% \setminted{
% 	mathescape,
% 	% linenos,
% 	autogobble,
% 	bgcolor=black!5,
% 	tabsize=2,
% 	fontsize=\small
% }

% \usepackage{multicol}
\DeclareMathOperator{\init}{in}
\DeclareMathOperator{\HF}{HF}

\title{Dimensión y polinomios de Hilbert}
\date{\DTMdate{2025-11-19}}

\author{José Cuevas Barrientos}
\email{josecuevasbtos@uc.cl}
\urladdr{https://josecuevas.xyz/teach/2025-2-alg/}

\logo{../puc_negro.png}
\institution{Pontificia Universidad Católica de Chile y Universidad de Chile}
\department{Facultad de Matemáticas}
\course{Álgebra II}
\coursecode{MPG3201}
\professor{José Samper}

\begin{document}

\maketitle

\section{Ejercicios}
\begin{enumerate}
	\item Calcule el polinomio de Hilbert afín de los siguientes ideales:
		\begin{enumerate}
			\item $(x^3 - y^2) \nsl k[x, y]$.
			\item $(x^3y^2 + 3x^2y^2 + y^3 + 1) \nsl k[x, y]$.
			\item $(x^3yz^5, xy^3z^2) \nsl k[x, y, z]$.
			\item $(x^3 - yz^2, y^4 - x^2yz) \nsl k[x, y, z]$.
		\end{enumerate}

	\item Decimos que un polinomio $f(t) \in \Q[t]$ es \emph{numérico} si para todo entero $n \in \Z$, su evaluación $f(n) \in \Z$ es
		también entera.
		Pruebe que todo polinomio numérico es de la forma
		\[
			f(t) = \sum_{j=0}^{d} a_j\binom{t}{j}, \qquad a_j \in \Z
		\]
		donde $\displaystyle \binom{t}{j} = \frac{t(t-1)\cdots(t-j+1)}{j!}$.

	\item Pruebe que si $\mathfrak{a \subseteq b} \nsl k[\vec x]$ son ideales, entonces $\dim \VV(\mathfrak{a}) \ge \dim
		\VV(\mathfrak{b})$.
		Teniendo inclusión estricta $\mathfrak{a \subset b}$ muestre ejemplos donde las dimensiones coinciden o no.

	\item
		\begin{enumerate}
			\item Sea $(a_1, \dots, a_n) \in \A^n(k)$ un punto en el espacio afín.
				Pruebe que $\dim\{ (a_1, \dots, a_n) \} = 0$.
			\item Dado un polinomio $f(x, y)$ no constante pruebe que la dimensión de $\VV(f)$ es 1.
				(Eslógan: \textquote{las hipersuperficies en el plano son curvas (posiblemente reducibles).})
			\item\label{ex:krull_hauptidealsatz}\lookright
				Más generalmente, pruebe que si $f(\vec x) \in k[x_1, \dots, x_n]$ es no constante, entonces la
				dimensión de la hipersuperficie $\VV(f)$ es $n - 1$.
		\end{enumerate}

	\item Un polinomio $f$ en $R := k[\vec x]$ se dice \emph{homogéneo} si todos sus monomios no nulos tienen el mismo grado (total).
		Un ideal $\mathfrak{a}$ en $k[\vec x]$ se dice \strong{homogéneo} si está generado por polinomios homogéneos.
		\begin{enumerate}
			\item Defina la \emph{función de Hilbert proyectiva} de un ideal homogéneo $\mathfrak{a}$ como
				\[
					\HF_{R/\mathfrak{a}}(s) := \dim_k R_s - \dim_k \mathfrak{a}_s,
				\]
				donde el subíndice $\mathfrak{b}_s$ denota el $k$-subespacio vectorial de los polinomios homogéneos en
				$\mathfrak{b}$ de grado (total) $s$.

				Pruebe que $\HF_{R/\mathfrak{a}}(s) = \HF^{\rm af}_{R/\mathfrak{a}}(s) - \HF^{\rm
				af}_{R/\mathfrak{a}}(s-1)$.

			\item Sea $f \in R$ un polinomio no necesariamente homogéneo.
				Definamos su \textquote{homogenización en la nueva indeterminada $t$} como
				\[
					f^{\rm hom}(\vec x, t) := t^{\deg f}f\left( \frac{x_1}{t}, \dots, \frac{x_n}{t} \right).
				\]
				Pruebe que la función de Hilbert afín del ideal $(f) \nsl R$ es igual a la función de Hilbert proyectiva del
				ideal $(f^{\rm hom}) \nsl R[t] = k[\vec x, t]$.
		\end{enumerate}
\end{enumerate}

\begin{additional}
\nocite{cox:ideals}
\section{Comentarios adicionales}
El ejercicio~\ref{ex:krull_hauptidealsatz} es un caso particular del \textquote{teorema de ideales principales de Krull} (vea
% \citeauthor{matsumura:ring}~\cite{matsumura:ring})
\citeauthor{eisenbud:commutative}~\cite{eisenbud:commutative})
o también conocido por su nombre en alemán, \emph{Hauptidealsatz}.

\printbibliography
\end{additional}

\end{document}
