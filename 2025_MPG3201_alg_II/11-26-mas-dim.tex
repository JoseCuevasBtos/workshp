\documentclass[11pt, reqno]{amsart}

\usepackage{../ayud-template}
\input{../general.tex}

\usepackage{tikz}
\usetikzlibrary{babel,cd}

\hyphenation{Harts-horne}

% \usepackage{minted}
% \usemintedstyle{colorful}
% \setminted{
% 	mathescape,
% 	% linenos,
% 	autogobble,
% 	bgcolor=black!5,
% 	tabsize=2,
% 	fontsize=\small
% }

% \usepackage{multicol}
\DeclareMathOperator{\init}{in}
\DeclareMathOperator{\HF}{HF}

\title{Dimensión II}
\date{\DTMdate{2025-11-26}}

\author{José Cuevas Barrientos}
\email{josecuevasbtos@uc.cl}
\urladdr{https://josecuevas.xyz/teach/2025-2-alg/}

\logo{../puc_negro.png}
\institution{Pontificia Universidad Católica de Chile y Universidad de Chile}
\department{Facultad de Matemáticas}
\course{Álgebra II}
\coursecode{MPG3201}
\professor{José Samper}

\begin{document}

\maketitle

\section*{Recordatorio}
Será útil en los ejercicios ocupar la siguiente proposición (Prop.~10 de \cite[511]{cox:ideals}, \S 9.4):
\begin{displayquote}
	Sea $V \subseteq \A^n(k)$ (resp.\ $V \subseteq \PP^n(k)$) una variedad\footnotemark{} algebraica.
	\footnotetext{Para mí, todas las variedades son irreducibles y, para quienes sepan esquemas, \emph{reducidas}.}
	Entonces dado un polinomio (homogéneo) $f \in k[\vec x]$ que no se anula en todo $V$, se cumple que $\dim(V \cap \VV(f)) = \dim V - 1$.
	Además, si $W \subset V$ es una subvariedad propia, entonces $\dim W < \dim V$.
\end{displayquote}
Como comentamos la semana pasada, este es un caso del teorema de ideales principales de Krull.

\section{Ejercicios}
\begin{enumerate}
	\item Considere $\mathfrak{a} := (xz - x^2, yz - xy) \nsl k[x, y, z]$.
		\begin{enumerate}
			\item Pruebe que $\mathfrak{a} \cap k[z] = 0$.
				Verifique o contradiga que $\mathfrak{a} \cap k[x_i, x_j] = 0$ donde $\{ x_i, x_j \}$ recorre todas las parejas $\{ \{ x, y \}, \{ x, z \}, \{ y, z \} \}$.
			\item ¿Qué puede decir de $\dim\VV(\mathfrak{a})$?
			\item ¿Es $\mathfrak{a}$ un ideal radical?
		\end{enumerate}
		\nocite{cox:ideals}

	\item Sea $f \in R := k[x_0, \dots, x_n]$ homogéneo de grado $r > 0$.
		\begin{enumerate}
			\item Calcule el polinomio de Hilbert (proyectivo) del ideal $(f)$.
			\item Sea $\mathfrak{a}$ un ideal y suponga que $f$ no es un divisor de cero en $R/\mathfrak{a}$.
				Pruebe que el polinomio de Hilbert $\mathfrak{a} + (f)$ solo depende de $\mathfrak{a}$ y $r$.
		\end{enumerate}

	\item\lookright
		Sea $V \subseteq \PP^n(k)$ una variedad
		proyectiva.
		\begin{enumerate}
			% \item Pruebe que dado $f \in k[x_0, \dots, x_n]$ existen finitos ideales primos \emph{minimales} que le contienen.
			\item Pruebe que, si $\dim V > 0$, entonces existe una subvariedad $W \subseteq V$ de $\dim W = \dim V - 1$.
				\begin{hint}
					Para ello, puede resultar útil el siguiente resultado de álgebra conmutativa:
					\begin{displayquote}
						El conjunto de primos ($\subseteq$-)minimales de un anillo noetheriano es finito
						(Ex.~8 de \cite{atiyah:commutative}, Ch.~6).
						\qedhere
					\end{displayquote}
				\end{hint}

			\item\label{ex:dimension_th}
				Pruebe que si $\dim V = m$, entonces existe una cadena de $m+1$ subvariedades distintas:
				\[
					V_0 \subset V_1 \subset \cdots \subset V_m = V
				\]
				y que no hay cadenas más largas.
		\end{enumerate}

	\item Sea $F \colon \A^m(k) \to \A^n(k)$ un morfismo regular (i.e., dado en coordenadas por polinomios),
		y sean $V := \VV(\mathfrak{a}) \subseteq \A^m(k)$ y $W := \overline{\Img(F|_V)} \subseteq \A^n(k)$ subconjuntos algebraicos afines.
		Así, $F$ se restringe a un morfismo \emph{dominante} (cuya imagen es densa) $F|_V \colon V \to W$.
		Pruebe que $\dim V \ge \dim W$ y dé ejemplos de cuando hay desigualdad estricta.

		\begin{sol}
			Por clases ya ha visto que hay $G_1, \dots, G_d \colon W \to \A^1(k)$ morfismos regulares que son algebraicamente independientes.
			Cada uno de ellos se escribe como una función racional $G_j(y_1, \dots, y_n) = p_j(\vec y)/q_j(\vec y)$, donde $p_j,
			q_j \in k[y_1, \dots, y_n] = k[\A^n]$ y donde $q_j$ no se anula en ningún punto de $W$.

			Luego podemos cambiar $G_j$ por $QG_j$, donde $Q := q_1\cdots q_n$, ya que si los nuevos morfismos regulares son
			algebraicamente dependientes, existe un polinomio homogéneo
			\[
				0 = \sum_{\vec a} c_{\vec a} (QG_1(\vec y))^{a_1}\cdots(QG_d(\vec y))^{a_d}
				= Q(\vec y)^e \sum_{\vec a} c_{\vec a} G_1(\vec y)^{a_1}\cdots G_d(\vec y)^{a_d},
			\]
			que se anula en todo $\vec y \in W$.
			Pero $Q$ no se anula en ningún $\vec y \in W$, lo que nos da una relación polinomial en $G_j$ que es absurda.

			Así, suponemos que los $G_j$'s son polinomios en $k[\vec y]$ y son algebraicamente independientes como morfismos de $W$.
			Queremos ver que\footnotemark{} $F\circ G_j \colon V \to \A^1(k)$ también son morfismos algebraicamente independientes.
			\footnotetext{Yo compongo al revés.}
			Si no, existe una relación polinomial
			\[
				0 = \sum_{\vec a} c_{\vec a} G_1(F(\vec x))^{a_1} \cdots G_d(F(\vec x))^{a_d},
			\]
			que se anula en todo $\vec x \in V$.
			Si $F$ fuera sobreyectivo, vemos que es válida para todo $\vec y \in W$, pero no es siempre cierto.

			En cualquier caso, nos dice que la imagen de $F$ cae en el cerrado afín $\VV( \sum_{\vec a} c_{\vec a}
			G_1^{a_1}\cdots G_d^{a_d} )$ y, como $W$ es la clausura de la imagen, tenemos que
			\[
				W \subseteq \VV\mathopen{}\left( \sum_{\vec a} c_{\vec a} G_1^{a_1}\cdots G_d^{a_d} \right)\mathclose{},
			\]
			es decir, todo punto de $W$ se anula en dicha expresión, lo que significa que debe ser el polinomio nulo; es decir,
			que los $F\circ G_j$ son algebraicamente independientes.
			Así, $d \le \dim V$ (porque la dimensión es la máxima cantidad de morfismos algebraicamente independientes).
		\end{sol}
\end{enumerate}

\begin{additional}
\appendix
\section{Comentarios adicionales}
El ejercicio~\ref{ex:dimension_th} muestra que la dimensión topológica o combinatórica de $V$ (en sentido de
% \cite{ega-iv1}, \S 0.14.1)
\citeauthor{hartshorne:algebraic}~\cite[5]{hartshorne:algebraic})
coincide con la \textquote{dimensión} que ya hemos definido (mediante polinomios de\break Hilbert).

\printbibliography
\end{additional}

\end{document}
