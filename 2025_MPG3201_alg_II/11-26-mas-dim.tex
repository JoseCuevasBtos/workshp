\documentclass[11pt, reqno]{amsart}

\usepackage{../ayud-template}
\input{../general.tex}

\usepackage{tikz}
\usetikzlibrary{babel,cd}

\hyphenation{Harts-horne}

% \usepackage{minted}
% \usemintedstyle{colorful}
% \setminted{
% 	mathescape,
% 	% linenos,
% 	autogobble,
% 	bgcolor=black!5,
% 	tabsize=2,
% 	fontsize=\small
% }

% \usepackage{multicol}
\DeclareMathOperator{\init}{in}
\DeclareMathOperator{\HF}{HF}

\title{Dimensión II}
\date{\DTMdate{2025-11-26}}

\author{José Cuevas Barrientos}
\email{josecuevasbtos@uc.cl}
\urladdr{https://josecuevas.xyz/teach/2025-2-alg/}

\logo{../puc_negro.png}
\institution{Pontificia Universidad Católica de Chile y Universidad de Chile}
\department{Facultad de Matemáticas}
\course{Álgebra II}
\coursecode{MPG3201}
\professor{José Samper}

\begin{document}

\maketitle

\section{Ejercicios}
\begin{enumerate}
	\item Considere $\mathfrak{a} := (xz - x^2, yz - xy) \nsl k[x, y, z]$.
		\begin{enumerate}
			\item Pruebe que $\mathfrak{a} \cap k[z] = 0$.
				¿Podría darse que $\mathfrak{a} \cap k[x, z] = 0$ o $\mathfrak{a} \cap k[y, z] = 0$?
			\item Muestre que $\mathfrak{a} \cap k[x, y] = 0$,
				pero $\mathfrak{a} \ne 0$.
			\item ¿Qué puede decir de $\dim\VV(\mathfrak{a})$?
		\end{enumerate}
		\nocite{cox:ideals}

	\item Sea $f \in R := k[x_0, \dots, x_n]$ homogéneo de grado $r > 0$.
		\begin{enumerate}
			\item Calcule el polinomio de Hilbert (proyectivo) del ideal $(f)$.
			\item Sea $\mathfrak{a}$ un ideal y suponga que $f$ no es un divisor de cero en $R/\mathfrak{a}$.
				Pruebe que el polinomio de Hilbert $\mathfrak{a} + (f)$ solo depende de $\mathfrak{a}$ y $r$.
		\end{enumerate}

	\item\lookright
		Sea $V \subseteq \PP^n(k)$ una variedad%
		\footnote{Para mí, todas las variedades son irreducibles y, para quienes sepan esquemas, \emph{reducidas}.}
		proyectiva.
		\begin{enumerate}
			% \item Pruebe que dado $f \in k[x_0, \dots, x_n]$ existen finitos ideales primos \emph{minimales} que le contienen.
			\item Pruebe que, si $\dim V > 0$, entonces existe una subvariedad $W \subseteq V$ de $\dim W = \dim V - 1$.
				\begin{hint}
					Para ello, pueden resultar útiles el hecho de que para todo $f \in k[x_0, \dots, x_n]$ que no se
					anula idénticamente en $V$ se cumple que $\dim(V \cap \VV(f)) = \dim V - 1$; así como el siguiente
					resultado de álgebra conmutativa:
					\begin{displayquote}
						El conjunto de primos ($\subseteq$-)minimales de un anillo noetheriano es finito
						(Ex.~8 de \cite{atiyah:commutative}, Ch.~6).
						\qedhere
					\end{displayquote}
				\end{hint}

			\item\label{ex:dimension_th}
				Pruebe que si $\dim V = m$, entonces existe una cadena de $m+1$ subvariedades distintas:
				\[
					V_0 \subset V_1 \subset \cdots \subset V_m = V
				\]
				y que no hay cadenas más largas.
		\end{enumerate}

	\item Sea $F \colon \A^m(k) \to \A^n(k)$ un morfismo regular y sea $V = \overline{\Img F} \subseteq \A^n(k)$ la clausura de su imagen
		(\textquote{$F$ es una parametrización racional de $V$}).
		Pruebe que $m \ge \dim V$, y dé un ejemplo donde $m > \dim V$.
\end{enumerate}

\begin{additional}
\appendix
\section{Comentarios adicionales}
El ejercicio~\ref{ex:dimension_th} muestra que la dimensión topológica o combinatórica de $V$ (en sentido de
% \cite{ega-iv1}, \S 0.14.1)
\citeauthor{hartshorne:algebraic}~\cite[5]{hartshorne:algebraic})
coincide con la \textquote{dimensión} que ya hemos definido (mediante polinomios de\break Hilbert).

\printbibliography
\end{additional}

\end{document}
