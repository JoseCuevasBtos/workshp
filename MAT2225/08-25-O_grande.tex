\documentclass[11pt, reqno]{amsart}

\usepackage[spanish]{babel}
\usepackage[LGR, T1]{fontenc}
\usepackage[utf8]{inputenc}

\input{../general.tex}
% \input{../graphics.tex}

\makeatletter
\def\emailaddrname{\textit{Correo electrónico}}
\def\subtitle#1{\gdef\@subtitle{#1}}
\def\@subtitle{}

% Metadata
\def\logo#1{\gdef\@logo{#1}}
\def\@logo{}
\def\institution#1{\gdef\@institution{#1}}
\def\@institution{}
\def\department#1{\gdef\@department{#1}}
\def\@department{}
\def\professor#1{\gdef\@professor{#1}}
\def\@professor{}
\def\course#1{\gdef\@course{#1}}
\def\@course{}
\def\coursecode#1{\gdef\@coursecode{#1}}
\def\@coursecode{}

\renewcommand{\maketitle}{
\begin{center}
	\small
	\renewcommand{\arraystretch}{1.2}
	\begin{tabular}{cp{.37\textwidth}p{0.44\textwidth}}
		% \hline
		\multirow{5}{*}{\includegraphics[height=2.0cm]{\@logo}}
	  & \multicolumn{2}{c}{ \makecell{{\bfseries \@institution} \\ \@department} } \\
	  % & \multicolumn{2}{c|}{{\bfseries\@institution} \\ \@department} \\
	  \cline{2-3}
	  & \textbf{Profesor:} \@professor & \textbf{Ayudante:} \authors \\
	  % \cline{2-3}
	  & \textbf{Curso:} \@course & \textbf{Sigla:} \@coursecode \\
	  % \cline{2-3}
	  & \multicolumn{2}{l}{ \textbf{Fecha:} \@date } \\
	  % \hline
	\end{tabular}
	\\[\baselineskip]
	% {}
	% \vspace{2\baselineskip}
	{\bfseries\Large\@title}
	\ifx\@subtitle\@empty\else
		\\[1ex]
		\large\mdseries\@subtitle
	\fi
\end{center}
}
\makeatother

\usepackage{multirow, makecell}

\usepackage[
	reversemp,
	letterpaper,
	% marginpar=2cm,
	% marginsep=1pt,
	margin=2.3cm
]{geometry}
\usepackage{fontawesome}
% \makeatletter
% \@reversemargintrue
% \makeatother

% Símbolos al margen, necesitan doble compilación
\newcommand{\hard}{\marginnote{\faFire}}
\newcommand{\hhard}{\marginnote{\faFire\faFire}}

% Dependencias para los teoremas
\usepackage{xifthen}
\def\@thmdep{}
\newcommand{\thmdep}[1]{
	\ifthenelse{\isempty{#1}}
	{\def\@thmdep{}}
	{\def\@thmdep{ (#1)}}
}
\newcommand{\thmstyle}{\color{thm}\sffamily\bfseries}

% ===== Estilos de Teoremas ==========
\newtheoremstyle{axiomstyle}
	{0.3cm}
	{0.3cm}
	{\normalfont}
	{0.5cm}
	{\bfseries\scshape}
	{:}
	{4pt}
	{\thmname{#1}\thmnote{ #3}\thmnumber{ (#2)}}
\newtheoremstyle{styleC}
	{0.5cm}
	{0.5cm}
	{\normalfont}
	{0.5cm}
	{\bfseries}
	{:}
	{4pt}
	{\thmname{#1\textrm{\@thmdep}}\thmnumber{ #2}\thmnote{ (#3)}}

% ====== Teoremas (sin borde) ===========
\theoremstyle{axiomstyle}
\newtheorem*{axiom}{Axioma}

% ====== Teoremas (sin borde) ==================
\theoremstyle{styleC}
\newtheorem{thm}{Teorema}[section]
\newtheorem{mydef}[thm]{Definición}
\newtheorem{prop}[thm]{Proposición}
\newtheorem{cor}[thm]{Corolario}
\newtheorem{lem}[thm]{Lema}
\newtheorem{con}[thm]{Conjetura}

\newtheorem*{prob}{Problema}
\newtheorem*{sol}{Solución}
\newtheorem*{obs}{Observación}
\newtheorem*{ex}{Ejemplo}

% \usepackage{tcolorbox}
% \newtcbox{bluebox}[1][]{enhanced jigsaw, 
%   sharp corners,
%   frame hidden,
%   nobeforeafter,
%   listing only,
%   #1} % comando para crear cajas de colores

\expandafter\let\expandafter\oldproof\csname\string\proof\endcsname
\let\oldendproof\endproof
\renewenvironment{proof}[1][\proofname]{%
  \oldproof[\scshape Demostración:]%
}{\oldendproof} % comando para redefinir la caja de la demostración
\newenvironment{hint}[1][\proofname]{%
  \oldproof[\scshape Pista:]%
}{\oldendproof} % comando para redefinir la caja de la demostración

% colores utilizados
\definecolor{numchap}{RGB}{249,133,29}
\definecolor{chap}{RGB}{6,129,204}
\definecolor{sec}{RGB}{204,0,0}
\definecolor{thm}{RGB}{106,176,240}
\definecolor{thmB}{RGB}{32,31,31}
\definecolor{part}{RGB}{212,66,66}

% ====== Diseño de los titulares ===============
\usepackage[explicit]{titlesec} % para personalizar el documento, la opción <<explicit>> hace que el texto de los titulares sea un objeto interactuable

\titleformat{\subsection}[runin]
	{\bfseries}
	{\textrm{\S}\thesubsection}
	{1ex}
	{#1.}

\setlist[enumerate,1]{label=\arabic*., ref=\arabic*} % Enumerate standards

% \usepackage{diagbox}

\title{$O$ grande, $o$ chica}
\date{25 de agosto de 2023}

\author[José Cuevas]{José Cuevas Barrientos}
\email{josecuevasbtos@uc.cl}

\logo{../puc_negro.png}
\institution{Pontificia Universidad Católica de Chile}
\department{Facultad de Matemáticas}
\course{Teoría de Números}
\coursecode{MAT2225}
\professor{Héctor Pastén}

\begin{document}

\maketitle

\section{Órdenes de magnitud}
\begin{enumerate}
	\item (Ejemplo iluminador) Demuestre que
		$$ \sum_{n \le x} \frac{1}{2^n} = 1 + O\left( \frac{1}{2^x} \right). $$
	\item Empleando sumas parciales, demuestre que:
		\begin{enumerate}
			\item $\displaystyle H(x) := \sum_{n \le x} \frac{1}{n} = \log x + \gamma + O\left( \frac{1}{x} \right),$ donde $ \gamma$ es una constante
				llamada la \textit{constante de Euler-Mascheroni}.
			\item $\displaystyle T(x) := \sum_{n \le x} \log n = x\log x - x + O(\log x)$.
		\end{enumerate}

	\item \textbf{Problema del círculo de Gauss:} Sea $r(n)$ la cantidad de formas de escribir $n$ como suma de dos cuadrados
		(e.g., para $n = 5$ tenemos que $r(5) = 8$ pues $(\pm 1)^2 + (\pm 4)^2 = 5$ nos da cuatro formas e intercambiar
		los factores nos da cuatro más).
		Demuestre que:
		$$ \frac{1}{x} \sum_{n\le x} r(n) = \pi + O\left( \frac{1}{\sqrt{x}} \right). $$

		\begin{hint}
			El nombre \textit{problema del \underline{círculo}} es sugerente.
		\end{hint}
	\item Demuestre que, para $s > 1$ se tiene la siguiente identidad:
		$$ \sum_{n=1}^{\infty} \frac{\mu(n)}{n^s} = \frac{1}{\zeta(s)}. $$
\end{enumerate}

\begin{thm}[cotas de Chebyshev]
	Para $x$ suficientemente grande se tiene
	$$ c_1 \le \frac{\pi(x)}{x / \log x} \le c_2, $$
	para algunos $0 < c_1 < c_2$.%
	\footnote{\citeauthor{hua:number}~\cite[82]{hua:number} lo demuestra con $c_1 = 1/8$ y $c_2 = 12$.
	Dependiendo del autor, las cotas varían, pero lo interesante es que varias aplicaciones no dependen del valor exacto.}
	En consecuencia, $\pi(x) \asymp x/\log x$.%
	\footnote{Recuérdese que $\log x$ \textit{siempre} denota el logaritmo natural.}
\end{thm}
Las cotas de Chebyshev tienen la facultad de ser <<elementales>> (en el sentido de que no requieren métodos complejos, por ejemplo)
y ser una buena versión preliminar del teorema de los números primos.
Como consecuencia:

\begin{enumerate}[resume]
	\item \hard
		\textbf{Primer teorema de Mertens:} Demuestre que
		$$ \sum_{p\le x} \frac{\log p}{p} = \log x + O(1), $$
		(donde el subíndice $p$ siempre recorre los números primos.)

		Para facilitar el ejercicio realice los siguientes pasos:
		\begin{enumerate}[(i)]
			\item Demuestre que cuando $n$ es entero
				$$ T(n) = \log(n!) = \sum_{p \le x} \left( \left\lfloor \frac np \right\rfloor + \left\lfloor \frac{n}{p^2} \right\rfloor
				+ \cdots \right) \log p. $$
			\item Demuestre que
				$$ \frac{x}{p} - 1 < \left\lfloor \frac np \right\rfloor + \left\lfloor \frac{n}{p^2} \right\rfloor + \cdots
				< \frac{x}{p} + \frac{x}{p(p - 1)}. $$
			\item Demuestre que $\sum_{p \le x} \log p \le c_2 x$.
			\item Concluya el enunciado.
		\end{enumerate}

	\item \hard
		\textbf{Segundo teorema de Mertens:} Demuestre que
		$$ \sum_{p\le x} \frac{1}{p} = \log\log x + M + O\left( \frac{1}{\log x} \right), $$
		donde $M$ es una constante llamada la \textit{constante de Mertens}.
\end{enumerate}

\section{Aplicaciones del teorema de los números primos}
Las cotas de Chebyshev nos dan un acercamiento al siguiente resultado:
\begin{thm}[de los números primos]
	$\displaystyle \pi(x) \sim \frac{x}{\log x}$.
\end{thm}

Hay varias demostraciones del resultado anterior.
Las primeras, originales de J. Hadamard y C. J. de la Vallée Poussin, siguen la línea de B. Riemann empleando métodos de análisis complejo.
Años más tarde, A. Selberg y P. Erd\H os dieron una demostración \textit{elemental} (sin análisis complejo), pero que se considera mucho menos ilustrativa.
Una exposición breve y elemental del teorema se encuentra en \citeauthor{richter21PNT}~\cite{richter21PNT}.

Veamos algunas aplicaciones:
\begin{enumerate}[resume]
	\item Sea $p_n$ la sucesión de los números primos en orden creciente.
		Demuestre que $p_n \sim n\log n$.
	\item Para todo $\epsilon > 0$ existe $x_0 > 0$ tal que para todo $x \ge x_0$
		siempre existe un primo $p$ tal que $x < p \le (1 + \epsilon)x$.
	\item Para cada natural $N$ existe un primo tal que (en base decimal) sus primeras cifras coinciden con (todas) las de $N$.
\end{enumerate}

\section{Comentarios adicionales y problemas abiertos}
\begin{itemize}
	\item Los números $H(x)$ se conocen como \textit{números harmónicos}.
	\item La constante de Euler-Mascheroni vale $\gamma \approx 0.5772156649...$
		Se cree que $\gamma$ es trascendente y, por tanto, irracional; pero ambas afirmaciones están sin demostrar.
	\item La constante de Mertens $M \approx 0.261497212847642...$ sufre de la misma suerte: también se desconoce si es irracional o trascendente.
	\item Existen grandes carreras por optimizar aproximaciones en la teoría analítica de números.
		Vamos a dar el ejemplo con el problema del círculo de Gauss.
		Se cree que
		$$ \sum_{n \le x} r(n) = \pi x + O\left( x^{\frac{1}{4} + \epsilon} \right). $$
	\item Más aún, se cree que la cota anterior es lo más aguda posible, es decir, se cree que es falso:
		$$ \sum_{n \le x} r(n) \ne \pi x + O\left( x^{\frac{1}{4} - \epsilon} \right). $$
	\item El problema 8 puede considerarse una especie de refinamiento del postulado de Bertrand:

		\begin{thm}[postulado de Bertrand]
			Para todo $n \ge 2$ existe un primo $p$ tal que $n \le p < 2n$.
		\end{thm}

		Este resultado se puede demostrar mediante técnicas desarrolladas a estas alturas del curso.
		Una demostración sencilla se encuentra por ejemplo en \citeauthor{hua:number}~\cite[82-85]{hua:number}.
\end{itemize}

\section{Comentarios ayudantía previa}
\begin{enumerate}
	\item[3.] Además de ocupar que $17 \mid q_8$ uno puede ocupar más sencillamente que $31 \mid q_1$.
		En efecto, el procedimiento siguiente es el mismo ya que
		$$ q_{1 + 30k} = 1 + \frac{10}{3}(10^{1 + 30k} - 1) \equiv 1 + \frac{10}{3}(10^1 - 1) \equiv 0 \pmod{31}, $$
		y como $q_{1 + 30k}$ con $k \ge 1$ es mayor estricto de 31, concluimos que es compuesto.

	\item[4.] El tiempo no dió en la sesión original, reproduzco mi solución.

		\newcommand{\mmid}{\parallel}
		La idea es proceder por inducción.
		El caso base es $f(1) = 1 + 17 = 9\cdot 2$ por lo que $9 \mmid f(1)$.
		Para $n > 2$ sea $x$ tal que $3^{n-1} \mmid f(x)$, es decir, $f(x) = 3^{n-1}a$ con $3 \nmid a$.
		Buscamos $y$ tal que $3^n \mmid f(y)$, o lo que es lo mismo, podemos mirar módulo $3^{n+1}$:
		Nótese que
		\begin{align*}
			f(x + b3^{n-2}) &= 17 + x^3 + 3 \cdot x^2 \cdot b3^{n-2} + 3 \cdot x \cdot b^23^{2(n-2)} + b^33^{3(n-2)} \\
					&\equiv 3^{n-1}(a + bx^2) \pmod{3^{n+1}},
		\end{align*}
		ahora buscamos manipular $a + bx^2$ de modo que sea divisible por 3, pero no por 9.
		Primero, observe que como $0 \equiv f(x) \equiv x^3 - 1 \pmod{3}$ entonces $3 \nmid x$.
		Así que, $x$ puede ser $1, 2, 4, 5, 7, 8 \pmod 9$ y, su cuadrado puede ser $1, 4, 7 \pod 9$.

		Como $x^2$ es coprimo con 9, entonces $x^2$ tiene inversa en $\Z/9\Z$ y, por tanto, $b = \frac{3 - a}{x^2}$ funciona
		(¡donde <<$/x^2$>> significa multiplicar por una inversa módulo 9!).

		Para hacer la cancelación de $3^{2(n - 2)}$ necesitamos las cotas $1 + 2(n - 2) \ge n + 1$ y $3(n - 2) \ge n + 1$,
		las cuales se satisfacen para $n \ge 4$, no obstante la técnica igual funciona (aunque no incondicionalmente)
		Los casos restantes se cubren por $3^4 = 81 \mmid 81 = f(1 + 3)$ y $3^3 = 27 \mmid -108 = f(4 - 9)$.
		% Como $3 \nmid a$, el mismo razonamiento aplica para $a$.
		% Finalmente una inspección rápida concluye que siempre existe $b$ tal que $a + bx^2$ sea $3, 6 \pod 9$:
		% \begin{center}
		% 	\begin{tabular}{c|*{6}{c}}
		% 		\diagbox{$x^2$}{$a$} & 1 & 2 & 4 & 5 & 7 & 8 \\
		% 		\hline
		% 		1  & 2 & 1 & 2 & 7 & 5 & 4 \\ 
		% 		4  & 5 & 1 & 5 & 4 & 2 & 1 \\ 
		% 		7  & 2 & 4 & 8 & 1 & 2 & 1 \\ 
		% 	\end{tabular}
		% \end{center}
\end{enumerate}

\nocite{hua:number}
\printbibliography[title={Referencias y lecturas adicionales}]

\end{document}
