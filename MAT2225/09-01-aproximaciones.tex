\documentclass[11pt, reqno]{amsart}

\usepackage[spanish]{babel}
\usepackage[LGR, T1]{fontenc}
\usepackage[utf8]{inputenc}

../LaTeX/general.tex
% \input{../graphics.tex}

\makeatletter
\def\emailaddrname{\textit{Correo electrónico}}
\def\subtitle#1{\gdef\@subtitle{#1}}
\def\@subtitle{}

% Metadata
\def\logo#1{\gdef\@logo{#1}}
\def\@logo{}
\def\institution#1{\gdef\@institution{#1}}
\def\@institution{}
\def\department#1{\gdef\@department{#1}}
\def\@department{}
\def\professor#1{\gdef\@professor{#1}}
\def\@professor{}
\def\course#1{\gdef\@course{#1}}
\def\@course{}
\def\coursecode#1{\gdef\@coursecode{#1}}
\def\@coursecode{}

\renewcommand{\maketitle}{
\begin{center}
	\small
	\renewcommand{\arraystretch}{1.2}
	\begin{tabular}{cp{.37\textwidth}p{0.44\textwidth}}
		% \hline
		\multirow{5}{*}{\includegraphics[height=2.0cm]{\@logo}}
	  & \multicolumn{2}{c}{ \makecell{{\bfseries \@institution} \\ \@department} } \\
	  % & \multicolumn{2}{c|}{{\bfseries\@institution} \\ \@department} \\
	  \cline{2-3}
	  & \textbf{Profesor:} \@professor & \textbf{Ayudante:} \authors \\
	  % \cline{2-3}
	  & \textbf{Curso:} \@course & \textbf{Sigla:} \@coursecode \\
	  % \cline{2-3}
	  & \multicolumn{2}{l}{ \textbf{Fecha:} \@date } \\
	  % \hline
	\end{tabular}
	\\[\baselineskip]
	% {}
	% \vspace{2\baselineskip}
	{\bfseries\Large\@title}
	\ifx\@subtitle\@empty\else
		\\[1ex]
		\large\mdseries\@subtitle
	\fi
\end{center}
}
\makeatother

\usepackage{multirow, makecell}

\usepackage[
	reversemp,
	letterpaper,
	% marginpar=2cm,
	% marginsep=1pt,
	margin=2.3cm
]{geometry}
\usepackage{fontawesome}
% \makeatletter
% \@reversemargintrue
% \makeatother

% Símbolos al margen, necesitan doble compilación
\newcommand{\hard}{\marginnote{\faFire}}
\newcommand{\hhard}{\marginnote{\faFire\faFire}}

% Dependencias para los teoremas
\usepackage{xifthen}
\def\@thmdep{}
\newcommand{\thmdep}[1]{
	\ifthenelse{\isempty{#1}}
	{\def\@thmdep{}}
	{\def\@thmdep{ (#1)}}
}
\newcommand{\thmstyle}{\color{thm}\sffamily\bfseries}

% ===== Estilos de Teoremas ==========
\newtheoremstyle{axiomstyle}
	{0.3cm}
	{0.3cm}
	{\normalfont}
	{0.5cm}
	{\bfseries\scshape}
	{:}
	{4pt}
	{\thmname{#1}\thmnote{ #3}\thmnumber{ (#2)}}
\newtheoremstyle{styleC}
	{0.5cm}
	{0.5cm}
	{\normalfont}
	{0.5cm}
	{\bfseries}
	{:}
	{4pt}
	{\thmname{#1\textrm{\@thmdep}}\thmnumber{ #2}\thmnote{ (#3)}}

% ====== Teoremas (sin borde) ===========
\theoremstyle{axiomstyle}
\newtheorem*{axiom}{Axioma}

% ====== Teoremas (sin borde) ==================
\theoremstyle{styleC}
\newtheorem{thm}{Teorema}[section]
\newtheorem{mydef}[thm]{Definición}
\newtheorem{prop}[thm]{Proposición}
\newtheorem{cor}[thm]{Corolario}
\newtheorem{lem}[thm]{Lema}
\newtheorem{con}[thm]{Conjetura}

\newtheorem*{prob}{Problema}
\newtheorem*{sol}{Solución}
\newtheorem*{obs}{Observación}
\newtheorem*{ex}{Ejemplo}

% \usepackage{tcolorbox}
% \newtcbox{bluebox}[1][]{enhanced jigsaw, 
%   sharp corners,
%   frame hidden,
%   nobeforeafter,
%   listing only,
%   #1} % comando para crear cajas de colores

\expandafter\let\expandafter\oldproof\csname\string\proof\endcsname
\let\oldendproof\endproof
\renewenvironment{proof}[1][\proofname]{%
  \oldproof[\scshape Demostración:]%
}{\oldendproof} % comando para redefinir la caja de la demostración
\newenvironment{hint}[1][\proofname]{%
  \oldproof[\scshape Pista:]%
}{\oldendproof} % comando para redefinir la caja de la demostración

% colores utilizados
\definecolor{numchap}{RGB}{249,133,29}
\definecolor{chap}{RGB}{6,129,204}
\definecolor{sec}{RGB}{204,0,0}
\definecolor{thm}{RGB}{106,176,240}
\definecolor{thmB}{RGB}{32,31,31}
\definecolor{part}{RGB}{212,66,66}

% ====== Diseño de los titulares ===============
\usepackage[explicit]{titlesec} % para personalizar el documento, la opción <<explicit>> hace que el texto de los titulares sea un objeto interactuable

\titleformat{\subsection}[runin]
	{\bfseries}
	{\textrm{\S}\thesubsection}
	{1ex}
	{#1.}

\setlist[enumerate,1]{label=\arabic*., ref=\arabic*} % Enumerate standards

% \usepackage{diagbox}

\title{Aproximaciones diofantinas}
\subtitle{Números irracionales y la abundancia de números de Liouville}
\date{1 de septiembre de 2023}

\author[José Cuevas]{José Cuevas Barrientos}
\email{josecuevasbtos@uc.cl}

\logo{../puc_negro.png}
\institution{Pontificia Universidad Católica de Chile}
\department{Facultad de Matemáticas}
\course{Teoría de Números}
\coursecode{MAT2225}
\professor{Héctor Pastén}

\begin{document}

\maketitle

\section{Aproximación y el teorema de Dirichlet}
El típico argumento de las cajitas se generaliza a lo siguiente:
\begin{enumerate}
	\item \hard
		\textbf{Aproximación multidimensional de Dirichlet:}
		Sean $\alpha_{11}, \alpha_{12}, \dots, \alpha_{m,r}$ números reales donde $m, r > 0$ son enteros fijos, y sean $M > 0$ otro entero.
		Demuestre que existen $m$ enteros $u_1, \dots, u_m$ y $r$ enteros $v_1, \dots, v_r$ (no todos nulos) con cada $|v_j| < M^{m/r}$
		tales que
		$$ \forall 1 \le j \le m \quad \left| \sum_{k=1}^{r} \alpha_{jk} v_k - u_j \right| < \frac{1}{M}. $$
\end{enumerate}

La teoría de aproximación diofantina permite encontrar varios ejemplos directos de números irra\-cionales:%
\footnote{Sin recurrir, por supuesto, al típico argumento de que los irracionales existen por que $|\R| > |\Q|$.}
\begin{enumerate}[resume]
	\item Demuestre que un número real $\alpha$ es irracional syss para todo $\epsilon > 0$ existen enteros $u, v$
		tales que $0 < |\alpha u - v| < \epsilon$.
	% \item Demuestre que si $\alpha$ es irracional existen infinitas fracciones (en forma reducida) $u/v$ tales que $|\alpha - u/v| < 1/v^2$.
	\item Sea $2 \le g_1 \le g_2 \le g_3 \le \cdots$ una sucesión creciente de naturales tales que $\lim_n g_n = \infty$
		y sea $(z_n)_{n\in\N}$ una sucesión de 0s y 1s, donde $z_n = 1$ infinitas veces.
		Demuestre que el número real
		$$ \alpha := \sum_{n=1}^{\infty} \frac{z_n}{g_1 g_2 \cdots g_n} = \frac{z_1}{g_1} + \frac{z_2}{g_1g_2} + \cdots, $$
		es irracional (¿por qué siempre está bien definido?).

		Nótese que esto da una generalización del típico argumento de que $e$ es irracional.
		También argumente porque las hipótesis ($2 \le g_1$; $g_n \to \infty$; $z_n \ne 0$ infinitas veces) son necesarias.
	\item \hard
		(Fermat) Sea $d > 0$ un natural libre de cuadrados.
		Demuestre que la ecuación de Pell $x^2 - dy^2 = 1$ posee al menos una solución $(x, y) \ne (\pm 1, 0)$ con coeficientes enteros.

		\begin{hint}
			Conviértalo en un problema de aproximar $\sqrt{d}$.
		\end{hint}
\end{enumerate}

\section{El teorema de Liouville y sus consecuencias}
\begin{thm}[Liouville]
	Sea $\alpha \in \R$ un número algebraico de grado $d$.%
	\footnote{Vale decir, que existe un polinomio irreducible $f(x) \in \Q[x]$ de grado $d$ tal que $f(\alpha) = 0$.}
	Entonces existen finitas fracciones reducidas $u/v \in \Q$ tales que $|\alpha - u/v| < 1/v^d$.
\end{thm}
\begin{mydef}
	Se dice que un número real $\alpha$ es \strong{de Liouville} si para cada $k \ge 1$ entero existe una fracción reducida $u/v$
	tal que $|\alpha - u/v| < 1/v^k$.
	Se denota por $\mathcal{L}$ el conjunto de los números de Liouville.
\end{mydef}

\begin{enumerate}[resume]
	\item Demuestre que los números de Liouville son trascendentes.
	\item Demuestre que $\mathcal{L}$ es un conjunto denso y no numerable de $\R$.
	\item Demuestre lo siguiente:
		\begin{enumerate}
			\item $\Q + \mathcal{L} = \mathcal{L}$ (donde la suma de conjuntos significa el conjunto de elementos $\alpha + \beta$,
				con $\alpha \in \Q, \beta \in \mathcal{L}$).
			\item $\Q^\times \cdot \mathcal{L} = \mathcal{L}$.
			\item \hard
				(Erd\H os \cite{erdos62liouville}) $\mathcal{L + L} = \R$.

				\begin{hint}
					Por los incisos anteriores redúzcase a probar que todo $\alpha \in (0, 1)$ está en $\mathcal{L + L}$.
					Este caso sale escribiendo $\alpha$ de manera conveniente y generalizando el ejemplo prototípico
					de un número de Liouville (vale decir, $0.1100001 00000 00000 00000 0001...$)
				\end{hint}
		\end{enumerate}
\end{enumerate}

\section{Comentarios adicionales}
\begin{itemize}
	\item
		Con $m = r = 1$ en el ejercicio 1, uno obtiene que
		$$ \left| \alpha - \frac{u}{v} \right| = \frac{1}{Mv} < \frac{1}{v^2}. $$
		Si exigimos que $\alpha$ sea irracional, entonces un teorema de Hurwitz demuestra que existen infinitas fracciones
		reducidas $u/v$ tales que
		$$ \left| \alpha - \frac{u}{v} \right| < \frac{1}{\sqrt{5}v^2}, $$
		y, además, está aproximación es aguda en el sentido de que cambiando $\sqrt{5}$ por un denominador más grande uno pierde la infinitud.

		Éstas fracciones pueden obtenerse mediante convergentes de la expansión en fracción continua simple (cfr. \cite[256-257]{hua:number}).

	% \item Si además fijamos un número $\beta$ no entero, entonces para un $\alpha$ irracional existen infinitas fracciones reducidas $u/v$ tales que
	% 	$$ |\alpha v - \beta - u| < \frac{1}{4|v|}, $$
	% 	lo cual es una considerable mejora.
	% 	J.H. Grace probó que el denominador $4$ es agudo.

	\item El problema 7 demuestra que el conjunto $\mathcal{L}$ es bastante curioso.
		Uno puede demostrar que $\mathcal{L}$ tiene medida (de Lebesgue) 0, de modo que acrecenta el interés.
		Se dice que un subconjunto $W \subseteq \mathcal{L}$ es un \strong{conjunto de Erd\H os-Liouville} si $W + W = \R$;
		el artículo reciente de \citeauthor{chalebgwa2023sets}~\cite{chalebgwa2023sets} prueba que hay $2^{\mathfrak{c}}$ conjuntos de Erd\H os-Liouville,
		en otras palabras, que hay tantos subconjuntos de $\R$ como conjuntos de Erd\H os-Liouville.
\end{itemize}

\nocite{hlawka:number}
\printbibliography[title={Referencias y lecturas adicionales}]

\end{document}
