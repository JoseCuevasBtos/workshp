\documentclass[11pt, reqno]{amsart}

\usepackage[spanish]{babel}
\usepackage[LGR, T1]{fontenc}
\usepackage[utf8]{inputenc}

../LaTeX/general.tex
% \input{../graphics.tex}

\makeatletter
\def\emailaddrname{\textit{Correo electrónico}}
\def\subtitle#1{\gdef\@subtitle{#1}}
\def\@subtitle{}

% Metadata
\def\logo#1{\gdef\@logo{#1}}
\def\@logo{}
\def\institution#1{\gdef\@institution{#1}}
\def\@institution{}
\def\department#1{\gdef\@department{#1}}
\def\@department{}
\def\professor#1{\gdef\@professor{#1}}
\def\@professor{}
\def\course#1{\gdef\@course{#1}}
\def\@course{}
\def\coursecode#1{\gdef\@coursecode{#1}}
\def\@coursecode{}

\renewcommand{\maketitle}{
\begin{center}
	\small
	\renewcommand{\arraystretch}{1.2}
	\begin{tabular}{cp{.37\textwidth}p{0.44\textwidth}}
		% \hline
		\multirow{5}{*}{\includegraphics[height=2.0cm]{\@logo}}
	  & \multicolumn{2}{c}{ \makecell{{\bfseries \@institution} \\ \@department} } \\
	  % & \multicolumn{2}{c|}{{\bfseries\@institution} \\ \@department} \\
	  \cline{2-3}
	  & \textbf{Profesor:} \@professor & \textbf{Ayudante:} \authors \\
	  % \cline{2-3}
	  & \textbf{Curso:} \@course & \textbf{Sigla:} \@coursecode \\
	  % \cline{2-3}
	  & \multicolumn{2}{l}{ \textbf{Fecha:} \@date } \\
	  % \hline
	\end{tabular}
	\\[\baselineskip]
	% {}
	% \vspace{2\baselineskip}
	{\bfseries\Large\@title}
	\ifx\@subtitle\@empty\else
		\\[1ex]
		\large\mdseries\@subtitle
	\fi
\end{center}
}
\makeatother

\usepackage{multirow, makecell}

\usepackage[
	reversemp,
	letterpaper,
	% marginpar=2cm,
	% marginsep=1pt,
	margin=2.3cm
]{geometry}
\usepackage{fontawesome}
% \makeatletter
% \@reversemargintrue
% \makeatother

% Símbolos al margen, necesitan doble compilación
\newcommand{\hard}{\marginnote{\faFire}}
\newcommand{\hhard}{\marginnote{\faFire\faFire}}

% Dependencias para los teoremas
\usepackage{xifthen}
\def\@thmdep{}
\newcommand{\thmdep}[1]{
	\ifthenelse{\isempty{#1}}
	{\def\@thmdep{}}
	{\def\@thmdep{ (#1)}}
}
\newcommand{\thmstyle}{\color{thm}\sffamily\bfseries}

% ===== Estilos de Teoremas ==========
\newtheoremstyle{axiomstyle}
	{0.3cm}
	{0.3cm}
	{\normalfont}
	{0.5cm}
	{\bfseries\scshape}
	{:}
	{4pt}
	{\thmname{#1}\thmnote{ #3}\thmnumber{ (#2)}}
\newtheoremstyle{styleC}
	{0.5cm}
	{0.5cm}
	{\normalfont}
	{0.5cm}
	{\bfseries}
	{:}
	{4pt}
	{\thmname{#1\textrm{\@thmdep}}\thmnumber{ #2}\thmnote{ (#3)}}

% ====== Teoremas (sin borde) ===========
\theoremstyle{axiomstyle}
\newtheorem*{axiom}{Axioma}

% ====== Teoremas (sin borde) ==================
\theoremstyle{styleC}
\newtheorem{thm}{Teorema}[section]
\newtheorem{mydef}[thm]{Definición}
\newtheorem{prop}[thm]{Proposición}
\newtheorem{cor}[thm]{Corolario}
\newtheorem{lem}[thm]{Lema}
\newtheorem{con}[thm]{Conjetura}

\newtheorem*{prob}{Problema}
\newtheorem*{sol}{Solución}
\newtheorem*{obs}{Observación}
\newtheorem*{ex}{Ejemplo}

% \usepackage{tcolorbox}
% \newtcbox{bluebox}[1][]{enhanced jigsaw, 
%   sharp corners,
%   frame hidden,
%   nobeforeafter,
%   listing only,
%   #1} % comando para crear cajas de colores

\expandafter\let\expandafter\oldproof\csname\string\proof\endcsname
\let\oldendproof\endproof
\renewenvironment{proof}[1][\proofname]{%
  \oldproof[\scshape Demostración:]%
}{\oldendproof} % comando para redefinir la caja de la demostración
\newenvironment{hint}[1][\proofname]{%
  \oldproof[\scshape Pista:]%
}{\oldendproof} % comando para redefinir la caja de la demostración

% colores utilizados
\definecolor{numchap}{RGB}{249,133,29}
\definecolor{chap}{RGB}{6,129,204}
\definecolor{sec}{RGB}{204,0,0}
\definecolor{thm}{RGB}{106,176,240}
\definecolor{thmB}{RGB}{32,31,31}
\definecolor{part}{RGB}{212,66,66}

% ====== Diseño de los titulares ===============
\usepackage[explicit]{titlesec} % para personalizar el documento, la opción <<explicit>> hace que el texto de los titulares sea un objeto interactuable

\titleformat{\subsection}[runin]
	{\bfseries}
	{\textrm{\S}\thesubsection}
	{1ex}
	{#1.}

\setlist[enumerate,1]{label=\arabic*., ref=\arabic*} % Enumerate standards

% \usepackage{diagbox}

\title{Pell y sus amigos}
% \subtitle{Roth, Thue y ¿$abc$?}
\date{15 de septiembre de 2023}

\author[José Cuevas]{José Cuevas Barrientos}
\email{josecuevasbtos@uc.cl}

\logo{../puc_negro.png}
\institution{Pontificia Universidad Católica de Chile}
\department{Facultad de Matemáticas}
\course{Teoría de Números}
\coursecode{MAT2225}
\professor{Héctor Pastén}

\begin{document}

\maketitle

\section{La ecuación de Pell}
Dado $d > 0$ libre de cuadrados, una \strong{ecuación de Pell} es una ecuación diofántica del tipo $x^2 - dy^2 = 1$
y nos interesan soluciones enteras de $x, y$, donde las soluciones \textit{triviales} son $(x, y) := (\pm 1, 0)$.
Un aspecto interesante de la ecuación de Pell es que existe una <<operación>> para construir más soluciones:
\begin{equation}
	(a, b) * (x, y) := (ax - dby, ay + bx).
	\label{eqn:pell_operation}
\end{equation}

\begin{enumerate}
	\item Considere la ecuación de Pell negativa dada por
		\[
			\mathcal{P}_-\colon \quad x^2 - dy^2 = -1.
		\]
		\begin{enumerate}
			\item Demuestre que la operación \eqref{eqn:pell_operation} aplicada a dos soluciones de $\mathcal{P}_-$
				nos da una solución de la ecuación de Pell usual (con $+1$).
			\item No obstante, esta condición no es muy buena.
				Encuentre algún $d$ para el cual la ecuación de Pell negativa no posee soluciones.
			\item Más generalmente, nótese que si $u^2 + dv^2 = a$ y $x^2 + dy^2 = b$, entonces
				$$ (ux - dvy)^2 + d(vx + uy)^2 = ab. $$
				De modo que las ecuaciones de Pell generalizadas $x^2 + dy^2 = b$ pueden o no tener soluciones o tener infinitas.
		\end{enumerate}
	\item Sea $d$ libre de cubos.
		Demuestre que la ecuación $x^n - dy^n = 1$ con $n \ge 3$ admite a lo sumo finitas soluciones.
	\item La \strong{ecuación de Markoff}\index{ecuación!de Markoff} está dada por
		\begin{equation}
			x^2 + y^2 + z^2 = 3xyz,
			\label{eqn:markoff}
		\end{equation}
		donde la \textit{solución trivial} es $(x, y, z) = (0, 0, 0)$.
		Demuestre lo siguiente:
		\begin{enumerate}
			\item Dada una solución $(a, b, c)$ de \eqref{eqn:markoff}, entonces $(a, b, 3ab - c)$ es otra solución.
			\item Todas las soluciones positivas no triviales a la ecuación de Markoff están generadas por $(1, 1, 1)$
				empleando la regla del inciso anterior.
		\end{enumerate}
\end{enumerate}

% \section{<<Mejor>> perspectiva}
\textbf{<<Mejor>> perspectiva.}
¿Por qué la ecuación de Pell tiene una estructura de grupo asociada?
Quizá algo que desvele un poco el misterio está en trabajar en $\Z[ \sqrt{d} ] = \{ x + y\sqrt{d} : x, y\in\Z \} \subseteq \C$.
Aquí la ecuación de Pell se revela como que
$$ 1 = x^2 - dy^2 = (x - y\sqrt{d})(x + y\sqrt{d}) =: \galnorm(x + y\sqrt{d}). $$
Nótese que si $\alpha, \beta \in \Z[ \sqrt{d} ]$ entonces $\galnorm(\alpha \beta) = \galnorm(\alpha) \galnorm(\beta)$ (esto es el contenido del ejercicio 1.$c$).

\begin{enumerate}[resume]
	\item \hard
		\textbf{Resolver la ecuación de Pell generalizada.} Sea $\omega = u + v\sqrt{d} \in \Z[ \sqrt{d} ]$ tal que $\galnorm(\omega) = 1$
		(es decir, es una solución de la ecuación de Pell) con $u, v > 0$.
		Demuestre que toda solución a la ecuación de Pell generalizada $x^2 - dy^2 = n$ es de la forma $(a + b\sqrt{d}) \omega^2$ donde
		$$ |a| \le \frac{ \sqrt{|n|}(\sqrt{\omega} + 1) }{2},
		\qquad |b| \le \frac{ \sqrt{|n|}(\sqrt{\omega} + 1) }{2\sqrt{d}}. $$
		\begin{enumerate}
			\item Para el primer paso defina $L(x + y\sqrt{d}) := ( \log|x + y\sqrt{d}|, \log|x - y\sqrt{d}| )$
				y verifique que $L(\alpha \beta) = L(\alpha) + L(\beta)$.
			\item Verifique que el conjunto $\{ L(\alpha) : \alpha \in \Z[ \sqrt{d} ] \}$ es cerrado bajo adición y multiplicación
				por escalares en $\Z$.
			\item Si ahora multiplicamos por $\R$, demuestre que el conjunto $\{ r \, L(\alpha) : \alpha \in \Z[\sqrt{d}], r \in \R \}$
				es un $\R$-espacio vectorial.
				Más aún, argumente el por qué es $\R^2$ y dé una base.
			\item Demuestre que una solución $x^2 - dy^2 = n$ satisface:
				$$ L(x + y \sqrt{d}) = \frac{\log|n|}{2}(1, 1) + c \, L(\omega), $$
				para algún $c \in \R$.
			\item Empleando el inciso anterior aproxime adecuadamente $L((x + y\sqrt{d}) \omega^k)$ (osea elija un $k$ apropiado)
				para concluir el enunciado.
		\end{enumerate}
\end{enumerate}

\section{El multiverso de Pell}
Aquí incluímos problemas en los que la ecuación de Pell hace una aparición eventual y llamativa.
\begin{enumerate}[resume]
	\item Un \strong{número triangular} es aquello de la forma
		$$ n = 1 + 2 + \cdots + m = \frac{m(m + 1)}{2}. $$
		Deduzca un método para encontrar \textit{todos} los números triangulares que son cuadrados perfectos.

		\begin{hint}
			Para agilizar calculos, le adelantamos que los dos primeros números triangulares que son cuadrados son 1 y
			\begin{equation}
				36 = 1 + 2 + 3 + \cdots + 8 = \frac{8 \cdot 9}{2}.
				\tqedhere
			\end{equation}
		\end{hint}

	\item Demuestre que existen infinitas ternas de números consecutivos $(a, a+1, a+2)$ tales que cada uno es suma de dos cuadrados.
		Encuentre alguna terna distinta de $(0, 1, 2)$.
\end{enumerate}

\section{Comentarios adicionales}
La escritura matemática dedicada al estudio de la ecuación de Pell es bastante rica.
Comencemos por un tecnicismo: es probable que el nombre <<ecuación de Pell>> sea poco acertado pues Pell no aportó mucho a las ecuaciones,
y aquellas que sí trató tampoco eran ninguna hazaña;
al parecer el nombre fue popularizado por Euler, quien lo confundió al leer el libro de Álgebra de Wallis, quien cita varias veces a Pell.
Históricamente los griegos ya conocían casos particulares de la ecuación de Pell, Teón de Esmirna trató la ecuación $x^2 - 2y^2 = 1$ y
descubrió su operación en este caso; y Diofanto describió las ecuaciones con $d \in \{ 26, 30 \}$.
Luego de ellos, el indio Brahmagupta descubrió en toda su generalidad la operación subyacente y otros matemáticos indios descubrieron el método
\textit{chakravala} para encontrar eficientemente soluciones.
Finalmente, cabe destacar el aporte de Fermat al demostrar que las ecuaciones de Pell siempre admiten alguna solución no trivial que,
aplicando la operación, induce la existencia de infinitas soluciones.
Algunos le llaman a la ecuación <<de Pell-Fermat>>, pero tratar de cambiarle el nombre sería como tratar de cambiarle el nombre América porque Vespucio
no fue el primer conquistador en llegar.

\begin{itemize}
	\item ¿Cómo se encuentra eficientemente una solución no trivial de la ecuación de Pell?
		Inmediatamante notamos que la solución satisface
		\[
			\left( \frac{x}{y} - \sqrt{d} \right)\left( \frac{x}{y} + \sqrt{d} \right) = \frac{x^2}{y^2} - d = \frac{1}{y^2},
		\]
		de modo que $\pm x$ satisface que $\left| \frac{x}{y} - \sqrt{d} \right| \le \frac{1}{y^2}$.
		Es decir, una solución de Pell está relacionada con una buena aproximación de $\sqrt{d}$ y, coincidentalmente, la teoría de las fracciones continuas
		ofrece tales aproximaciones.
		Para más información revise \citeauthor{granville:masterclass}~\cite[427-435]{granville:masterclass}, \S11B.

	\item La demostración del ejercicio 3 puede mejorarse ligeramente a
		$$ |a| \le \frac{ \sqrt{|n|}(\sqrt{\omega} + 1/\sqrt{\omega})}{2}, \qquad |b| \le \frac{ \sqrt{|n|}(\sqrt{\omega} + 1/\sqrt{\omega})}{2\sqrt{d}}; $$
		véase \citeauthor{conrad:pell}~\cite{conrad:pell}.

	\item El ejercicio 4 es parte de una pregunta diofantina más general: ¿cuándo es un coeficiente binomial un cuadrado?
		Los casos triviales son $\binom{n}{0} = 1^2$, y $\binom{n^2}{1} = n^2$.
		Es fácil notar que uno no pierde generalidad estudiando $\binom{n + k}{k}$ con $k \le n$, en este contexto nosotros estudiamos $\binom{n + 2}{2}$.
		El caso $\binom{n + 3}{3}$ induce una curva elíptica tras cambios de coordenadas y uno puede probar que $\binom{n + j}{j}$ con $j \ge 4$
		solo admite finitas soluciones.
\end{itemize}

\nocite{granville:masterclass}
\printbibliography[title={Referencias y lecturas adicionales}]

\end{document}
