\documentclass[11pt, reqno]{amsart}

\usepackage[spanish]{babel}
\usepackage[LGR, T1]{fontenc}
\usepackage[utf8]{inputenc}

\input{../general.tex}
% \input{../graphics.tex}

\makeatletter
\def\emailaddrname{\textit{Correo electrónico}}
\def\subtitle#1{\gdef\@subtitle{#1}}
\def\@subtitle{}

% Metadata
\def\logo#1{\gdef\@logo{#1}}
\def\@logo{}
\def\institution#1{\gdef\@institution{#1}}
\def\@institution{}
\def\department#1{\gdef\@department{#1}}
\def\@department{}
\def\professor#1{\gdef\@professor{#1}}
\def\@professor{}
\def\course#1{\gdef\@course{#1}}
\def\@course{}
\def\coursecode#1{\gdef\@coursecode{#1}}
\def\@coursecode{}

\renewcommand{\maketitle}{
\begin{center}
	\small
	\renewcommand{\arraystretch}{1.2}
	\begin{tabular}{cp{.37\textwidth}p{0.44\textwidth}}
		% \hline
		\multirow{5}{*}{\includegraphics[height=2.0cm]{\@logo}}
	  & \multicolumn{2}{c}{ \makecell{{\bfseries \@institution} \\ \@department} } \\
	  % & \multicolumn{2}{c|}{{\bfseries\@institution} \\ \@department} \\
	  \cline{2-3}
	  & \textbf{Profesor:} \@professor & \textbf{Ayudante:} \authors \\
	  % \cline{2-3}
	  & \textbf{Curso:} \@course & \textbf{Sigla:} \@coursecode \\
	  % \cline{2-3}
	  & \multicolumn{2}{l}{ \textbf{Fecha:} \@date } \\
	  % \hline
	\end{tabular}
	\\[\baselineskip]
	% {}
	% \vspace{2\baselineskip}
	{\bfseries\Large\@title}
	\ifx\@subtitle\@empty\else
		\\[1ex]
		\large\mdseries\@subtitle
	\fi
\end{center}
}
\makeatother

\usepackage{multirow, makecell}

\usepackage[
	reversemp,
	letterpaper,
	% marginpar=2cm,
	% marginsep=1pt,
	margin=2.3cm
]{geometry}
\usepackage{fontawesome}
% \makeatletter
% \@reversemargintrue
% \makeatother

% Símbolos al margen, necesitan doble compilación
\newcommand{\hard}{\marginnote{\faFire}}
\newcommand{\hhard}{\marginnote{\faFire\faFire}}

% Dependencias para los teoremas
\usepackage{xifthen}
\def\@thmdep{}
\newcommand{\thmdep}[1]{
	\ifthenelse{\isempty{#1}}
	{\def\@thmdep{}}
	{\def\@thmdep{ (#1)}}
}
\newcommand{\thmstyle}{\color{thm}\sffamily\bfseries}

% ===== Estilos de Teoremas ==========
\newtheoremstyle{axiomstyle}
	{0.3cm}
	{0.3cm}
	{\normalfont}
	{0.5cm}
	{\bfseries\scshape}
	{:}
	{4pt}
	{\thmname{#1}\thmnote{ #3}\thmnumber{ (#2)}}
\newtheoremstyle{styleC}
	{0.5cm}
	{0.5cm}
	{\normalfont}
	{0.5cm}
	{\bfseries}
	{:}
	{4pt}
	{\thmname{#1\textrm{\@thmdep}}\thmnumber{ #2}\thmnote{ (#3)}}

% ====== Teoremas (sin borde) ===========
\theoremstyle{axiomstyle}
\newtheorem*{axiom}{Axioma}

% ====== Teoremas (sin borde) ==================
\theoremstyle{styleC}
\newtheorem{thm}{Teorema}[section]
\newtheorem{mydef}[thm]{Definición}
\newtheorem{prop}[thm]{Proposición}
\newtheorem{cor}[thm]{Corolario}
\newtheorem{lem}[thm]{Lema}
\newtheorem{con}[thm]{Conjetura}

\newtheorem*{prob}{Problema}
\newtheorem*{sol}{Solución}
\newtheorem*{obs}{Observación}
\newtheorem*{ex}{Ejemplo}

% \usepackage{tcolorbox}
% \newtcbox{bluebox}[1][]{enhanced jigsaw, 
%   sharp corners,
%   frame hidden,
%   nobeforeafter,
%   listing only,
%   #1} % comando para crear cajas de colores

\expandafter\let\expandafter\oldproof\csname\string\proof\endcsname
\let\oldendproof\endproof
\renewenvironment{proof}[1][\proofname]{%
  \oldproof[\scshape Demostración:]%
}{\oldendproof} % comando para redefinir la caja de la demostración
\newenvironment{hint}[1][\proofname]{%
  \oldproof[\scshape Pista:]%
}{\oldendproof} % comando para redefinir la caja de la demostración

% colores utilizados
\definecolor{numchap}{RGB}{249,133,29}
\definecolor{chap}{RGB}{6,129,204}
\definecolor{sec}{RGB}{204,0,0}
\definecolor{thm}{RGB}{106,176,240}
\definecolor{thmB}{RGB}{32,31,31}
\definecolor{part}{RGB}{212,66,66}

% ====== Diseño de los titulares ===============
\usepackage[explicit]{titlesec} % para personalizar el documento, la opción <<explicit>> hace que el texto de los titulares sea un objeto interactuable

\titleformat{\subsection}[runin]
	{\bfseries}
	{\textrm{\S}\thesubsection}
	{1ex}
	{#1.}

\setlist[enumerate,1]{label=\arabic*., ref=\arabic*} % Enumerate standards

% \usepackage{diagbox}

\title{Principio local-global}
% \subtitle{Roth, Thue y ¿$abc$?}
\date{29 de septiembre de 2023}

\author[José Cuevas]{José Cuevas Barrientos}
\email{josecuevasbtos@uc.cl}

\logo{../puc_negro.png}
\institution{Pontificia Universidad Católica de Chile}
\department{Facultad de Matemáticas}
\course{Teoría de Números}
\coursecode{MAT2225}
\professor{Héctor Pastén}

\begin{document}

\maketitle

\section{Reciprocidad cuadrática}
Comenzaremos por presentar unos ejercicios básicos de reciprocidad cuadrática para poder sacarle mayor provecho a los resultados del principio local-global.
\begin{mydef}
	Sea $n > 0$ un entero.
	Una \strong{raíz primitiva módulo $n$} es un número $g$ coprimo a $n$,
	tal que para todo número coprimo $a$ a $n$ existe algún $m > 0$ tal que $g^m \equiv a \pmod n$.
\end{mydef}
Otra manera de verlo es que el conjunto $U_n := (\Z/n\Z)^\times$ unidades módulo $n$ está formado precisamente por las clases de congruencia coprimas con $n$
y que es un grupo con el producto.
Una raíz primitiva es, entonces, un generador de $U_n$.

\begin{enumerate}
	\item Demuestre las siguientes afirmaciones:
		\begin{enumerate}
			\item Para todo primo $p$ existe una raíz primitiva módulo $p$.
				\begin{hint}
					Emplee el pequeño teorema de Fermat.
				\end{hint}
			\item Si $g$ es una raíz primitiva módulo $p$, entonces $g$ o $g + p$ es una raíz primitiva módulo $p^2$.
			\item No obstante, no todos los enteros admiten una raíz primitiva módulo $n$; de un ejemplo.
		\end{enumerate}
\end{enumerate}

\begin{mydef}
	Sea $g$ una raíz primitiva módulo $n$.
	Dado $a$ coprimo con $n$ defina%
	\footnote{A veces se denota $\log_g$ y se llama \textit{logaritmo discreto}.}
	$\ind_g(a) := m$ como el mínimo natural (incluyendo $m = 0$) tal que $g^a \equiv 1 \pmod n$.
\end{mydef}
\begin{enumerate}[resume]
	\item Demuestre que si $g_1, g_2$ son dos raíces primitivas módulo $n > 2$,
		entonces para todo $h$ coprimo con $n$, se cumple que $\ind_{g_1}(h), \ind_{g_2}(h)$ tienen igual paridad.
\end{enumerate}

\begin{mydef}
	Sea $p$ un número primo y sea $g$ una raíz primitiva módulo $p$.
	Se define el \strong{símbolo de Legendre} como:
	$$ \left( \frac{a}{p} \right) :=
	\begin{cases}
		0, & p \mid a, \\
		(-1)^{\ind_g(a)}, & p \nmid a.
	\end{cases} $$
\end{mydef}
El ejercicio anterior prueba que el símbolo de Legendre está bien definido.

\begin{enumerate}[resume]
	\item Sea $p$ un número primo. Demuestre las siguientes:
		\begin{enumerate}
			\item Para un número $a$ coprimo a $n$ tenemos que $(a/p) = 1$ syss existe un número $h$ tal que $h^2 \equiv a \pmod p$.
				En este caso, decimos que $a$ es un \strong{residuo cuadrático módulo $p$}.
			\item Para $a, b$ coprimos a $n$ se tiene que $(ab/p) = (a/p) (b/p)$.
			\item Demuestre que si $p \equiv 1 \pmod 4$, entonces $-1$ es un residuo cuadrático módulo $p$.
		\end{enumerate}
\end{enumerate}

\section{Principio local-global}
% Sea $f(\vec x) = 0$ una ecuación diofántica.
\begin{mydef}
	Una \strong{H-solución módulo $p$} de $f(x) = 0$ es un entero $a$ tal que $f(a) \equiv 0 \pmod p$, pero $f'(a) \not\equiv 0 \pmod p$;
	donde $f'(x)$ es la derivada (formal) de $f(x)$.
\end{mydef}

Un \strong{principio local-global} es un criterio bajo el cual una determinada ecuación diofántica $f(x_1, \dots, x_n) = 0$,
donde $f(\vec x)$ tiene coeficientes en $\Z$, tiene soluciones enteras si tiene soluciones en $\R$ y tiene H-soluciones módulo $p$
para todo $p$.
Recuérdese:
\begin{thm}[Hasse-Minkowski]
	Las formas cuadráticas satisfacen un principio local-global.
	% Vale decir, dado un polinomio $f(x_1, \dots, x_n) = 0$ 
\end{thm}

\begin{enumerate}[resume]
	\item 
		\begin{enumerate}
			\item Empleando el principio local-global demuestre que todo primo $p \equiv 1 \pmod 4$ es suma de dos cuadrados.
				(Aquí puede asumir que la existencia de una solución racional implica la existencia de una solución entera.)
			\item Empleando la identidad
				\begin{equation}
					(a^2 + b^2)(c^2 + d^2) = (ac - bd)^2 + (ad + bc)^2,
					\label{eqn:two_squares}
				\end{equation}
				concluya una mejor clase de números que se pueden escribir como suma de dos cuadrados.
		\end{enumerate}
	\item Demuestre que el principio local-global falla en la ecuación
		$$ (x^2 - 2)(x^2 - 17)(x^2 - 34) = 0, $$
		es decir, que admite soluciones reales y H-soluciones para todo $p$, pero no admite soluciones racionales.
	\item Demuestre que el principio local-global falla en la ecuación $x^2 + 11y^2 = 3$.
\end{enumerate}

\section{Comentarios adicionales}
La expresión <<H-solución>> es de mi autoría y está allí para referenciar un uso escondido del \textit{lema de Hensel}.
Una versión más precisa del principio local-global es que relaciona la existencia de soluciones en los enteros $p$-ádicos $\Z_p$ con soluciones enteras.
Una buena introducción, a mi juicio, yace en el libro \citeauthor{cassels:local_fields}~\cite{cassels:local_fields}.

El ejemplo de una \textit{forma} (i.e., polinomio homogéneo) aguda que contradice el principio local-global es el \strong{ejemplo de Selmer}:
$$ 3x^3 + 4y^3 + 5z^3 = 0, $$
pero la demostración es más larga e involucra un mejor manejo del lema de Hensel (cfr. \citeauthor{conrad:local_global}~\cite{conrad:local_global}).

El paso de <<existe solución racional>> a <<existe solución entera>> puede formalizarse mejor con el siguiente resultado:
\begin{lem}[Davenport-Cassels]
	Sea $a \in \Z$ tal que la ecuación $a = x_1^2 + \cdots + x_n^2$ tiene soluciones en $\Q$.
	Entonces la misma ecuación tiene soluciones en $\Z$.
\end{lem}
No obstante, se suele escribir este resultado para $n = 3$ ya que para $n = 4$ es un teorema de Legendre que la ecuación siempre admite solución;
para $n = 1$ es trivial y para $n = 2$ es conocida la clasificación de los enteros que son sumas de dos cuadrados.
La demostración se puede ver en \citeauthor{rajwade:squares}~\cite{rajwade:squares}.

Finalmente, la identidad~\eqref{eqn:two_squares} pierde cierto misterio si pensamos $a^2 + b^2 = |a + \ui b|^2$, donde $\ui = \sqrt{-1}$.
Si hacemos el mismo juego en los cuaterniones y en los octoniones, obtendremos identidades parecidas para sumas de 4 y de 8 cuadrados resp.;
la busqueda de estas identidades es un problema interesante conocido como el \textit{problema de Hurwitz}, ya resuelto y expuesto de manera elemental
en \cite{rajwade:squares}.

\nocite{cassels:local_fields, conrad:local_global, rajwade:squares}
\printbibliography[title={Referencias y lecturas adicionales}]

\end{document}
