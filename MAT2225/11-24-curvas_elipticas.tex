\documentclass[11pt, reqno]{amsart}

\usepackage[spanish]{babel}
\usepackage[LGR, T1]{fontenc}
\usepackage[utf8]{inputenc}

../LaTeX/general.tex
% \input{../graphics.tex}

\makeatletter
\def\emailaddrname{\textit{Correo electrónico}}
\def\subtitle#1{\gdef\@subtitle{#1}}
\def\@subtitle{}

% Metadata
\def\logo#1{\gdef\@logo{#1}}
\def\@logo{}
\def\institution#1{\gdef\@institution{#1}}
\def\@institution{}
\def\department#1{\gdef\@department{#1}}
\def\@department{}
\def\professor#1{\gdef\@professor{#1}}
\def\@professor{}
\def\course#1{\gdef\@course{#1}}
\def\@course{}
\def\coursecode#1{\gdef\@coursecode{#1}}
\def\@coursecode{}

\renewcommand{\maketitle}{
\begin{center}
	\small
	\renewcommand{\arraystretch}{1.2}
	\begin{tabular}{cp{.37\textwidth}p{0.44\textwidth}}
		% \hline
		\multirow{5}{*}{\includegraphics[height=2.0cm]{\@logo}}
	  & \multicolumn{2}{c}{ \makecell{{\bfseries \@institution} \\ \@department} } \\
	  % & \multicolumn{2}{c|}{{\bfseries\@institution} \\ \@department} \\
	  \cline{2-3}
	  & \textbf{Profesor:} \@professor & \textbf{Ayudante:} \authors \\
	  % \cline{2-3}
	  & \textbf{Curso:} \@course & \textbf{Sigla:} \@coursecode \\
	  % \cline{2-3}
	  & \multicolumn{2}{l}{ \textbf{Fecha:} \@date } \\
	  % \hline
	\end{tabular}
	\\[\baselineskip]
	% {}
	% \vspace{2\baselineskip}
	{\bfseries\Large\@title}
	\ifx\@subtitle\@empty\else
		\\[1ex]
		\large\mdseries\@subtitle
	\fi
\end{center}
}
\makeatother

\usepackage{multirow, makecell}

\usepackage[
	reversemp,
	letterpaper,
	% marginpar=2cm,
	% marginsep=1pt,
	margin=2.3cm
]{geometry}
\usepackage{fontawesome}
% \makeatletter
% \@reversemargintrue
% \makeatother

% Símbolos al margen, necesitan doble compilación
\newcommand{\hard}{\marginnote{\faFire}}
\newcommand{\hhard}{\marginnote{\faFire\faFire}}

% Dependencias para los teoremas
\usepackage{xifthen}
\def\@thmdep{}
\newcommand{\thmdep}[1]{
	\ifthenelse{\isempty{#1}}
	{\def\@thmdep{}}
	{\def\@thmdep{ (#1)}}
}
\newcommand{\thmstyle}{\color{thm}\sffamily\bfseries}

% ===== Estilos de Teoremas ==========
\newtheoremstyle{axiomstyle}
	{0.3cm}
	{0.3cm}
	{\normalfont}
	{0.5cm}
	{\bfseries\scshape}
	{:}
	{4pt}
	{\thmname{#1}\thmnote{ #3}\thmnumber{ (#2)}}
\newtheoremstyle{styleC}
	{0.5cm}
	{0.5cm}
	{\normalfont}
	{0.5cm}
	{\bfseries}
	{:}
	{4pt}
	{\thmname{#1\textrm{\@thmdep}}\thmnumber{ #2}\thmnote{ (#3)}}

% ====== Teoremas (sin borde) ===========
\theoremstyle{axiomstyle}
\newtheorem*{axiom}{Axioma}

% ====== Teoremas (sin borde) ==================
\theoremstyle{styleC}
\newtheorem{thm}{Teorema}[section]
\newtheorem{mydef}[thm]{Definición}
\newtheorem{prop}[thm]{Proposición}
\newtheorem{cor}[thm]{Corolario}
\newtheorem{lem}[thm]{Lema}
\newtheorem{con}[thm]{Conjetura}

\newtheorem*{prob}{Problema}
\newtheorem*{sol}{Solución}
\newtheorem*{obs}{Observación}
\newtheorem*{ex}{Ejemplo}

% \usepackage{tcolorbox}
% \newtcbox{bluebox}[1][]{enhanced jigsaw, 
%   sharp corners,
%   frame hidden,
%   nobeforeafter,
%   listing only,
%   #1} % comando para crear cajas de colores

\expandafter\let\expandafter\oldproof\csname\string\proof\endcsname
\let\oldendproof\endproof
\renewenvironment{proof}[1][\proofname]{%
  \oldproof[\scshape Demostración:]%
}{\oldendproof} % comando para redefinir la caja de la demostración
\newenvironment{hint}[1][\proofname]{%
  \oldproof[\scshape Pista:]%
}{\oldendproof} % comando para redefinir la caja de la demostración

% colores utilizados
\definecolor{numchap}{RGB}{249,133,29}
\definecolor{chap}{RGB}{6,129,204}
\definecolor{sec}{RGB}{204,0,0}
\definecolor{thm}{RGB}{106,176,240}
\definecolor{thmB}{RGB}{32,31,31}
\definecolor{part}{RGB}{212,66,66}

% ====== Diseño de los titulares ===============
\usepackage[explicit]{titlesec} % para personalizar el documento, la opción <<explicit>> hace que el texto de los titulares sea un objeto interactuable

\titleformat{\subsection}[runin]
	{\bfseries}
	{\textrm{\S}\thesubsection}
	{1ex}
	{#1.}

\setlist[enumerate,1]{label=\arabic*., ref=\arabic*} % Enumerate standards

% \DeclareFieldFormat[book]{title}{\textit{#1}\addperiod}

\title{Curvas elípticas}
% \subtitle{Guía para el autoestopista diofántico}
\date{24 de noviembre de 2023}

\DeclareMathOperator{\PGL}{PGL}
\DeclareMathOperator{\Per}{Per}

\author[José Cuevas]{José Cuevas Barrientos}
\email{josecuevasbtos@uc.cl}

\logo{../puc_negro.png}
\institution{Pontificia Universidad Católica de Chile}
\department{Facultad de Matemáticas}
\course{Teoría de Números}
\coursecode{MAT2225}
\professor{Héctor Pastén}

\begin{document}

\maketitle

\section{Con Sage}
Una \strong{curva elíptica} $E$ sobre un cuerpo $k$ admite varias definiciones equivalentes:
\begin{enumerate}
	\item Es una curva proyectiva, suave, con un punto $k$-racional isomorfa a una curva de la forma $\VV_+(f) \subseteq \PP^2(k)$,
		donde $f$ es homogéneo de grado 3.
	\item Es una subvariedad suave de $\PP^2(k)$ dada por una ecuación de Weierstrass:
		$$ y^2z + a_1xyz = x^3 + a_2x^2z + a_4xz^2 + a_6z^3. $$
	\item Es una curva proyectiva cuyos puntos $K$-racionales forman un grupo abeliano
		(en particular, $E(K)$ debe ser no vacío, pues los grupos son no vacíos).
\end{enumerate}
\begin{thm}[Mordell-Weil, 1929]
	Sea $E$ una curva elíptica sobre $\Q$.
	Entonces su grupo de puntos racionales $E(\Q)$ es finitamente generado.
\end{thm}

\begin{mydef}
	Un elemento $g$ de un grupo abeliano $G$ se dice \strong{de torsión} si $g + \cdots + g = n\cdot g = 0$ para algún $n \ge 1$.
\end{mydef}
El siguiente criterio es útil:
\begin{thm}[débil de Lutz-Nagell, 1937]
	Sea $E$ una curva elíptica sobre $\Q$ dada por una ecuación de Weierstrass en $\PP^2(\Q)$.
	Todos los puntos racionales de torsión tienen coordenadas enteras.
\end{thm}

\begin{enumerate}
	% \item \textbf{Cotas débiles de Hasse-Weil:}
	% 	Sea $p$ un primo\footnotemark{} que se escribe de la forma $p = a^2 + 3b^2$, donde $a \equiv 2 \pmod 2$, y sea $g$ una raíz primitiva
	% 	módulo $p$ (vid. ayudantía del 29 de sept.).
	% 	\footnotetext{Por la teoría de formas cuadráticas, todo primo $p \equiv 1 \pmod 3$ admite susodicha representación.}
	% 	Considere la curva afín $E_\ell \colon x^2 = y^3 + \ell$, donde $p \nmid \ell$, y llamamemos $N_p(\ell)$ a la cantidad de puntos
	% 	de $E_\ell$.
	% 	\begin{enumerate}
	% 		\item Definiendo
	% 			$$ S_\ell = \sum_{n=1}^{p} \binom{n^3 + \ell}{p}, $$
	% 			donde $\binom{0}{p} := 0$.
	% 			Demuestre que $N_p(\ell) = p + S_\ell$.
	% 		\item Sea $L := \ell r^3$ y defínase $T_\ell := \binom{\ell}{p} S_\ell$.
	% 			Demuestre que $T_L \equiv T_\ell$.
	% 	\end{enumerate}
	\item Encuentre una fórmula para todas las sucesiones de tres cuadrados en progresión aritmética.
	\item Demuestre que no hay cuatro cuadrados en progresión aritmética.
	\item Encuentre todas las sucesiones de tres cubos coprimos en progresión aritmética.
\end{enumerate}

\section{Sin Sage}
\begin{enumerate}[resume]
	\item Encuentre una ecuación de Weierstrass para la curva elíptica $E \colon x^3 + y^3 = z^3$.
	\item Demuestre que la curva elíptica $y^2 = x^3 + x^2 + 4$ tiene infinitas soluciones.
\end{enumerate}

\nocite{silverman:elliptic}
\printbibliography[title={Referencias y lecturas adicionales}]

\end{document}
