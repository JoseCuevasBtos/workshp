\usepackage[spanish]{babel}
\usepackage[LGR, T1]{fontenc}
\usepackage[utf8]{inputenc}

../LaTeX/general.tex
% \input{../graphics.tex}

\makeatletter
\def\emailaddrname{\textit{Correo electrónico}}
\def\subtitle#1{\gdef\@subtitle{#1}}
\def\@subtitle{}

% Metadata
\def\logo#1{\gdef\@logo{#1}}
\def\@logo{}
\def\institution#1{\gdef\@institution{#1}}
\def\@institution{}
\def\department#1{\gdef\@department{#1}}
\def\@department{}
\def\professor#1{\gdef\@professor{#1}}
\def\@professor{}
\def\course#1{\gdef\@course{#1}}
\def\@course{}
\def\coursecode#1{\gdef\@coursecode{#1}}
\def\@coursecode{}

\renewcommand{\maketitle}{
\begin{center}
	\small
	\renewcommand{\arraystretch}{1.2}
	\begin{tabular}{cp{.37\textwidth}p{0.44\textwidth}}
		% \hline
		\multirow{5}{*}{\includegraphics[height=2.0cm]{\@logo}}
	  & \multicolumn{2}{c}{ \makecell{{\bfseries \@institution} \\ \@department} } \\
	  % & \multicolumn{2}{c|}{{\bfseries\@institution} \\ \@department} \\
	  \cline{2-3}
	  & \textbf{Profesor:} \@professor & \textbf{Ayudante:} \authors \\
	  % \cline{2-3}
	  & \textbf{Curso:} \@course & \textbf{Sigla:} \@coursecode \\
	  % \cline{2-3}
	  & \multicolumn{2}{l}{ \textbf{Fecha:} \@date } \\
	  % \hline
	\end{tabular}
	\\[\baselineskip]
	% {}
	% \vspace{2\baselineskip}
	{\bfseries\Large\@title}
	\ifx\@subtitle\@empty\else
		\\[1ex]
		\large\mdseries\@subtitle
	\fi
\end{center}
}
\makeatother

\usepackage{multirow, makecell}

\usepackage[
	reversemp,
	letterpaper,
	% marginpar=2cm,
	% marginsep=1pt,
	margin=2.3cm
]{geometry}
\usepackage{fontawesome}
% \makeatletter
% \@reversemargintrue
% \makeatother

% Símbolos al margen, necesitan doble compilación
\newcommand{\hard}{\marginnote{\faFire}}
\newcommand{\hhard}{\marginnote{\faFire\faFire}}

% Dependencias para los teoremas
\usepackage{xifthen}
\def\@thmdep{}
\newcommand{\thmdep}[1]{
	\ifthenelse{\isempty{#1}}
	{\def\@thmdep{}}
	{\def\@thmdep{ (#1)}}
}
\newcommand{\thmstyle}{\color{thm}\sffamily\bfseries}

% ===== Estilos de Teoremas ==========
\newtheoremstyle{axiomstyle}
	{0.3cm}
	{0.3cm}
	{\normalfont}
	{0.5cm}
	{\bfseries\scshape}
	{:}
	{4pt}
	{\thmname{#1}\thmnote{ #3}\thmnumber{ (#2)}}
\newtheoremstyle{styleC}
	{0.5cm}
	{0.5cm}
	{\normalfont}
	{0.5cm}
	{\bfseries}
	{:}
	{4pt}
	{\thmname{#1\textrm{\@thmdep}}\thmnumber{ #2}\thmnote{ (#3)}}

% ====== Teoremas (sin borde) ===========
\theoremstyle{axiomstyle}
\newtheorem*{axiom}{Axioma}

% ====== Teoremas (sin borde) ==================
\theoremstyle{styleC}
\newtheorem{thm}{Teorema}[section]
\newtheorem{mydef}[thm]{Definición}
\newtheorem{prop}[thm]{Proposición}
\newtheorem{cor}[thm]{Corolario}
\newtheorem{lem}[thm]{Lema}
\newtheorem{con}[thm]{Conjetura}
\newtheorem*{sol}{Solución}
\newtheorem*{obs}{Observación}
\newtheorem*{ex}{Ejemplo}

\newtcbox{bluebox}[1][]{enhanced jigsaw, 
  sharp corners,
  frame hidden,
  nobeforeafter,
  listing only,
  #1} % comando para crear cajas de colores

\expandafter\let\expandafter\oldproof\csname\string\proof\endcsname
\let\oldendproof\endproof
\renewenvironment{proof}[1][\proofname]{%
  \oldproof[\scshape Demostración:]%
}{\oldendproof} % comando para redefinir la caja de la demostración
\newenvironment{hint}[1][\proofname]{%
  \oldproof[\scshape Pista:]%
}{\oldendproof} % comando para redefinir la caja de la demostración

% colores utilizados
\definecolor{numchap}{RGB}{249,133,29}
\definecolor{chap}{RGB}{6,129,204}
\definecolor{sec}{RGB}{204,0,0}
\definecolor{thm}{RGB}{106,176,240}
\definecolor{thmB}{RGB}{32,31,31}
\definecolor{part}{RGB}{212,66,66}

% ====== Diseño de los titulares ===============
\usepackage[explicit]{titlesec} % para personalizar el documento, la opción <<explicit>> hace que el texto de los titulares sea un objeto interactuable

\titleformat{\subsection}[runin]
	{\bfseries}
	{\textrm{\S}\thesubsection}
	{1ex}
	{#1.}

\setlist[enumerate,1]{label=\arabic*., ref=\arabic*} % Enumerate standards
